\documentclass[a4class]{report}
\usepackage[T1]{fontenc}
\usepackage[utf8]{inputenc}
\usepackage[italian]{babel}
\usepackage{booktabs}%package per la gestione delle Tabelle
\usepackage{caption}%Package per le Tabelle
\usepackage{amsmath,amssymb}%Package per la gestione della Matematica
\setlength{\parindent}{0pt}%Per evitare i rientri dei capoversi
\newcommand{\numberset}{\mathbb}
\newcommand{\N}{\numberset{N}}
\newcommand{\R}{\numberset{R}}
\newcommand{\Q}{\numberset{Q}}



\begin{document}
\chapter{Limiti di Funzioni}
Si definisce come \textit{limite}

\begin{description}
  \item[limite finito al finito \quad $c,l \in \R$]
        $\lim{x \to c}{f(x)} = l \if \forall \epsilon \exists \gamma : \forall x \not = c\ |x-c| < \gamma \if |f(x)-l| < \epsilon$
  \item[limite infinito al finito \quad $c \in \R$]
        $\lim{x \to c}{f(x)} = \pm \infty \if \forall k>0 \exists \gamma >0 : \forall x \not =c\ |x-c|<\gamma \if f(x) > K$
  \item[limite finito all'infinito \quad $l \in \R$]
        $\lim{x \to \infty}{f(x)} = l \if \forall \epsilon>0 \exists K>0 : \forall x > K \if |f(x)-l| < \epsilon$
  \item[limite infinito all'infinito]
        $\lim{x \to \infty}{f(x)} = \infty \if \forall K>0 \exists H>0 : \forall x > H \if f(x) > K$
\end{description}

%Inserire Algebra dei Limiti con i teoremi da fare

\subsection{Asintoti}
Le funzioni possono presentare 3 tipologie di Asintoti, scopribili mediante il calcolo dei limiti come segue:
\begin{description}
  \item[Asintoto Verticale]
        Si dice che f ha un asintoto verticale di equazione $x = c c \in \R$ per $x \to c$ se $\lim{x \to c}{f(x)} = \pm \infty$
  \item[Asintoto Orizzontale]: Si dice che $f$ ha un asintoto orizzontale di equazione $y=l$
                              per $x \to \pm \infty$ se $\lim{x \to \pm \infty}{f(x)} = l$
  \item[Asintoto obliquo]Si dice che $f$ ammette asintoto obliquo di equazione $y = mx + q$
        con $m \noteq 0$ per $x \to \pm \infty$ se $\lim{x \to \pm \infty}{f(x)-(mx+q)} = 0$
        La funzione $f(x)$ ammette asintoto obliquo se e solo se valgono le seguenti condizioni:
        \begin{enumerate}
          \item esiste finito $\lim{x \to \pm \infty}{\frac{f(x)}{x}} = m \noteq 0$
          \item esiste finito $\lim{x \to \pm \infty}{f(x) - mx} = q$
        \end{enumerate}
\end{description}


\section{Continuità}
Se $f:I \mapsto \R$ e $c \in I$ si dice che $f$ è \textit{continua} in $c$ se esiste $\lim{x \to c}{f(x)} = f(c)$.
Una funzione $f$ è continua in $I$ se e solo se è continua in tutti i punti dell'intervallo.
Tutte le funzioni elementari sono continue in tutti i punti del loro dominio.
%Inserire grafici funzioni continue

%inserire tipologie punti di Discontinuità

\subsection{Algebra delle funzioni continue}
Siano $f,g$ due funzioni definite almeno in un intorno di $x_0 \in \R$ e continue in $x_0$.
\begin{enumerate}
  \item $f(x) \pm g(x)$ è continua in $x_0$
  \item $f(x)g(x)$ è continua in $x_0$
  \item $\frac{f(x)}{g(x)}$ è continua in $x_0$ con $g(x) \noteq 0$
\end{enumerate}

%Da fare la dimostrazione

%teorema Algebra delle funzioni composte
Sia $g$ una funzione, definita almeno in un intorno di $x_0$ e continua in $x_0$, e sia
$f$ una funzione definita almeno in un intorno di $g(x_0)$ e continua in $g(x_0)$, allora
$f o g$ è definita almeno in un intorno di $x_0$ ed è continua in $x_0$
%Fare la dimostrazione

%Fare tutti i teoremi sulla continuità
\end{document}
