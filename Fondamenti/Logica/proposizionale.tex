%File latex per il capitolo sulla logica proposizionale Classica
\chapter{Logica Proposizionale}

%Definizione di Logica
La logica è lo studio	del ragionamento e dell’argomentazione e, in particolare,
dei	procedimenti inferenziali, rivolti a chiarire	quali	procedimenti di pensiero
siano	validi e quali no.

Vi sono molteplici tipologie di logiche, come ad esempio la logica classica e le logiche costruttive,
tutte accomunate di essere composte da 3 elementi:

%Elementi di una Logica
\begin{itemize}
  \item \textbf{Linguaggio}:insieme di simboli utilizzati nella Logica per definire le cose
  \item \textbf{Sintassi}:insieme di regole che determina quali elementi appartengono o meno al linguaggio
  \item \textbf{Semantica}:permette di dare un significato alle formule del linguaggio e determinare
        se rappresentano o meno la verità.
\end{itemize}

Noi ci occupiamo della logica Classica che si compone in \textsc{logica proposizionale} e
\textit{logica predicativa}.

La Logica proposizionale è un tipo di logica Classica che presenta come caratteristica quella
di essere un linguaggio limitato in quanto si possono esprimere soltanto proposizioni senza
avere la possibilità di estenderla a una classe di persone.

\section{Linguaggio e Sintassi}
Il linguaggio di una logica proposizionale è composto dai seguenti elementi:

%Elementi linguaggio logica proposizionale
\begin{itemize}
  \item Variabili Proposizionali: $P,Q,R \dots$
  \item Connettivi Proposizionali: $\land, \lor, \neg, \Rightarrow, \iff$
  \item Simboli Ausiliari: (,)
  \item Costanti: $T,F$
\end{itemize}

La sintassi di un linguaggio è composta da una serie di formule ben formate($FBF$) definite
induttivamente nel seguente modo:

%definizione formule ben formate
\begin{enumerate}
  \item Le costanti e le variabili proposizionali $\in FBF$.
  \item Se $A$ e $B \in FBF$ allora $(A \land B)$,$(A \lor B)$,$(\neg A)$,$(A \Leftarrow B)$,
        $(A \iff B)$,$TA$ e $FA$ sono delle formule ben formate.
  \item nient'altro è una formula
\end{enumerate}

Esempio:\newline
$(P \land Q) \in Fbf$  è una formula ben formata\newline
$(PQ \land R) \not \in Fbf$ in quanto non si rispetta la sintassi del linguaggio definita.\newline

%Manca Definizione delle sottoformule
Sia $A \in FBF$, l'insieme delle sottoformule di $A$ è definito come segue:
\begin{enumerate}
\item Se $A$ è una costante o variabile proposizionale allora A stessa è la sua sottoformula
\item Se $A$ è una formula del tipo $(\neg A')$ allora le sottoformule di A sono A stessa e le sottoformule di $A'$;
      $\neg$ è detto connettivo principale e $A'$ sottoformula immediata di A.
\item Se $A$ è una formula del tipo $B o C$ dove $o$ è un connettivo binario della logica proposizionale e B ed C due formule;
      le sottoformule di A sono A stessa e le sottoformule di B e C;o è il connettivo principale e B e C sono le due sottoformule immediate di A.
\end{enumerate}


É possibile ridurre ed eliminare delle parentesi attraverso l'introduzione della
precedenza tra gli operatori, che è definita come segue:\newline
$\neg, \land, \lor, \Rightarrow,\iff$.

In assenza di parentesi una formula va parentizzata privileggiando le sottoformule
i cui connettivi principali hanno la precedenza più alta.\newline
In caso di parità di precedenza vi è la convenzione di associare da destra a sinistra.

Esempio:\newline
$\neg A \land (\neg B \Rightarrow C) \lor D$ diventa
$((\neg A) \land ((\neg B) \Rightarrow C) \lor D)$.

%Semantica Logica Proposizionale
\section{Semantica}
%Cercare migliore definizione di Semantica
La semantica di una logica consente di dare un significato alle formule del Linguaggio
attraverso le tabelle di verità.\newline
Si definisce $v(T) = 1$ e $v(F) = 0$ per cui $1$ rappresenta la verità mentre lo $0$
la falsità di una variabile,sottoformula e formula.

I connettivi della Logica Proposizionale hanno la seguente tabella di verità:\newline

%Tabella di Verità degli operatori
$\begin{array}{ccccccc}
\toprule
\text{A} & \text{B} & A \land B & A \lor B & \neg A & A \Rightarrow B & A \iff B \\
\midrule
    0 & 0 & 0 & 0 & 1 & 1 & 1 \\
    0 & 1 & 0 & 1 & 1 & 1 & 0 \\
    1 & 0 & 0 & 1 & 0 & 0 & 0 \\
    1 & 1 & 1 & 1 & 0 & 1 & 1 \\
\bottomrule
\end{array}$\newline

Una formula nella logica proposizionale può essere di tre diversi tipi:

%Tipologie di formule
\begin{itemize}
  \item Tautologica: la formula è soddisfatta da qualsiasi valutazione della formula
  \item Soddisfacibile non Tautologica:la formula è soddisfatta da qualche valutazione
        della formula ma non da tutte
  \item Contaddizione: la formula non viene soddisfatta da qualsiasi valutazione della formula
\end{itemize}

Esempio:\newline
Formula $A \land \neg A$ \quad contraddizione

%Tabella di Verità
$\begin{array}{ccc}
\toprule A & \neg A & A \land \neg A \\
\midrule
        0 & 1 & 0 \\
        1 & 0 & 0 \\
\bottomrule
\end{array}$\newpage

Formula $Z = (A \land B) \lor C$  soddisfacibile non Tautologica

%Tabella di Verità
$\begin{array}{ccccc}
\toprule A & B & C & A \land B & (A \land B) \lor C \\
\midrule
         0 & 0 & 0 & 0 & 0 \\
         0 & 0 & 1 & 0 & 1 \\
         0 & 1 & 0 & 0 & 0 \\
         0 & 1 & 1 & 0 & 1 \\
         1 & 0 & 0 & 0 & 0 \\
         1 & 0 & 1 & 0 & 1 \\
         1 & 1 & 0 & 1 & 1 \\
         1 & 1 & 1 & 1 & 1 \\
\bottomrule
\end{array}$\newline

Formula $X = (A \land B) \Rightarrow (\neg A \land C)$ \quad Soddisfacibile  non tautologica\newline

%Tabella di Verità della formula X
$\begin{array}{ccccccc}
\toprule A & B & C & \neg A & A \land B & \neg A \land C & X\\
\midrule
         0 & 0 & 0 & 1 & 0 & 0 & 1 \\
         0 & 0 & 1 & 1 & 0 & 1 & 1 \\
         0 & 1 & 0 & 1 & 0 & 0 & 1 \\
         0 & 1 & 1 & 1 & 0 & 1 & 1 \\
         1 & 0 & 0 & 0 & 0 & 0 & 1 \\
         1 & 0 & 1 & 0 & 0 & 0 & 1 \\
         1 & 1 & 0 & 0 & 1 & 0 & 0 \\
         1 & 1 & 1 & 0 & 1 & 0 & 0 \\
\bottomrule
\end{array}$ \newline

Formula $Y = \neg(A \land B) \iff (A \lor B \Rightarrow C)$ soddisfacibile non Tautologica

%Tabella di Verità
$\begin{array}{cccccccc}
\toprule
A & B & C & A \land B & \neg(A \land B) & A \lor B & (A \lor B) \Rightarrow C & Y \\
\midrule
0 & 0 & 0 & 0 & 1 & 0 & 1 & 1 \\
0 & 0 & 1 & 0 & 1 & 0 & 1 & 1 \\
0 & 1 & 0 & 0 & 1 & 1 & 0 & 0 \\
0 & 1 & 1 & 0 & 1 & 1 & 1 & 1 \\
1 & 0 & 0 & 0 & 1 & 1 & 0 & 0 \\
1 & 0 & 1 & 0 & 1 & 1 & 1 & 1 \\
1 & 1 & 0 & 1 & 0 & 1 & 0 & 1 \\
1 & 1 & 1 & 1 & 0 & 1 & 1 & 0 \\
\bottomrule
\end{array}$

\section{Sistema Deduttivo}
Il sistema deduttivo è un metodo di calcolo che manipola proposizioni, senza la
necessità di ricorrere ad altri aspetti della logica(nessuna necessità di ricorrere all'interpretazione).\newline
Nella logica proposizionale, tramite i teoremi di completezza e correttezza, esiste
una corrispondenza tra le formule derivanti dal sistema deduttivo e le formule verificabili
tramite la semantica della logica.

I sistemi deduttivi della logica proposizionale sono i seguenti:

\begin{itemize}
  \item \textsc{Sistema deduttivo Hilbertiano}:non viene analizzato!!!
  \item \textsc{Metodo dei Tableau}
  \item \textsc{Risoluzione Proposizionale}:non viene analizzato!!!
\end{itemize}

(Da migliorare)(inserire definizione di dimostrazione e teorema)
Una dimostrazione di una formula di una logica può venire tramite:

\begin{itemize}
  \item  \textbf{Metodo diretto}: Data un'ipotesi, attraverso una serie di passi
          si riesce a dimostrare la correttezza della Tesi
  \item \textbf{Metodo per assurdo}(non sempre accettato in tutte le logiche):
        Si nega la tesi ed attraverso una serie di passi si riesce a dimostrare
        la negazione delle ipotesi.
\end{itemize}

\subsection{Tableau Proposizionali}
Il metodo dei Tableau è stato introdotto da Hintikka e Beth alla fine degli anni '50
e poi ripresi successivamente da Smullyan.

I tableau sono degli alberi,la cui radice è l'enunciato in esame, e gli altri nodi
sono costruiti attraverso l'applicazione di una serie di regole,fino ad arrivare
alle formule atomiche come radici.

Le regole dei Tableau sono le seguenti:\newline
T $\land$

$\begin{array}{c}
\toprule
S,T A \land B \\
\midrule
S, TA, TB \\
\bottomrule
\end{array}$

T $\lor$

$\begin{array}{c}
\toprule
S,T A \lor B \\
\midrule
S, TA/ S, TB \\
\bottomrule
\end{array}$

T $\neg$

$\begin{array}{c}
\toprule
S,T \neg A \\
\midrule
S, FA \\
\bottomrule
\end{array}$

T $\Rightarrow$

$\begin{array}{c}
\toprule
S,T A \Rightarrow B \\
\midrule
S, FA / S, TB \\
\bottomrule
\end{array}$

T $\iff$(da fare)

F $\land$

$\begin{array}{c}
\toprule
S,F A \land B \\
\midrule
S, FA/ S, TB \\
\bottomrule
\end{array}$

F $\lor$

$\begin{array}{c}
\toprule
S,F A \lor B \\
\midrule
S, FA, FB \\
\bottomrule
\end{array}$

F $\neg$

$\begin{array}{c}
\toprule
S,F \neg A \\
\midrule
S, TA \\
\bottomrule
\end{array}$

F $\Rightarrow$

$\begin{array}{c}
\toprule
S,F A \Rightarrow B \\
\midrule
S, TA, FB \\
\bottomrule
\end{array}$

F $\iff$(da fare)

%Dimostrazione di tableau
Il metodo dei Tableau è un metodo dei sistemi deduttivi, che permette attraverso
l'applicazione di una serie di regole, di capire la tipologia della formula.

%Metodi per capire il tipo della Formula
\begin{tabular}{cccc}
\toprule Tipologia & Fare Tableau per & Chiuso? & Aperto? \\
\midrule
         Teorema & $\neg A$ & Si & No \\
         Soddisfacibile & $A$ & No & Si \\
         Contradditoria & $A$ & Si & No \\
\bottomrule
\end{tabular}

Esempio: \newline
Formula: $ \neg(A \lor B) \rightarrow (\neg A \land \neg B)$
%Da fare dimostrazione con il Pacchetto bussproof

%Da fare definizione di di Tableau,Tableau dimostrazione,rami chiusi ed aperti del Tableau
