%File latex per il capitolo sulla logica proposizionale Classica
\chapter{Logica Proposizionale}

%Definizione di Logica
La logica è lo studio del ragionamento e dell’argomentazione e, in particolare,
dei	procedimenti inferenziali, rivolti a chiarire quali	procedimenti di pensiero siano validi e quali no.

Vi sono molteplici tipologie di logiche, come ad esempio la logica classica e le logiche costruttive,
tutte accomunate di essere composte da 3 elementi:

%Elementi di una Logica
\begin{itemize}
  \item \textbf{Linguaggio}:insieme di simboli utilizzati nella Logica per definire le cose
  \item \textbf{Sintassi}:insieme di regole che determina quali elementi appartengono o meno al linguaggio
  \item \textbf{Semantica}:permette di dare un significato alle formule del linguaggio e determinare
        se rappresentano o meno la verità.
\end{itemize}

Noi ci occupiamo della logica Classica che si compone in \textsc{logica proposizionale} e
\textit{logica predicativa}.

La Logica proposizionale è un tipo di logica Classica che presenta come caratteristica quella
di essere un linguaggio limitato in quanto si possono esprimere soltanto proposizioni senza
avere la possibilità di estenderla a una classe di persone.

\section{Linguaggio e Sintassi}
Il linguaggio di una logica proposizionale è composto dai seguenti elementi:

%Elementi linguaggio logica proposizionale
\begin{itemize}
  \item Variabili Proposizionali: $P,Q,R \dots$
  \item Connettivi Proposizionali: $\land, \lor, \neg, \rightarrow, \iff$
  \item Simboli Ausiliari: (,)
  \item Costanti: $T,F$
\end{itemize}

La sintassi di un linguaggio è composta da una serie di formule ben formate($FBF$) definite
induttivamente nel seguente modo:
%definizione formule ben formate
\begin{enumerate}
  \item Le costanti e le variabili proposizionali $\in FBF$.
  \item Se $A$ e $B \in FBF$ allora $(A \land B)$,$(A \lor B)$,$(\neg A)$,$(A \rightarrow B)$,
        $(A \iff B)$,$TA$ e $FA$ sono delle formule ben formate.
  \item nient'altro è una formula
\end{enumerate}

%DA INSERIRE IN UN FILE  A PARTE
Esempio:\newline
$(P \land Q) \in Fbf$  è una formula ben formata\newline
$(PQ \land R) \not \in Fbf$ in quanto non si rispetta la sintassi del linguaggio definita.\newline

%Definizione delle sottoformule
Sia $A \in FBF$, l'insieme delle sottoformule di $A$ è definito come segue:
\begin{enumerate}
\item Se $A$ è una costante o variabile proposizionale allora A stessa è la sua sottoformula
\item Se $A$ è una formula del tipo $(\neg A')$ allora le sottoformule di A sono A stessa e le sottoformule di $A'$;
      $\neg$ è detto connettivo principale e $A'$ sottoformula immediata di A.
\item Se $A$ è una formula del tipo $B o C$ dove $o$ è un connettivo binario della logica proposizionale e B ed C due formule;
      le sottoformule di A sono A stessa e le sottoformule di B e C;o è il connettivo principale e B e C sono le due sottoformule immediate di A.
\end{enumerate}


É possibile ridurre ed eliminare delle parentesi attraverso l'introduzione della
precedenza tra gli operatori, che è definita come segue:\newline
$\neg, \land, \lor, \rightarrow,\iff$.

In assenza di parentesi una formula va parentizzata privileggiando le sottoformule
i cui connettivi principali hanno la precedenza più alta.\newline
In caso di parità di precedenza vi è la convenzione di associare da destra a sinistra.

Esempio:\newline
$\neg A \land (\neg B \rightarrow C) \lor D$ diventa
$((\neg A) \land ((\neg B) \rightarrow C) \lor D)$.

%Semantica Logica Proposizionale
\section{Semantica}
La semantica di una logica consente di dare un significato e un interpretazione
 alle formule del Linguaggio attraverso le tabelle di verità.\newline
Si definisce $v(T) = 1$ e $v(F) = 0$ per cui $1$ rappresenta la verità mentre lo $0$
la falsità di una variabile,sottoformula e formula.

I connettivi della Logica Proposizionale hanno la seguente tabella di verità:\newline
%Tabella di Verità degli operatori
$\begin{array}{ccccccc}
\toprule
\text{A} & \text{B} & A \land B & A \lor B & \neg A & A \Rightarrow B & A \iff B \\
\midrule
    0 & 0 & 0 & 0 & 1 & 1 & 1 \\
    0 & 1 & 0 & 1 & 1 & 1 & 0 \\
    1 & 0 & 0 & 1 & 0 & 0 & 0 \\
    1 & 1 & 1 & 1 & 0 & 1 & 1 \\
\bottomrule
\end{array}$\newline

Una formula nella logica proposizionale può essere di tre diversi tipi:
%Tipologie di formule
\begin{description}
    \item[Tautologica]: la formula è soddisfatta da qualsiasi valutazione della Formula
    \item[Soddisfacibile non Tautologica]:la formula è soddisfatta da qualche valutazione
                        della formula ma non da tutte.
    \item[Contraddizione]:la formula non viene mai soddisfatta
\end{description}

Esempio:\newline
Formula $A \land \neg A$ \quad contraddizione

%Tabella di Verità
$\begin{array}{ccc}
\toprule A & \neg A & A \land \neg A \\
\midrule
        0 & 1 & 0 \\
        1 & 0 & 0 \\
\bottomrule
\end{array}$\newpage

Formula $Z = (A \land B) \lor C$  soddisfacibile non Tautologica

%Tabella di Verità
$\begin{array}{ccccc}
\toprule A & B & C & A \land B & (A \land B) \lor C \\
\midrule
         0 & 0 & 0 & 0 & 0 \\
         0 & 0 & 1 & 0 & 1 \\
         0 & 1 & 0 & 0 & 0 \\
         0 & 1 & 1 & 0 & 1 \\
         1 & 0 & 0 & 0 & 0 \\
         1 & 0 & 1 & 0 & 1 \\
         1 & 1 & 0 & 1 & 1 \\
         1 & 1 & 1 & 1 & 1 \\
\bottomrule
\end{array}$\newline

Formula $X = (A \land B) \Rightarrow (\neg A \land C)$ \quad Soddisfacibile  non tautologica\newline

%Tabella di Verità della formula X
$\begin{array}{ccccccc}
\toprule A & B & C & \neg A & A \land B & \neg A \land C & X\\
\midrule
         0 & 0 & 0 & 1 & 0 & 0 & 1 \\
         0 & 0 & 1 & 1 & 0 & 1 & 1 \\
         0 & 1 & 0 & 1 & 0 & 0 & 1 \\
         0 & 1 & 1 & 1 & 0 & 1 & 1 \\
         1 & 0 & 0 & 0 & 0 & 0 & 1 \\
         1 & 0 & 1 & 0 & 0 & 0 & 1 \\
         1 & 1 & 0 & 0 & 1 & 0 & 0 \\
         1 & 1 & 1 & 0 & 1 & 0 & 0 \\
\bottomrule
\end{array}$ \newline

Formula $Y = \neg(A \land B) \iff (A \lor B \Rightarrow C)$ soddisfacibile non Tautologica

%Tabella di Verità
$\begin{array}{cccccccc}
\toprule
A & B & C & A \land B & \neg(A \land B) & A \lor B & (A \lor B) \Rightarrow C & Y \\
\midrule
0 & 0 & 0 & 0 & 1 & 0 & 1 & 1 \\
0 & 0 & 1 & 0 & 1 & 0 & 1 & 1 \\
0 & 1 & 0 & 0 & 1 & 1 & 0 & 0 \\
0 & 1 & 1 & 0 & 1 & 1 & 1 & 1 \\
1 & 0 & 0 & 0 & 1 & 1 & 0 & 0 \\
1 & 0 & 1 & 0 & 1 & 1 & 1 & 1 \\
1 & 1 & 0 & 1 & 0 & 1 & 0 & 1 \\
1 & 1 & 1 & 1 & 0 & 1 & 1 & 0 \\
\bottomrule
\end{array}$

\subsection{Modelli e decidibilità}
Si definisce \emph{modello}, indicato con $M \models A$, tutte le valutazioni booleane
che rendono vera la formula $A$.
Si definisce \emph{contromodello}, indicato con , tutte le valutazioni booleane
che rendono falsa la formula $A$.

La logica proposizionale è decidibile (posso sempre verificare il significato di una formula).
Esiste infatti una procedura effettiva che stabilisce la validità o no di una formula, o se questa
ad esempio è una tautologia.
In particolare il verificare se una proposizione è tautologica o meno è l’operazione di decibi-
lità principale che si svolge nel calcolo proposizonale.

\begin{defi}
    Se $M \models A$ per tutti gli $M$, allora $A$ è una tautologia e si indica $\models A$
\end{defi}

\begin{defi}
    Se $M \models A$ per qualche $M$, allora $A$ è soddisfacibile
\end{defi}

\begin{defi}
Se $M \models A$ non è soddisfatta da nessun $M$, allora $A$ è insoddisfacibile
\end{defi}

\section{Sistema Deduttivo}
Il sistema deduttivo è un metodo di calcolo che manipola proposizioni, senza la
necessità di ricorrere ad altri aspetti della logica.\newline
Nella logica proposizionale, tramite i teoremi di completezza e correttezza, esiste
una corrispondenza tra le formule derivanti dal sistema deduttivo e le formule verificabili
tramite la semantica della logica.

I sistemi deduttivi della logica proposizionale sono i seguenti:
\begin{description}
    \item[Sistema deduttivo Hilbertiano]: non viene analizzato
    \item[Metodo dei Tableau]
    \item [Risoluzione Proposizionale]:non viene analizzato !!!!
\end{description}

\begin{defi}
Una sequenza di formule $A_1,\dots,A_n$ di $\Lambda$ è una \emph{dimostrazione} se
per ogni $i$ compreso tra $1$ e $n$, $A_i$ è un assioma di $\Lambda$ oppure una
conseguenza diretta di una formula precedente.
\end{defi}

\begin{defi}
Una formula $A$ di una logica $\Lambda$ è detta \emph{teorema} di $\Lambda$,indicata
con $\vdash A$ se esiste una dimostrazione di $\Lambda$ che ha $A$ come ultima formula
\end{defi}

Una dimostrazione di una formula di una logica può venire tramite:
\begin{itemize}
  \item  \textbf{Metodo diretto}: Data un'ipotesi, attraverso una serie di passi
          si riesce a dimostrare la correttezza della Tesi
  \item \textbf{Metodo per assurdo}(non sempre accettato in tutte le logiche):
        Si nega la tesi ed attraverso una serie di passi si riesce a dimostrare
        la negazione delle ipotesi.
\end{itemize}

\begin{thm}
    Un apparato deduttivo $R$ è completo se, per ogni formula $A \in Fbf$, $\vdash A$
    implica $\models A$
\end{thm}

\begin{thm}
    Un apparato deduttivo $R$ è corretto se, per ogni formula $A \in Fbf$, $\models A$
    implica $\vdash A$
\end{thm}
\subsection{Tableau Proposizionali}
Il metodo dei Tableau è stato introdotto da Hintikka e Beth alla fine degli anni '50
e poi ripresi successivamente da Smullyan.

I tableau sono degli alberi,la cui radice è l'enunciato in esame, e gli altri nodi
sono costruiti attraverso l'applicazione di una serie di regole,fino ad arrivare
alle formule atomiche come radici.

Le regole dei Tableau sono le seguenti:\newline
T $\land$

$\begin{array}{c}
\toprule
S,T A \land B \\
\midrule
S, TA, TB \\
\end{array}$

T $\lor$

$\begin{array}{c}
\toprule
S,T A \lor B \\
\midrule
S, TA/ S, TB \\
\end{array}$

T $\neg$

$\begin{array}{c}
\toprule
S,T \neg A \\
\midrule
S, FA \\
\end{array}$

T $\Rightarrow$

$\begin{array}{c}
\toprule
S,T A \Rightarrow B \\
\midrule
S, FA / S, TB \\
\end{array}$

T $\iff$(da fare)

F $\land$

$\begin{array}{c}
\toprule
S,F A \land B \\
\midrule
S, FA/ S, TB \\
\end{array}$

F $\lor$

$\begin{array}{c}
\toprule
S,F A \lor B \\
\midrule
S, FA, FB \\
\end{array}$

F $\neg$

$\begin{array}{c}
\toprule
S,F \neg A \\
\midrule
S, TA \\
\end{array}$

F $\Rightarrow$

$\begin{array}{c}
\toprule
S,F A \Rightarrow B \\
\midrule
S, TA, FB \\
\end{array}$

F $\iff$(da fare)

%Dimostrazione di tableau
Il metodo dei Tableau è un metodo dei sistemi deduttivi, che permette attraverso
l'applicazione di una serie di regole, di capire la tipologia della formula.
%Metodi per capire il tipo della Formula
\begin{tabular}{cccc}
\toprule Tipologia & Fare Tableau per & Chiuso? & Aperto? \\
\midrule
         Teorema & $\neg A$ & Si & No \\
         Soddisfacibile & $A$ & No & Si \\
         Contradditoria & $A$ & Si & No \\
\bottomrule
\end{tabular}

%Da fare definizione di di Tableau,Tableau dimostrazione,rami chiusi ed aperti del Tableau

\section{Equivalenze Logiche}
Due proposizioni $A$ e $B$ sono logicamente equivalenti se e solo se A e B hanno
la stessa valutazione booleana.

Nella logica proposizionale sono definite le seguenti equivalenze logiche, indicate con $\equiv$,:
\begin{enumerate}
    \item Idempotenza
            \begin{align*}
                A \lor B & \equiv & A \\
                A \land B & \equiv & A \\
            \end{align*}
    \item Associavità
            \begin{align*}
                A \lor (B \lor C) & \equiv & (A \lor B) \lor C \\
                A \land (B \land C) & \equiv & (A \land B) \land C \\
                A \iff (B \iff C) & \equiv & (A \iff B) \iff C \\
            \end{align*}
    \item Commutatività
            \begin{align*}
                A \lor B & \equiv & B \lor A \\
                A \land B & \equiv & B \land A \\
                A \iff B & \equiv B \iff A \\
            \end{align*}
    \item Distribuitività
            \begin{align*}
                A \lor (B \land C) & \equiv & (A \lor B) \land (A \lor C)\\
                A \land (B \lor C) & \equiv & (A \land B \lor (A \land C) \\
            \end{align*}
    \item Assorbimento
            \begin{align*}
                A \lor (A \land B) & \equiv & A \\
                A \land (A \lor B) & \equiv & A \\
            \end{align*}
    \item Doppia negazione
                \begin{equation}
                    \neg \neg A \equiv A
                \end{equation}
    \item Leggi di De Morgan
            \begin{align*}
                \neg (A \lor B) & \equiv & \neg A \land \neg B \\
                \neg(A \land B) & \equiv & \neg A \lor \neg B \\
            \end{align*}
    \item Terzo escluso
            \begin{equation}
                A \lor \neg A \equiv T
            \end{equation}
    \item Contrapposizione
            \begin{equation}
                A \rightarrow B \equiv \neg B \rightarrow \neg A
            \end{equation}
    \item Contraddizione
            \begin{equation}
                A \land \neg A \equiv F
            \end{equation}
\end{enumerate}

\section{Completezza di insiemi di Connettivi}
Un insieme di connettivi logici è completo se mediante i suoi connettivi si può
esprimere un qualunque altro connettivo.
Nella logica proposizionale valgono anche le seguenti equivalenze, utili per ridurre il linguaggio,:
\begin{align*}
    (A \rightarrow B) & \equiv & (\neg A \lor B) \\
    (A \lor B) & \equiv & \neg(\neg A \land \neg B) \\
    (A \land B) & \equiv & \neg(\neg A \lor \neg B) \\
    (A \iff B) & \equiv & (A \rightarrow B) \land (B \rightarrow A) \\
\end{align*}

L'insieme dei connettivi $\{ \neg,\lor,\land \}$, $\{ \neg,\land \}$ e $\{ \neg,\lor \}$ sono completi
e ciò è facilmente dimostrabile utilizzando le seguenti equivalenze logiche
