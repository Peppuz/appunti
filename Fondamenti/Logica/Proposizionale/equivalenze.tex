\section{Equivalenze Logiche}
Due proposizioni $A$ e $B$ sono logicamente equivalenti se e solo se A e B hanno
la stessa valutazione booleana.

Nella logica proposizionale sono definite le seguenti equivalenze logiche, indicate con $\equiv$,:
\begin{enumerate}
    \item Idempotenza
            \begin{align*}
                A \lor B & \equiv & A \\
                A \land B & \equiv & A \\
            \end{align*}
    \item Associavità
            \begin{align*}
                A \lor (B \lor C) & \equiv & (A \lor B) \lor C \\
                A \land (B \land C) & \equiv & (A \land B) \land C \\
                A \iff (B \iff C) & \equiv & (A \iff B) \iff C \\
            \end{align*}
    \item Commutatività
            \begin{align*}
                A \lor B & \equiv & B \lor A \\
                A \land B & \equiv & B \land A \\
                A \iff B & \equiv B \iff A \\
            \end{align*}
    \item Distribuitività
            \begin{align*}
                A \lor (B \land C) & \equiv & (A \lor B) \land (A \lor C)\\
                A \land (B \lor C) & \equiv & (A \land B \lor (A \land C) \\
            \end{align*}
    \item Assorbimento
            \begin{align*}
                A \lor (A \land B) & \equiv & A \\
                A \land (A \lor B) & \equiv & A \\
            \end{align*}
    \item Doppia negazione
                \begin{equation}
                    \neg \neg A \equiv A
                \end{equation}
    \item Leggi di De Morgan
            \begin{align*}
                \neg (A \lor B) & \equiv & \neg A \land \neg B \\
                \neg(A \land B) & \equiv & \neg A \lor \neg B \\
            \end{align*}
    \item Terzo escluso
            \begin{equation}
                A \lor \neg A \equiv T
            \end{equation}
    \item Contrapposizione
            \begin{equation}
                A \rightarrow B \equiv \neg B \rightarrow \neg A
            \end{equation}
    \item Contraddizione
            \begin{equation}
                A \land \neg A \equiv F
            \end{equation}
\end{enumerate}
