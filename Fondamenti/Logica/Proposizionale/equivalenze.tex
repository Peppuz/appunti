\section{Equivalenze Logiche}
Due proposizioni $A$ e $B$ sono logicamente equivalenti se e solo se A e B hanno
la stessa valutazione booleana.

Nella logica proposizionale sono definite le seguenti equivalenze logiche, indicate con $\equiv$,:
\begin{enumerate}
    \item Idempotenza
            \begin{align*}
                A \lor A  \equiv  A \\
                A \land A  \equiv  A \\
            \end{align*}
    \item Associavità
            \begin{align*}
                A \lor (B \lor C) \equiv  (A \lor B) \lor C \\
                A \land (B \land C)  \equiv  (A \land B) \land C
            \end{align*}
    \item Commutatività
            \begin{align*}
                A \lor B  \equiv  B \lor A \\
                A \land B  \equiv  B \land A
            \end{align*}
    \item Distribuitività
            \begin{align*}
                A \lor (B \land C)  \equiv & (A \lor B) \land (A \lor C)\\
                A \land (B \lor C)  \equiv & (A \land B \lor (A \land C)
            \end{align*}
    \item Assorbimento
            \begin{align*}
                A \lor (A \land B)  \equiv  A
                A \land (A \lor B)  \equiv  A
            \end{align*}
    \item Doppia negazione
                \begin{equation*}
                    \neg \neg A \equiv A
                \end{equation*}
    \item Leggi di De Morgan
            \begin{align*}
                \neg (A \lor B)  \equiv  \neg A \land \neg B \\
                \neg(A \land B)  \equiv  \neg A \lor \neg B
            \end{align*}
    \item Terzo escluso
            \begin{equation*}
                A \lor \neg A \equiv T
            \end{equation*}
    \item Contrapposizione
            \begin{equation*}
                A \rightarrow B \equiv \neg B \rightarrow \neg A
            \end{equation*}
    \item Contraddizione
            \begin{equation*}
                A \land \neg A \equiv F
            \end{equation*}
\end{enumerate}

\subsection{Dimostrazione Equivalenze Logiche}
In questo sottoparagrafo vengono svolte le dimostrazioni delle equivalenze logiche
attraverso l'utilizzo del metodo dei Tableaux, anche se si poteva utilizzare la tavola di verità.
\begin{itemize}
    \item $A \lor A \equiv A$
          %Dimostrazione Idempotenza OR
\begin{proof}
\begin{equation*}
\begin{prooftree}
\hypo{F A \lor A \iff A}
\infer1 {T A \lor A,FA / F A \lor A,TA}
\infer1 {TA,FA/TA,FA/FA,FA,TA}
\end{prooftree}
\end{equation*}
Il tableaux chiude in tutti i rami quindi è una tautologia
\end{proof}

    \item $A \land A \equiv A$
          %Dimostrazione Idempotenza AND
\begin{proof}
\begin{equation*}
\begin{prooftree}
\hypo{F A \land A \iff A}
\infer1 {T A \land A,FA / F A \land A,TA}
\infer1 {TA,TA,FA/TA,FA/FA,TA}
\end{prooftree}
\end{equation*}
Il tableaux chiude in tutti i rami quindi è una tautologia
\end{proof}

    \item $A \lor (B \lor C) \equiv  (A \lor B) \lor C$
          %Dimostrazione Associativà OR
\begin{proof}
\begin{equation*}
\begin{prooftree}
\hypo{A \lor (B \lor C) \iff (A \lor B) \lor C}
\infer1 {T (A \lor (B \lor C)),F ((A \lor B) \lor C)/F (A \lor (B \lor C)),T ((A \lor B) \lor C)}
\infer1 {T A \lor (B \lor C),F (A \lor B),FC/FA,F (B \lor C),T (A \lor B) \lor C}
\infer1 {T A \lor (B \lor C),FA,FB,FC/FA,FB,FC,T (A \lor B) \lor C}
\infer1 {TA,FA,FB,FC/T (B \lor C),FA,FB,FC/FA,FB,FC,TC/FA,FB,FC,T (A \lor B)}
\infer1 {TA,FA,FB,FC/TB,FA,FB,FC/TC,FA,FB,FC/FA,FB,FC,TC/FA,FB,FC,TA/FA,FB,FC,TB}
\end{prooftree}
\end{equation*}
Tutti i rami chiudono quindi è una tautologia.
\end{proof}

    \item $A \land (B \land C) \equiv (A \land B) \land C$
          La dimostrazione è simile a quella per l'Associavità dell'Or
    \item $A \land B \equiv B \land A$
          %Dimostrazione Commutatività dell'And
\begin{proof}
\begin{equation*}
\begin{prooftree}
\hypo{A \land B \iff B \land A}
\infer1 {T (A \land B),F (B \land A)/F (A \land B),T (B \land A)}
\infer1 {TA,TB,F (B \land A)/F (A \land B),TB,TA}
\infer1 {TA,TB,FB/TA,TB,FA/FA,TB,TA/FB,TB,TA}
\end{prooftree}
\end{equation*}
Tutti i rami del tableaux chiudono per cui è una tautologia.
\end{proof}

    \item $A \lor B \equiv B \lor A$
          La dimostrazione è simile a quella per la Commutatività dell'And.
    \item $A \land (B \lor C) \equiv (A \land B) \lor (A \land C)$
          %Dimostrazione Distribuitività And
\begin{proof}
\begin{equation*}
\begin{prooftree}
\hypo{F A \land (B \lor C) \iff (A \land B) \lor (A \land C)}
\infer1 {T A \land (B \lor C),F (A \land B) \lor (A \land C)/F A \land (B \lor C),T (A \land B) \lor (A \land C)}
\end{prooftree}
\end{equation*}
Decido di dividere l'analisi dei vari sottorami del Tableau per chiarezza per cui
analizzo ora il sottoramo $T A \land (B \lor C),F (A \land B) \lor (A \land C)$
\begin{equation*}
\begin{prooftree}
\hypo{T A \land (B \lor C),F (A \land B) \lor (A \land C)}
\infer1 {TA,T B \lor C,F (A \land B) \lor (A \land C)}
\infer1 {TA,T B \lor C,F A \land B,F A \land C}
\infer1 {TA,T B \lor C,FA,F A \land C/TA,T B \lor C,FB,F A \land C}
\infer1{\times/TA,TB,FB,F A \land C/TA,TC,FB,F A \land C}
\infer1 {\times/\times/TA,TC,FB,FA/TA,TC,FB,FC}
\end{prooftree}
\end{equation*}
Il primo ramo del tableaux chiude per cui bisogna analizzare il secondo sottoramo
$F A \land (B \lor C),T (A \land B) \lor (A \land C)$
\begin{equation*}
\begin{prooftree}
\hypo{F A \land (B \lor C),T (A \land B) \lor (A \land C)}
\infer1 {FA,T (A \land B) \lor (A \land C)/F B \lor C,T (A \land B) \lor (A \land C)}
\infer1 {FA,T A \land B/FA,T A \land C/FB,FC,T (A \land B) \lor (A \land C)}
\infer1 {FA,TA,TB/FA,TA,TC/FB,FC,T A \land B/FB,FC,T A \land C}
\infer1 {\times/\times/FB,FC,TA,TB/FB,FC,TA,TC}
\end{prooftree}
\end{equation*}
Anche il secondo ramo chiude per cui la formula è una tautologia.
\end{proof}

    \item $A \lor (B \land C)  \equiv (A \lor B) \land (A \lor C)$
          La dimostrazione è simile alla distribuitività dell'And
    \item $A \lor (A \land B) \equiv A$
          %Dimostrazione Assorbimento Or
\begin{proof}
\begin{equation*}
\begin{prooftree}
\hypo{F A \lor (A \land B) \iff A}
\infer1 {T A \lor (A \land B),FA/F A \lor (A \land B),TA}
\infer1 {TA,FA/T A \land B,FA/FA,TA,F A \land B}
\infer1 {\times/TA,TB,FA/\times}
\end{prooftree}
\end{equation*}
Tutti i rami del Tableau chiudono per cui la formula è una tautologia.
\end{proof}

    \item $A \land (A \lor B)  \equiv  A$
          %Dimostrazione assorbimento And
\begin{proof}
\begin{equation*}
\begin{prooftree}
\hypo{F A \land (A \lor B) \iff A}
\infer1 {F A \land (A \lor B),TA/T A \land(A \lor B),FA}
\infer1 {FA,TA/F A \lor B,TA/TA T A \lor B,FA}
\infer1{\times/FA,FB,TA/\times}
\end{prooftree}
\end{equation*}
Tutti i rami del Tableaux chiudono per cui la formula è una tautologia.
\end{proof}

    \item $\neg \neg A \equiv A$
          %Dimostrazione Doppio Escluso
\begin{proof}
\begin{equation*}
\begin{prooftree}
\hypo{F \neg \neg A \iff A}
\infer1 {T \neg \neg A,FA/F \neg \neg A,TA}
\infer1 {F \neg A,FA/T \neg A,TA}
\infer1 {TA,FA/FA,TA}
\end{prooftree}
\end{equation*}
I due rami del Tableaux chiudono per cui è una tautologia.
\end{proof}

    \item $\neg(A \land B)  \equiv  \neg A \lor \neg B$
          %Dimostrazione deMorgan dell'And
\begin{proof}
\begin{equation*}
\begin{prooftree}
\hypo{F \neg(A \land B)  \iff  \neg A \lor \neg B}
\infer1 {F \neg(A \land B),T \neg A \lor \neg B/T \neg (A \land B),F \neg A \lor \neg B}
\infer1 {T A \land B,T \neg A \lor \neg B/T \neg(A \land B),F \neg A,F \neg B}
\infer1 {TA,TB,T \neg A \lor \neg B/TA,TB,T \neg(A \land B)}
\infer1 {TA,TB,T \neg A/TA,TB,T \neg B/TA,TB,F (A \land B)}
\infer1{TA,TB,FA/TA,TB,FB/TA,TB,FA/TA,TB,FB}
\end{prooftree}
\end{equation*}
Tutti i rami chiudono per cui è una formula tautologica.
\end{proof}

    \item $\neg (A \lor B)  \equiv  \neg A \land \neg B$
          La dimostrazione è simile a quella della deMorgan dell'And.
    \item $A \lor \neg A \equiv T$
          Per definizione di $\lor$ si nota che $A \lor \neg A$ è sempre vero
    \item $\neg (A \lor B)  \equiv  \neg A \land \neg B$
          %Dimostrazione contrapposizione
\begin{proof}
\begin{equation*}
\begin{prooftree}
\hypo{F \neg (A \lor B)  \iff  \neg A \land \neg B}
\infer1 {T \neg(A \lor B),F \neg A \land \neg B/F \neg(A \lor B),T \neg A \land \neg B}
\infer1 {F A \lor B,F \neg A \land \neg B/T A \lor B,T \neg A \land \neg B}
\infer1 {FA,FB,F \neg A \land \neg B/T A \lor B,T \neg A,T \neg B}
\infer1 {FA,FB,F \neg A/FA,FB,F \neg B/FA,FB,T A \lor B}
\infer1 {FA,FB,TA/FA,FB,TB/FA,FB,TA/FA,FB,TB}
\end{prooftree}
\end{equation*}
Tutti i rami chiudi per cui è una tautologia.
\end{proof}

    \item $A \land \neg A \equiv F$
          \begin{proof}
             Per definizione di $\land$ si nota che è impossibile che si possa avere
             $A$ e $\neg A$ uguali.
         \end{proof}
\end{itemize}
