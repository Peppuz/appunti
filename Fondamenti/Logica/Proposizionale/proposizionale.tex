%File latex per il capitolo sulla logica proposizionale Classica
\chapter{Logica Proposizionale}
%Definizione di Logica
La logica è lo studio del ragionamento e dell’argomentazione e, in particolare,
dei	procedimenti inferenziali, rivolti a chiarire quali	procedimenti di pensiero siano validi e quali no.
Vi sono molteplici tipologie di logiche, come ad esempio la logica classica e le logiche costruttive,
tutte accomunate di essere composte da 3 elementi:

%Elementi di una Logica
\begin{itemize}
  \item \textbf{Linguaggio}:insieme di simboli utilizzati nella Logica per definire le cose
  \item \textbf{Sintassi}:insieme di regole che determina quali elementi appartengono o meno al linguaggio
  \item \textbf{Semantica}:permette di dare un significato alle formule del linguaggio e determinare
        se rappresentano o meno la verità.
\end{itemize}

Noi ci occupiamo della logica Classica che si compone in \textsc{logica proposizionale} e
\textit{logica predicativa}.

La Logica proposizionale è un tipo di logica Classica che presenta come caratteristica quella
di essere un linguaggio limitato in quanto si possono esprimere soltanto proposizioni senza
avere la possibilità di estenderla a una classe di persone.

%Sintassi proposizionale
\section{Linguaggio e Sintassi}
Un linguaggio predicativo $L$ è composto dai seguenti insiemi di simboli:
\begin{enumerate}
    \item Insieme di variabili individuali(infiniti) $x,y,z,\dots$
    \item Connettivi logici $\land \lor \neg \rightarrow \iff$
    \item Quantificatori esistenziali $\forall \exists$
    \item Simboli ( , )
    \item Costanti proposizionali $T,F$
    \item Simbolo di uguaglianza $=$,eventualmente assente
\end{enumerate}
Questa è la parte del linguaggio tipica di ogni linguaggio del primo ordine poi
ogni linguaggio definisce la propria segnatura ossia definisce in maniera autonomo:
\begin{enumerate}
    \item Insiemi di simboli di costante $a,b,c,\dots$
    \item Simboli di funzione con arieta $f,g,h,\dots$
    \item Simboli di predicato $P,Q,Z,\dots$ con arietà
\end{enumerate}

%Inserire Esempio
Esempio:Linguaggio della teoria degli insiemi \newline
Costante:$\emptyset$\newline
Predicati:$\in(x,y)$, $=(x,y)$

Esempio:Linguaggio della teoria dei Numeri \newline
Costante:$0$ \newline
Predicati:$<(x,y)$,$=(x,y)$ \newline
Funzioni:$succ(x)$,$+(x,y)$,$*(x,y)$

%Definizione di Termini e Formule ben formate
Per definire le formule ben formate della logica Predicativa bisogna prima definire
l'insieme di termini e le formule atomiche.

\begin{defi}
    L'insieme $TERM$ dei termini è definito induttivamente come segue
    \begin{enumerate}
        \item Ogni variabile e costante sono dei Termini
        \item Se $t_1 \dots t_n$ sono dei termini e $f$ è un simbolo di funzione di arietà $n$
              allora $f(t_1,\dots,t_n)$ è un termine
    \end{enumerate}
\end{defi}

\begin{defi}
    L'insieme $ATOM$ delle formule atomiche è definito come:
    \begin{enumerate}
        \item $T$ e $F$ sono degli atomi
        \item Se $t_1$ e $t_2$ sono dei termini, allora $t_1 = t_2$ è un atomo
        \item Se $t_1,\dots,t_n$ sono dei termini e $P$ è un predicato a $n$ argomenti,
              allora $P(t_1,\dots,t_n)$ è un atomo.
    \end{enumerate}
\end{defi}

\begin{defi}
    L'insieme delle formule ben formate($FBF$) di $L$ è definito induttivamente come
    \begin{enumerate}
        \item Ogni atomo è una formula
        \item Se $A,B \in FBF$, allora $\neg A$, $A \land B$,$A \lor B$,$A \rightarrow B$
              e $A \iff B$ appartengono alle formule ben formate
        \item Se $A \in FBF$ e $x$ è una variabile, allora $\forall x A$ e $\exists x A$
              appartengono alle formule ben formate
        \item Nient'altro è una formula
    \end{enumerate}
\end{defi}

%Inserire Esempi

%Precedenza degli Operatori
La precedenza tra gli operatori logici è definita nella logica predicativa come segue
$\forall,\exists,\neg,\land,\lor,\rightarrow,\iff$ e si assume che gli operatori associno a destra.

%Inserire Esempi

%Variabili legate e chiuse
\begin{defi}
    L'insieme $var(t)$ delle variabili di un termine $t$ è definito come segue:
    \begin{itemize}
        \item $var(t) = \{t \}$, se $t$ è una variabile
        \item $var(t) = \emptyset$ se $t$ è una costante
        \item $var(f(t_1,\dots,t_n)) = \bigcup _{i = 1} ^n var(t_i)$
        \item $var(R(t_1,\dots,t_n)) = \bigcup _{i = 1} ^ n var(t_i)$
    \end{itemize}
\end{defi}

%Termini chiusi ed aperti
Si definisce \emph{Termine chiuso}, un termine che non contiene variabili altrimenti
si definisce il termine come \emph{chiuso}.\newline
Le variabili nei termini e nelle formule atomiche possono essere libere
 in quanto gli unici operatori che "legano" le variabili sono i quantificatori.

Il campo di azione dei quantificatori si riferisce soltanto alla parte in cui
si applica il quantificatore per cui una variabile si dice \emph{libera}
se non ricade nel campo di azione di un quantificatore altrimenti la variabile si dice \emph{vincolata}.


%Semantica Logica Proposizionale
\section{Semantica}
La semantica di una logica consente di dare un significato e un interpretazione
 alle formule del Linguaggio.\newline

\begin{defi}
Sia data una formula proposizionale $P$ e sia ${P_1,\dots,P_n}$, l'insieme degli
atomi che compaiono nella formula $A$.Si definisce come \emph{interpretazione} una
funzione $v:\{P_1,\dots,P_n\} \mapsto \{T,F\}$ che attribuisce un valore di verità
a ciascun atomo della formula $A$.
\end{defi}

I connettivi della Logica Proposizionale hanno i seguenti valori di verità:
%Tabella di Verità degli operatori
$\begin{array}{ccccccc}
\toprule
\text{A} & \text{B} & A \land B & A \lor B & \neg A & A \Rightarrow B & A \iff B \\
\midrule
    F & F & F & F & T & T & T \\
    F & T & F & T & T & T & F \\
    T & F & F & T & F & F & F \\
    T & T & T & T & F & T & T \\
\bottomrule
\end{array}$\newline

Essendo ogni formula $A$ definita mediante un unico albero sintattico, l'interpretazione $v$
è ben definito e ciò comporta che data una formula $A$ e un interpretazione $v$,
eseguita la definizione induttiva dei valori di verità, si ottiene un unico $v(A)$.

Una formula nella logica proposizionale può essere di tre diversi tipi:
%Tipologie di formule
\begin{description}
    \item[valida o tautologica]: la formula è soddisfatta da qualsiasi valutazione della Formula
    \item[Soddisfacibile non Tautologica]:la formula è soddisfatta da qualche valutazione
                        della formula ma non da tutte.
    \item[falsibicabile]:la formula non è soddisfatta da qualche valutazione della formula.
    \item[Contraddizione]:la formula non viene mai soddisfatta
\end{description}

\begin{thm}
$A$ è una formula valida se e solo se $\neg A$ è insoddisfacibile.
$A$ è soddisfacibile se e solo se $\neg A$ è falsibicabile
\end{thm}

%Fare la dimostrazione

Esempio:\newline
Formula $A \land \neg A$ \quad contraddizione

%Tabella di Verità
$\begin{array}{ccc}
\toprule A & \neg A & A \land \neg A \\
\midrule
        0 & 1 & 0 \\
        1 & 0 & 0 \\
\bottomrule
\end{array}$\newpage

Formula $Z = (A \land B) \lor C$  soddisfacibile non Tautologica

%Tabella di Verità
$\begin{array}{ccccc}
\toprule A & B & C & A \land B & (A \land B) \lor C \\
\midrule
         0 & 0 & 0 & 0 & 0 \\
         0 & 0 & 1 & 0 & 1 \\
         0 & 1 & 0 & 0 & 0 \\
         0 & 1 & 1 & 0 & 1 \\
         1 & 0 & 0 & 0 & 0 \\
         1 & 0 & 1 & 0 & 1 \\
         1 & 1 & 0 & 1 & 1 \\
         1 & 1 & 1 & 1 & 1 \\
\bottomrule
\end{array}$\newline

Formula $X = (A \land B) \Rightarrow (\neg A \land C)$ \quad Soddisfacibile  non tautologica\newline

%Tabella di Verità della formula X
$\begin{array}{ccccccc}
\toprule A & B & C & \neg A & A \land B & \neg A \land C & X\\
\midrule
         0 & 0 & 0 & 1 & 0 & 0 & 1 \\
         0 & 0 & 1 & 1 & 0 & 1 & 1 \\
         0 & 1 & 0 & 1 & 0 & 0 & 1 \\
         0 & 1 & 1 & 1 & 0 & 1 & 1 \\
         1 & 0 & 0 & 0 & 0 & 0 & 1 \\
         1 & 0 & 1 & 0 & 0 & 0 & 1 \\
         1 & 1 & 0 & 0 & 1 & 0 & 0 \\
         1 & 1 & 1 & 0 & 1 & 0 & 0 \\
\bottomrule
\end{array}$ \newline

Formula $Y = \neg(A \land B) \iff (A \lor B \Rightarrow C)$ soddisfacibile non Tautologica

%Tabella di Verità
$\begin{array}{cccccccc}
\toprule
A & B & C & A \land B & \neg(A \land B) & A \lor B & (A \lor B) \Rightarrow C & Y \\
\midrule
0 & 0 & 0 & 0 & 1 & 0 & 1 & 1 \\
0 & 0 & 1 & 0 & 1 & 0 & 1 & 1 \\
0 & 1 & 0 & 0 & 1 & 1 & 0 & 0 \\
0 & 1 & 1 & 0 & 1 & 1 & 1 & 1 \\
1 & 0 & 0 & 0 & 1 & 1 & 0 & 0 \\
1 & 0 & 1 & 0 & 1 & 1 & 1 & 1 \\
1 & 1 & 0 & 1 & 0 & 1 & 0 & 1 \\
1 & 1 & 1 & 1 & 0 & 1 & 1 & 0 \\
\bottomrule
\end{array}$

\subsection{Modelli e decidibilità}
Si definisce \emph{modello}, indicato con $M \models A$, tutte le valutazioni booleane
che rendono vera la formula $A$.
Si definisce \emph{contromodello}, indicato con , tutte le valutazioni booleane
che rendono falsa la formula $A$.

La logica proposizionale è decidibile (posso sempre verificare il significato di una formula).
Esiste infatti una procedura effettiva che stabilisce la validità o no di una formula, o se questa
ad esempio è una tautologia.
In particolare il verificare se una proposizione è tautologica o meno è l’operazione di decibi-
lità principale che si svolge nel calcolo proposizonale.

\begin{defi}
    Se $M \models A$ per tutti gli $M$, allora $A$ è una tautologia e si indica $\models A$
\end{defi}

\begin{defi}
    Se $M \models A$ per qualche $M$, allora $A$ è soddisfacibile
\end{defi}

\begin{defi}
Se $M \models A$ non è soddisfatta da nessun $M$, allora $A$ è insoddisfacibile
\end{defi}


%Sistema Deduttivo
\section{Sistema Deduttivo}
Il sistema deduttivo è un metodo di calcolo che manipola proposizioni, senza la
necessità di ricorrere ad altri aspetti della logica.\newline
Nella logica proposizionale, tramite i teoremi di completezza e correttezza, esiste
una corrispondenza tra le formule derivanti dal sistema deduttivo e le formule verificabili
tramite la semantica della logica.

I sistemi deduttivi della logica proposizionale sono i seguenti:
\begin{description}
    \item[Sistema deduttivo Hilbertiano]: non viene analizzato
    \item[Metodo dei Tableaux]
    \item [Risoluzione Proposizionale]:non viene analizzato !!!!
\end{description}

\begin{defi}
Una sequenza di formule $A_1,\dots,A_n$ di $\Lambda$ è una \emph{dimostrazione} se
per ogni $i$ compreso tra $1$ e $n$, $A_i$ è un assioma di $\Lambda$ oppure una
conseguenza diretta di una formula precedente.
\end{defi}

\begin{defi}
Una formula $A$ di una logica $\Lambda$ è detta \emph{teorema} di $\Lambda$,indicata
con $\vdash A$ se esiste una dimostrazione di $\Lambda$ che ha $A$ come ultima formula
\end{defi}

Una dimostrazione di una formula di una logica può venire tramite:
\begin{itemize}
  \item  \textbf{Metodo diretto}: Data un'ipotesi, attraverso una serie di passi
          si riesce a dimostrare la correttezza della Tesi
  \item \textbf{Metodo per assurdo}(non sempre accettato in tutte le logiche):
        Si nega la tesi ed attraverso una serie di passi si riesce a dimostrare
        la negazione delle ipotesi.
\end{itemize}

\begin{thm}
    Un apparato deduttivo $R$ è completo se, per ogni formula $A \in Fbf$, $\vdash A$
    implica $\models A$
\end{thm}

\begin{thm}
    Un apparato deduttivo $R$ è corretto se, per ogni formula $A \in Fbf$, $\models A$
    implica $\vdash A$
\end{thm}
\subsection{Tableau Proposizionali}
Il metodo dei Tableau è stato introdotto da Hintikka e Beth alla fine degli anni '50
e poi ripresi successivamente da Smullyan.
Per poter comprendere e capire i Tableaux dobbiamo introdurre una serie di definizioni:
\begin{defi}
Per ogni formula $A$, $\{A,\neg A\}$ è una coppia di formule complementari in cui
$A$ è il complemento di $\neg A$
\end{defi}

\begin{defi}
Un letterale è un atomo o la sua negazione.Se $p$ è un atomo allora $\{p,\neg p\}$
è una coppia di letterali complementari.
\end{defi}

I tableau sono degli alberi,la cui radice è l'enunciato in esame, e gli altri nodi
sono costruiti attraverso l'applicazione di una serie di regole,fino ad arrivare
alle formule atomiche come radici.

I tableaux proposizionali si dividono in due tipologie di formule(e quindi regole):
le $\alpha$ regole e le $\beta$ regole;le $\alpha$ formule sono di tipo congiuntivo
ed è soddisfatta se e soltanto se le sottoformule $\alpha_1$ e $\alpha_2$ sono entrambe soddisfatte
mentre le $\beta$ formule sono di tipo disgiuntivo e sono soddisfatte se e soltanto
se almeno una delle due sottoformule $\beta_1$ e $\beta_2$ è soddisfatta.

Le regole dei Tableau sono le seguenti:
%T AND
\begin{equation*}
%T AND
\begin{prooftree}
\hypo{S,T (A \land B)}
\infer1 {S,TA,TB}
\end{prooftree}
\quad T \land \qquad
%F AND
\begin{prooftree}
\hypo{S,F (A \land B)}
\infer1 {S,FA/S,FB}
\end{prooftree}
F \ \land
\end{equation*}

\begin{equation*}
%T OR
\begin{prooftree}
\hypo{S,T (A \lor B)}
\infer1 {S,TA / S,TB}
\end{prooftree}
\quad T \lor \qquad
%F OR
\begin{prooftree}
\hypo{S,F (A \lor B)}
\infer1 {S,FA,FB}
\end{prooftree}
F \ \lor
\end{equation*}

\begin{equation*}
%T NOT
\begin{prooftree}
\hypo{S,T (\neg A)}
\infer1 {S,FA}
\end{prooftree}
\quad T \neg \qquad
%F NOT
\begin{prooftree}
\hypo{S,F (\neg A)}
\infer1 {S,TA}
\end{prooftree}
F \ \neg
\end{equation*}

\begin{equation*}
%T ->
\begin{prooftree}
\hypo{S,T (A \rightarrow B)}
\infer1 {S,FA / S,TB}
\end{prooftree}
\quad T \rightarrow \qquad
%F ->
\begin{prooftree}
\hypo{S,F (A \rightarrow B)}
\infer1 {S,TA,FB}
\end{prooftree}
F \ \rightarrow
\end{equation*}

\begin{equation*}
%T <-->
\begin{prooftree}
\hypo{S,T (A \iff B)}
\infer1 {S,TA,TB/S,FA,FB}
\end{prooftree}
\quad T \iff \qquad
%F <-->
\begin{prooftree}
\hypo{S,F (A \iff B)}
\infer1{S,TA,FB/S,FA,TB}
\end{prooftree}
F \ \iff
\end{equation*}

Le $\beta$ regole sono quelle che creano due sottoformule indicate nella regola con $/$
mentre, per esclusione, le $\alpha$ regole sono quelle in cui si crea soltanto una sottoformula.

%Definizione induttiva di costruzione di un tableaux
Il tableaux di una formula $A$ inizialmente è composto da un solo nodo, la radice, etichettata
dalla formula $A$.
Il tableaux si costruisce induttivamente come segue:
si sceglie una foglia $l$ non etichettata dell'albero che verrà etichettata da un
insieme di formule $U(l)$ definite come:
\begin{enumerate}
  \item Se $U(l)$ è un insieme di letterali, si controlla se sono presenti una coppia
        di letterali complementati in $U(l)$;
        in caso siano presenti si marca la foglia come chiusa $\times$ altrimenti la foglia è aperta
  \item Se $U(l)$ non è un insieme di letterali si sceglie una formula in $U(l)$ tramite:
        \begin{itemize}
          \item Se la formula è un $\alpha$-formula $B$ si crea un nuovo nodo $l'$,
                figlio di $l$, e lo si etichetta come $U(l') = (U(l) - \{B\}) \cup \{\alpha_1,alpha_2\}$
          \item Se la formula è un $\beta$-formula $C$ si creano due nodi $l'$ e $l''$,
                figli di $l$ con $l'$ etichettato come: $U(l') = (U(l) - \{C\}) \cup \{\beta_1\}$
                mentre $l''$ è etichettata come $U(l'') = (U(l) - \{C\}) \cup \{\beta_2\}$
        \end{itemize}
\end{enumerate}
Il tableaux termina quando tutti i rami sono etichettati come chiusi e/o aperti.

%Definizione di Tableaux completo
\begin{defi}
Si definisce un tableaux \emph{completo} se la sua costruzione è complementata.
Il tableaux si dice \emph{chiuso} se tutte le foglie sono segnate come chiuse
altrimenti il tableaux è \emph{aperto}
\end{defi}

%Dimostrazione di tableau
Il metodo dei Tableau è un metodo dei sistemi deduttivi, che permette attraverso
l'applicazione di una serie di regole, di capire la tipologia della formula.
%Metodi per capire il tipo della Formula
\begin{tabular}{cccc}
\toprule Tipologia & Fare Tableau per & Chiuso? & Aperto? \\
\midrule
         Teorema & $\neg A$ & Si & No \\
         Soddisfacibile & $A$ & No & Si \\
         Contradditoria & $A$ & Si & No \\
\bottomrule
\end{tabular}

Esempio:$C \rightarrow (P \rightarrow ((Q \rightarrow \neg P) \lor (C \rightarrow P)))$
\begin{equation*}
\begin{prooftree}
\hypo{F C \rightarrow (P \rightarrow ((Q \rightarrow \neg P) \lor (C \rightarrow P)))}
\infer1 {TC,F (P \rightarrow ((Q \rightarrow \neg P) \lor (C \rightarrow P)))}
\infer1 {TC,TP,F (Q \rightarrow \neg P) \lor (C \rightarrow P)}
\infer1 {TC,TP,F (Q \rightarrow \neg P),F (C \rightarrow P)}
\infer1{TC,TP,F (Q \rightarrow \neg P),TC,FP}
\end{prooftree}
\end{equation*}
Il tableaux chiude in quanto non può essere contemporaneamente $TP$ e $FP$ per cui
la formula è una tautologia.

Esempio:
%Esempio Tableaux Predicativo
Formula: $(\forall x F(x) \lor \exists G(x)) \rightarrow (\exists x (F(x) \lor G(x)))$
\begin{proof}
\begin{equation*}
\begin{prooftree}
\hypo{F \forall x F(x) \lor \exists G(x) \rightarrow \exists x (F(x) \lor G(x))}
\infer1{T \forall x F(x) \lor \exists G(x),F \exists x (F(x) \lor G(x))}
\infer1{T \forall x F(x),F \exists x (F(x) \lor G(x))/T \exists G(x),F \exists x (F(x) \lor G(x))}
\infer1{T F(a),F \exists x (F(x) \lor G(x)),T \forall x \dots/T G(a),F \exists x (F(x) \lor G(x))}
\infer1{T F(a),F F(a) \lor G(a),T \forall \dots,F \exists \dots/T G(a),F F(a) \lor G(a),F \exists \dots}
\infer1{T F(a),F F(a),F G(a),T \forall \dots,F \exists \dots/T G(a),F F(a),F G(a),F \exists \dots}
\end{prooftree}
\end{equation*}
Il tableaux contraddizione chiude per cui la formula è una tautologia.
\end{proof}


%Completezza dei Tableaux
%Dimostrazione completezza Tableaux

%Dimostrazione correttezza Tableaux


%Equivalenze Logiche
\section{Equivalenze Logiche}
Due proposizioni $A$ e $B$ sono logicamente equivalenti se e solo se A e B hanno
la stessa valutazione booleana.

Nella logica proposizionale sono definite le seguenti equivalenze logiche, indicate con $\equiv$,:
\begin{enumerate}
    \item Idempotenza
            \begin{align*}
                A \lor B & \equiv & A \\
                A \land B & \equiv & A \\
            \end{align*}
    \item Associavità
            \begin{align*}
                A \lor (B \lor C) & \equiv & (A \lor B) \lor C \\
                A \land (B \land C) & \equiv & (A \land B) \land C \\
                A \iff (B \iff C) & \equiv & (A \iff B) \iff C \\
            \end{align*}
    \item Commutatività
            \begin{align*}
                A \lor B & \equiv & B \lor A \\
                A \land B & \equiv & B \land A \\
                A \iff B & \equiv B \iff A \\
            \end{align*}
    \item Distribuitività
            \begin{align*}
                A \lor (B \land C) & \equiv & (A \lor B) \land (A \lor C)\\
                A \land (B \lor C) & \equiv & (A \land B \lor (A \land C) \\
            \end{align*}
    \item Assorbimento
            \begin{align*}
                A \lor (A \land B) & \equiv & A \\
                A \land (A \lor B) & \equiv & A \\
            \end{align*}
    \item Doppia negazione
                \begin{equation}
                    \neg \neg A \equiv A
                \end{equation}
    \item Leggi di De Morgan
            \begin{align*}
                \neg (A \lor B) & \equiv & \neg A \land \neg B \\
                \neg(A \land B) & \equiv & \neg A \lor \neg B \\
            \end{align*}
    \item Terzo escluso
            \begin{equation}
                A \lor \neg A \equiv T
            \end{equation}
    \item Contrapposizione
            \begin{equation}
                A \rightarrow B \equiv \neg B \rightarrow \neg A
            \end{equation}
    \item Contraddizione
            \begin{equation}
                A \land \neg A \equiv F
            \end{equation}
\end{enumerate}


%Completezza dei Connettivi
\section{Completezza di insiemi di Connettivi}
Un insieme di connettivi logici è completo se mediante i suoi connettivi si può
esprimere un qualunque altro connettivo.
Nella logica proposizionale valgono anche le seguenti equivalenze, utili per ridurre il linguaggio,:
\begin{align*}
    (A \rightarrow B) & \equiv & (\neg A \lor B) \\
    (A \lor B) & \equiv & \neg(\neg A \land \neg B) \\
    (A \land B) & \equiv & \neg(\neg A \lor \neg B) \\
    (A \iff B) & \equiv & (A \rightarrow B) \land (B \rightarrow A) \\
\end{align*}

L'insieme dei connettivi $\{ \neg,\lor,\land \}$, $\{ \neg,\land \}$ e $\{ \neg,\lor \}$ sono completi
e ciò è facilmente dimostrabile utilizzando le seguenti equivalenze logiche
