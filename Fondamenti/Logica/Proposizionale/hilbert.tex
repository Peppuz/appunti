%Appunti sul sistema deduttivo Hilbertiano
Il sistema deduttivo Hilbertiano è un sistema assiomatico in cui si individuano
una serie di proposizione, detti assiomi, da cui partire per dimostrare altre proposizioni,
limitando quindi il numero di regole di inferenza.

Nel sistema hilbertiano sono presenti 3 sistemi di Assiomi:
\begin{itemize}
    \item $(A \rightarrow (B \rightarrow A))$
    \item $(A \rightarrow (B \rightarrow C)) \rightarrow ((A \rightarrow B) \rightarrow (A \rightarrow C))$
    \item $(B \rightarrow \neg A) \rightarrow ((B \rightarrow A) \rightarrow \neg B)$
\end{itemize}
Nel sistema Hilbertiano c'è soltanto una regola di inferenza:
\begin{equation*}
\begin{prooftree}
\hypo{A \quad A \rightarrow B}
\infer1{B}
\end{prooftree}
\end{equation*}

Let U be a set of formulas and A a formula. The notation U 
A means that the formulas in U are assumptions in the proof of A. A proof is a
sequence of lines U i  φ i , such that for each i, U i ⊆ U , and φ i is an axiom, a
previously proved theorem, a member of U i or can be derived by MP from previous
lines U i  φ i , U i  φ i , where i , i < i.
Rule 3.13 (Deduction rule)
U ∪ {A}  B
.
U  A → B
We must show that this derived rule is sound, that is, that the use of the derived
rule does not increase the set of provable theorems in H . This is done by showing
how to transform any proof using the rule into one that does not use the rule. There-
fore, in principle, any proof that uses the derived rule could be transformed to one
that uses only the three axioms and MP.
Theorem 3.14 (Deduction theorem) The deduction rule is a sound derived rule.

A more precise terminology would be to say that  A → (B → A)
is an axiom scheme that is a shorthand for an infinite number of axioms obtained by
replacing the ‘variables’ A and B with actual formulas, for example:
A
B
A



  



((p ∨ ¬ q) ↔ r) → ( ¬ (q ∧ ¬ r) → ((p ∨ ¬ q) ↔ r) ).
Similarly,  A → A is a theorem scheme that is a shorthand for an infinite number
of theorems that can be proved in H , including, for example:
 ((p ∨ ¬ q) ↔ r) → ((p ∨ ¬ q) ↔ r).
We will not retain this precision in our presentation because it will always clear
%Assiomi e regole di inferenza


%Esercizi ed Esempio
