%Parte sulla deduzione di Gentzen(detta anche Deduzione Naturale)
\section{Deduzione di Gentzen}
Il sistema deduttivo di Gentzen, introdotto come si può dedurre dal nome dal logico
tedesco Gerhard Gentzen nel 1930, presenta delle analogie con il sistema dei Tableaux.
Può sembrare non famigliare in quanto ha soltanto un tipo di assioma e molte regole
di inferenza, al contrario delle teorie matematiche, e per cui si può rappresentare
la deduzione mediante alberi, cosa che la rende simile ai tableaux.

\begin{defi}
Si definisce \emph{assioma} del sistema di Gentzen, un insieme di espressioni $U$
in cui compare una coppia complementare, ossia una coppia del tipo $(P,\neg P)$,
mentre si definisce \emph{regole di inferenza} una regola dedotta da 1 o 2 insiemi
di formule $U1$ e $U2$.
\end{defi}

Nel sistema Gentzen sono definite delle regole, come avvenuto nei Tableaux, per
riuscire a desumere le regole di inferenza e rappresentano l'inverso dei tableaux,
 come si può vedere dalla seguente tabella di regole del sistema di Gentzen:

%Inserire le regole di Gentzen

%Esempio sistema di Gentzen
Esempio:Dimostrare in Gentzen $(p ∨ q) → (q ∨ p)$
\begin{proof}


%Analogie tra Gentzen e i Tableaux
Let U be a set of formulas and let Ū be the set of complements of
formulas in U . Then  U in G if and only if there is a closed semantic tableau
for Ū .

Let A be a formula in propositional logic. Then  A in G if and only
if there is a closed semantic tableau for ¬ A.


%Inserire Esercizi
