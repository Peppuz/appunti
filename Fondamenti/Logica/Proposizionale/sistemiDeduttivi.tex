\section{Sistema Deduttivo}
Il sistema deduttivo è un metodo di calcolo che manipola proposizioni, senza la
necessità di ricorrere ad altri aspetti della logica.\newline
Nella logica proposizionale, tramite i teoremi di completezza e correttezza, esiste
una corrispondenza tra le formule derivanti dal sistema deduttivo e le formule verificabili
tramite la semantica della logica.

I sistemi deduttivi della logica proposizionale sono i seguenti:
\begin{description}
    \item[Sistema deduttivo Hilbertiano]: non viene analizzato
    \item[Metodo dei Tableaux]
    \item[Sistema Deduttivo di Gentzen]:presenta analogie con i Tableaux
    \item [Risoluzione Proposizionale]:non viene analizzato !!!!
\end{description}

\begin{defi}
Una sequenza di formule $A_1,\dots,A_n$ di $\Lambda$ è una \emph{dimostrazione} se
per ogni $i$ compreso tra $1$ e $n$, $A_i$ è un assioma di $\Lambda$ oppure una
conseguenza diretta di una formula precedente.
\end{defi}

\begin{defi}
Una formula $A$ di una logica $\Lambda$ è detta \emph{teorema} di $\Lambda$,indicata
con $\vdash A$ se esiste una dimostrazione di $\Lambda$ che ha $A$ come ultima formula
\end{defi}

Una dimostrazione di una formula di una logica può venire tramite:
\begin{itemize}
  \item  \textbf{Metodo diretto}: Data un'ipotesi, attraverso una serie di passi
          si riesce a dimostrare la correttezza della Tesi
  \item \textbf{Metodo per assurdo}(non sempre accettato in tutte le logiche):
        Si nega la tesi ed attraverso una serie di passi si riesce a dimostrare
        la negazione delle ipotesi.
\end{itemize}

\begin{thm}
    Un apparato deduttivo $R$ è completo se, per ogni formula $A \in Fbf$, $\vdash A$
    implica $\models A$
\end{thm}

\begin{thm}
    Un apparato deduttivo $R$ è corretto se, per ogni formula $A \in Fbf$, $\models A$
    implica $\vdash A$
\end{thm}
\subsection{Tableau Proposizionali}
Il metodo dei Tableau è stato introdotto da Hintikka e Beth alla fine degli anni '50
e poi ripresi successivamente da Smullyan.

I tableau sono degli alberi,la cui radice è l'enunciato in esame, e gli altri nodi
sono costruiti attraverso l'applicazione di una serie di regole,fino ad arrivare
alle formule atomiche come radici.

Le regole dei Tableau sono le seguenti:
%T AND
\begin{equation*}
%T AND
\begin{prooftree}
\hypo{S,T (A \land B)}
\infer1 {S,TA,TB}
\end{prooftree}
\quad T \land \qquad
%F AND
\begin{prooftree}
\hypo{S,F (A \land B)}
\infer1 {S,FA/S,FB}
\end{prooftree}
F \ \land
\end{equation*}

\begin{equation*}
%T OR
\begin{prooftree}
\hypo{S,T (A \lor B)}
\infer1 {S,TA / S,TB}
\end{prooftree}
\quad T \lor \qquad
%F OR
\begin{prooftree}
\hypo{S,F (A \lor B)}
\infer1 {S,FA,FB}
\end{prooftree}
F \ \lor
\end{equation*}

\begin{equation*}
%T NOT
\begin{prooftree}
\hypo{S,T (\neg A)}
\infer1 {S,FA}
\end{prooftree}
\quad T \neg \qquad
%F NOT
\begin{prooftree}
\hypo{S,F (\neg A)}
\infer1 {S,TA}
\end{prooftree}
F \ \neg
\end{equation*}

\begin{equation*}
%T ->
\begin{prooftree}
\hypo{S,T (A \rightarrow B)}
\infer1 {S,FA / S,TB}
\end{prooftree}
\quad T \rightarrow \qquad
%F ->
\begin{prooftree}
\hypo{S,F (A \rightarrow B)}
\infer1 {S,TA,FB}
\end{prooftree}
F \ \rightarrow
\end{equation*}

\begin{equation*}
%T <-->
\begin{prooftree}
\hypo{S,T (A \iff B)}
\infer1 {S,TA,TB/S,FA,FB}
\end{prooftree}
\quad T \iff \qquad
%F <-->
\begin{prooftree}
\hypo{S,F (A \iff B)}
\infer1{S,TA,FB/S,FA,TB}
\end{prooftree}
F \ \iff
\end{equation*}

%Dimostrazione di tableau
Il metodo dei Tableau è un metodo dei sistemi deduttivi, che permette attraverso
l'applicazione di una serie di regole, di capire la tipologia della formula.
%Metodi per capire il tipo della Formula
\begin{tabular}{cccc}
\toprule Tipologia & Fare Tableau per & Chiuso? & Aperto? \\
\midrule
         Teorema & $\neg A$ & Si & No \\
         Soddisfacibile & $A$ & No & Si \\
         Contradditoria & $A$ & Si & No \\
\bottomrule
\end{tabular}

Esempio:$C \rightarrow (P \rightarrow ((Q \rightarrow \neg P) \lor (C \rightarrow P)))$
\begin{equation*}
\begin{prooftree}
\hypo{F C \rightarrow (P \rightarrow ((Q \rightarrow \neg P) \lor (C \rightarrow P)))}
\infer1 {TC,F (P \rightarrow ((Q \rightarrow \neg P) \lor (C \rightarrow P)))}
\infer1 {TC,TP,F (Q \rightarrow \neg P) \lor (C \rightarrow P)}
\infer1 {TC,TP,F (Q \rightarrow \neg P),F (C \rightarrow P)}
\infer1{TC,TP,F (Q \rightarrow \neg P),TC,FP}
\end{prooftree}
\end{equation*}
Il tableaux chiude in quanto non può essere contemporaneamente $TP$ e $FP$ per cui
la formula è una tautologia.

Esempio:
%Rappresentazione esempio Grafo
\begin{tikzpicture}
\GraphInit[vstyle=normal]
\tikzset{EdgeStyle/.style={<->,bend right}}
\Vertex[x = 0,y = 0]{Marco}
\Vertex[x = -1,y = -3]{Laura}
\Vertex[x = 1,y = 2]{Luca}
\Vertex[x = -2,y = 2]{Sara}
\Vertex[x = 3,y = -3]{Chiara}
\Vertex[x = -5,y = 0]{Amilcare}
\Vertex[x = 4,y = 2]{Achille}
\Vertex[x = 2,y = -5]{Astolfo}
\Loop(Marco)
\Loop(Laura)
\Loop(Luca)
\Loop(Sara)
\Loop(Chiara)
\Loop(Amilcare)
\Loop(Achille)
\Loop(Astolfo)
\Edges(Marco,Laura,Marco,Luca,Marco,Chiara,Marco,Sara,Marco)
\end{tikzpicture}

