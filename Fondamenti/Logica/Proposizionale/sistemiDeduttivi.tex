\section{Sistema Deduttivo}
Il sistema deduttivo è un metodo di calcolo che manipola proposizioni, senza la
necessità di ricorrere ad altri aspetti della logica.\newline
Nella logica proposizionale, tramite i teoremi di completezza e correttezza, esiste
una corrispondenza tra le formule derivanti dal sistema deduttivo e le formule verificabili
tramite la semantica della logica.

I sistemi deduttivi della logica proposizionale sono i seguenti:
\begin{description}
    \item[Sistema deduttivo Hilbertiano]: non viene analizzato
    \item[Metodo dei Tableau]
    \item [Risoluzione Proposizionale]:non viene analizzato !!!!
\end{description}

\begin{defi}
Una sequenza di formule $A_1,\dots,A_n$ di $\Lambda$ è una \emph{dimostrazione} se
per ogni $i$ compreso tra $1$ e $n$, $A_i$ è un assioma di $\Lambda$ oppure una
conseguenza diretta di una formula precedente.
\end{defi}

\begin{defi}
Una formula $A$ di una logica $\Lambda$ è detta \emph{teorema} di $\Lambda$,indicata
con $\vdash A$ se esiste una dimostrazione di $\Lambda$ che ha $A$ come ultima formula
\end{defi}

Una dimostrazione di una formula di una logica può venire tramite:
\begin{itemize}
  \item  \textbf{Metodo diretto}: Data un'ipotesi, attraverso una serie di passi
          si riesce a dimostrare la correttezza della Tesi
  \item \textbf{Metodo per assurdo}(non sempre accettato in tutte le logiche):
        Si nega la tesi ed attraverso una serie di passi si riesce a dimostrare
        la negazione delle ipotesi.
\end{itemize}

\begin{thm}
    Un apparato deduttivo $R$ è completo se, per ogni formula $A \in Fbf$, $\vdash A$
    implica $\models A$
\end{thm}

\begin{thm}
    Un apparato deduttivo $R$ è corretto se, per ogni formula $A \in Fbf$, $\models A$
    implica $\vdash A$
\end{thm}
\subsection{Tableau Proposizionali}
Il metodo dei Tableau è stato introdotto da Hintikka e Beth alla fine degli anni '50
e poi ripresi successivamente da Smullyan.

I tableau sono degli alberi,la cui radice è l'enunciato in esame, e gli altri nodi
sono costruiti attraverso l'applicazione di una serie di regole,fino ad arrivare
alle formule atomiche come radici.

Le regole dei Tableau sono le seguenti:\newline
T $\land$

$\begin{array}{c}
\toprule
S,T A \land B \\
\midrule
S, TA, TB \\
\end{array}$

T $\lor$

$\begin{array}{c}
\toprule
S,T A \lor B \\
\midrule
S, TA/ S, TB \\
\end{array}$

T $\neg$

$\begin{array}{c}
\toprule
S,T \neg A \\
\midrule
S, FA \\
\end{array}$

T $\Rightarrow$

$\begin{array}{c}
\toprule
S,T A \Rightarrow B \\
\midrule
S, FA / S, TB \\
\end{array}$

T $\iff$(da fare)

F $\land$

$\begin{array}{c}
\toprule
S,F A \land B \\
\midrule
S, FA/ S, TB \\
\end{array}$

F $\lor$

$\begin{array}{c}
\toprule
S,F A \lor B \\
\midrule
S, FA, FB \\
\end{array}$

F $\neg$

$\begin{array}{c}
\toprule
S,F \neg A \\
\midrule
S, TA \\
\end{array}$

F $\Rightarrow$

$\begin{array}{c}
\toprule
S,F A \Rightarrow B \\
\midrule
S, TA, FB \\
\end{array}$

F $\iff$(da fare)

%Dimostrazione di tableau
Il metodo dei Tableau è un metodo dei sistemi deduttivi, che permette attraverso
l'applicazione di una serie di regole, di capire la tipologia della formula.
%Metodi per capire il tipo della Formula
\begin{tabular}{cccc}
\toprule Tipologia & Fare Tableau per & Chiuso? & Aperto? \\
\midrule
         Teorema & $\neg A$ & Si & No \\
         Soddisfacibile & $A$ & No & Si \\
         Contradditoria & $A$ & Si & No \\
\bottomrule
\end{tabular}

%Da fare definizione di di Tableau,Tableau dimostrazione,rami chiusi ed aperti del Tableau
