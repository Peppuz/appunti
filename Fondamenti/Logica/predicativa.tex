%Capitolo sulla Logica Predicativa
\chapter{Logica Predicativa}

\section{Semantica}
Il connettivo $A \iff B$ equivale per definizione ad $(A \rightarrow B) \land (B \rightarrow A)$ per cui quando si usa il $\iff$ si utilizza l'equivalenza
definita
Tableaux di $F A \iff B$ utilizzare la definizione data 

Struttura di Interpretazione(Semantica) %Cercare meglio la definizione!!!!!!!!!
$S = <D,I> $
D(dominio) è un insieme finito/infinito di costanti
I(interpretazione) è una funzione che associa a simboli e formule del linguaggio un significato a partire dai simboli primitivi del linguaggio
(costanti,predicati e funzioni) cioè l'interpretazione varierà a seconda della signatura del linguaggio

Si può interpretare questa struttura come una funzione che associa ad ogni costante una costante del Linguaggio 
$C_i \mapsto C_d : C_d \in D$

LogicA Aritmetica di Peano
Linguaggio: $0,s(x),+,*,=$ è la segnatura

Dominio: tutti e numeri Naturali
L'unica costante del linguaggio $0$ viene associato la costante del Dominio $0$. Può sembrare strano ma sono dei numeri diversi in quanto 
$0 \in Linguaggio \mapsto 0 \in N$ ossia si associa ad un elemento del linguaggio una sua interpretazione


In Logica Proposizionale abbiamo sviluppato indipendente la parte sintattica e la parte semantica e con il Teorema di correttezza li abbiamo messi assiemi
Anche a livello predicativo è possibile definibile in maniera chiara la semantica ed è definito il teorema di Completezza

In logica chiamiamo Numerale la costante del linguaggio e Numero la sua interpretazione

Ho un simbolo di funzione con n argomenti $f ^ n _L(t1...t_n) \mapsto I(t_k)$ quando $t_k = f(t_1....t_n)$ ossia associa ad ogni elemento della funzione una sua interpretazione

$I(P(t_1....t_n) = <t_1....t_n> $ dove $I(P) \mapsto R ^n$ P è l'insieme delle n-uple che stanno nella relazione definita

Il successore viene interpretato come la funzione successore

Esempio:
$s s(0) \mapsto 2 \in N$ in quanto 0 lo interpreto come 0, il successore di 0 come 1 e il successore di 1 è 2 e questo due è l'interpretazione di s(s(0))

Esempio Interpretazione relazione
Il Predicato = viene interpretato come l'insieme delle coppie che hanno i numeri uguali
$= \mapsto <0,0>,<1,1>,<2,2>,......<n,n>$

Esempio: 
x + y = z    + è una funzione del linguaggio
Fissato il $s(0) e s(s(0))$
$s(0) + s(s(0)) \mapsto 3 \in N$ è l'interpretazione di s(0) e l'interpretazione del s(s(0))

Come si interpretano le formule contenenti delle variabili libere?
$\eta$ che associa alle variabili libere un'interpretazione  allora grazie a questa funzione $\eta$ possiamo valutare la formula come vera o falsa

Es:
$x \mapsto s(0)$
$y \mapsto 0$
$z \mapsto s(0)$
Allora x + y = z con questi valori è vera

Formula aperta ha senso chiedersi 
Si dice che una formula aperta    se esiste un assegnamento alle variabile che la rende vera
Si dice che una formula aperta se non esiste un assegnamento alle variabile che la rende vera

Data una struttura $S $ e $\eta |-- P(t1...t_n)$ si dice che è vera se esiste un interpretazione di P(t1...tn) che la rende vera

Se una formula è chiusa allora non necessita della funzione $\eta$ in quanto l'interpretazione è unica

$S,\eta |-- A \land B$ è vera se $S,\eta |-- A \land S,\eta |-- B$

$S,\eta |-- \exists x A(x) \iff S,\eta |-- A(a) con a \in D$  S contiene il Dominio

$S,\eta |-- \forall x A(x) \iff S,\eta |-- A(a) \forall a \in D$

Dirò che una formula è valida se è sempre valide qualsiasi struttura

Tutte le formule dimostrate con il metodo dei Tableau devono essere valide

Modello è un interpretazione che rende vere le formule di una teoria
Tutte le formule del primo Ordine dimostrabili sono sempre valide per il teorema di Completezza
Teoria è fatta da un insieme di assiomi e da una logica $T = Ax + L$
Ad esempio prendendo la teoria dell'Aritmetica di Peano gli diamo una serie di assiomi e una logica 
i cui modelli sono tutte le formule che soddisfano gli assiomi e la logica

Poi si può modificare la logica e in quel caso mescolando assiomi con logiche diverse otteniamo Teorie diverse

Sono degli assiomi dell'Aritmetica di Peano (da sistemare)
$\forall x \neg S(x) = 0
\forall x x + 0 = x
\forall x,y x + s(y) = s(x + y)
\forall x x * 0 = 0
\forall x,y x * s(y) = (x * y) + x 
\forall x,y (s(x) = s(y)) \rightarrow x = y$


