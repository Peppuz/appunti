%Capitolo sulla Logica Predicativa
\chapter{Logica Predicativa}
La logica Predicativa, detta anche logica del primo ordine, si ha la possibilità
di predicare le proprietà di una classe di individui.\newline
É una logica semidecidibile, in quanto è ricorsivamente enumerabile ma non ricorsivo,
per cui non sempre tramite una sequenza di passi si riesce a capire la tipologia di formula.

%Linguaggio e sintassi predicativa
\input{Logica/Predicativa/sintassi}

%Sistemi deduttivi predicativi
\section{Sistemi Deduttivi predicativi}
Comprendere la tipologia della formula tramite la sua semantica nella logica predicativa
è più complesso rispetto alla semantica della logica proposizionale per cui l'apparato deduttivo
e la sua correttezza e completezza rispetto alla semantica assumono particolare rilevanza.

\begin{defi}
Una sostituzione è una funzione $\sigma:VAR \mapsto TERM$ definita induttivamente:
\end{defi}
\begin{enumerate}
    \item $T\sigma = T$ e $F \sigma = F$
    \item se $c$ è una costante allora $c \sigma = c$
    \item se $x$ è una variabile allora $x \sigma = x$
    \item se $f$ è un simbolo di funzione di arietà $n$ allora
          $f(t_1,\dots,t_n)\sigma = f(t_1\sigma,\dots,t_n\sigma)$
\end{enumerate}
La sostituzione $\sigma$ di un termine $t$ al posto di un simbolo di variabile $x$
è indicata da $\sigma = \{t/x \}$.

\begin{lem}
Se $t$ è un termine e $\sigma$ è una sostituzione, allora $t\sigma$ è un termine
\end{lem}
%Da fare dimostrazione

\begin{defi}
    Data una formula $S$, la formula $S\sigma$ è definita nel seguente modo:
\end{defi}
\begin{enumerate}
    \item se $S = P(t_1,\dots,t_n)$ è una formula atomica allora
          allora $S\sigma = P(t_1\sigma,\dots,t_n\sigma)$
    \item se $S = (\neg A)$ allora $S \sigma = \neg(A \sigma)$
    \item se $S = (A \circ B)$ allora $S \sigma = A\sigma \circ B \sigma$
    \item se $S = (\forall x A)$ allora $S \sigma = \forall x (A  \sigma _ x)$
    \item se $S = (\exists x A)$ allora $S \sigma = \exists x (A \sigma _ x)$
\end{enumerate}
%Tableau predicativi
\subsection{Tableau Predicativi}
I tableau predicativi è un apparato deduttivo che permette di stabilire la tipologia
di una formula del linguaggio mediante l'applicazione di una serie di regole.
Mantiene le stesse modalità di dimostrazione dei tableaux proposizionale e le stesse
definizioni di espansioni del Tableaux.

$\begin{array}{cc}
\toprule S, T \exists x A(x) & \qquad S, F \exists x A(x) \\
\midrule S,T A(a) \text{con a un nuovo simbolo di variabile} & \qquad S,F A(a),F \exists x A(x) \\
\end{array}$

$\begin{array}{c}
\toprule S,T \forall x A(x) \\
\midrule S, T A(a), T \forall x A(x)\\
\end{array}$

\qquad $\begin{array}{c}
\toprule S,F \forall x A(x) \\
\midrule S,F A(a) \text{con a un nuovo simbolo di variabile} \\
\end{array}$

Nelle tableuax regole $T \exists$ e $F \forall$ bisogna introdurre un nuovo simbolo di
variabile, in quanto non si poteva conoscere prima dell'applicazione della regole
il valore della variabile, mentre nelle altre due si può utilizzare qualsiasi variabile.
Il fatto di portarsi dietro la formula nell'applicazione delle regole dei tableaux
porta alla semidecidibilità.

Nei tableau predicativi l'ordine di applicazione delle regole, quando è possibile
sceglierlo, cambia la chiusura o meno del Tableaux.

Le euristiche nell'applicazione delle regole dei Tableaux sono le seguenti:
\begin{itemize}
    \item Applicare prima le regole che non ramificano il tableaux
    \item Applicare prima le regole che vincolano all'introduzione di nuovi parametri
    \item Quando si ha la possibilità di scegliere il parametro, conviene sceglierlo uno già scelto.
\end{itemize}
%Tableau formule

%Esempi


%Semantica Predicativa
\section{Semantica}
La semantica di una logica consente di dare un significato e un interpretazione
 alle formule del Linguaggio attraverso le tabelle di verità.\newline
Si definisce $v(T) = 1$ e $v(F) = 0$ per cui $1$ rappresenta la verità mentre lo $0$
la falsità di una variabile,sottoformula e formula.

I connettivi della Logica Proposizionale hanno la seguente tabella di verità:\newline
%Tabella di Verità degli operatori
$\begin{array}{ccccccc}
\toprule
\text{A} & \text{B} & A \land B & A \lor B & \neg A & A \Rightarrow B & A \iff B \\
\midrule
    0 & 0 & 0 & 0 & 1 & 1 & 1 \\
    0 & 1 & 0 & 1 & 1 & 1 & 0 \\
    1 & 0 & 0 & 1 & 0 & 0 & 0 \\
    1 & 1 & 1 & 1 & 0 & 1 & 1 \\
\bottomrule
\end{array}$\newline

Una formula nella logica proposizionale può essere di tre diversi tipi:
%Tipologie di formule
\begin{description}
    \item[Tautologica]: la formula è soddisfatta da qualsiasi valutazione della Formula
    \item[Soddisfacibile non Tautologica]:la formula è soddisfatta da qualche valutazione
                        della formula ma non da tutte.
    \item[Contraddizione]:la formula non viene mai soddisfatta
\end{description}

Esempio:\newline
Formula $A \land \neg A$ \quad contraddizione

%Tabella di Verità
$\begin{array}{ccc}
\toprule A & \neg A & A \land \neg A \\
\midrule
        0 & 1 & 0 \\
        1 & 0 & 0 \\
\bottomrule
\end{array}$\newpage

Formula $Z = (A \land B) \lor C$  soddisfacibile non Tautologica

%Tabella di Verità
$\begin{array}{ccccc}
\toprule A & B & C & A \land B & (A \land B) \lor C \\
\midrule
         0 & 0 & 0 & 0 & 0 \\
         0 & 0 & 1 & 0 & 1 \\
         0 & 1 & 0 & 0 & 0 \\
         0 & 1 & 1 & 0 & 1 \\
         1 & 0 & 0 & 0 & 0 \\
         1 & 0 & 1 & 0 & 1 \\
         1 & 1 & 0 & 1 & 1 \\
         1 & 1 & 1 & 1 & 1 \\
\bottomrule
\end{array}$\newline

Formula $X = (A \land B) \Rightarrow (\neg A \land C)$ \quad Soddisfacibile  non tautologica\newline

%Tabella di Verità della formula X
$\begin{array}{ccccccc}
\toprule A & B & C & \neg A & A \land B & \neg A \land C & X\\
\midrule
         0 & 0 & 0 & 1 & 0 & 0 & 1 \\
         0 & 0 & 1 & 1 & 0 & 1 & 1 \\
         0 & 1 & 0 & 1 & 0 & 0 & 1 \\
         0 & 1 & 1 & 1 & 0 & 1 & 1 \\
         1 & 0 & 0 & 0 & 0 & 0 & 1 \\
         1 & 0 & 1 & 0 & 0 & 0 & 1 \\
         1 & 1 & 0 & 0 & 1 & 0 & 0 \\
         1 & 1 & 1 & 0 & 1 & 0 & 0 \\
\bottomrule
\end{array}$ \newline

Formula $Y = \neg(A \land B) \iff (A \lor B \Rightarrow C)$ soddisfacibile non Tautologica

%Tabella di Verità
$\begin{array}{cccccccc}
\toprule
A & B & C & A \land B & \neg(A \land B) & A \lor B & (A \lor B) \Rightarrow C & Y \\
\midrule
0 & 0 & 0 & 0 & 1 & 0 & 1 & 1 \\
0 & 0 & 1 & 0 & 1 & 0 & 1 & 1 \\
0 & 1 & 0 & 0 & 1 & 1 & 0 & 0 \\
0 & 1 & 1 & 0 & 1 & 1 & 1 & 1 \\
1 & 0 & 0 & 0 & 1 & 1 & 0 & 0 \\
1 & 0 & 1 & 0 & 1 & 1 & 1 & 1 \\
1 & 1 & 0 & 1 & 0 & 1 & 0 & 1 \\
1 & 1 & 1 & 1 & 0 & 1 & 1 & 0 \\
\bottomrule
\end{array}$

\subsection{Modelli e decidibilità}
Si definisce \emph{modello}, indicato con $M \models A$, tutte le valutazioni booleane
che rendono vera la formula $A$.
Si definisce \emph{contromodello}, indicato con , tutte le valutazioni booleane
che rendono falsa la formula $A$.

La logica proposizionale è decidibile (posso sempre verificare il significato di una formula).
Esiste infatti una procedura effettiva che stabilisce la validità o no di una formula, o se questa
ad esempio è una tautologia.
In particolare il verificare se una proposizione è tautologica o meno è l’operazione di decibi-
lità principale che si svolge nel calcolo proposizonale.

\begin{defi}
    Se $M \models A$ per tutti gli $M$, allora $A$ è una tautologia e si indica $\models A$
\end{defi}

\begin{defi}
    Se $M \models A$ per qualche $M$, allora $A$ è soddisfacibile
\end{defi}

\begin{defi}
Se $M \models A$ non è soddisfatta da nessun $M$, allora $A$ è insoddisfacibile
\end{defi}


%Equivalenze Logiche
\subsection{Equivalenze logiche}
Nella logica predicativa si definisce due formule semanticamente equivalenti,
indicato con $P \equiv Q$, se hanno gli stessi modelli.
Le equivalenze della logica predicativa sono le seguenti:
\begin{enumerate}
    \item $\forall x P \equiv \neg \exists x \neg P$
    \item $\neg \forall x P \equiv \exists x \neg P$
    \item $\exists x P \equiv \neg \forall x \neg P$
    \item $\neg \exists x P \equiv \forall x \neg P$
    \item $\forall x \forall y P \equiv \forall y \forall x P$
    \item $\exists x \exists y P \equiv \exists y \exists x P$
    \item $\forall x(P_1 \land P_2) \equiv \forall x P_1 \land \forall x P_2$
    \item $\exists x(P_1 \lor P_2) \equiv \exists x P_1 \lor \exists x P_2$
\end{enumerate}

%Teorie logiche del primo Ordine
\input{Logica/Predicativa/teorie}

%Traduzione da italiano a linguaggio formale
\section{Traduzione linguaggio Naturale in linguaggio predicativo}

%Frase Tutti i docenti hanno un età maggiore di 24 anni
Esercizio: Tutti i docenti hanno un età maggiore di 24 anni

Costanti: $24$
Predicati:$Docente(x)$,$>(x,y)$
Funzioni:$eta(x)$

$\forall x (Docente(x) \rightarrow eta(x) > 24)$

Esercizi:Tutti i docenti hanno una chiave d'accesso all'edificio U6

Costanti:$U6$
Predicati:$Docente(x)$,$Avere(y)$,$Chiave(x,y)$
Funzioni:non presente

\begin{equation*}
    \forall x (Docente(x) \land \exists y (Chiave(y,U6) \land Avere(y)))
\end{equation*}

Esercizio:Tutti i canali televisivi con una share maggiore del 10\% sono
          considerati canali principali

Costanti:$10\%$
Predicati:$Canale(x)$,$>(x,y)$,$CanalePrincipale(x)$
Funzioni:$share(x)$
\begin{equation*}
    \forall x (Canale(x) \land share(x) > 10\% \rightarrow CanalePrincipale(x))
\end{equation*}

Esercizio:il fratello di Marco ha copiato il compito ed è stato respinto

Costanti:$Marco$,$compito$
Predicati:$Copiare(x,y)$,$Uomo(x)$,$Bocciato(x)$,$Fratello(x,y)$
Funzioni: non presenti
\begin{equation*}
    \exists x (Uomo(x) \land Fratello(x,Marco) \land Copiare(x,compito) \rightarrow Bocciato(x))
\end{equation*}

Esercizio: Tutte le sere gli studenti ascoltano musica Uzbeka e bevono caffè

Costanti:$musicaUzbeka$,$caffè$
Predicati:$Studenti(x)$,$Ascoltare(x,y)$,$Bere(x,y)$,$Sera(y)$
Funzioni:non presenti
\begin{equation*}
\forall x,y (Studente(x) \land Sera(y) \rightarrow (Ascoltare(x,musicaUzbeka) \land Bere(x,caffè)))
\end{equation*}

Esercizio:Gli studenti che non si iscrivono all'appello di Fondamenti non possono svolgere l'esame

Costanti:$Fondamenti$
Predicati:$Studente(x)$,$Iscrivere(x,y)$,$Svolgere(x,y)$,$Esame(y)$
Funzioni:
\begin{equation*}
\forall x (Studente(x) \land \neg Iscrivere(x,Fondamenti) \rightarrow
\exists y (Esame(y) \land \neg Svolgere(x,y)))
\end{equation*}

Esercizio:Tutti i professori fanno esami

Costanti: non presenti \newline
Predicati:$Professore(x)$,$Fare(x,y)$,$Esame(y)$ \newline
Funzioni: non presenti
\begin{equation*}
    \forall x (Professore(x) \rightarrow \exists y(Esame(y) \land Fare(x,y)))
\end{equation*}

Esercizio: Se uno studente non è iscritto via Sifa ad un appello non può fare l'esame

Costanti: non presenti \newline
Predicati:$Studente(x)$,$Iscritto(x,y)$,$Appello(y)$,$Esame(x)$ \newline
Funzioni: non presenti
\begin{equation*}
    \forall x (Studenti(x) \land \exists y(Appello(y) \land \neg Iscritto(x,y)) \rightarrow \neg Esame(x))
\end{equation*}

Esercizio: il voto di un esame universitario va da 0 a 30 e lode

Costanti:$0$ e $30L$ \newline
Predicati:$Esame(x)$,$>=(x,y)$,$<=(x,y)$ \newline
Funzioni:$voto(x)$
\begin{equation*}
    \forall x (Esame(x) \rightarrow voto(x) >= 0 \land voto(x) <= 30L)
\end{equation*}

Esercizio:Tutti i docenti sono sposati con una donna antipatica

Costanti: non presenti \newline
Predicati:$Docente(x)$,$Donna(y)$,$Sposati(x,y)$,$Antipatica(y)$ \newline
Funzioni: non presenti
\begin{equation*}
    \forall x (Docente(x) \rightarrow \exists y(Donna(y) \land Antipatica(y) \land Sposati(x,y)))
\end{equation*}

Esercizio:Marco ha un capo magnanimo

Costanti:$Marco$ \newline
Predicati:$Capo(x,y)$,$Magnanimo(x)$ \newline
Funzioni: non presenti \newline
\begin{equation*}
    Capo(x,Marco) \land Magnanimo(x)
\end{equation*}

Esercizio:L'everest è la montagna più alta al mondo

Costanti:$Everest$ \newline
Predicati:$Montagna(x)$,$<(x,y)$ \newline
Funzioni:$altezza(x)$
\begin{equation*}
    \forall x (Montagna(x) \rightarrow altezza(x) < altezza(Everest))
\end{equation*}

Esercizio:Se ogni amico di Mario è amico di Diego e Pietro non è amico di
          Mario, allora Pietro non è amico di Diego

Costanti:$Mario$,$Diego$,$Pietro$ \newline
Predicati:$Amico(x,y)$ \newline
Funzioni:non presenti
\begin{equation*}
    \forall x (((Amico(x,Mario) \rightarrow Amico(x,Diego)) \land \neg Amico(Pietro,Mario))
                \rightarrow \neg Amico(Pietro,Diego))
\end{equation*}

Esercizio:Se tutti i filosofi intelligenti sono curiosi, e solo i tedeschi sono
filosofi intelligenti, allora, se ci sono filosofi intelligenti, qualche
tedesco è curioso.

Esercizio:Se tutti gli studenti sono persone serie, tutti gli studenti sono
studiosi e tutte le persone serie e studiose non fanno tardi la
sera, allora se esiste qualcuno che fa tardi la sera, non tutti sono
studenti

Esercizio:Tutti i ragazzi e le ragazze iscritti all’università sanno usare il
          computer o conoscono qualcuno che lo sa usare

