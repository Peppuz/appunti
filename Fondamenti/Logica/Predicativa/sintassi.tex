\section{Linguaggio e Sintassi}
Un linguaggio predicativo $L$ è composto dai seguenti insiemi di simboli:
\begin{enumerate}
    \item Insieme di variabili individuali(infiniti) $x,y,z,\dots$
    \item Connettivi logici $\land \lor \neg \rightarrow \iff$
    \item Quantificatori esistenziali $\forall \exists$
    \item Simboli ( , )
    \item Costanti proposizionali $T,F$
    \item Simbolo di uguaglianza $=$,eventualmente assente
\end{enumerate}
Questa è la parte del linguaggio tipica di ogni linguaggio del primo ordine poi
ogni linguaggio definisce la propria segnatura ossia definisce in maniera autonomo:
\begin{enumerate}
    \item Insiemi di simboli di costante $a,b,c,\dots$
    \item Simboli di funzione con arieta $f,g,h,\dots$
    \item Simboli di predicato $P,Q,Z,\dots$ con arietà
\end{enumerate}

%Inserire Esempio

%Definizione di Termini e Formule ben formate
Per definire le formule ben formate della logica Predicativa bisogna prima definire
l'insieme di termini e le formule atomiche.

\begin{defi}
    L'insieme $TERM$ dei termini è definito induttivamente come segue
    \begin{enumerate}
        \item Ogni variabile e costante sono dei Termini
        \item Se $t_1 \dots t_n$ sono dei termini e $f$ è un simbolo di funzione di arietà $n$
              allora $f(t_1,\dots,t_n)$ è un termine
    \end{enumerate}
\end{defi}

\begin{defi}
    L'insieme $ATOM$ delle formule atomiche è definito come:
    \begin{enumerate}
        \item $T$ e $F$ sono degli atomi
        \item Se $t_1$ e $t_2$ sono dei termini, allora $t_1 = t_2$ è un atomo
        \item Se $t_1,\dots,t_n$ sono dei termini e $P$ è un predicato a $n$ argomenti,
              allora $P(t_1,\dots,t_n)$ è un atomo.
    \end{enumerate}
\end{defi}

\begin{defi}
    L'insieme delle formule ben formate($FBF$) di $L$ è definito induttivamente come
    \begin{enumerate}
        \item Ogni atomo è una formula
        \item Se $A,B \in FBF$, allora $\neg A$, $A \land B$,$A \lor B$,$A \rightarrow B$
              e $A \iff B$ appartengono alle formule ben formate
        \item Se $A \in FBF$ e $x$ è una variabile, allora $\forall x A$ e $\exists x A$
              appartengono alle formule ben formate
        \item Nient'altro è una formula
    \end{enumerate}
\end{defi}

%Inserire Esempi

%Precedenza degli Operatori
La precedenza tra gli operatori logici è definita nella logica predicativa come segue
$\forall,\exists,\neg,\land,\lor,\rightarrow,\iff$ e si assume che gli operatori associno a destra.

%Inserire Esempi

%Variabili legate e chiuse
\begin{defi}
    L'insieme $var(t)$ delle variabili di un termine $t$ è definito come segue:
    \begin{itemize}
        \item $var(t) = \{t \}$, se $t$ è una variabile
        \item $var(t) = \emptyset$ se $t$ è una costante
        \item $var(f(t_1,\dots,t_n)) = \bigcup _{i = 1} ^n var(t_i)$
        \item $var(R(t_1,\dots,t_n)) = \bigcup _{i = 1} ^ n var(t_i)$
    \end{itemize}
\end{defi}

%Termini chiusi ed aperti
Si definisce \emph{Termine chiuso}, un termine che non contiene variabili altrimenti
si definisce il termine come \emph{chiuso}.\newline
Le variabili nei termini e nelle formule atomiche possono essere libere
 in quanto gli unici operatori che "legano" le variabili sono i quantificatori.

Il campo di azione dei quantificatori si riferisce soltanto alla parte in cui
si applica il quantificatore per cui una variabile si dice \emph{libera}
se non ricade nel campo di azione di un quantificatore altrimenti la variabile si dice \emph{vincolata}.
