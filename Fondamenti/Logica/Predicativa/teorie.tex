\section{Teorie del Primo Ordine}
Una teoria è un insieme di formule di un linguaggio del primo ordine $L$ e la teoria
la definiamo a partire dalla relazione $\models$

\begin{defi}
    Sia $\sum$ un insieme di formule di $L$, che chiameremo assiomi, si definisce
\emph{teoria}, l'insieme $T_ {\sum}$ delle formule $\phi$ di $L$ tali che $\sum \models \phi$
\end{defi}

\begin{defi}
    Una teoria T è un insieme di enunciati chiuso rispetto alla conseguenza logica
    ovvero $T \models \phi$ implica $\phi \in T$
\end{defi}

\begin{defi}
    Una teoria del primo ordine T è \emph{completa} se per ogni formula $\phi \in L$
è verificata una e una sola delle due:$T \models \phi$ o $T \models \neg \phi$.
\end{defi}

Per rappresentare un particolare dominio, ad esempio i numeri naturali, i grafi,
dobbiamo definire degli assiomi che permettono di catturare la struttura e il comportamento
degli oggetti del dominio che intendiamo trattare.

Una teoria, come l'aritmetica di Peano, la teoria dei gruppi ed eccetera, è un insieme
di assiomi che descrivono certe proprietà degli oggetti che si definiscono.\newline
Gli assiomi hanno il compito di restringere la classe dei modelli della logica
del primo ordine ai modelli degli oggetti che si vogliono trattare.

Consideriamo il linguaggio della teoria dei numeri, definito negli appunti, per
rappresentare l'\emph{aritmetica di Peano} si usano i seguenti assiomi:
\begin{enumerate}
    \item $\forall x \neg (s(x) = 0)$
    \item $\forall x \neg(x < 0)$
    \item $\forall x,y (x < s(y) \rightarrow (x < y \lor x = y))$
    \item $\forall x,y (x < y \lor x = y \lor x > y)$
    \item $\forall x \ x + 0 = x$
    \item $\forall x,y \ x + s(y) = s(x+y)$
    \item $\forall x x * 0 = 0$
    \item $\forall x,y x * s(y) = (x * y) + x$
    \item $\forall x,y (s(x) = s(y)) \rightarrow x = y$
    \item $P(0)$ dove $P$ indica i numeri Pari
    \item $\neg D(0)$ dove $D$ indica i numeri Dispari
    \item $\forall x (P(x) \rightarrow \neg P(s(x)))$
    \item $\forall x (D(x) \rightarrow \neg D(S(x)))$
\end{enumerate}
