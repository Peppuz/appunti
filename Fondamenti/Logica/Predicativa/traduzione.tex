%Paragrafo sulla traduzione dall'italiano a un linguaggio formale della logica del 1° ordine
\section{Traduzione in linguaggio formale}
La traduzione in linguaggio formale della logica predicativa consiste nel formalizzare
le frasi della lingua naturale, in particolare l'italiano per noi italiani, in
formule della logica proposizionale attraverso la definizione della realtà da rappresentare.

Per rappresentare le frasi del linguaggio naturale in frasi formali bisogna definire:
\begin{enumerate}
    \item quali sono le eventuali costanti della frase da tradurre
    \item quali sono le eventuali funzioni della frase da tradurre
    \item quali sono i predicati della frase da tradurre
\end{enumerate}

Le costanti sono rappresentati nel linguaggio naturale da sostantivi mentre i
predicati e le funzioni sono rappresentati da forme verbali.
Alcuni esempi di rappresentazione da italiano a linguaggio formale sono i seguenti:

Esempio: Tutti gli uomini sono mortali,Socrate è un uomo allora Socrate è un mortale

Costanti:Socrate \newline
Predicati:$Uomo(x),Mortale(x)$ \newline
Funzioni: non presenti \newline
\begin{equation*}
\forall x ((Uomo(x) \rightarrow Mortale(x)) \land Uomo(Socrate) \rightarrow Mortale(Socrate))
\end{equation*}

Esempio: un cugino di Marco non ha cani

Costanti: $Marco$\newline
Predicati:$Cugino(x,y),Avere(x,y),Cane(y)$\newline
Funzioni: non sono presenti
\begin{equation*}
\exists x (Cugino(x,Marco) \land \forall y (Cane(y) \rightarrow \neg Avere(x,y)))
\end{equation*}

Esempio: Ogni treno ha un numero identificativo

Costanti: non presenti \newline
Predicati: $Treno(x)$,$Avere(x)$\newline
Funzioni:$id(x)$
\begin{equation*}
\forall x (Treno(x) \rightarrow Avere(id(x)))
\end{equation*}
