%Capitolo sui Numeri ordinali
\section{Ordinali}
I numeri ordinali sono usati per indicare la posizione di un elemento di una sequenza
ordinata, al contrario dei numeri cardinali che indicano la dimensione di un insieme.

Un insieme $\alpha$ è un ordinale se sono verificate le seguenti condizioni:
\begin{enumerate}
    \item $\beta \in \alpha$ implica $\beta \subset \alpha$
    \item $\beta \in \alpha$ e $\gamma \in \alpha$ implica che sia verificata una delle seguenti:
          \begin{itemize}
              \item $\beta = \gamma \land \beta \not \in \gamma \land \gamma \not \in \beta$
              \item $\beta \neq \gamma \land \beta \in \gamma \land \gamma \not \in \beta$
              \item $\beta \neq \gamma \land \beta \not \in \gamma \land \gamma \in \beta$
          \end{itemize}
    \item $\beta \subset \alpha$ e $\beta \neq \emptyset$ implica che esiste un $\gamma \in \alpha$
          tale che $\beta \cap \gamma = \emptyset$
\end{enumerate}

Se $\alpha$ è un ordinale allora il suo successore, indicato con $succ(\alpha)$,
è il numero ordinale $succ(\alpha) = \alpha \cup \{\alpha\}$ e non ci sono altri
numeri ordinali tra $\alpha$ e $succ(\alpha)$.

Esistono degli ordinali che non hanno successori, chiamati \emph{ordinali limite}
ed il più piccolo ordinale limite è $\omega$.
Una definizione formale di ordinale limite è la seguente:
Un ordinale $\lambda$ è un ordinale limite se per ogni $\alpha < \lambda$ si ha $succ(\alpha) < \lambda$.
