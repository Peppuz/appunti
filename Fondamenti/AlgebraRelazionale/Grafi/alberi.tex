\section{DAG ed Alberi}
Si definisce come \emph{DAG}(Directed Acyclic Graph), un grafo orientato senza cicli.
%Inserire esempi degli DAG

Si definisce come \emph{Albero}, un DAG connesso con un solo nodo sorgente,detto \emph{radice},
in cui ogni nodo diverso dalla radice ha un solo nodo entrante.\newline
I nodi privi di archi entranti sono detti \emph{foglie} dell'albero.

%Inserire esempi e proprietà degli Alberi
%Paragrafo sugli Alberi
\section{Alberi}
Si definisce come \emph{Albero libero}, un DAG connesso con un solo nodo sorgente, detto \emph{radice},
in cui ogni nodo diverso dalla radice ha un solo nodo entrante.\newline
I nodi privi di archi entranti sono detti \emph{foglie} dell'albero.

%Inserire esempi e proprietà degli Alberi
%%Paragrafo sugli Alberi
\section{Alberi}
Si definisce come \emph{Albero libero}, un DAG connesso con un solo nodo sorgente, detto \emph{radice},
in cui ogni nodo diverso dalla radice ha un solo nodo entrante.\newline
I nodi privi di archi entranti sono detti \emph{foglie} dell'albero.

%Inserire esempi e proprietà degli Alberi
%%Paragrafo sugli Alberi
\section{Alberi}
Si definisce come \emph{Albero libero}, un DAG connesso con un solo nodo sorgente, detto \emph{radice},
in cui ogni nodo diverso dalla radice ha un solo nodo entrante.\newline
I nodi privi di archi entranti sono detti \emph{foglie} dell'albero.

%Inserire esempi e proprietà degli Alberi
%\input{Esempi/alberi}Esempi Alberi!!!!

L'albero è una struttura matematica importantissima in informatica utilizzata per
rappresentare una serie di situazioni, come ad esempio organizzazioni gerarchiche di dati,
procedimenti enumerativi o decisionali, e ve ne esistono un'infinita di implementazioni di alberi
però iniziamo ad analizzare per prima gli alberi liberi binari.

%Inserire immagine albero binario

%Metodi di visita di un albero
I metodi di visita di un albero binario sono 3:
\begin{itemize}
  \item inorder: si visiona prima il sottoalbero sinistro poi il nodo e infine il sottoalbero destro
  \item preorder: si visiona prima il nodo poi i suoi sottoalberi
  \item postorder: si visionano prima i sottoalberi ed infine il nodo
\end{itemize}
Il primo metodo di visita viene usato soprattutto negli alberi binari di ricerca per
stampare gli elementi dell'albero in maniera crescente mentre in un albero binario
normale la scelta di quale metodo di visita utilizzare è inifluente e ogni programmatore
sceglie nell'utilizzo quale metodo di visita utilizzare per stampare l'albero.
Il cammino dalla radice ad un elemento foglia dell'albero richiede al massimo $O(h)$,
in cui $h$ è l'altezza dell'albero, in quanto richiede di scendere di livello fino
ad arrivare alle foglie, che si trovano al livello $h$.

La specifica di un albero binario, in cui ogni implementazione per essere valida deve prevedere:\newline
\textbf{Item} search(Tree T,Item k);\newline
\textbf{void} insert(Tree T,Item x);\newline
\textbf{Item} delete(Tree T,Item x);\newline
\textbf{Item} mininum(Tree T);\newline
\textbf{Item} maxinum(Tree T);\newline
\textbf{Item} predecessor(Tree T,Item x);\newline
\textbf{Item} successor(Tree T,Item x);\newline
Esempi Alberi!!!!

L'albero è una struttura matematica importantissima in informatica utilizzata per
rappresentare una serie di situazioni, come ad esempio organizzazioni gerarchiche di dati,
procedimenti enumerativi o decisionali, e ve ne esistono un'infinita di implementazioni di alberi
però iniziamo ad analizzare per prima gli alberi liberi binari.

%Inserire immagine albero binario

%Metodi di visita di un albero
I metodi di visita di un albero binario sono 3:
\begin{itemize}
  \item inorder: si visiona prima il sottoalbero sinistro poi il nodo e infine il sottoalbero destro
  \item preorder: si visiona prima il nodo poi i suoi sottoalberi
  \item postorder: si visionano prima i sottoalberi ed infine il nodo
\end{itemize}
Il primo metodo di visita viene usato soprattutto negli alberi binari di ricerca per
stampare gli elementi dell'albero in maniera crescente mentre in un albero binario
normale la scelta di quale metodo di visita utilizzare è inifluente e ogni programmatore
sceglie nell'utilizzo quale metodo di visita utilizzare per stampare l'albero.
Il cammino dalla radice ad un elemento foglia dell'albero richiede al massimo $O(h)$,
in cui $h$ è l'altezza dell'albero, in quanto richiede di scendere di livello fino
ad arrivare alle foglie, che si trovano al livello $h$.

La specifica di un albero binario, in cui ogni implementazione per essere valida deve prevedere:\newline
\textbf{Item} search(Tree T,Item k);\newline
\textbf{void} insert(Tree T,Item x);\newline
\textbf{Item} delete(Tree T,Item x);\newline
\textbf{Item} mininum(Tree T);\newline
\textbf{Item} maxinum(Tree T);\newline
\textbf{Item} predecessor(Tree T,Item x);\newline
\textbf{Item} successor(Tree T,Item x);\newline
Esempi Alberi!!!!

L'albero è una struttura matematica importantissima in informatica utilizzata per
rappresentare una serie di situazioni, come ad esempio organizzazioni gerarchiche di dati,
procedimenti enumerativi o decisionali, e ve ne esistono un'infinita di implementazioni di alberi
però iniziamo ad analizzare per prima gli alberi liberi binari.

%Inserire immagine albero binario

%Metodi di visita di un albero
I metodi di visita di un albero binario sono 3:
\begin{itemize}
  \item inorder: si visiona prima il sottoalbero sinistro poi il nodo e infine il sottoalbero destro
  \item preorder: si visiona prima il nodo poi i suoi sottoalberi
  \item postorder: si visionano prima i sottoalberi ed infine il nodo
\end{itemize}
Il primo metodo di visita viene usato soprattutto negli alberi binari di ricerca per
stampare gli elementi dell'albero in maniera crescente mentre in un albero binario
normale la scelta di quale metodo di visita utilizzare è inifluente e ogni programmatore
sceglie nell'utilizzo quale metodo di visita utilizzare per stampare l'albero.
Il cammino dalla radice ad un elemento foglia dell'albero richiede al massimo $O(h)$,
in cui $h$ è l'altezza dell'albero, in quanto richiede di scendere di livello fino
ad arrivare alle foglie, che si trovano al livello $h$.

La specifica di un albero binario, in cui ogni implementazione per essere valida deve prevedere:\newline
\textbf{Item} search(Tree T,Item k);\newline
\textbf{void} insert(Tree T,Item x);\newline
\textbf{Item} delete(Tree T,Item x);\newline
\textbf{Item} mininum(Tree T);\newline
\textbf{Item} maxinum(Tree T);\newline
\textbf{Item} predecessor(Tree T,Item x);\newline
\textbf{Item} successor(Tree T,Item x);\newline
%Esempi Alberi!!!!

%Inserire caratteristiche e Nomenclatura degli Alberi
Le tipologie di Alberi sono:
\begin{description}
    \item[Albero Binario]: albero con al più due rami per nodo 
    \item[Albero Strettamento Binario]: albero da cui da ogni nodo partono proprio due rami 
    \item[Albero Bilanciato]: albero in cui tutti i cammini hanno la stessa lunghezza
\end{description}

%Fare definizione anche delle foreste

%Alberi binari di Ricerca e metodo di visita degli alberi
\subsection{Alberi binari di ricerca e metodo di visita}
Dato un insieme di nodi in cui è definita una relazione d'ordine, si definisce come
\emph{albero di ricerca} un albero in cui tutti i nodi della radice sinistra sono
minori della radice e tutti i nodi a destra della radice sono maggiori e ogni sottoalbero
è anch'esso un albero di ricerca.

%Fare Esempio

Il metodo di visità di Albero di Ricerca è il seguente:\newline
Partendo dalla radice per visitare tutti i nodi dell'albero bisogna:
\begin{itemize}
    \item visita del sottoalbero sinistro
    \item visita della radice
    \item visita del sottoalbero destro
\end{itemize}
Si continua ad applicare la visita al sottoalbero fino a quanto si arriva a visitare la radice
attraverso la ricorsione

%Fare esempio
