\section{Struttura relazionale}
Una struttura relazionale $SR$ è una $n$-upla in cui il primo componente è un Insieme
non vuoto $S$ e le rimanenti componenti sono relazioni su $S^n$.

\subsection{Tipologie di Ordinamenti}
\begin{description}
    \item[PREORDINE]: è una struttura relazionale $(S,R)$ in cui $S$ è un insieme non vuoto
          e $R$ è una relazione binaria \emph{riflessiva} e \emph{transitiva} su $S x S$
    \item[ORDINE STRETTO]: è una struttura relazionale $(S,R)$ in cui $R$ è una
          relazione binaria \emph{irriflessiva} e \emph{transitiva} su $S x S$
    \item[ORDINE LARGO(POSET)]: è una struttura relazionale $(S,R)$ in cui $R$ è una
          relazione binaria \emph{riflessiva,antisimmetrica} e \emph{transitiva} su $S x S$.
\end{description}

Una relazione $R$ è un ordinamento sull'insieme $S$ se e solo se $\forall x,y \in S$
vale solo una delle proprietà di \emph{tricotomia}:
\begin{itemize}
    \item $x = y \land \neg(xRy) \land \neg(yRx)$
    \item $xRy \land x \neq y \land \neg(yRx)$
    \item $yRx \land x \neq y \land \neg(xRy)$
\end{itemize}

\begin{prop}
Se $(S,R)$ è una struttura relazionale anche $(S,R^{-1})$ rappresenta la stessa
struttura relazionale chimata e $(S,R^{-1})$ è il \emph{duale} di $(S,R)$.
\end{prop}

%Dimostrazione della Proposizione!!!!!

%Proposizione lunghezza cicli di un grafo del poset + dimostrazione
\begin{prop}
Il grafo di un Poset non ha cicli di lunghezza maggiore di 1
\end{prop}

%Ordine Lessicografico


%Diagramma di Hasse
\subsection{Diagramma di Hasse}
Il grafo di un poset può essere rappresentato dal \emph{diagramma di Hasse}, un grafo
con le seguenti proprietà:
\begin{itemize}
    \item si prescinde dal disegnare i cappi in quanto il poset è una relazione riflessiva
    \item si prescinde dal disegnare gli archi definiti per transitività
    \item il grafo si legge dal basso verso l'alto
\end{itemize}

esempio

\begin{prop}
Dato un poset $(S,R)$ e un insieme $X \subseteq S$ si hanno le seguenti proprietà:
\begin{description}
    \item[massimale]: un elemento $s \in S$ è massimale se $\not \exists s' \in S : s \leq s'$
    \item[minimale]: un elemento $s \in S$ è minimale se $\not \exists s' \in S : s \geq s'$
    \item[maggiorante]: un elemento $s \in S$ è un maggiorante se $\forall x \in X$ si ha $s \geq s$
    \item[minorante]: un elemento $s \in S$ è un minorante se $\forall x \in X$ si ha $s \leq x$
    \item[minimo maggiorante]:un elemento $s \in S$ è un minimo maggiorante, indicato con $\sqcup X$,
          se è un maggiorante e per ogni altro maggiorante $s'$ di $X$ si ha $s \leq s'$.
    \item[massimo minorante]: un elemento $s \in S$ è un massimo minorante, indicato con $\sqcap X$,
          se è un minorante e per ogni altro minorante $s'$ di $X$ si ha $s' \leq s$
    \item[massimo]: se $\sqcup X \in X$ si dice che $\sqcup X$ è un massimo.
    \item[minimo]: se $\sqcap X \in X$ si dice che $\sqcap X$ è un minimo.
\end{description}
\end{prop}
%Proprietà con dimostrazioni

%Definizione di Buon Ordinamento

%Reticoli
\subsection{Reticoli}
Un \emph{Reticolo} è un poset $(S,R)$ in cui per ogni coppia $x,y \in S$ esistono
il massimo minorante(indicato con $x \sqcap y$) e il minimo maggiorante(indicato con $x \sqcup y$).
Se un poset $(S,R)$ è un reticolo anche il suo poset duale è un reticolo.
