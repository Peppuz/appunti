\section{Grafi}
Si definisce come \emph{Grafo} $G$ una coppia $<V,E>$ in cui $V$ è l'insieme
dei \textbf{vertici} o \textbf{Nodi},indicanti gli elementi, invece $E$
 è l'insieme degli \textbf{archi}, indicanti la relazione esistentre tra i vertici del grafo.\newline
In un grafo orientato è ammesso il cappio ossia archi che escono ed entrano dallo stesso vertice.

Se in grafo tutti gli archi presentano un ordinamento, ossia si definisce una direzione
tra i 2 vertici di un grafo,si definisce \emph{Grafo orientato}
altrimenti si definisce il grafo come \emph{Grafo non orientato}.
%Inserire Esempi

Un arco che congiunge $V_i$ a $V_j$ si dice \emph{uscente} da $V_i$ ed \emph{entrante} in $V_j$.

\subsection{Nomenclatura}
In un grafo si definisce:
\begin{description}
    \item[NODO SORGENTE]: nodi in cui non si hanno archi entranti
    \item[NODO POZZO]:nodi in cui non si hanno archi uscenti
    \item[NODO ISOLATO]: nodi in cui non si hanno archi entranti ed uscenti
    \item[GRADO DI ENTRATA]:è il numero di archi entranti in un nodo
    \item[GRADO DI USCITA]: è il numero degli archi uscenti da un nodo
    \item[CAMMINO da $V_{in}$ a $V_{fin}$]:è una sequenza finita di nodi $<V_1,V_2,\dots,V_n>$
     con $V_1 = V_{in}$ e $V_n = V_{fin}$, dove ciascun nodo è collegato al successivo da un arco orientato
    \item[SEMICAMMINO da $V_{in}$ a $V_{fin}$]: è una sequenza finita di nodi
     $<V_1,V_2,\dots,V_n>$ con $V_1 = V_{in}$ e $V_n = V_{fin}$, dove ciascun nodo
     è collegato al successivo da un arco non orientato.
    \item[CONNESSO]:un grafo in cui dati due nodi $V_a$ e $V_b$, con $V_a \neq V_b$,
                    esiste un semicammino tra di essi.
    \item[CICLO]intorno un nodo $V$ è un cammino in cui $V = V_{in} = V_{fin}$
    \item[SEMICICLO]intorno un nodo $V$ è un semicammino in cui $V = V_{in} = V_{fin}$
    \item[CAPPIO]intorno ad un nodo è un cammino di lunghezza 1 in cui $V_in = V_{fin}$
\end{description}

%Inserire esempi ed esercizi
%Inserire Proprietà Grafi

%Paragrafo sui Dag e gli Alberi
%Paragrafo sugli Alberi
\section{Alberi}
Si definisce come \emph{Albero libero}, un DAG connesso con un solo nodo sorgente, detto \emph{radice},
in cui ogni nodo diverso dalla radice ha un solo nodo entrante.\newline
I nodi privi di archi entranti sono detti \emph{foglie} dell'albero.

%Inserire esempi e proprietà degli Alberi
%%Paragrafo sugli Alberi
\section{Alberi}
Si definisce come \emph{Albero libero}, un DAG connesso con un solo nodo sorgente, detto \emph{radice},
in cui ogni nodo diverso dalla radice ha un solo nodo entrante.\newline
I nodi privi di archi entranti sono detti \emph{foglie} dell'albero.

%Inserire esempi e proprietà degli Alberi
%%Paragrafo sugli Alberi
\section{Alberi}
Si definisce come \emph{Albero libero}, un DAG connesso con un solo nodo sorgente, detto \emph{radice},
in cui ogni nodo diverso dalla radice ha un solo nodo entrante.\newline
I nodi privi di archi entranti sono detti \emph{foglie} dell'albero.

%Inserire esempi e proprietà degli Alberi
%\input{Esempi/alberi}Esempi Alberi!!!!

L'albero è una struttura matematica importantissima in informatica utilizzata per
rappresentare una serie di situazioni, come ad esempio organizzazioni gerarchiche di dati,
procedimenti enumerativi o decisionali, e ve ne esistono un'infinita di implementazioni di alberi
però iniziamo ad analizzare per prima gli alberi liberi binari.

%Inserire immagine albero binario

%Metodi di visita di un albero
I metodi di visita di un albero binario sono 3:
\begin{itemize}
  \item inorder: si visiona prima il sottoalbero sinistro poi il nodo e infine il sottoalbero destro
  \item preorder: si visiona prima il nodo poi i suoi sottoalberi
  \item postorder: si visionano prima i sottoalberi ed infine il nodo
\end{itemize}
Il primo metodo di visita viene usato soprattutto negli alberi binari di ricerca per
stampare gli elementi dell'albero in maniera crescente mentre in un albero binario
normale la scelta di quale metodo di visita utilizzare è inifluente e ogni programmatore
sceglie nell'utilizzo quale metodo di visita utilizzare per stampare l'albero.
Il cammino dalla radice ad un elemento foglia dell'albero richiede al massimo $O(h)$,
in cui $h$ è l'altezza dell'albero, in quanto richiede di scendere di livello fino
ad arrivare alle foglie, che si trovano al livello $h$.

La specifica di un albero binario, in cui ogni implementazione per essere valida deve prevedere:\newline
\textbf{Item} search(Tree T,Item k);\newline
\textbf{void} insert(Tree T,Item x);\newline
\textbf{Item} delete(Tree T,Item x);\newline
\textbf{Item} mininum(Tree T);\newline
\textbf{Item} maxinum(Tree T);\newline
\textbf{Item} predecessor(Tree T,Item x);\newline
\textbf{Item} successor(Tree T,Item x);\newline
Esempi Alberi!!!!

L'albero è una struttura matematica importantissima in informatica utilizzata per
rappresentare una serie di situazioni, come ad esempio organizzazioni gerarchiche di dati,
procedimenti enumerativi o decisionali, e ve ne esistono un'infinita di implementazioni di alberi
però iniziamo ad analizzare per prima gli alberi liberi binari.

%Inserire immagine albero binario

%Metodi di visita di un albero
I metodi di visita di un albero binario sono 3:
\begin{itemize}
  \item inorder: si visiona prima il sottoalbero sinistro poi il nodo e infine il sottoalbero destro
  \item preorder: si visiona prima il nodo poi i suoi sottoalberi
  \item postorder: si visionano prima i sottoalberi ed infine il nodo
\end{itemize}
Il primo metodo di visita viene usato soprattutto negli alberi binari di ricerca per
stampare gli elementi dell'albero in maniera crescente mentre in un albero binario
normale la scelta di quale metodo di visita utilizzare è inifluente e ogni programmatore
sceglie nell'utilizzo quale metodo di visita utilizzare per stampare l'albero.
Il cammino dalla radice ad un elemento foglia dell'albero richiede al massimo $O(h)$,
in cui $h$ è l'altezza dell'albero, in quanto richiede di scendere di livello fino
ad arrivare alle foglie, che si trovano al livello $h$.

La specifica di un albero binario, in cui ogni implementazione per essere valida deve prevedere:\newline
\textbf{Item} search(Tree T,Item k);\newline
\textbf{void} insert(Tree T,Item x);\newline
\textbf{Item} delete(Tree T,Item x);\newline
\textbf{Item} mininum(Tree T);\newline
\textbf{Item} maxinum(Tree T);\newline
\textbf{Item} predecessor(Tree T,Item x);\newline
\textbf{Item} successor(Tree T,Item x);\newline
Esempi Alberi!!!!

L'albero è una struttura matematica importantissima in informatica utilizzata per
rappresentare una serie di situazioni, come ad esempio organizzazioni gerarchiche di dati,
procedimenti enumerativi o decisionali, e ve ne esistono un'infinita di implementazioni di alberi
però iniziamo ad analizzare per prima gli alberi liberi binari.

%Inserire immagine albero binario

%Metodi di visita di un albero
I metodi di visita di un albero binario sono 3:
\begin{itemize}
  \item inorder: si visiona prima il sottoalbero sinistro poi il nodo e infine il sottoalbero destro
  \item preorder: si visiona prima il nodo poi i suoi sottoalberi
  \item postorder: si visionano prima i sottoalberi ed infine il nodo
\end{itemize}
Il primo metodo di visita viene usato soprattutto negli alberi binari di ricerca per
stampare gli elementi dell'albero in maniera crescente mentre in un albero binario
normale la scelta di quale metodo di visita utilizzare è inifluente e ogni programmatore
sceglie nell'utilizzo quale metodo di visita utilizzare per stampare l'albero.
Il cammino dalla radice ad un elemento foglia dell'albero richiede al massimo $O(h)$,
in cui $h$ è l'altezza dell'albero, in quanto richiede di scendere di livello fino
ad arrivare alle foglie, che si trovano al livello $h$.

La specifica di un albero binario, in cui ogni implementazione per essere valida deve prevedere:\newline
\textbf{Item} search(Tree T,Item k);\newline
\textbf{void} insert(Tree T,Item x);\newline
\textbf{Item} delete(Tree T,Item x);\newline
\textbf{Item} mininum(Tree T);\newline
\textbf{Item} maxinum(Tree T);\newline
\textbf{Item} predecessor(Tree T,Item x);\newline
\textbf{Item} successor(Tree T,Item x);\newline

