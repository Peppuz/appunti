%File per gli appunti sull'Induzione
\chapter{Induzione}
L'induzione è un importante strumento per la definizione di nuovi insiemi,come
ad esempio l'insieme delle FBF(Formule ben Formate), e la dimostrazione di determinate
proprietà di un insieme.

%Dimostrazione per induzione(Da migliorare)
\section{Principio di Induzione}
Il principio di Induzione si utilizza per dimostrare la correttezza di determinate
proprietà dell'insieme dei numeri Naturali.

Il principio di induzione viene definito nel seguente modo:\newline
Data una proposizione $P(x)$ valida per $\forall x \in N$ bisogna:
\begin{enumerate}
  \item \textbf{Caso Base}: Verificare $P(i)$ con $i$ indicante il primo elemento valido della proposizione
  \item \textbf{Passo Induttivo}: Supposto $P(x)$ vero  bisogna verificare la verità di $P(x+1)$
\end{enumerate}

%Definizione Principio di Induzione Completo
Il principio di Induzione completo è definito nel seguente modo:
\begin{defi}
Sia A(n) una asserzione per ogni elemento $n \geq i \in \N$. Supponendo che:
\begin{itemize}
    \item $A(i)$ è vera (Caso Base)
    \item $\forall m \in \N$, se $A(k)$ è vera $\forall k$,con $0 < k < m$, ne segue
          che è vera $A(m)$
\end{itemize}
Allora $\forall n \in N$,$A(n)$ è vera
\end{defi}

Esempio:\newline
\begin{thm}$\displaystyle \sum_{i = 0} ^ n i = \frac{n(n + 1)}{2}$ \end{thm}

\begin{proof}
Per $n = 0 \quad \displaystyle \sum_{i = 0} ^ 0 i = \frac{0(0 + 1)}{2} \quad 0 = 0$ vero

Se $\displaystyle \sum_{i = 0} ^ n i = \frac{n(n+1)}{2}$ allora
$\displaystyle \sum_{i = 0} ^ {n+1} i = \frac{(n+1)(n+2)}{2}$
\begin{equation*}
\begin{split}
  \sum_{i = 0}^{n+1} i & = \sum_{i = 0} ^ n + (n+1) \\
                     & = \frac{n(n+1)}{2} + (n+1) \\
                     & = \frac{n(n+1) + 2(n+1)}{2} \\
                     & = \frac{(n+1)(n+2)}{2} \\
\end{split}
\end{equation*}
\end{proof}

Esempio:\newline
\begin{thm}
 $\displaystyle \sum_{i = 1}^n 2i-1 = n^2$
\end{thm}

\begin{proof}
Per $n = 1 \quad \displaystyle \sum_{i = 1}^1 2i-1 = 1^2 \quad 1 = 1$ è vero

Se $\displaystyle \sum_{i = 1}^n 2i-1 = n^2$ allora
$\displaystyle \sum_{i = 1}^{n+1} 2i-1 = (n+1)^2$

\begin{equation*}
\begin{split}
  \sum_{i=1}^{n+1} 2i-1 & = \sum_{i=1}^n 2i-1 + 2(n+1) - 1 \\
         & = n^2 + 2n + 1 \\
         & (n+1)^2 \\
\end{split}
\end{equation*}
\end{proof}

\begin{thm}
    $\displaystyle n! \geq 2^{(n-1)} \quad \forall n \in \N$
\end{thm}
%Dimostrazione
\begin{proof}
Per $n = 0$ si ha $0! \geq 2^{0-1}$ ossia $ 1 \geq 1/2$ che è sempre verificato

Se $n! \geq 2^{(n-1)}$ allora $(n+1)! \geq 2^n$
\begin{equation*}
\begin{split}
(n+1)! \geq 2^{n+1-1} & n! (n+1) \geq 2^{n-1} * 2 \\
                      & n! \frac{(n+1)}{2} \geq 2^{n-1} \\
\frac{(n+1)}{2} \geq 0 \forall n \in \N \text{per cui lo si può togliere e si rimane a}\\
                     & n! \geq 2^{n-1} \text{verificato per ipotesi} \\
\end{split}
\end{equation*}
\end{proof}

%Definizione induttiva
\section{Definizione Induttiva}
L'induzione permette anche di definire nuovi insiemi nel seguente modo:
\begin{enumerate}
  \item si definisce un insieme di "oggetti base" appartenenti all'insieme.
  \item si definisce un insieme di operazioni di costruzione che, applicate ad elementi
        dell'insieme, producono nuovi elementi dell'insieme.
  \item nient'altro appartiene all'insieme definito.
\end{enumerate}

%Inserire Esempi
Esempio:Definizione induttiva di numeri naturali\newline
\begin{enumerate}
  \item $0 \in N$
  \item Se $x \in N$ allora $s(x) \in N$
  \item Nient'altro appartiene ai numeri naturali
\end{enumerate}

Esempio:espressione in Java
\begin{enumerate}
    \item le variabili e le costanti sono delle espressioni
    \item se $E_1$ e $E_2$ sono delle espressioni ed $op$ è un operatore binario,
          allora $E_1 op E_2$ è un espressione
    \item se $E_1$ e $E_2$ sono delle espressioni e $op$ è un operatore unario,
          allora $op E_1$ è un espressione
    \item nient'altro è un espressione
\end{enumerate}

%Ricorsione definizione
\section{Ricorsione}
La ricorsione è funzione definitoria che consiste nel definire un insieme
di elementi base e di definire gli altri elementi mediante il richiamo di se stessa,
fino ad arrivare ai casi base.

Esempio:
la definizione ricorsiva del fattoriale è definita come segue:
\begin{equation*}
    n! = \begin{cases} 1 \quad n = 0 \lor n = 1 \\ n * (n-1)! \quad n > 1\\
\end{cases}
\end{equation*}

la definizione del coefficiente binomiale è definita come segue:
\begin{equation*}
    (^n _ k) = \begin{cases} 1 \quad n = k \lor k = 0 \\
                             n \quad k = n-1 \lor k = 1 \\
                             (^{n-1} _{k-1}) + (^{n-1} _k) \quad \text{altrimenti} \\
                \end{cases}
\end{equation*}

la definizione di $somma:Z x Z \mapsto Z$ è la seguente:
\begin{equation*}
    somma(a,b) = \begin{cases} a \quad b = 0 \\
                               somma(b,a) \quad a < b \\
                               1 + somma(b-1) \quad b > 0\\
                               -1 + somma(b+1) \quad b < 0 \\
                  \end{cases}
\end{equation*}

%Stringhe
\section{Stringhe}
La definizione induttiva dell'insieme delle stringhe su un alfabeto è la seguente:
\begin{itemize}
    \item $\epsilon$ è una stringa con $\epsilon = $ stringa vuota
    \item se $X$ è una stringa e $c$ è un carattere allora $X+c$ è una stringa
\end{itemize}

Sulle stringhe possiamo definire ricorsivamente delle funzioni:
\begin{itemize}
    \item $lunghezza:STR \mapsto \N$ indicante la lunghezza delle stringhe definita come:
            \begin{equation*}
                lunghezza(x) = \begin{cases} 0 \quad \text{se} x = \epsilon \\
                                             1 + lunghezza(x') \text{se} x = x' + c \\
                                \end{cases}
            \end{equation*}
    \item
\end{itemize}
