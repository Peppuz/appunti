%File per gli appunti sull'Induzione
\chapter{Induzione}
L'induzione è un importante strumento per la definizione di nuovi insiemi,come
ad esempio l'insieme delle FBF(Formule ben Formate), e la dimostrazione di determinate
proprietà di un insieme.

%Dimostrazione per induzione(Da migliorare)
\section{Principio di Induzione}
Il principio di Induzione si utilizza per dimostrare la correttezza di determinate
proprietà dell'insieme dei numeri Naturali.

Il principio di induzione viene definito nel seguente modo:\newline
Data una proposizione $P(x)$ valida per $\forall x \in N$ bisogna:
\begin{enumerate}
  \item \textbf{Caso Base}: Verificare $P(i)$ con $i$ indicante i primi elementi della proposizione
  \item \textbf{Passo Induttivo}: Supposto $P(x)$ vero  bisogna verificare la verità di $P(x+1)$
\end{enumerate}

Esempio:\newline
\begin{thm}$\displaystyle \sum_{i = 0} ^ n i = \frac{n(n + 1)}{2}$ \end{thm}

\begin{proof}
\textbf{Base} $n = 0 \quad \displaystyle \sum_{i = 0} ^ 0 i = \frac{0(0 + 1)}{2} \quad 0 = 0$ vero

\textbf{Ipotesi Induttiva}: $\displaystyle \sum_{i = 0} ^ n i = \frac{n(n+1)}{2}$ \newline
\textbf{Tesi Induttiva}: $\displaystyle \sum_{i = 0} ^ {n+1} i = \frac{(n+1)(n+2)}{2}$

\begin{equation*}
\begin{split}
  \sum_{i = 0}^n+1 i & = \sum_{i = 0} ^ n + (n+1) \\
                     & = \frac{n(n+1)}{2} + (n+1) \\
                     & = \frac{n(n+1) + 2(n+1)}{2} \\
                     & = \frac{(n+1)(n+2)}{2} \\
\end{split}
\end{equation*}
\end{proof}

Esempio:\newline
\begin{thm}
 $\displaystyle \sum_{i = 1}^n 2i-1 = n^2$
\end{thm}

\begin{proof}
Dimostrazione\newline
\textbf{Base} $n = 1 \quad \displaystyle \sum_{i = 1}^1 2i-1 = 1^2 \quad 1 = 1$ è vero

\textbf{Ipotesi Induttiva}: $\displaystyle \sum_{i = 1}^n 2i-1 = n^2$ \newline
\textbf{Tesi Induttiva}: $\displaystyle \sum_{i = 1}^{n+1} 2i-1 = (n+1)^2$

\begin{equation*}
\begin{split}
  \sum_{i=1}^{n+1} 2i-1 & = \sum_{i=1}^n 2i-1 + 2(n+1) - 1 \\
         & = n^2 + 2n + 1 = (n+1)^2 \\
\end{split}
\end{equation*}
\end{proof}

%Definizione induttiva
\section{Definizione Induttiva}
L'induzione permette anche di definire nuovi insiemi nel seguente modo:

\begin{enumerate}
  \item si definisce un insieme di "oggetti base" appartenenti all'insieme.
  \item si definisce un insieme di operazioni di costruzione che, applicate ad elementi
        dell'insieme, producono nuovi elementi dell'insieme.
  \item nient'altro appartiene all'insieme definito.
\end{enumerate}

%Inserire Esempi
Esempio:Definizione induttiva di numeri naturali\newline
\begin{enumerate}
  \item $0 \in N$
  \item Se $x \in N$ allora $s(x) \in N$
  \item Nient'altro appartiene ai numeri naturali
\end{enumerate}

%Ricorsione definizione

%Stringhe
