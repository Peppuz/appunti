\section{Algebra Booleana}
Un \emph{algebra di boole} è un reticolo $(B,R)$ in cui valgono le seguenti proprietà:
\begin{enumerate}
    \item
\end{enumerate}

L'algebra booleana si può definire anche in maniera assiomatica come segue:
\begin{defi}
    Sia $B$ un insieme, un algebra di Boole è una sestupla $(B,\cup,\cap,^{'},0,1)$
dove $\cup$ e $\cap$ sono due operazioni binarie su $B$, $^{'}$ è un operazione unaria su $B$,
$0$ e $1$ sono due elementi distinti di $B$.
\end{defi}

\begin{defi}
    In un algebra di Boole si dice \emph{duale} di un enunciato scambiando $\cup$
    con $\cap$ e $0$ con $1$
\end{defi}

\begin{thm}Nella algebra di Boole valgono le seguenti proprietà, presi qualsiasi $x,y,z \in B$:
\end{thm}
\begin{enumerate}
    \item $x \cap 0 = 0$
    \item $x \cup 1 = 1$
    \item $x \cup 0 = x$
    \item $x \cap 1 = x $
    \item $x \cap (x \cup y) = x$ (Assorbimento)
    \item $x \cup (x \cap y) = x$ (Assorbimento)
    \item $x \cup y = y \cup x $ (Commutativa)
    \item $x \cap y = y \cap x $ (Commutativa)
    \item $x \cup (y \cap z) = (x \cup y) \cap (x \cup z) $ (Distribitiva)
    \item $x \cap (y \cup z) = (x \cap y) \cup (x \cap z) $ (Distribuitiva)
    \item $x \cap (y \cap z) = (x \cap y) \cap z$ (Associatività)
    \item $x \cup (y \cup z) = (x \cup y) \cup z$ (Associatività)
    \item $(x \cap y)' = (x' \cup y')$ (Legge di De Morgan)
    \item $(x \cup y)' = x' \cap y' $ (Legge di de Morgan)
    \item $x \cap y = (x' \cup y')'$
    \item $x \cup y = (x' \cap y')'$
    \item $1' = 0$
    \item $0' = 1$
    \item $x \cup x' = 1$
    \item $x \cap x' = 0$
\end{enumerate}

%Definizione gruppi e algebra booleana
\subsection{Algebra dei gruppi}
Un \emph{semigruppo} è una coppia $(S,\circ)$ dove $S$ è un insieme e $\circ$ è
un operazione binaria associativa su $S$;un semigruppo è detto \emph{commutativo}
se $\circ$ è un operazione commutativa.

Un \emph{monoide} è un semigruppo in cui esiste,ed è unico, un'unita destra e sinistra per $\circ$,
ossia un elemento $u \in S$ per cui vale $u \circ x = x \circ u = x \forall x \in S$.

Un \emph{gruppo} è un monoide in cui $\forall x \in S$ esiste,ed è unico, un'elemento inverso,
tale che: $x \circ x^{-1} = x^{-1} \circ x = u$

Dati due semigruppi $(X,\circ _x)$ e $(Y,\circ_y)$, una funzione $h$ è detta
\emph{omomorfismo di semigruppi} se e solo $\forall x_1,x_2 \in X$ si ha che:
$h(x_1)\circ_y h(x_2) = h(x_1 \circ_x x_2)$.
Se $h$ è una relazione biettiva, allora $h$ è detta \emph{isomorfismo di semigruppi}.
