%Capitolo sulle relazioni
\chapter{Relazioni}
Si definisce \textit{relazione n-aria} un sottoinsieme del prodotto cartesiano
rappresentato da tutte le coppie che rispettano la relazione voluta tra gli $n$ insiemi.
Si definisce \textit{arietà} di una relazione il numero e il tipo degli argomente
di una relazione.

%Definizione Dominio e Codominio di una relazione
\textbf{Dominio}:insieme degli elementi $x$ tali che $<x,y> \in R$ per qualsiasi $y$.
\textbf{Codominio}:insieme degli elementi $y$ tali che $<x,y> \in R$ per qualsiasi $x$.

%Definizione Relazione Complementare ed Inversa
Esempio:\newline
$A$ x $B = \{<1,1>,<1,4>,<1,5>,<2,1>,<2,4>,<2,5>,<3,1>,<3,4>,<3,5>\} $ \newline
$R \subseteq A$ x $B$\newline
R = \{<1,1>,<1,4>,<1,5>,<2,4>,<2,5>,<3,4>,<3,5>\}

$R \subseteq B$ x $A$ \newline
R = \{<1,1>,<1,2>,<1,3>\}

%proprietà delle Relazioni
Data una relazione $R$ definita su un dominio $S$ si definiscono le seguenti proprietà:

\begin{itemize}
  \item Riflessiva $\iff \forall x \in S \quad xRx$
  \item Irriflessiva $\iff \forall x \in S \quad x \not R x$
  \item simmetrica $\iff  xRy \rightarrow yRx$
  \item asimmetrica: $\iff xRy \rightarrow y \not R x$
  \item antisimmetrica: $\iff xRy \land yRx \rightarrow x = y$
  \item transitiva: $\iff xRy \land yRz \rightarrow xRz$
\end{itemize}

%Inserire Esempi

Sulle relazioni si possono applicare le usuali operazioni insiemistiche quindi, ad esempio,
date $R_1 \subseteq S x T$ e $R_2 \subseteq S x T$ anche $R_1 \cup R_2$ è una relazione su $S x T$.

%Definizione Relazione Complementare e Relazione Inversa
Data una relazione binaria $R \subseteq S x T$ definiamo \emph{relazione complementare}
$\bar{R} \subseteq S x T$ come $x \bar{R}y \iff <x,y> \not \in R$.
Per definizione si ha $\bar{\bar{R}} = R$ e $R \cup \bar{R} = S x T$.

Data una relazione binaria $R \subseteq S x T$ esiste sempre la \emph{relazione inversa}
$R^-1 = \{<y,x> | <x,y> \in R\} \subseteq T x S$.
Per definizione $(R ^ -1) ^ -1 = R$

%Proprietà relazioni
Proprietà delle relazioni Riflessive
Date due relazioni $R$ e $R'$ definite su $S$ si ha:
\begin{enumerate}
    \item se $R$ è riflessiva anche $R^-1$ è Riflessiva
    \item $R$ è riflessiva se e solo se $\bar{R}$ è riflessiva
    \item se $R$ e $R'$ sono riflessive anche $R \cup R'$ e $R \cap R'$ sono riflessive
\end{enumerate}
%mancano le dimostrazioni

Proprietà relazioni Simmetriche
Date due relazioni $R$ e $R'$ definite su $S$, si ha:
\begin{enumerate}
    \item $R$ è simmetrica se e solo se $R = R^-1$
    \item se $R$ è simmetrica anche $R^-1$ e $\bar{R}$ sono simmetriche
    \item $R$ è antisimmetrico se e solo se $R \cap R^-1 \subseteq \wp S$
    \item $R$ è asimmetrica se e solo se $R \cap R^-1 = \emptyset$
    \item se $R$ e $R'$ sono simmetriche anche $R \cup R'$ e $R \cap R'$ sono simmetriche
\end{enumerate}
%Mancano le dimostrazioni

Proprietà relazioni Transitive
Siano $R$ e $R'$ due relazioni su $S$, se $R$ e $R'$ sono transitive anche $R \cap R'$ è transitiva.
%Manca la dimostrazione

%Modalità di Rappresentazione delle Relazioni
\section{Rappresentazione di Relazioni}
Vi sono diverse modalità di rappresentazione delle relazioni,il cui metodo migliore
dipendono dall'arietà della relazione, che sono:
\begin{description}
    \item[Tabella a $n$ colonne] è una matrice a due dimensioni con righe,rappresentanti
          gli elementi, e colonne, indicanti gli insiemi; è conveniente utilizzare
          quando l'arietà della relazione è $\geq 2$.
    \item[Grafo Bipartito] è un grafo in cui si elencano gli elementi di tutti gli insiemi
         e si usano delle frecce, chiamate archi, per indicare l'associazione tra gli elementi.
         E' meglio utilizzare il grafo bipartito soltanto per le relazioni binarie.
    \item[Matrice Booleana] è una matrice $M_R$ a valori \{0,1\} composta da $n$ righe e $m$ colonn
    \item[Grafi] modalità di rappresentazione di relazioni binarie(spiegate in un paragrafo successivo)
\end{description}

\subsection{Tabelle}
Inserire esempi!!!!!!!!!!!!

\subsection{Grafo Bipartito}
Inserire esempi con risoluzione!!!!

\subsecion{Matrice Booleana}
La \emph{Matrice booleana} è una matrice $M_R$,composta da $n$ righe e $m$ colonne,
i cui elementi sono definiti come $m_{ij} = \{ ^{1 \iff <s_i,t_j \in R} _{0 \text{altrimenti}}$

Inserire esempi!!!!

Da una matrice booleana si possono determinare facilmente le proprietà
di una relazione $R$,definita su $S$, soprattutto la proprietà simmetrica e la riflessiva.

La \emph{Matrice Complementare} $M_{\bar{R}}$ è costituita dai seguenti elementi
$\bar{m}_{ij} = \{ ^ 1 \iff m_{ij} = 0 _ 0 \iff m_{ij} = 1$

La \emph{Matrice inversa} $M_{R ^-1}$ è la trasporta della matrice $M_R$.
%Da fare la dimostrazione

Date due matrici $A$ e $B$,embrambe di $n x m$ elementi, si definiscono 3 operazioni:
\begin{description}
    \item[MEET $A \sqcap B = C$]: è una matrice booleana i cui elementi sono:
    $c_{ij} = \{ ^ 1 \quad a_{ij} = 1 \vand b_{ij} = 1 _ 0 \quad a_{ij} = 0 \vor b_{ij} = 0$
    \item[JOIN $A \sqcup B = C$]: è una matrice booleana i cui elementi sono:
    $c{ij} = \{ ^ 1 \quad a_{ij} = 1 \vor b_{ij} = 1 _ 0 \quad a_{ij} = 0 \vand b_{ij} = 0$
    \item[PRODOTTO BOOLEANO $A \odot B$]: è una matrice booleana $n x p$, i cui elementi sono:
    $c_{ij} = \{1 \quad \text{se per qualche} k(1 \leq k \leq m) \text{si ha} a_{ik} = 1 \vand b_{kj} = 1
    _ 0 \quad \text{altrimenti}$
\end{description}

Inserire Esempi

%Sezione Grafi
\section{Grafi}
Si definisce come \emph{Grafo} $G$ una coppia $(V,E)$ in cui $V$ è l'insieme
dei \textbf{vertici} o \textbf{Nodi},indicanti gli elementi, invece $E$
 è l'insieme degli \textbf{archi}, indicanti la relazione esistente tra i vertici del grafo.\newline
In un grafo orientato è ammesso il cappio ossia archi che escono ed entrano dallo stesso vertice.\newline
Numero di archi = numero di nodi - 1

Se in grafo tutti gli archi presentano un ordinamento, ossia si definisce una direzione
tra i 2 vertici di un grafo,si definisce \emph{Grafo orientato}
altrimenti si definisce il grafo come \emph{non orientato}.
%Inserire Esempi

Un arco che congiunge $V_i$ a $V_j$ si dice \emph{uscente} da $V_i$ ed \emph{entrante} in $V_j$.

\subsection{Nomenclatura}
In un grafo si definisce:
\begin{description}
    \item[NODO SORGENTE]: nodi in cui non si hanno archi entranti
    \item[NODO POZZO]:nodi in cui non si hanno archi uscenti
    \item[NODO ISOLATO]: nodi in cui non si hanno archi entranti né uscenti
    \item[GRADO DI ENTRATA]: è il numero di archi entranti in un nodo
    \item[GRADO DI USCITA]: è il numero degli archi uscenti da un nodo
    \item[CAMMINO da $V_{in}$ a $V_{fin}$]:è una sequenza finita di nodi $(V_1,V_2,\dots,V_n)$
     con $V_1 = V_{in}$ e $V_n = V_{fin}$, dove ciascun nodo è collegato al successivo da un arco orientato
    \item[SEMICAMMINO da $V_{in}$ a $V_{fin}$]: è una sequenza finita di nodi
     $(V_1,V_2,\dots,V_n)$ con $V_1 = V_{in}$ e $V_n = V_{fin}$, dove ciascun nodo
     è collegato al successivo da un arco non orientato.
    \item[CONNESSO]:un grafo in cui dati due nodi $V_a$ e $V_b$, con $V_a \neq V_b$,
                    esiste un semicammino tra di essi.
    \item[CICLO]intorno un nodo $V$ è un cammino in cui $V = V_{in} = V_{fin}$
    \item[SEMICICLO]intorno un nodo $V$ è un semicammino in cui $V = V_{in} = V_{fin}$
    \item[CAPPIO]intorno ad un nodo è un cammino di lunghezza 1 in cui $V_in = V_{fin}$
\end{description}

%Inserire esempi ed esercizi
%Esempio Tableaux Predicativo
Formula: $(\forall x F(x) \lor \exists G(x)) \rightarrow (\exists x (F(x) \lor G(x)))$
\begin{proof}
\begin{equation*}
\begin{prooftree}
\hypo{F \forall x F(x) \lor \exists G(x) \rightarrow \exists x (F(x) \lor G(x))}
\infer1{T \forall x F(x) \lor \exists G(x),F \exists x (F(x) \lor G(x))}
\infer1{T \forall x F(x),F \exists x (F(x) \lor G(x))/T \exists G(x),F \exists x (F(x) \lor G(x))}
\infer1{T F(a),F \exists x (F(x) \lor G(x)),T \forall x \dots/T G(a),F \exists x (F(x) \lor G(x))}
\infer1{T F(a),F F(a) \lor G(a),T \forall \dots,F \exists \dots/T G(a),F F(a) \lor G(a),F \exists \dots}
\infer1{T F(a),F F(a),F G(a),T \forall \dots,F \exists \dots/T G(a),F F(a),F G(a),F \exists \dots}
\end{prooftree}
\end{equation*}
Il tableaux contraddizione chiude per cui la formula è una tautologia.
\end{proof}


% Proprietà Grafi
\subsection{Proprietà dei Grafi}
Le proprietà delle relazioni si riflettono in proprietà dei grafi.

\begin{defi}
Sia $G$ una relazione binaria su un insieme $V$
\end{defi}
\begin{enumerate}
    \item Se $G$ è riflessiva allora il corrispettivo grafo avrà un cappio intorno ogni nodo
    \item Se $G$ è una relazione irriflessiva allora nel grafo non ci sono cappi
    \item Se $G$ è una relazione simmetrica allora il grafo non è orientato
    \item Se $G$ è una relazione asimmetrica allora tra due nodi non ci sarà mai un arco e il suo inverso
    \item Se $G$ è una relazione transitiva allora nel grafo qualora vi siano gli archi
          tra $x_1 \mapsto x_2$ e tra $x_2 \mapsto x_3$ vi è l'arco tra $x_1 \mapsto x_3$
\end{enumerate}

%Paragrafo sui Dag e gli Alberi
%Paragrafo sugli Alberi
\section{Alberi}
Si definisce come \emph{Albero libero}, un DAG connesso con un solo nodo sorgente, detto \emph{radice},
in cui ogni nodo diverso dalla radice ha un solo nodo entrante.\newline
I nodi privi di archi entranti sono detti \emph{foglie} dell'albero.

%Inserire esempi e proprietà degli Alberi
%%Paragrafo sugli Alberi
\section{Alberi}
Si definisce come \emph{Albero libero}, un DAG connesso con un solo nodo sorgente, detto \emph{radice},
in cui ogni nodo diverso dalla radice ha un solo nodo entrante.\newline
I nodi privi di archi entranti sono detti \emph{foglie} dell'albero.

%Inserire esempi e proprietà degli Alberi
%\input{Esempi/alberi}Esempi Alberi!!!!

L'albero è una struttura matematica importantissima in informatica utilizzata per
rappresentare una serie di situazioni, come ad esempio organizzazioni gerarchiche di dati,
procedimenti enumerativi o decisionali, e ve ne esistono un'infinita di implementazioni di alberi
però iniziamo ad analizzare per prima gli alberi liberi binari.

%Inserire immagine albero binario

%Metodi di visita di un albero
I metodi di visita di un albero binario sono 3:
\begin{itemize}
  \item inorder: si visiona prima il sottoalbero sinistro poi il nodo e infine il sottoalbero destro
  \item preorder: si visiona prima il nodo poi i suoi sottoalberi
  \item postorder: si visionano prima i sottoalberi ed infine il nodo
\end{itemize}
Il primo metodo di visita viene usato soprattutto negli alberi binari di ricerca per
stampare gli elementi dell'albero in maniera crescente mentre in un albero binario
normale la scelta di quale metodo di visita utilizzare è inifluente e ogni programmatore
sceglie nell'utilizzo quale metodo di visita utilizzare per stampare l'albero.
Il cammino dalla radice ad un elemento foglia dell'albero richiede al massimo $O(h)$,
in cui $h$ è l'altezza dell'albero, in quanto richiede di scendere di livello fino
ad arrivare alle foglie, che si trovano al livello $h$.

La specifica di un albero binario, in cui ogni implementazione per essere valida deve prevedere:\newline
\textbf{Item} search(Tree T,Item k);\newline
\textbf{void} insert(Tree T,Item x);\newline
\textbf{Item} delete(Tree T,Item x);\newline
\textbf{Item} mininum(Tree T);\newline
\textbf{Item} maxinum(Tree T);\newline
\textbf{Item} predecessor(Tree T,Item x);\newline
\textbf{Item} successor(Tree T,Item x);\newline
Esempi Alberi!!!!

L'albero è una struttura matematica importantissima in informatica utilizzata per
rappresentare una serie di situazioni, come ad esempio organizzazioni gerarchiche di dati,
procedimenti enumerativi o decisionali, e ve ne esistono un'infinita di implementazioni di alberi
però iniziamo ad analizzare per prima gli alberi liberi binari.

%Inserire immagine albero binario

%Metodi di visita di un albero
I metodi di visita di un albero binario sono 3:
\begin{itemize}
  \item inorder: si visiona prima il sottoalbero sinistro poi il nodo e infine il sottoalbero destro
  \item preorder: si visiona prima il nodo poi i suoi sottoalberi
  \item postorder: si visionano prima i sottoalberi ed infine il nodo
\end{itemize}
Il primo metodo di visita viene usato soprattutto negli alberi binari di ricerca per
stampare gli elementi dell'albero in maniera crescente mentre in un albero binario
normale la scelta di quale metodo di visita utilizzare è inifluente e ogni programmatore
sceglie nell'utilizzo quale metodo di visita utilizzare per stampare l'albero.
Il cammino dalla radice ad un elemento foglia dell'albero richiede al massimo $O(h)$,
in cui $h$ è l'altezza dell'albero, in quanto richiede di scendere di livello fino
ad arrivare alle foglie, che si trovano al livello $h$.

La specifica di un albero binario, in cui ogni implementazione per essere valida deve prevedere:\newline
\textbf{Item} search(Tree T,Item k);\newline
\textbf{void} insert(Tree T,Item x);\newline
\textbf{Item} delete(Tree T,Item x);\newline
\textbf{Item} mininum(Tree T);\newline
\textbf{Item} maxinum(Tree T);\newline
\textbf{Item} predecessor(Tree T,Item x);\newline
\textbf{Item} successor(Tree T,Item x);\newline


