%Capitolo sulle relazioni
\chapter{Relazioni}
Si definisce \textit{relazione n-aria} un sottoinsieme del prodotto cartesiano
rappresentato da tutte le coppie che rispettano la relazione voluta tra gli $n$ insiemi.
Si definisce \textit{arietà} di una relazione il numero e il tipo degli argomente
di una relazione.

%Definizione Dominio e Codominio di una relazione
\textbf{Dominio}:insieme degli elementi $x$ tali che $<x,y> \in R$ per qualsiasi $y$.
\textbf{Codominio}:insieme degli elementi $y$ tali che $<x,y> \in R$ per qualsiasi $x$.

%Definizione Relazione Complementare ed Inversa
Esempio:\newline
$A$ x $B = \{<1,1>,<1,4>,<1,5>,<2,1>,<2,4>,<2,5>,<3,1>,<3,4>,<3,5>\} $ \newline
$R \subseteq A$ x $B$\newline
R = \{<1,1>,<1,4>,<1,5>,<2,4>,<2,5>,<3,4>,<3,5>\}

$R \subseteq B$ x $A$ \newline
R = \{<1,1>,<1,2>,<1,3>\}

%proprietà delle Relazioni
Data una relazione $R$ definita su un dominio $S$ si definiscono le seguenti proprietà:

\begin{itemize}
  \item Riflessiva $\iff \forall x \in S \quad xRx$
  \item Irriflessiva $\iff \forall x \in S \quad x \not R x$
  \item Simmetrica $\iff  xRy \rightarrow yRx$
  \item Asimmetrica: $\iff xRy \rightarrow y \not R x$
  \item Antisimmetrica: $\iff xRy \land yRx \rightarrow x = y$
  \item Transitiva: $\iff xRy \land yRz \rightarrow xRz$
\end{itemize}

Sulle relazioni si possono applicare le usuali operazioni insiemistiche quindi, ad esempio,
date $R_1 \subseteq S x T$ e $R_2 \subseteq S x T$ anche $R_1 \cup R_2$ è una relazione su $S x T$.

%Definizione Relazione Complementare e Relazione Inversa
Data una relazione binaria $R \subseteq S x T$ definiamo \emph{relazione complementare}
$\bar{R} \subseteq S x T$ come $x \bar{R}y \iff <x,y> \not \in R$.
Per definizione si ha $\bar{\bar{R}} = R$ e $R \cup \bar{R} = S x T$.

Data una relazione binaria $R \subseteq S x T$ esiste sempre la \emph{relazione inversa}
$R^-1 = \{<y,x> | <x,y> \in R\} \subseteq T x S$.
Per definizione $(R ^ -1) ^ -1 = R$

%Proprietà relazioni
\begin{defi}
Date due relazioni $R$ e $R'$ definite su $S$ si ha:
\begin{enumerate}
    \item se $R$ è riflessiva anche $R^-1$ è Riflessiva
    \item $R$ è riflessiva se e solo se $\bar{R}$ è riflessiva
    \item se $R$ e $R'$ sono riflessive anche $R \cup R'$ e $R \cap R'$ sono riflessive
\end{enumerate}
\end{defi}
%mancano le dimostrazioni

\begin{defi}
Date due relazioni $R$ e $R'$ definite su $S$, si ha:
\begin{enumerate}
    \item $R$ è simmetrica se e solo se $R = R^-1$
    \item se $R$ è simmetrica anche $R^-1$ e $\bar{R}$ sono simmetriche
    \item $R$ è antisimmetrico se e solo se $R \cap R^-1 \subseteq \wp S$
    \item $R$ è asimmetrica se e solo se $R \cap R^-1 = \emptyset$
    \item se $R$ e $R'$ sono simmetriche anche $R \cup R'$ e $R \cap R'$ sono simmetriche
\end{enumerate}
\end{defi}
%Mancano le dimostrazioni

\begin{defi}
Siano $R$ e $R'$ due relazioni su $S$, se $R$ e $R'$ sono transitive anche $R \cap R'$ è transitiva.
\end{defi}
%Manca la dimostrazione

%Modalità di Rappresentazione delle Relazioni
\section{Rappresentazione di Relazioni}
Vi sono diverse modalità di rappresentazione delle relazioni,il cui metodo migliore
dipendono dall'arietà della relazione, che sono:
\begin{description}
    \item[Tabella a $n$ colonne] è una matrice a due dimensioni con righe,rappresentanti
          gli elementi, e colonne, indicanti gli insiemi; è conveniente utilizzare
          quando l'arietà della relazione è $\geq 2$.
    \item[Grafo Bipartito] è un grafo in cui si elencano gli elementi di tutti gli insiemi
         e si usano delle frecce, chiamate archi, per indicare l'associazione tra gli elementi.
         E' meglio utilizzare il grafo bipartito soltanto per le relazioni binarie.
    \item[Matrice Booleana] è una matrice $M_R$ a valori \{0,1\} composta da $n$ righe e $m$ colonn
    \item[Grafi] modalità di rappresentazione di relazioni binarie(spiegate in un paragrafo successivo)
\end{description}

\subsection{Tabelle}
Si vuole definire la relazione $anagrafica \subseteq Cognomi x Nomi x Date x Luoghi$

\begin{tabular}{cccc}
\toprule
C & N & D & L \\
\midrule
Moscato & Ugo & 10/1/57 & Milano \\
Iaquinta & Vincenzo & 13/5/68 & Udine \\
Inzaghi & Filippo & 23/5/78 & Piacenza \\
\bottomrule
\end{tabular}

\subsection{Grafo Bipartito}
Dati due insiemi $A = \{ 1,2,3 \}$ e $B = \{ 4,7,1 \}$ si definisce la relazione
$R \subset A x B = \{ (1,7),(1,1),(2,4),(3,7),(3,4) \}$.


\subsection{Matrice Booleana}
La \emph{Matrice booleana} è una matrice $M_R$,composta da $n$ righe e $m$ colonne,
i cui elementi sono definiti come $m_{ij} = \{ ^{1 \iff <s_i,t_j \in R} _{0 \text{altrimenti}}$

Inserire esempi!!!!

Da una matrice booleana si possono determinare facilmente le proprietà
di una relazione $R$,definita su $S$, soprattutto la proprietà simmetrica e la riflessiva.

La \emph{Matrice Complementare} $M_{\bar{R}}$ è costituita dai seguenti elementi
$\bar{m}_{ij} = \{ ^ 1 \iff m_{ij} = 0 _ 0 \iff m_{ij} = 1$

La \emph{Matrice inversa} $M_{R ^-1}$ è la trasporta della matrice $M_R$.
%Da fare la dimostrazione

Date due matrici $A$ e $B$,embrambe di $n x m$ elementi, si definiscono 3 operazioni:
\begin{description}
    \item[MEET $A \sqcap B = C$]: è una matrice booleana i cui elementi sono:
$c_{ij} = \{ \begin{cases} 1 \quad a_{ij} = 1 \land b_{ij} = 1 \\ 0 \quad a_{ij} = 0 \lor b_{ij} = 0 \end{cases}$
    \item[JOIN $A \sqcup B = C$]: è una matrice booleana i cui elementi sono:
    $c{ij} = \{ ^ 1 \quad a_{ij} = 1 \lor b_{ij} = 1 _ 0 \quad a_{ij} = 0 \land b_{ij} = 0$
    \item[PRODOTTO BOOLEANO $A \odot B$]: è una matrice booleana $n x p$, i cui elementi sono:
    $c_{ij} = \{1 \quad \text{se per qualche} k(1 \leq k \leq m) \text{si ha} a_{ik} = 1 \land b_{kj} = 1
    _ 0 \quad \text{altrimenti}$
\end{description}

Inserire Esempi

%Sezione Composizione di relazioni
\section{Composizione di Relazioni}
Data una relazione $R_1 \subseteq S x T$ e una relazione $R_2 \subseteq T x Q$,
si definisce come relazione composta $R_1 \circ R_2 \subseteq S x Q$ come segue
$(a,c) \in R_1 \circ R_2 \iff \exists b \in T | (a,b) \in R_1 \land (b,c) \in R_2$

Esempio:
Siano $S = \{ a,b \}, R_1 = \{ (a,a),(a,b),(b,b) \}$ e $R_2 = \{ (a,b),(b,a),(b,b) \}$
$R_1 \circ R_2 = \{ (a,a),(a,b),(b,a),(b,b) \}$ \newline
$R_2 \circ R_1 = \{ (a,b),(b,a),(b,b) \}$

\begin{prop}
La composizione di Relazioni non è commutativa ma invece è associativa
\end{prop}
%Fare la Dimostrazione

\begin{thm}
Se $R_1 \subseteq S x T$ e $R_2 \subseteq T x Q$, allora $(R_1 \circ R_2)^-1 = R_1^-1 \circ R_2^-1$
\end{thm}
%Fare Dimostrazione

%Potenza di una relazione

%Sezione Grafi
\section{Grafi}
Si definisce come \emph{Grafo} $G$ una coppia $(V,E)$ in cui $V$ è l'insieme
dei \textbf{vertici} o \textbf{Nodi},indicanti gli elementi, invece $E$
 è l'insieme degli \textbf{archi}, indicanti la relazione esistente tra i vertici del grafo.\newline
In un grafo orientato è ammesso il cappio ossia archi che escono ed entrano dallo stesso vertice.\newline
Numero di archi = numero di nodi - 1

Se in grafo tutti gli archi presentano un ordinamento, ossia si definisce una direzione
tra i 2 vertici di un grafo,si definisce \emph{Grafo orientato}
altrimenti si definisce il grafo come \emph{non orientato}.
%Inserire Esempi

Un arco che congiunge $V_i$ a $V_j$ si dice \emph{uscente} da $V_i$ ed \emph{entrante} in $V_j$.

\subsection{Nomenclatura}
In un grafo si definisce:
\begin{description}
    \item[NODO SORGENTE]: nodi in cui non si hanno archi entranti
    \item[NODO POZZO]:nodi in cui non si hanno archi uscenti
    \item[NODO ISOLATO]: nodi in cui non si hanno archi entranti né uscenti
    \item[GRADO DI ENTRATA]: è il numero di archi entranti in un nodo
    \item[GRADO DI USCITA]: è il numero degli archi uscenti da un nodo
    \item[CAMMINO da $V_{in}$ a $V_{fin}$]:è una sequenza finita di nodi $(V_1,V_2,\dots,V_n)$
     con $V_1 = V_{in}$ e $V_n = V_{fin}$, dove ciascun nodo è collegato al successivo da un arco orientato
    \item[SEMICAMMINO da $V_{in}$ a $V_{fin}$]: è una sequenza finita di nodi
     $(V_1,V_2,\dots,V_n)$ con $V_1 = V_{in}$ e $V_n = V_{fin}$, dove ciascun nodo
     è collegato al successivo da un arco non orientato.
    \item[CONNESSO]:un grafo in cui dati due nodi $V_a$ e $V_b$, con $V_a \neq V_b$,
                    esiste un semicammino tra di essi.
    \item[CICLO]intorno un nodo $V$ è un cammino in cui $V = V_{in} = V_{fin}$
    \item[SEMICICLO]intorno un nodo $V$ è un semicammino in cui $V = V_{in} = V_{fin}$
    \item[CAPPIO]intorno ad un nodo è un cammino di lunghezza 1 in cui $V_in = V_{fin}$
\end{description}

%Inserire esempi ed esercizi
%Esempio Tableaux Predicativo
Formula: $(\forall x F(x) \lor \exists G(x)) \rightarrow (\exists x (F(x) \lor G(x)))$
\begin{proof}
\begin{equation*}
\begin{prooftree}
\hypo{F \forall x F(x) \lor \exists G(x) \rightarrow \exists x (F(x) \lor G(x))}
\infer1{T \forall x F(x) \lor \exists G(x),F \exists x (F(x) \lor G(x))}
\infer1{T \forall x F(x),F \exists x (F(x) \lor G(x))/T \exists G(x),F \exists x (F(x) \lor G(x))}
\infer1{T F(a),F \exists x (F(x) \lor G(x)),T \forall x \dots/T G(a),F \exists x (F(x) \lor G(x))}
\infer1{T F(a),F F(a) \lor G(a),T \forall \dots,F \exists \dots/T G(a),F F(a) \lor G(a),F \exists \dots}
\infer1{T F(a),F F(a),F G(a),T \forall \dots,F \exists \dots/T G(a),F F(a),F G(a),F \exists \dots}
\end{prooftree}
\end{equation*}
Il tableaux contraddizione chiude per cui la formula è una tautologia.
\end{proof}


% Proprietà Grafi
\subsection{Proprietà dei Grafi}
Le proprietà delle relazioni si riflettono in proprietà dei grafi.

\begin{defi}
Sia $G$ una relazione binaria su un insieme $V$
\end{defi}
\begin{enumerate}
    \item Se $G$ è riflessiva allora il corrispettivo grafo avrà un cappio intorno ogni nodo
    \item Se $G$ è una relazione irriflessiva allora nel grafo non ci sono cappi
    \item Se $G$ è una relazione simmetrica allora il grafo non è orientato
    \item Se $G$ è una relazione asimmetrica allora tra due nodi non ci sarà mai un arco e il suo inverso
    \item Se $G$ è una relazione transitiva allora nel grafo qualora vi siano gli archi
          tra $x_1 \mapsto x_2$ e tra $x_2 \mapsto x_3$ vi è l'arco tra $x_1 \mapsto x_3$
\end{enumerate}

%Paragrafo sui Dag e gli Alberi
%Paragrafo sugli Alberi
\section{Alberi}
Si definisce come \emph{Albero libero}, un DAG connesso con un solo nodo sorgente, detto \emph{radice},
in cui ogni nodo diverso dalla radice ha un solo nodo entrante.\newline
I nodi privi di archi entranti sono detti \emph{foglie} dell'albero.

%Inserire esempi e proprietà degli Alberi
%%Paragrafo sugli Alberi
\section{Alberi}
Si definisce come \emph{Albero libero}, un DAG connesso con un solo nodo sorgente, detto \emph{radice},
in cui ogni nodo diverso dalla radice ha un solo nodo entrante.\newline
I nodi privi di archi entranti sono detti \emph{foglie} dell'albero.

%Inserire esempi e proprietà degli Alberi
%\input{Esempi/alberi}Esempi Alberi!!!!

L'albero è una struttura matematica importantissima in informatica utilizzata per
rappresentare una serie di situazioni, come ad esempio organizzazioni gerarchiche di dati,
procedimenti enumerativi o decisionali, e ve ne esistono un'infinita di implementazioni di alberi
però iniziamo ad analizzare per prima gli alberi liberi binari.

%Inserire immagine albero binario

%Metodi di visita di un albero
I metodi di visita di un albero binario sono 3:
\begin{itemize}
  \item inorder: si visiona prima il sottoalbero sinistro poi il nodo e infine il sottoalbero destro
  \item preorder: si visiona prima il nodo poi i suoi sottoalberi
  \item postorder: si visionano prima i sottoalberi ed infine il nodo
\end{itemize}
Il primo metodo di visita viene usato soprattutto negli alberi binari di ricerca per
stampare gli elementi dell'albero in maniera crescente mentre in un albero binario
normale la scelta di quale metodo di visita utilizzare è inifluente e ogni programmatore
sceglie nell'utilizzo quale metodo di visita utilizzare per stampare l'albero.
Il cammino dalla radice ad un elemento foglia dell'albero richiede al massimo $O(h)$,
in cui $h$ è l'altezza dell'albero, in quanto richiede di scendere di livello fino
ad arrivare alle foglie, che si trovano al livello $h$.

La specifica di un albero binario, in cui ogni implementazione per essere valida deve prevedere:\newline
\textbf{Item} search(Tree T,Item k);\newline
\textbf{void} insert(Tree T,Item x);\newline
\textbf{Item} delete(Tree T,Item x);\newline
\textbf{Item} mininum(Tree T);\newline
\textbf{Item} maxinum(Tree T);\newline
\textbf{Item} predecessor(Tree T,Item x);\newline
\textbf{Item} successor(Tree T,Item x);\newline
Esempi Alberi!!!!

L'albero è una struttura matematica importantissima in informatica utilizzata per
rappresentare una serie di situazioni, come ad esempio organizzazioni gerarchiche di dati,
procedimenti enumerativi o decisionali, e ve ne esistono un'infinita di implementazioni di alberi
però iniziamo ad analizzare per prima gli alberi liberi binari.

%Inserire immagine albero binario

%Metodi di visita di un albero
I metodi di visita di un albero binario sono 3:
\begin{itemize}
  \item inorder: si visiona prima il sottoalbero sinistro poi il nodo e infine il sottoalbero destro
  \item preorder: si visiona prima il nodo poi i suoi sottoalberi
  \item postorder: si visionano prima i sottoalberi ed infine il nodo
\end{itemize}
Il primo metodo di visita viene usato soprattutto negli alberi binari di ricerca per
stampare gli elementi dell'albero in maniera crescente mentre in un albero binario
normale la scelta di quale metodo di visita utilizzare è inifluente e ogni programmatore
sceglie nell'utilizzo quale metodo di visita utilizzare per stampare l'albero.
Il cammino dalla radice ad un elemento foglia dell'albero richiede al massimo $O(h)$,
in cui $h$ è l'altezza dell'albero, in quanto richiede di scendere di livello fino
ad arrivare alle foglie, che si trovano al livello $h$.

La specifica di un albero binario, in cui ogni implementazione per essere valida deve prevedere:\newline
\textbf{Item} search(Tree T,Item k);\newline
\textbf{void} insert(Tree T,Item x);\newline
\textbf{Item} delete(Tree T,Item x);\newline
\textbf{Item} mininum(Tree T);\newline
\textbf{Item} maxinum(Tree T);\newline
\textbf{Item} predecessor(Tree T,Item x);\newline
\textbf{Item} successor(Tree T,Item x);\newline



%Relazioni di Equivalenza
\section{Relazioni di Equivalenza}
Si definisce $R$ una \emph{relazione di equivalenza} se e solo se la relazione binaria
$R$ è riflessiva,simmetrica e transitiva.

%Inserire Esempi

Data una relazione di equivalenza $R$ definita su un insieme $S$, si definisce
\emph{classe di equivalenza} di un elemento $x \in S$ come $[x] = \{y | <x,y> \in R \}$

\begin{thm}
Se $R$ è una relazione di equivalenza su $S$, allora le classi di Equivalenza
generate da $R$ partizionano $S$
\end{thm}
%Fare la Dimostrazione

Data una relazione di equivalenza in S, la partizione che essa determina si dice
\emph{insieme quoziente} di $S$ rispetto alla relazione di equivalenza e si indica con $S/$

%Ordinamento
\section{Struttura relazionale}
Una struttura relazionale $SR$ è una $n$-upla in cui il primo componente è un Insieme
non vuoto $S$ e le rimanenti componenti sono relazioni su $S^n$.

%Tipologie di Ordinamenti
\subsection{Tipologie di Ordinamenti}
\begin{description}
    \item[PREORDINE]: è una struttura relazionale $(S,R)$ in cui $S$ è un insieme non vuoto
          e $R$ è una relazione binaria \emph{riflessiva} e \emph{transitiva} su $S x S$
    \item[ORDINE STRETTO]: è una struttura relazionale $(S,R)$ in cui $R$ è una
          relazione binaria \emph{irriflessiva} e \emph{transitiva} su $S x S$
    \item[ORDINE LARGO(POSET)]: è una struttura relazionale $(S,R)$ in cui $R$ è una
          relazione binaria \emph{riflessiva, antisimmetrica} e \emph{transitiva} su $S x S$.
\end{description}

Una relazione $R$ è un ordinamento sull'insieme $S$ se e solo se $\forall x,y \in S$
vale solo una delle proprietà di \emph{tricotomia}:
\begin{itemize}
    \item $x = y \land x \slashed{R} y \land y \slashed{R} x$
    \item $xRy \land x \neq y \land y \slashed{R} x$
    \item $(yRx) \land x \neq y \land x \slashed{R} y$
\end{itemize}

Inserire Esempii!!!!!!

\begin{prop}
Se $(S,R)$ è una struttura relazionale anche $(S,R^{-1})$ rappresenta la stessa
struttura relazionale chimata e $(S,R^{-1})$ è il \emph{duale} di $(S,R)$.
\end{prop}

%Dimostrazione della Proposizione!!!!!

%Proposizione lunghezza cicli di un grafo del poset + dimostrazione
\begin{prop}
Il grafo di un Poset non ha cicli di lunghezza maggiore di 1
\end{prop}

%Ordine Lessicografico

%Diagramma di Hasse
\subsection{Diagramma di Hasse}
Il grafo di un poset può essere rappresentato dal \emph{diagramma di Hasse}, un grafo
con le seguenti proprietà:
\begin{itemize}
    \item si prescinde dal disegnare i cappi in quanto il poset è una relazione riflessiva
    \item si prescinde dal disegnare gli archi definiti per transitività
    \item il grafo si legge dal basso verso l'alto
\end{itemize}

esempio


%Elementi estremali
%Chiusura Transitiva
Sia $(S,R)$ una struttura relazionale con $R$ è una relazione binaria su $S$.
La \emph{chiusura transitiva} di $R$ è la più piccola relazione transitiva $R'$ in cui $R \subset R'$.
La \emph{chiusura riflessiva} di $R$ è la più piccola relazione riflessiva $R'$ in cui $R \subset R'$

\begin{prop}
Dato un poset $(S,R)$ e un insieme $X \subseteq S$ si hanno le seguenti proprietà:
\end{prop}
\begin{enumerate}
    \item[elem. massimale]: un elemento $s \in S$ è massimale se $\not \exists s' \in S : s \leq s'$
    \item[elem. minimale]: un elemento $s \in S$ è minimale se $\not \exists s' \in S : s \geq s'$
    \item[maggiorante]: un elemento $s \in S$ è un maggiorante se $\forall x \in X$ si ha $s \geq s$
    \item[minorante]: un elemento $s \in S$ è un minorante se $\forall x \in X$ si ha $s \leq x$
    \item[minimo maggiorante]:un elemento $s \in S$ è un minimo maggiorante, indicato con $\sqcup X$,
          se è un maggiorante e per ogni altro maggiorante $s'$ di $X$ si ha $s \leq s'$.
    \item[massimo minorante]: un elemento $s \in S$ è un massimo minorante, indicato con $\sqcap X$,
          se è un minorante e per ogni altro minorante $s'$ di $X$ si ha $s' \leq s$
    \item[massimo]: se $\sqcup X \in X$ si dice che $\sqcup X$ è un massimo.
    \item[minimo]: se $\sqcap X \in X$ si dice che $\sqcap X$ è un minimo.
\end{enumerate}
%Proprietà con dimostrazioni

%Definizione di Buon Ordinamento
\begin{defi}
    Un poset è detto buon ordinamento se e solo se per ogni sottoinsieme non vuoto esiste
    $\cap X$ e tale poset è detto \emph{ben formato}
\end{defi}

Inserire Esempi !!!!


%Reticoli
\subsection{Reticoli}
Un \emph{Reticolo} è un poset $(S,R)$ in cui per ogni coppia $x,y \in S$ esistono
il massimo minorante(indicato con $x \sqcap y$) e il minimo maggiorante(indicato con $x \sqcup y$).
Se un poset $(S,R)$ è un reticolo anche il suo poset duale è un reticolo e se due
reticoli sono isomorfi come poset allora i reticoli sono detti \emph{isomorfi}.

\begin{prop}
    Se $(L_1,\leq)$ e $(L_2,\leq)$ sono reticoli, anche $(L_1 X L_2,\leq)$ lo è,
    con ordine parziale prodotto
\end{prop}

%Proprietà del Reticolo
\begin{defi}
Sia $(L,\leq)$ un reticolo. Presi comunque $a,b,c \in L$ valgono le seguenti proprietà:
\end{defi}
\begin{enumerate}
    \item $a \leq a \cup b$ e $b \leq a \cup b$
    \item Se $a \leq c$ e $b \leq c$, allora $a \cup b \leq c$
    \item $a \cap b \leq a$ e $a \cap b \leq b$
    \item Se $c \leq a$ e $c \leq b$ allora $c \leq a \cap b$
    \item $a \cup a = a$ (Idempotenza)
    \item $a \cap a = a$ (Idempotenza)
    \item $a \cup b = b \cup a$ (Commutativa)
    \item $a \cap b = b \cap a$ (Commutativa)
    \item $a \cup (b \cup c) = (a \cup b) \cup c$ (Associativa)
    \item $a \cap (b \cap b) = (a \cap b) \cap c$ (Associativa)
    \item $a \cup(a \cap b) = a$ (Assorbimento)
    \item $a \cap (a \cup b) = a$ (Assorbimento)
\end{enumerate}

%Reticolo Completo
\begin{defi}
    Se ogni sottoinsieme di un reticolo possiede un minimo maggiorante e un massimo minorante
    allora il reticolo si definisce \emph{completo}.
\end{defi}

%Reticolo limitato
\begin{defi}
    Un reticolo è definito \emph{limitato} se esiste un minimo e un massimo assoluti.
\end{defi}

%Reticolo distribuitivo
\begin{defi}
    Un reticolo è detto \emph{distribuitivo} se valgono le seguenti proprietà:
\end{defi}
\begin{enumerate}
    \item $a \cap (b \cup c) = (a \cap b) \cup (a \cap c)$
    \item $a \cup (b \cap c) = (a \cup b) \cap (a \cup c)$
\end{enumerate}

%Reticolo Complementato
\begin{defi}
    Sia $(L,\leq)$ un reticolo distribuitivo limitato, con massimo $1$ e minimo $0$,
e sia $a \in L$, allora $a'$ è il \emph{complemento},il quale se esiste è unico,
 di $a$ se è rispettata la seguente proprietà: $a \cup a' = 1$ e $a \cap a' = 0$.
\end{defi}

\begin{defi}
Un reticolo $(L,\leq)$ è detto \emph{complementato} se è limitato e ogni suo elemento ha il complemento.
\end{defi}


