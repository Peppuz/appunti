%Capitolo sulle relazioni
\chapter{Relazioni}
Si definisce \textit{relazione n-aria} un sottoinsieme del prodotto cartesiano
rappresentato da tutte le coppie che rispettano la relazione voluta tra gli $n$ insiemi.
Si definisce \textit{arietà} di una relazione il numero e il tipo degli argomente
di una relazione.

%Definizione Dominio e Codominio di una relazione
\textbf{Dominio}:insieme degli elementi $x$ tali che $<x,y> \in R$ per qualsiasi $y$.
\textbf{Codominio}:insieme degli elementi $y$ tali che $<x,y> \in R$ per qualsiasi $x$.

%Definizione Relazione Complementare ed Inversa
Esempio:\newline
$A$ x $B = \{<1,1>,<1,4>,<1,5>,<2,1>,<2,4>,<2,5>,<3,1>,<3,4>,<3,5>\} $ \newline
$R \subseteq A$ x $B$\newline
R = \{<1,1>,<1,4>,<1,5>,<2,4>,<2,5>,<3,4>,<3,5>\}

$R \subseteq B$ x $A$ \newline
R = \{<1,1>,<1,2>,<1,3>\}

%proprietà delle Relazioni
Data una relazione $R$ definita su un dominio $S$ si definiscono le seguenti proprietà:

\begin{itemize}
  \item Riflessiva $\iff \forall x \in S \quad xRx$
  \item Irriflessiva $\iff \forall x \in S \quad x \not R x$
  \item simmetrica $\iff  xRy \rightarrow yRx$
  \item asimmetrica: $\iff xRy \rightarrow y \not R x$
  \item antisimmetrica: $\iff xRy \land yRx \rightarrow x = y$
  \item transitiva: $\iff xRy \land yRz \rightarrow xRz$
\end{itemize}

%Inserire Esempi

Sulle relazioni si possono applicare le usuali operazioni insiemistiche quindi, ad esempio,
date $R_1 \subseteq S x T$ e $R_2 \subseteq S x T$ anche $R_1 \cup R_2$ è una relazione su $S x T$.



%Proprietà relazioni


%Modalità di Rappresentazione delle Relazioni
