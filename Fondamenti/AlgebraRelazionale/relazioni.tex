%Capitolo sulle relazioni
\chapter{Relazioni}
Si definisce \textit{relazione n-aria} un sottoinsieme del prodotto cartesiano
rappresentato da tutte le coppie che rispettano la relazione voluta tra gli $n$ insiemi.
Si definisce \textit{arietà} di una relazione il numero e il tipo degli argomente
di una relazione.

%Definizione Dominio e Codominio di una relazione
\textbf{Dominio}:insieme degli elementi $x$ tali che $<x,y> \in R$ per qualsiasi $y$.
\textbf{Codominio}:insieme degli elementi $y$ tali che $<x,y> \in R$ per qualsiasi $x$.

%Definizione Relazione Complementare ed Inversa
Esempio:\newline
$A$ x $B = \{<1,1>,<1,4>,<1,5>,<2,1>,<2,4>,<2,5>,<3,1>,<3,4>,<3,5>\} $ \newline
$R \subseteq A$ x $B$\newline
R = \{<1,1>,<1,4>,<1,5>,<2,4>,<2,5>,<3,4>,<3,5>\}

$R \subseteq B$ x $A$ \newline
R = \{<1,1>,<1,2>,<1,3>\}

%proprietà delle Relazioni
Data una relazione $R$ definita su un dominio $S$ si definiscono le seguenti proprietà:

\begin{itemize}
  \item Riflessiva $\iff \forall x \in S \quad xRx$
  \item Irriflessiva $\iff \forall x \in S \quad x \not R x$
  \item simmetrica $\iff  xRy \rightarrow yRx$
  \item asimmetrica: $\iff xRy \rightarrow y \not R x$
  \item antisimmetrica: $\iff xRy \land yRx \rightarrow x = y$
  \item transitiva: $\iff xRy \land yRz \rightarrow xRz$
\end{itemize}

%Inserire Esempi

Sulle relazioni si possono applicare le usuali operazioni insiemistiche quindi, ad esempio,
date $R_1 \subseteq S x T$ e $R_2 \subseteq S x T$ anche $R_1 \cup R_2$ è una relazione su $S x T$.

%Definizione Relazione Complementare e Relazione Inversa
Data una relazione binaria $R \subseteq S x T$ definiamo \emph{relazione complementare}
$\bar{R} \subseteq S x T$ come $x \bar{R}y \iff <x,y> \not \in R$.
Per definizione si ha $\bar{\bar{R}} = R$ e $R \cup \bar{R} = S x T$.

Data una relazione binaria $R \subseteq S x T$ esiste sempre la \emph{relazione inversa}
$R^-1 = \{<y,x> | <x,y> \in R\} \subseteq T x S$.
Per definizione $(R ^ -1) ^ -1 = R$

%Proprietà relazioni
Proprietà delle relazioni Riflessive
Date due relazioni $R$ e $R'$ definite su $S$ si ha:
\begin{enumerate}
    \item se $R$ è riflessiva anche $R^-1$ è Riflessiva
    \item $R$ è riflessiva se e solo se $\bar{R}$ è riflessiva
    \item se $R$ e $R'$ sono riflessive anche $R \cup R'$ e $R \cap R'$ sono riflessive
\end{enumerate}
%mancano le dimostrazioni

Proprietà relazioni Simmetriche
Date due relazioni $R$ e $R'$ definite su $S$, si ha:
\begin{enumerate}
    \item $R$ è simmetrica se e solo se $R = R^-1$
    \item se $R$ è simmetrica anche $R^-1$ e $\bar{R}$ sono simmetriche
    \item $R$ è antisimmetrico se e solo se $R \cap R^-1 \subseteq \wp S$
    \item $R$ è asimmetrica se e solo se $R \cap R^-1 = \emptyset$
    \item se $R$ e $R'$ sono simmetriche anche $R \cup R'$ e $R \cap R'$ sono simmetriche
\end{enumerate}
%Mancano le dimostrazioni

Proprietà relazioni Transitive
Siano $R$ e $R'$ due relazioni su $S$, se $R$ e $R'$ sono transitive anche $R \cap R'$ è transitiva.
%Manca la dimostrazione

%Modalità di Rappresentazione delle Relazioni
\section{Rappresentazione di Relazioni}
Vi sono diverse modalità di rappresentazione delle relazioni,il cui metodo migliore
dipendono dall'arietà della relazione, che sono:
\begin{description}
    \item[Tabella a $n$ colonne] è una matrice a due dimensioni con righe,rappresentanti
          gli elementi, e colonne, indicanti gli insiemi; è conveniente utilizzare
          quando l'arietà della relazione è $\geq 2$.
    \item[Grafo Bipartito] è un grafo in cui si elencano gli elementi di tutti gli insiemi
         e si usano delle frecce, chiamate archi, per indicare l'associazione tra gli elementi.
         E' meglio utilizzare il grafo bipartito soltanto per le relazioni binarie.
    \item[Matrice Booleana] è una matrice $M_R$ a valori \{0,1\} composta da $n$ righe e $m$ colonn
    \item[Grafi] modalità di rappresentazione di relazioni binarie(spiegate in un paragrafo successivo)
\end{description}

\subsection{Tabelle}
Inserire esempi!!!!!!!!!!!!

\subsection{Grafo Bipartito}
Inserire esempi con risoluzione!!!!

\subsecion{Matrice Booleana}
La \emph{Matrice booleana} è una matrice $M_R$,composta da $n$ righe e $m$ colonne,
i cui elementi sono definiti come $m_{ij} = \{ ^{1 \iff <s_i,t_j \in R} _{0 \text{altrimenti}}$

Inserire esempi!!!!

Da una matrice booleana si possono determinare facilmente le proprietà
di una relazione $R$,definita su $S$, soprattutto la proprietà simmetrica e la riflessiva.

La \emph{Matrice Complementare} $M_{\bar{R}}$ è costituita dai seguenti elementi
$\bar{m}_{ij} = \{ ^ 1 \iff m_{ij} = 0 _ 0 \iff m_{ij} = 1$

La \emph{Matrice inversa} $M_{R ^-1}$ è la trasporta della matrice $M_R$.
%Da fare la dimostrazione

Date due matrici $A$ e $B$,embrambe di $n x m$ elementi, si definiscono 3 operazioni:
\begin{description}
    \item[MEET $A \sqcap B = C$]: è una matrice booleana i cui elementi sono:
    $c_{ij} = \{ ^ 1 \quad a_{ij} = 1 \vand b_{ij} = 1 _ 0 \quad a_{ij} = 0 \vor b_{ij} = 0$
    \item[JOIN $A \sqcup B = C$]: è una matrice booleana i cui elementi sono:
    $c{ij} = \{ ^ 1 \quad a_{ij} = 1 \vor b_{ij} = 1 _ 0 \quad a_{ij} = 0 \vand b_{ij} = 0$
    \item[PRODOTTO BOOLEANO $A \odot B$]: è una matrice booleana $n x p$, i cui elementi sono:
    $c_{ij} = \{1 \quad \text{se per qualche} k(1 \leq k \leq m) \text{si ha} a_{ik} = 1 \vand b_{kj} = 1
    _ 0 \quad \text{altrimenti}$
\end{description}

Inserire Esempi

%Sezione Grafi
\section{Grafi}
Si definisce come \emph{Grafo} $G$ una coppia $<V,E>$ in cui $V$ è l'insieme
dei \textbf{vertici} o \textbf{Nodi},indicanti gli elementi, invece $E$
 è l'insieme degli \textbf{archi}, indicanti la relazione esistentre tra i vertici del grafo.\newline
In un grafo orientato è ammesso il cappio ossia archi che escono ed entrano dallo stesso vertice.

Se in grafo tutti gli archi presentano un ordinamento, ossia si definisce una direzione
tra i 2 vertici di un grafo,si definisce \emph{Grafo orientato}
altrimenti si definisce il grafo come \emph{Grafo non orientato}.
%Inserire Esempi

Un arco che congiunge $V_i$ a $V_j$ si dice \emph{uscente} da $V_i$ ed \emph{entrante} in $V_j$.

\subsection{Nomenclatura}
In un grafo si definisce:
\begin{description}
    \item[NODO SORGENTE]: nodi in cui non si hanno archi entranti
    \item[NODO POZZO]:nodi in cui non si hanno archi uscenti
    \item[NODO ISOLATO]: nodi in cui non si hanno archi entranti ed uscenti
    \item[GRADO DI ENTRATA]:è il numero di archi entranti in un nodo
    \item[GRADO DI USCITA]: è il numero degli archi uscenti da un nodo
    \item[CAMMINO da $V_{in}$ a $V_{fin}$]:è una sequenza finita di nodi $<V_1,V_2,\dots,V_n>$
     con $V_1 = V_{in}$ e $V_n = V_{fin}$, dove ciascun nodo è collegato al successivo da un arco orientato
    \item[SEMICAMMINO da $V_{in}$ a $V_{fin}$]: è una sequenza finita di nodi
     $<V_1,V_2,\dots,V_n>$ con $V_1 = V_{in}$ e $V_n = V_{fin}$, dove ciascun nodo
     è collegato al successivo da un arco non orientato.
    \item[CONNESSO]:un grafo in cui dati due nodi $V_a$ e $V_b$, con $V_a \neq V_b$,
                    esiste un semicammino tra di essi.
    \item[CICLO]intorno un nodo $V$ è un cammino in cui $V = V_{in} = V_{fin}$
    \item[SEMICICLO]intorno un nodo $V$ è un semicammino in cui $V = V_{in} = V_{fin}$
    \item[CAPPIO]intorno ad un nodo è un cammino di lunghezza 1 in cui $V_in = V_{fin}$
\end{description}

%Inserire esempi ed esercizi
%Inserire Proprietà Grafi

\section{DAG ed Alberi}
Si definisce come \emph{DAG}(Directed Acyclic Graph), un grafo orientato senza cicli.
%Inserire esempi degli DAG

Si definisce come \emph{Albero}, un DAG connesso con un solo nodo sorgente,detto \emph{radice},
in cui ogni nodo diverso dalla radice ha un solo nodo entrante.\newline
I nodi privi di archi entranti sono detti \emph{foglie} dell'albero.

%Inserire esempi e proprietà degli Alberi
%Inserire caratteristiche e Nomenclatura degli Alberi

