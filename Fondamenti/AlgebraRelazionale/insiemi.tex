%Capitolo sugli insiemi
\chapter{Insiemi}
Gli insiemi sono collezioni di oggetti detti elementi, in cui si prescinde dall'
ordine e dalla ripetizione degli elementi.

Si dice $x \in A$ se l'elemento appartiene all'insieme altrimenti si usa $x \not \in A$.
L'insieme vuoto si indica con $\emptyset$ mentre se due insiemi sono composti dagli stessi
elementi si usa $A = B$.

Gli insiemi possono essere definiti in due maniere:

\begin{itemize}
  \item \textbf{Estensionale}:si elencano gli elementi di un insieme\newline
        Esempio:\newline
        $A = \{1,2,3\}$\newline
        $B = \{2,4,6\}$\newline
  \item \textbf{Intensionale}:si descrivono gli elementi che soddisfano una determinata proprietà\newline
        Esempio: \newline
        $D = \{x \in N | x < 100\}$
\end{itemize}

Dati due insiemi $S$ e $T$ si ha:
%Definizione sottoinsieme e insiemi uguali
\begin{multline}
  S \subset T = \{x | x \in S \Rightarrow x \in T \land S \not = T \} \\
  S = T  = \{x | x \in S \iff x \in T \} \\
  S \subseteq T = \{x | S \subset Q \lor S = T \}
\end{multline}

%Inserire Esempio

%Definizione di Cardinalità di un Insieme
Si definisce \textit{cardinalità} il numero degli elementi e si indica con $|A|$.\newline
Due insiemi $S$ e $T$ si dicono \textit{equipotenti}, indicato con $S \sim T$, se
essi sono in corrispondenza univoca.

%Insiemi Numerici
Gli insiemi numerici definiti nella Teoria degli insiemi sono i seguenti: \newline
\begin{itemize}
  \item $\mathbf{N}$: Insieme dei numeri Naturali
  \item $\mathbf{Z}$: insieme dei numeri interi
  \item $\mathbf{Q}$: insieme dei numeri razionali
  \item $\mathbf{R}$: insieme dei numeri reali
  \item $\mathbf{C}$: insieme dei numeri complessi
\end{itemize}

Gli insiemi $\mathbf{N}, \mathbf{Z}, \mathbf{Q}$ hanno la stessa cardinalità indicata con $\aleph$

Gli insiemi $\mathbf{R}$ e $\mathbf{C}$ hanno la stessa cardinalità indicata con $ 2 ^ \aleph$

%Tipologia di insiemi
\begin{itemize}
  \item \textbf{Insieme Finito}: Cercare Definizione
  \item \textbf{Insieme Transfinito}: Cercare Definizione
  \item \textbf{Insieme Numerabile}: Cercare Definizione
\end{itemize}

%Inserire Esempi

\section{Unione ed Intersezione}
L'unione di due insiemi $S \cup T$ è l'insieme formato degli elementi di S e degli
elementi di T.\newline
L'intersezione di due insiemi $S \cap T$ è l'insieme degli elementi presenti in
tutti e due gli insiemi.

$S \cup T = \{x | x \in S \lor x \in T \} $ \newline
$S \cap T = \{x | x \in S \land x \in T \} $

Esempio:\newline
A = \{"Rosso","Arancio","Giallo" \} \newline
B = \{"Verde","Giallo","Marrone" \} \newline
A $\cup$ B = \{"Rosso","Arancio","Giallo","Verde","Marrone" \}
A $\cap$ B = \{"Giallo" \}

L'unione e l'intersezione sono legate dalle proprietà distributive:

\begin{enumerate}
  \item $S_1 \cap (S_2 \cup S_3) = (S_1 \cap S_2) \cup (S_1 \cap S_3)$
  \item $S_1 \cup (S_2 \cap S_3) = (S_1 \cup S_2) \cap (S_1 \cup S_3)$
\end{enumerate}
%Mancano le proprietà dell'unione e dell'Intersezione con le dimostrazioni

\section{Complementare}
Dato un insieme $U$, detto Universo, si dice \textit{complemento} di $S$, indicato con $\bar{S}$,
la differenza di un sottoinsieme $S$ di $U$ rispetto ad $U$.\newline
$\bar{S} = \{x | x \in U \land x \not \in S \} $

Esempio: \newline
$U = \{1,2,3,4,5,6,7\} $\newline
$S = \{2,4,6\} $ \newline
$\bar{S} = \{1,3,5,7\} $

%Mancano le proprietà

\section{Differenza di Insiemi}
Dati 2 insiemi $S$ e $T$ chiamiamo $S \ T$ l'insieme \textit{differenza} costituito
da tutti gli elementi di S che non sono elementi di T. \newline
$S \ T = {x | x \in S \land x \not \in T} $

Esempio: \newline
$S = \{a,b,c,d,e\}$ \quad $T = \{a,c,f,g,e,h\}$\newline
$S \ T = \{b,d\}$

La \textit{differenza simmetrica} di due insiemi $S_1$ e $S_2$, indicata con $S_1 \triangle S_2$,
è definita come $S_1 \triangle S_2 = (S_1 \textbackslash S_2) \cup (S_2 \textbackslash S_1) $

%Mancano le proprietà

\section{Insieme delle Parti}
L'insieme delle Parti di un insieme $S$, indicato con $\wp S$, è l'insieme formato
da tutti i sottoinsiemi dell'insieme $S$. \newline
$\wp S = \{X | X \subseteq S\} $

\textbf{Definizione}: Se $S$ è composto da $n \geq 0$ elementi, il numero di elementi
di $\wp S$ è $2 ^ n$.

%Manca la Dimostrazione
%Manca la definizione di Cardinalità e dei diversi tipi di Insiemi Numerici

\section{Prodotto Cartesiano}
Dati 2 insiemi $A$ e $B$, non necessariamente  distinti, si definisce come \textit{Prodotto Cartesiano},
indicato con $A x B$, l'insieme di tutte le coppie in cui il primo elemento appartiene ad $A$
e il secondo elemento della coppia appartiene ad $B$.\newline
$A$ x $B = {<a,b> | a \in A \land b \in B} $

Esempio:\newline
$A = \{1,2,3\} \newline
B = \{1,4,5\} \newline
A$ x $B = \{<1,1>,<1,4>,<1,5>,<2,1>,<2,4>,<2,5>,<3,1>,<3,4>,<3,5>\} $
