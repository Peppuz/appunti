%Capitolo sugli insiemi
\chapter{Insiemi}
Gli insiemi sono collezioni di oggetti detti elementi, in cui si prescinde dall'
ordine e dalla ripetizione degli elementi e questa è la definizione ingenua.

Si dice $x \in A$ se l'elemento appartiene all'insieme altrimenti si usa $x \not \in A$.
L'insieme vuoto si indica con $\emptyset$ mentre se due insiemi hanno gli stessi elementi si usa $A = B$.

Gli insiemi possono essere definiti in due maniere:
\begin{itemize}
  \item \textbf{Estensionale}:si elencano gli elementi di un insieme\newline
        Esempio:\newline
        $A = \{1,2,3\}$\newline
        $B = \{2,4,6\}$\newline
  \item \textbf{Intensionale}:si descrivono gli elementi che soddisfano una determinata proprietà\newline
        Esempio: \newline
        $D = \{x \in N | x < 100\}$
\end{itemize}

Dati due insiemi $S$ e $T$ si ha:
%Definizione sottoinsieme e insiemi uguali
\begin{equation*}
  S \subset T = \{x | x \in S \Rightarrow x \in T \land S \not = T \} \\
  S = T  = \{x | x \in S \iff x \in T \} \\
  S \subseteq T = \{x | S \subset Q \lor S = T \}
\end{equation*}

%Inserire Esempio
Esempio:\newline
\begin{equation*}
\begin{split}
A = \{1,2,3,5,6,7,9,10 \} \\
B = \{1,3,5,6,9 \} \\
C = \{1,2,3,5,6,7,9,10 \} \\
A = B \text{Falso} \\
B \subseteq A \text{Vero} \\
A \subset C \text{Falso} \\
A \subseteq C \text{Vero} \\
\end{split}
\end{equation*}

%Definizione di Cardinalità di un Insieme
Si definisce \textit{cardinalità} il numero degli elementi e si indica con $|A|$.\newline
Due insiemi $S$ e $T$ si dicono \textit{equipotenti}, indicato con $S \sim T$, se
essi sono in corrispondenza univoca.

%Insiemi Numerici
Gli insiemi numerici definiti nella Teoria degli insiemi sono i seguenti: \newline
\begin{itemize}
  \item $\N$: Insieme dei numeri Naturali
  \item $\Z$: insieme dei numeri interi
  \item $\Q$: insieme dei numeri razionali
  \item $\R$: insieme dei numeri reali
  \item $\C$: insieme dei numeri complessi
\end{itemize}

Gli insiemi $\N, \Z, \Q$ hanno la stessa cardinalità indicata con $\aleph$,dimostrato da Georg Cantor.

Gli insiemi $\R$ e $\C$ hanno la stessa cardinalità indicata con $ 2 ^ \aleph$,
dimostrato da Georg Cantor mediante il principio di diagonalizzazione.

%Tecnica di diagonalizzazione
\subsection{Tecnica di diagonalizzazione}
La tecnica di diagonalizzazione è tecnica, inventata da Cantor, per dimostrare la
non numerabilità dei Numeri Reali.\newline
Essa consiste nel tentativo di costruire una bisezione tra un insieme $X$ ed $\N$
e verificare che qualche elemento di $X$ sfugge alla bisezione.

%Teorema di Cantor!!! DA DIMOSTRARE

\section{Unione ed Intersezione}
L'unione di due insiemi $S \cup T$ è l'insieme formato degli elementi di S e degli
elementi di T.\newline
L'intersezione di due insiemi $S \cap T$ è l'insieme degli elementi presenti in
tutti e due gli insiemi.

$S \cup T = \{x | x \in S \lor x \in T \} $ \newline
$S \cap T = \{x | x \in S \land x \in T \} $

Esempio:\newline
$A = \{"Rosso","Arancio","Giallo" \}$ \newline
B = \{"Verde","Giallo","Marrone" \} \newline
A $\cup$ B = \{"Rosso","Arancio","Giallo","Verde","Marrone" \}\newline
A $\cap$ B = \{"Giallo" \}

$S = \{1,2,5,4,3,7,6,9\}$
$T = \{5,4,2,9,11,34,6\}$
$S \cup T = \{1,2,3,4,5,6,7,9,11,34\}$
$S \cap T = \{2,4,5,6,9\}$

\begin{prop}
Proprietà dell'unione
\begin{enumerate}
\item $S \cup S = S$ \quad Idempotenza
\item $S \cup \emptyset = S$ \quad Elemento Neutro
\item $S_1 \cup S_2 = S_2 \cup S_1$ \quad Associatività
\item $S_1 \cup S_2 = S_2 \iff S_1 \subseteq S_2$
\item $(S_1 \cup S_2) \cup S_3 = S_1 \cup (S_2 \cup S_3)$ \quad Commutatività
\item $S_1 \subseteq S_1 \cup S_2$
\item $S_2 \subseteq S_1 \cup S_2$
\end{enumerate}
\end{prop}

%dimostrazione
\begin{proof}
\begin{enumerate}
    \item $S \cup S = \{ x | x \in S \lor x \in S \} = S$
    \item $S \cup \emptyset = \{ x | x \in \lor x \in \emptyset \} = S$
    \item $S_1 \cup S_2 = \{ x | x \in S_1 \lor x \in S_2 \}$ per definizione di Insieme
           si può scrivere anche $\{ x | x \in S_2 \lor x \in S_1 \} = S_2 \cup S_1$
    \item Prima dimostriamo che $S_1 \cup S_2 = S_2 \rightarrow S_1 \subseteq S_2$
          $S_1 \cup S_2 = \{ x | x \in S_1 \lor x \in S_2 \}$
          Per ipotesi sappiamo che $S_1 \cup S_2 = S_2$ per cui tutti gli elementi di $S_1$
          sono anche elementi di $S_2$ per cui si dimostra che $S_1 \subseteq S_2$.
          In maniera analoga si dimostra che $S_1 \subseteq S_2 \rightarrow S_1 \cup S_2 = S_2$.
    \item Da Fare
    \item Da Fare
    \item Da fare
\end{enumerate}
\end{proof}

\begin{prop}
Proprietà dell'Intersezione
\begin{enumerate}
  \item $S \cap S = S$
  \item $S \cap \emptyset = \emptyset$
  \item $S_1 \cap S_2 = S_2 \cap S_1$
  \item $(S_1 \cap S_2) \cap S_3 = S_1 \cap (S_2 \cap S_3)$
  \item $S_1 \cap S_2 = S_1 \iff S_1 \subseteq S_2$
  \item $S_1 \cap S_2 \subseteq S_1$
  \item $S_1 \cap S_2 \subseteq S_2$
\end{enumerate}
\end{prop}
%dimostrazione

\begin{proof}
    \item $S \cap S = \{ x | x \in S \land x \in S \} = S$
    \item $S \cap \emptyset = \{ x | x \in S \land x \in \emptyset \} = \emptyset$
    \item $S_1 \cap S_2 = \{ x | x \in S_1 \land x \in S_2 \}$ per definizione di insieme
           si può scrivere anche $\{ x | x \in S_2 \land x \in S_1 \} = S_2 \cap S_1 $
    \item Da Fare
    \item Da Fare
    \item Da Fare
    \item Da Fare
\end{proof}

L'unione e l'intersezione sono legate dalle proprietà distributive:

\begin{enumerate}
  \item $S_1 \cap (S_2 \cup S_3) = (S_1 \cap S_2) \cup (S_1 \cap S_3)$
  \item $S_1 \cup (S_2 \cap S_3) = (S_1 \cup S_2) \cap (S_1 \cup S_3)$
\end{enumerate}
%Dimostrazione

\section{Complementare}
Dato un insieme $U$, detto Universo, si dice \textit{complemento} di $S$, indicato con $\bar{S}$,
la differenza di un sottoinsieme $S$ di $U$ rispetto ad $U$.\newline
$\bar{S} = \{x | x \in U \land x \not \in S \} $

Esempio: \newline
$U = \{1,2,3,4,5,6,7\} $\newline
$S = \{2,4,6\} $ \newline
$\bar{S} = \{1,3,5,7\} $

\begin{prop}
Proprietà:
\begin{enumerate}
  \item $\bar{U} = \emptyset $
  \item $\bar{\emptyset} = U$
  \item $\bar{\bar{S}} = S$
  \item $\bar{(S_1 \cup S_2)} = \bar{S_1} \cap \bar{S_2}$
  \item $\bar{(S_1 \cap S_2)} = \bar{S_1} \cup \bar{S_2}$
  \item $S \cap \bar{S} = \emptyset$
  \item $S \cup \bar{S} = U$
  \item $S_1 = S_2 \iff \bar{S_1} = \bar{S_2}$
  \item $S_1 \subseteq S_2 \iff \bar{S_2} \subseteq \bar{S_1}$
\end{enumerate}
\end{prop}

\section{Differenza di Insiemi}
Dati 2 insiemi $S$ e $T$ chiamiamo $S \ T$ l'insieme \textit{differenza} costituito
da tutti gli elementi di S che non sono elementi di T. \newline
$S \ T = {x | x \in S \land x \not \in T} $

Esempio: \newline
$S = \{a,b,c,d,e\}$ \quad $T = \{a,c,f,g,e,h\}$\newline
$S \ T = \{b,d\}$

\begin{prop}
Proprietà Differenza
\begin{enumerate}
  \item $S \ S = \emptyset$
  \item $S \ \emptyset = S$
  \item $\emptyset \ S = \emptyset$
  \item $(S_1 \ S_2) \ S_3 = (S_1 \ S_3) \ S_2 = S_1 \ (S_2 \cup S_3)$
  \item $S_1 \ S_2 = S_1 \cap \bar{S_2}$
\end{enumerate}
\end{prop}

La \textit{differenza simmetrica} di due insiemi $S_1$ e $S_2$, indicata con $S_1 \triangle S_2$,
è definita come $S_1 \triangle S_2 = (S_1 \textbackslash S_2) \cup (S_2 \textbackslash S_1) $

%Mancano le proprietà
\begin{prop}
Proprietà Differenza simmetrica
\begin{enumerate}
  \item $S \triangle S = \emptyset$
  \item $S \triangle \emptyset = S$
  \item $S_1 \triangle S_2 = S_2 \triangle S_1$
  \item $S_1 \triangle S_2 = (S_1 \cap \bar{S_2}) \cup (S_2 \cap \bar{S_1})$
  \item $S_1 \triangle S_2 = (S_1 \cup S_2) \ (S_2 \cap S_1)$
\end{enumerate}
\end{prop}

%Partizione di un Insieme
\section{Partizione di un Insieme}
Dato un insieme non vuoto $S$, una partizione di $S$ è una famiglia $F$ di sottoinsiemi di $S$ tale che
\begin{enumerate}
    \item ogni elemento di $S$ appartiene a qualche elemento di $F$, ossia $\cup F = S$
    \item due elementi qualunque di $F$ sono disgiunti ossia $\cap F = \emptyset$
\end{enumerate}
La partizione non può avere come elemento l'insieme vuoto in quanto esso non
appartiene agli elementi dell'insieme A.

Esempio:
$A = \{ 1,2,3 \}$ \newline
$Par(A) = \{ \{ 1 \},\{2,3 \} \}$

%Funzione Caratteristica!!!!
\section{Funzione Caratteristica}
Sia $U$ un insieme assunto come Universo,si definisce come \emph{funzione caratteristica}
di un sottoinsieme $S \subseteq U$ come:
$car(S,x) = \begin{cases} 1 \quad x \in S \\ 0 \quad x \not \in S \end{cases}$

\section{Insieme delle Parti}
L'insieme delle Parti di un insieme $S$, indicato con $\wp S$, è l'insieme formato
da tutti i sottoinsiemi dell'insieme $S$. \newline
$\wp S = \{X | X \subseteq S\} $

\begin{defi}
Se $S$ è composto da $n \geq 0$ elementi, il numero di elementi di $\wp S$ è $2 ^ n$.
\end{defi}
%Manca la Dimostrazione
\begin{proof}
Supponendo di avere una sequenza binaria di 3 bit,le cui possibili combinazioni
vengono rappresentate da $2 ^ k$ con $k = numero di bit$.\newline
Prendendo un insieme $A = \{"a","b","c" \}$ e utilizzando la funzione caratteristica,
con la convenzione di indicare il primo elemento dell'insieme $A$ a destra, si nota
che le sequenze di bit sono uguali alla sequenza ottenuta utilizzando la funzione caratteristica.
\end{proof}

%Paragrafo sul Prodotto Cartesiano
\section{Prodotto Cartesiano,Coppie Ordinate e Sequenze}
Le coppie Ordinate sono una collezione di 2 oggetti in cui non si prescinde
dall'ordine e dalla ripetizione infatti $(a,b) \neq (b,a)$ e $(1,2,1,4) \neq (1,2,4)$.

Una \emph{n-upla} ordinata $x_1,\dots,x_n$ è definita come $(x1,\dots,x_n) = ( (x_1,\dots,x_{n-1}),x_n)$
dove $(x_1,\dots,x_{n-1})$ è una $(n-1)$-upla ordinata.

Si definisce come \emph{sequenza}, una sequenza di oggetti in cui non si prescinde
dalla multiplicità degli elementi..

Dati 2 insiemi $A$ e $B$, non necessariamente  distinti, si definisce come \textit{Prodotto Cartesiano},
indicato con $A x B$, l'insieme di tutte le coppie in cui il primo elemento appartiene ad $A$
e il secondo elemento della coppia appartiene ad $B$.\newline
$A \times B = \{(a,b) | a \in A \ \text{e} \ b \in B\} $
\begin{align*}
A = \{1,2,3\} \\
B = \{1,4,5\} \\
A \times B = \{(1,1),(1,4),(1,5),(2,1),(2,4),(2,5),(3,1),(3,4),(3,5)\} \\
S = \{ 4,14,56 \} \\
T = \{ 3,46,12 \} \\
S \times T = \{(4,3),(4,46),(4,12),(14,3),(14,46),(14,12),(56,3),(56,46),(56,12) \} \\
\end{align*}

%MultiInsiemi
\section{Multiinsiemi}
Si definisce come \emph{multiinsiemi} una collezione di elementi in cui si prescinde
dall'ordine ma non dalla multiplicità degli elementi.\newline
Si puo anche definire come una funzione $M:E \mapsto \N$ che associa ad ogni elemento
di un insieme $E$ finito o numerabile, un numero, appartenente ad $\N$ indicante
il numero di occorrenze dell'elemento di $E$ nel multiinsieme $M$.\newline
La cardinalità di un multiinsieme $M$ è definita come
\begin{equation*}
\displaystyle |M| = \sum{_{e_i \in E} ^ {|E|}} M(e_i)
\end{equation*}
\begin{align*}
S = (1,2,1,3,4,4,2,3,3) \\
T = (2,1,3,1,4,4,3,2,3)
\end{align*}
Sono due multiinsiemi uguali con cardinalità 9

\subsection{Operazioni su Multiinsiemi}
\begin{description}
    \item[Intersezione]: $M_1 \cap M_2 = M_3$ dove $M_3(e) = min(M_1(e),M_2(e)) \forall e \in E$
    \item[Unione]: $M_1 \cup M_2 = M_3$ dove $M_3(e) = max(M_1(e),M_2(e)) \forall e \in E$
    \item[Unione Disgiunta]: $M_1 \uplus M_2 = M_3$ dove $M_3(e) = M_1(e) + M_2(e) \forall e \in E$
\end{description}

%Inserire Esempi!!!!

