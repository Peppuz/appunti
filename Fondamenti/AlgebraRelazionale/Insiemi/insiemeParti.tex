\section{Funzione Caratteristica}
Sia $U$ un insieme assunto come Universo,si definisce come \emph{funzione caratteristica}
di un sottoinsieme $S \subseteq U$ come:
\begin{equation*}
    car(S,x) = \begin{cases} 1 \quad \text{se} \ x \in S \\
                             0 \quad \text{se} \ x \not \in S\\
               \end{cases}
\end{equation*}

\section{Insieme delle Parti}
L'insieme delle Parti di un insieme $S$, indicato con $\wp S$, è l'insieme formato
da tutti i sottoinsiemi dell'insieme $S$. \newline
$\wp S = \{X | X \subseteq S\} $
\begin{align*}
A = \{\emptyset,3,20 \} \\
\wp A = \{A,\{\emptyset\},\{ 3\},\{ 20 \},\{ \emptyset,3 \},\{\emptyset,20 \},\{3,20\},\emptyset \} \\
\end{align*}

\begin{align*}
S = \{ \emptyset, \{ \emptyset \}, a,b\} \\
\wp S = & \{ \emptyset,\{\emptyset \},\{\{\emptyset\}\},\{a\},\{b\}, \\
        & \ \{ \emptyset,\{\emptyset\}\}, \{ \emptyset,a\},\{ \emptyset,b\} \\
        & \ \{ \emptyset,\{\emptyset\},a\},\{\emptyset,\{\emptyset\},b\}, \\
        & \ \{ \{\emptyset\},a,b\},\{ \emptyset,a,b\},\{\{\emptyset\},a\},\{\{\emptyset\},b\} \} \\
\end{align*}

\begin{defi}
Se $S$ è composto da $n \geq 0$ elementi, il numero di elementi di $\wp S$ è $2 ^ n$.
\end{defi}
%Dimostrazione
\begin{proof}
Supponendo di avere una sequenza binaria di 3 bit,le cui possibili combinazioni
vengono rappresentate da $2 ^ k$ con $k = numBit$.\newline
Prendendo un insieme $A = \{'a','b','c' \}$ e utilizzando la funzione caratteristica,
con la convenzione di indicare il primo elemento dell'insieme $A$ a destra, si nota
che le sequenze di bit sono uguali alla sequenza ottenuta utilizzando la funzione caratteristica.
\end{proof}
