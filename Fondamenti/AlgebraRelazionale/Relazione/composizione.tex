\section{Composizione di Relazioni}
Data una relazione $R_1 \subseteq S x T$ e una relazione $R_2 \subseteq T \times Q$,
si definisce come relazione composta $R_1 \circ R_2 \subseteq S \times Q$ come segue
$(a,c) \in R_1 \circ R_2$ se e solo se  $\exists b \in T | (a,b) \in R_1 \land (b,c) \in R_2$

Siano $S = \{ a,b \}, R_1 = \{ (a,a),(a,b),(b,b) \}$ e $R_2 = \{ (a,b),(b,a),(b,b) \}$
$R_1 \circ R_2 = \{ (a,a),(a,b),(b,a),(b,b) \}$ \newline
$R_2 \circ R_1 = \{ (a,b),(b,a),(b,b) \}$

\begin{prop}
La composizione di Relazioni è associativa
\end{prop}
%Fare la Dimostrazione

\begin{thm}
Se $R_1 \subseteq S \times T$ e $R_2 \subseteq T \times Q$, allora $(R_1 \circ R_2)^-1 = R_1^-1 \circ R_2^-1$
\end{thm}
%Fare Dimostrazione

%Potenza di una relazione

Inserire Esempi!!!
