%Chiusura Transitiva
Sia $(S,R)$ una struttura relazionale con $R$ è una relazione binaria su $S$.\newline
La \emph{chiusura transitiva} di $R$ è la più piccola relazione transitiva $R'$ in cui $R \subseteq R'$.
La \emph{chiusura riflessiva} di $R$ è la più piccola relazione riflessiva $R'$ in cui $R \subseteq R'$

\begin{prop}
Dato un poset $(S,R)$ e un insieme $X \subseteq S$ si hanno le seguenti proprietà:
\end{prop}
\begin{enumerate}
    \item[elem. massimale]: un elemento $s \in S$ è massimale se $\not \exists s' \in S : s \leq s'$
    \item[elem. minimale]: un elemento $s \in S$ è minimale se $\not \exists s' \in S : s \geq s'$
    \item[maggiorante]: un elemento $s \in S$ è un maggiorante se $\forall x \in X$ si ha $s \geq x$
    \item[minorante]: un elemento $s \in S$ è un minorante se $\forall x \in X$ si ha $s \leq x$
    \item[minimo maggiorante]:un elemento $s \in S$ è un minimo maggiorante, indicato con $\sqcup X$,
          se è un maggiorante e per ogni altro maggiorante $s'$ di $X$ si ha $s \leq s'$.
    \item[massimo minorante]: un elemento $s \in S$ è un massimo minorante, indicato con $\sqcap X$,
          se è un minorante e per ogni altro minorante $s'$ di $X$ si ha $s' \leq s$
    \item[massimo]: se $\sqcup X \in X$ si dice che $\sqcup X$ è un massimo.
    \item[minimo]: se $\sqcap X \in X$ si dice che $\sqcap X$ è un minimo.
\end{enumerate}
%Proprietà con dimostrazioni

%Definizione di Buon Ordinamento
\begin{defi}
    Un poset è detto buon ordinamento se e solo se per ogni sottoinsieme non vuoto esiste
    $\cap X$ e tale poset è detto \emph{ben formato}
\end{defi}

Inserire Esempi !!!!
