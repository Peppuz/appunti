%Capitolo sulle funzioni
\chapter{Funzioni}
Si definisce \textit{funzione $f:S \mapsto T$} una relazione $f \subseteq SxT$
tale che $\forall x \in S$ esiste al più un $y \in T$ per cui $<x,y> \in f$.\newline
Se il $dom(f) = S$ la funzione si dice \emph{totale} altrimenti la funzione è \emph{parziale}.

%Tipologie di funzioni
\section{Tipologie di Funzioni}
Una funzione $f:S \mapsto T$ si dice:
\begin{description}
    \item[iniettiva]: $\forall x,y \in S x \neq y \rightarrow f(x) \neq f(y)$
    \item[suriettiva]: $\forall y \in T \exists x \in S : f(x) = y$
    \item[biettiva]: se la funzione è iniettiva e suriettiva
\end{description}

%Esempi
Esempi:
$+:\N x \N \mapsto \N$ è una funzione totale,suriettiva ma non iniettiva
$*:\N x \N \mapsto \N$ è una funzione totale,suriettiva ma non è iniettiva
$successore:\N \mapsto \N$ è una funzione totale ed è iniettiva ma non suriettiva
$successore:\Z \mapsto \Z$ è una funzione totale ed è biettiva

%Funzione inversa
Una funzione $f:S \mapsto T$ è detta \emph{invertibile} se la sua relazione inversa
$f ^ -1$ è essa stessa una funzione.\newline
Una funzione $f:S \mapsto T$ ammette una \emph{funzione inversa} $f ^ -1 :T \mapsto S$
se e solo se $f$ è una funzione iniettiva.

%Proprietà funzioni Inverse
\section{Proprietà funzioni inverse}
Sia $f:A \mapsto B$ invertibile, con funzione inversa $f ^ -1$:
\begin{enumerate}
    \item $f^-1$ è totale $\iff f$ è suriettiva
    \item $f$ è totale $\iff f^-1$ è suriettiva
\end{enumerate}

%Funzione Composta
Date due funzioni $f:S \mapsto T$ e $g:T \mapsto Q$ si definisce \emph{funzione composta}
$g \circ f:S \mapsto Q$ la funzione tale che $(g \circ f)(x) = g(f(x))$ per ogni $x \in S$.
La funzione composta $(g \circ f)(x)$ è definita se e solo se sono definite entrambe
$g(f(x))$ e $f(x)$

%Inserire esempi di funzioni composte

Siano $f:S \mapsto T$ e $g:T \mapsto Q$ invertibili. Allora $g \circ f$ è invertibile
e la sua inversa è $(g \circ f) ^ -1 = f^-1 \circ g ^ -1$.
%Inserire dimostrazione

%Operazione
\section{Operazioni}
Si definisce come \emph{operazione n-aria} su un insieme $S$, una funzione
$f:S^n \mapsto S$ con $n \geq 1$.
Se $f$ è un'operazione binaria su $S$, essa si può rappresentare anche mediante
la notazione infissa $x_1 f x_2$ invece di $f(x_1,x_2)$

%Inserire esempi!!!!
