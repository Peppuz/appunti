%Capitolo sulle funzioni
\chapter{Funzioni}
%Migliorare definizione di Funzione!!!
Si definisce \textit{funzione $f:S \mapsto T$} una relazione $f \subseteq SxT$
tale che $\forall x \in S$ esiste al più un $y \in T$ per cui $<x,y> \in f$.

Se il $dom(f) = S$ la funzione si dice \emph{totale} altrimenti la funzione è \emph{parziale}.


%Tipologie di funzioni
\section{Tipologie di Funzioni}
Una funzione $f:S \mapsto T$ si dice:
\begin{description}
    \item[iniettiva]: $\forall x,y \in S x \neq y \if f(x) \neq f(y)$
    \item[suriettiva]: $\forall y \in T \exists x \in S : f(x) = y$
    \item[biettiva]: se la funzione è iniettiva e suriettiva
\end{description}

%Inserire esempi funzioni
