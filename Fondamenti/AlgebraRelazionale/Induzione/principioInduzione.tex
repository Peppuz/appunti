%Dimostrazione per induzione(Da migliorare)
\section{Principio di Induzione}
Il principio di Induzione si utilizza per dimostrare la correttezza di determinate
proprietà dell'insieme dei numeri Naturali.

Il principio di induzione viene definito nel seguente modo:\newline
Data una proposizione $P(x)$ valida per $\forall x \in N$ bisogna:
\begin{enumerate}
  \item \textbf{Caso Base}: Sia $P(0)$ vero
  \item \textbf{Passo Induttivo}: Supposto $P(x)$ vero  bisogna verificare la verità di $P(x+1)$
\end{enumerate}

%Definizione Principio di Induzione Completo
Il principio di Induzione completo è definito nel seguente modo:
\begin{defi}
Sia A(n) una asserzione per ogni elemento $n \geq i \in \N$. Supponendo che:
\begin{itemize}
    \item $A(i)$ è vera (Caso Base)
    \item $\forall m \in \N$, se $A(k)$ è vera $\forall k$,con $0 < k < m$, ne segue
          che è vera $A(m)$
\end{itemize}
Allora $\forall n \in N$,$A(n)$ è vera
\end{defi}

%Esempi
%Esempio Tableaux Predicativo
Formula: $(\forall x F(x) \lor \exists G(x)) \rightarrow (\exists x (F(x) \lor G(x)))$
\begin{proof}
\begin{equation*}
\begin{prooftree}
\hypo{F \forall x F(x) \lor \exists G(x) \rightarrow \exists x (F(x) \lor G(x))}
\infer1{T \forall x F(x) \lor \exists G(x),F \exists x (F(x) \lor G(x))}
\infer1{T \forall x F(x),F \exists x (F(x) \lor G(x))/T \exists G(x),F \exists x (F(x) \lor G(x))}
\infer1{T F(a),F \exists x (F(x) \lor G(x)),T \forall x \dots/T G(a),F \exists x (F(x) \lor G(x))}
\infer1{T F(a),F F(a) \lor G(a),T \forall \dots,F \exists \dots/T G(a),F F(a) \lor G(a),F \exists \dots}
\infer1{T F(a),F F(a),F G(a),T \forall \dots,F \exists \dots/T G(a),F F(a),F G(a),F \exists \dots}
\end{prooftree}
\end{equation*}
Il tableaux contraddizione chiude per cui la formula è una tautologia.
\end{proof}

%Dimostrazione Tableaux Predicativo
Esempio:$\exists x (P(x) \lor Q(x)) \rightarrow (\exists x P(x) \lor \forall y Q(y))$
\begin{equation*}
\begin{prooftree}
\hypo{F \exists x (P(x) \lor Q(x)) \rightarrow (\exists x P(x) \lor \forall y Q(y))}
\infer1 {T \exists x (P(x) \lor Q(x)), F \exists x P(x) \lor \forall y Q(y)}
\infer1{T P(a) \lor Q(a), F \exists x P(x) \lor \forall y Q(y)}
\infer1 {T P(a) \lor Q(a), F \exists x P(x), F \forall y Q(y)}
\infer1 {T P(a) \lor Q(a), F \exists x P(x), F Q(b)}
\infer1 {T P(a),F \exists x P(x),F Q(b)/T Q(a),F Q(b),F \exists x P(x)}
\end{prooftree}
\end{equation*}
Il secondo ramo del tableau non potrà mai chiudere percui bisogna fare il T-Tableaux

\begin{equation*}
\begin{prooftree}
\hypo{T \exists x (P(x) \lor Q(x)) \rightarrow (\exists x P(x) \lor \forall y Q(y))}
\infer1 {F \exists x P(x) \lor Q(x)/T \exists x P(x) \lor \forall y Q(y)}
\infer1 {F \exists x P(x) \lor Q(x)/T \exists x P(x)/T \forall y Q(y)}
\end{prooftree}
\end{equation*}
Il tableaux non potrà mai chiudere in quanto il secondo e il terzo non generanno mai delle contraddizioni
per cui la formula è sodddisfacibile non tautologica.

Esempio:\newline
\begin{thm}
 $\displaystyle \sum_{i = 1}^n 2i-1 = n^2$
\end{thm}

\begin{proof}
Per $n = 1 \quad \displaystyle \sum_{i = 1}^1 2i-1 = 1^2 \quad 1 = 1$ è vero

Se $\displaystyle \sum_{i = 1}^n 2i-1 = n^2$ allora
$\displaystyle \sum_{i = 1}^{n+1} 2i-1 = (n+1)^2$

\begin{equation*}
\begin{split}
  \sum_{i=1}^{n+1} 2i-1 & = \sum_{i=1}^n 2i-1 + 2(n+1) - 1 \\
         & = n^2 + 2n + 1 \\
         & = (n+1)^2 \\
\end{split}
\end{equation*}
\end{proof}

\begin{thm}
    $\displaystyle n! \geq 2^{(n-1)} \quad \forall n \in \N$
\end{thm}
%Dimostrazione
\begin{proof}
Per $n = 0$ si ha $0! \geq 2^{0-1}$ ossia $ 1 \geq 1/2$ che è sempre verificato

Se $n! \geq 2^{(n-1)}$ allora $(n+1)! \geq 2^n$
\begin{equation*}
\begin{split}
(n+1)! \geq 2^{n+1-1} & = n! (n+1) \geq 2^{n-1} * 2 \\
                      & = n! \frac{(n+1)}{2} \geq 2^{n-1} \\
\end{split}
\end{equation*}
Essendo $\frac{(n+1)}{2} \geq 0 \forall n \in \N$ e $n! \geq 2^{n-1}$ verificato per ipotesi
si ricava che la proposizione è verificata.
\end{proof}

%Dimostrazione per induzione esempio5
\begin{thm}
    $\sum _ {k=0} ^ n (3k+1) = \frac{3n^2+5n+2}{2}$
\end{thm}

\begin{proof}
Caso base $n = 0$:
\begin{equation*}
    \sum _ {k = 0} ^ 0 (3k+1) = \frac{0+0+2}{2} \quad 1 = 1 \text{vero}
\end{equation*}
Caso passo:
\quad Ipotesi:$\sum _ {k=0} ^ n (3k+1) = \frac{3n^2+5n+2}{2}$ \newline
\quad Tesi:$\sum _ {k=0} ^ {n+1} (3k+1) = \frac{3(n+1)^2+5n+5+2}{2} = \frac{3n^2+11n+10}{2}$
\begin{equation*}
\begin{split}
\sum _ {k=0} ^ {n+1} (3k+1) & = \sum _ {k = 0} ^ n (3k+1) + 3(n+1)+1\\
                            & = \frac{3n^2+5n+2}{2} + 3n+4\\
                            & = \frac{3n^2+5n+2+6n+8}{2}\\
                            & = \frac{3n^2+11+10}{2}\\
\end{split}
\end{equation*}
\end{proof}

%Dimostrazione per Induzione esempio 6
\begin{thm}
    $\displaystyle \sum _{k=1} ^ n k^3 = \frac{n^2 (n+1)^2}{4}$
\end{thm}

%Dimostrazione
\begin{proof}
Caso base: $n = 1 \sum _{k=1} ^ 1 k^3 = \frac{1^2(1+1)^2}{4} 1 = \frac{4}{4} \text{vero}$

Ipotesi induttiva: $\sum _{k=1} ^ n k^3 = \frac{n^2 (n+1)^2}{4}$
Tesi induttiva: $\sum _{k=1} ^ {n+1} k^3 = \frac{(n+1)^2 (n+2)^2}{4}$
\begin{equation*}
\begin{split}
\sum _{k=1} ^{n+1} k^3 & = \sum _{k=1} ^ n k^3 + (n+1)^3 \\
                       & = \frac{n^2 (n+1)^2}{4} + (n+1)^3 \\
                       & = (n+1)^2 (\frac{n^2}{4} + (n+1)) \\
                       & = (n+1)^2 (\frac{n^2 + 4n + 4}{4}) \\
                       & = \frac{(n+1)^2 (n+2)^2}{4} \\
\end{split}
\end{equation*}
\end{proof}

