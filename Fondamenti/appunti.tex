\documentclass[a4paper,11pt,fleqn]{report}
\usepackage[T1]{fontenc}%Gestione Font
\usepackage[utf8]{inputenc}%Gestione Charset Utf8
\usepackage[italian]{babel}%Gestione sillabazione Italiana
\usepackage{amsmath}%Package per la gestione della matematica
\usepackage{amsthm}%Package per la gestione dei Teoremi della matematica
\usepackage{amsfonts}%Package per i Font Matematici
\usepackage{booktabs}%Package per la gestione delle Tabelle
\usepackage{caption}%Package per la gestione delle Tabelle
\usepackage{slashed}%Package per fare il simbolo di non relazione
\usepackage{tkz-graph}%Package per disegnare grafi
\usepackage{prooftrees}%Package per i Tableau
\setlength{\parindent}{0pt}%Toglie il rientro dei capoversi
\newtheorem{defi}{Definizione}%Definizione per avere la gestione delle definizioni
\newtheorem{prop}{Proposizione}
\newtheorem{lem}{Lemma}
\newtheorem{thm}{Teorema}
\newcommand{\numberset}{\mathbb}
\newcommand{\N}{\numberset{N}}
\newcommand{\Z}{\numberset{Z}}
\newcommand{\Q}{\numberset{Q}}
\newcommand{\R}{\numberset{R}}
\newcommand{\C}{\numberset{C}}

\begin{document}
\author{Marco Natali}
\title{Appunti di Fondamenti dell'Informatica}
\maketitle
%Capitolo sugli insiemi
\chapter{Insiemi}

%Definizione di Insieme
\chapter{Relazioni}
Si definisce \textit{relazione n-aria} un sottoinsieme del prodotto cartesiano
rappresentato da tutte le coppie che rispettano la relazione voluta tra gli $n$ insiemi,
che può essere definita in maniera estensionale e/o intensionale. \newline
Si definisce \textit{arietà} di una relazione il numero e il tipo degli argomente
di una relazione.

%Definizione Dominio e Codominio di una relazione
\textbf{Dominio}:insieme degli elementi $x$ tali che $(x,y) \in R$ per qualsiasi $y$.\newline
\textbf{Codominio}:insieme degli elementi $y$ tali che $(x,y) \in R$ per qualsiasi $x$.

\begin{align*}
A \times B & = \{(1,1),(1,4),(1,5),(2,1),(2,4),(2,5),(3,1),(3,4),(3,5)\} \\
R \subseteq A \times B  & = \{(1,1),(1,4),(1,5),(2,4),(2,5),(3,4),(3,5)\}\\
R \subseteq B \times A & = \{(1,1),(1,2),(1,3)\} \\
\end{align*}


%Insiemi Numerici

Gli insiemi numerici definiti nella Teoria degli insiemi sono i seguenti:
\begin{itemize}
  \item $\N$: insieme dei numeri naturali, comprendente anche lo $O$
  \item $\Z$: insieme dei numeri interi
  \item $\Q$: insieme dei numeri razionali
  \item $\R$: insieme dei numeri reali
  \item $\C$: insieme dei numeri complessi
\end{itemize}
Gli insiemi $\N, \Z, \Q$ hanno la stessa cardinalità indicata con $\aleph_0$,dimostrato da Georg Cantor.\newline
Gli insiemi $\R$ e $\C$ hanno la stessa cardinalità indicata con $ 2 ^ {\aleph_0}$,
dimostrato da Georg Cantor mediante il principio di diagonalizzazione.

%Tecnica di diagonalizzazione
\subsection{Tecnica di diagonalizzazione}
La tecnica di diagonalizzazione è una tecnica, inventata da Georg Cantor, per dimostrare la
non numerabilità dei Numeri Reali.\newline
Essa consiste nel tentativo di costruire una biiezione tra un insieme $X$ ed $\N$
e verificare che qualche elemento di $X$ sfugge alla biiezione.

\begin{thm}
    $2^{\aleph_0}$ è strettamente maggiore di $\aleph_0$
\end{thm}

%DA DIMOSTRARE


%Unione ed Intersezione
\section{Unione ed Intersezione}
L'unione di due insiemi $S \cup T$ è l'insieme formato degli elementi di S e degli
elementi di T.\newline
L'intersezione di due insiemi $S \cap T$ è l'insieme degli elementi presenti in
tutti e due gli insiemi.

$S \cup T = \{x | x \in S \ \text{o} \ x \in T \} $ \newline
$S \cap T = \{x | x \in S \ \text{e} \ x \in T \} $

Esempio:
\begin{align*}
A = & \{Rosso,Arancio,Giallo \} \\
B = & \{Verde,Giallo,Marrone \} \\
A \cup B = & \{Rosso,Arancio,Giallo,Verde,Marrone \} \\
A \cap B = & \{Giallo \} \\
S = \{1,2,5,4,3,7,6,9\}  & \ S \cup T = \{1,2,3,4,5,6,7,9,11,34\} \\
T = \{5,4,2,9,11,34,6\}  & \ S \cap T = \{2,4,5,6,9\} \\
C = \{3,9,15,21,27\}  & \ C \cup D = \{3,6,9,15,21,24,27,33,42,51,60\} \\
D = \{6,15,24,33,42,51,60\} & \ C \cap D = \{15\}\\
\end{align*}

%Proprietà dell'Unione
\begin{prop}
    L'unione gode delle seguenti proprietà:
\end{prop}
\begin{enumerate}
\item $S \cup S = S$ \quad Idempotenza
\item $S \cup \emptyset = S$ \quad Elemento Neutro
\item $S_1 \cup S_2 = S_2 \cup S_1$ \quad Associatività
\item $S_1 \cup S_2 = S_2 \iff S_1 \subseteq S_2$
\item $(S_1 \cup S_2) \cup S_3 = S_1 \cup (S_2 \cup S_3)$ \quad Commutatività
\item $S_1 \subseteq S_1 \cup S_2$
\item $S_2 \subseteq S_1 \cup S_2$
\end{enumerate}

%dimostrazione
%\begin{proof}
%\begin{enumerate}
%    \item $S \cup S = \{ x | x \in S \lor x \in S \} = S$
%    \item $S \cup \emptyset = \{ x | x \in \lor x \in \emptyset \} = S$
%    \item $S_1 \cup S_2 = \{ x | x \in S_1 \lor x \in S_2 \}$ per definizione di Insieme
%           si può scrivere anche $\{ x | x \in S_2 \lor x \in S_1 \} = S_2 \cup S_1$
%    \item Prima dimostriamo che $S_1 \cup S_2 = S_2 \rightarrow S_1 \subseteq S_2$
%          $S_1 \cup S_2 = \{ x | x \in S_1 \lor x \in S_2 \}$
%          Per ipotesi sappiamo che $S_1 \cup S_2 = S_2$ per cui tutti gli elementi di $S_1$
%          sono anche elementi di $S_2$ per cui si dimostra che $S_1 \subseteq S_2$.
%          In maniera analoga si dimostra che $S_1 \subseteq S_2 \rightarrow S_1 \cup S_2 = S_2$.
%    \item Da Fare
%    \item Da Fare
%    \item Da fare
%\end{enumerate}
%\end{proof}

%Proprietà dell'Intersezione
\begin{prop}
    L'intersezione gode delle seguenti proprietà:
\end{prop}
\begin{enumerate}
  \item $S \cap S = S$
  \item $S \cap \emptyset = \emptyset$
  \item $S_1 \cap S_2 = S_2 \cap S_1$
  \item $(S_1 \cap S_2) \cap S_3 = S_1 \cap (S_2 \cap S_3)$
  \item $S_1 \cap S_2 = S_1 \iff S_1 \subseteq S_2$
  \item $S_1 \cap S_2 \subseteq S_1$
  \item $S_1 \cap S_2 \subseteq S_2$
\end{enumerate}

%dimostrazione
%\begin{proof}
%    \item $S \cap S = \{ x | x \in S \land x \in S \} = S$
%    \item $S \cap \emptyset = \{ x | x \in S \land x \in \emptyset \} = \emptyset$
%    \item $S_1 \cap S_2 = \{ x | x \in S_1 \land x \in S_2 \}$ per definizione di insieme
%           si può scrivere anche $\{ x | x \in S_2 \land x \in S_1 \} = S_2 \cap S_1 $
%    \item Da Fare
%    \item Da Fare
%    \item Da Fare
%    \item Da Fare
%\end{proof}

\begin{prop}
L'unione e l'intersezione hanno le prop. distributive:
\end{prop}
\begin{enumerate}
  \item $S_1 \cap (S_2 \cup S_3) = (S_1 \cap S_2) \cup (S_1 \cap S_3)$
  \item $S_1 \cup (S_2 \cap S_3) = (S_1 \cup S_2) \cap (S_1 \cup S_3)$
\end{enumerate}

%Dimostrazione


%Complementare e Differenza
\section{Complementare}
Dato un insieme $U$, detto Universo, si dice \textit{complemento} di $S$, indicato con $\bar{S}$,
la differenza di un sottoinsieme $S$ di $U$ rispetto ad $U$.\newline
$\bar{S} = \{x | x \in U \ \text{e} \ x \not \in S \} $
Migliorare rappresentazione simbolo complementare

Esempio:
\begin{equation*}
U = \{1,2,3,4,5,6,7\}
S = \{2,4,6\}
\bar{S} = \{1,3,5,7\}
\end{equation*}


\begin{prop}
    Le proprietà della complementazione di un insieme sono :
\end{prop}
\begin{enumerate}
  \item $\bar{U} = \emptyset $
  \item $\bar{\emptyset} = U$
  \item $\overline{\bar{S}} = S$
  \item $\overline{(S_1 \cup S_2)} = \bar{S_1} \cap \bar{S_2}$
  \item $\overline{(S_1 \cap S_2)} = \bar{S_1} \cup \bar{S_2}$
  \item $S \cap \bar{S} = \emptyset$
  \item $S \cup \bar{S} = U$
  \item $S_1 = S_2 \ \text{se e solo se} \ \bar{S_1} = \bar{S_2}$
  \item $S_1 \subseteq S_2 \ \text{se e solo se} \ \overline{S_2} \subseteq \overline{S_1}$
\end{enumerate}

\section{Differenza di Insiemi}
Dati 2 insiemi $S$ e $T$ chiamiamo $S \setminus T$ l'insieme \textit{differenza} costituito
da tutti gli elementi di S che non sono elementi di T. \newline
$S \setminus T = \{x | x \in S \ \text{e} \ x \not \in T\} $

Esempio:
\begin{align*}
S = \{a,b,c,d,e\} & \ T = \{a,c,f,g,e,h\} \\
S \setminus T = \{b,d\} \\
\end{align*}
%Proprietà differenza
\begin{prop}
    La differenza di insiemi gode delle seguenti proprietà:
\end{prop}
\begin{enumerate}
  \item $S \setminus S = \emptyset$
  \item $S \setminus \emptyset = S$
  \item $\emptyset \setminus S = \emptyset$
  \item $(S_1 \setminus S_2) \setminus S_3 =
         (S_1 \setminus S_3) \setminus S_2 = S_1 \setminus (S_2 \cup S_3)$
  \item $S_1 \setminus S_2 = S_1 \cap \bar{S_2}$
\end{enumerate}

La \textit{differenza simmetrica} di due insiemi $S_1$ e $S_2$, indicata con $S_1 \triangle S_2$,
è definita come $S_1 \triangle S_2 = (S_1 \setminus S_2) \cup (S_2 \setminus S_1) $

% proprietà diff. simmetrica
\begin{prop}
Proprietà Differenza simmetrica
\end{prop}
\begin{enumerate}
  \item $S \triangle S = \emptyset$
  \item $S \triangle \emptyset = S$
  \item $S_1 \triangle S_2 = S_2 \triangle S_1$
  \item $S_1 \triangle S_2 = (S_1 \cap \bar{S_2}) \cup (S_2 \cap \bar{S_1})$
  \item $S_1 \triangle S_2 = (S_1 \cup S_2) \ (S_2 \cap S_1)$
\end{enumerate}


%Partizione di un Insieme
\section{Partizione di un Insieme}
Dato un insieme non vuoto $S$, una partizione di $S$ è una famiglia $F$ di sottoinsiemi di $S$ tale che
\begin{enumerate}
    \item ogni elemento di $S$ appartiene a qualche elemento di $F$, ossia $\cup F = S$
    \item due elementi qualunque di $F$ sono disgiunti ossia $\cap F = \emptyset$
\end{enumerate}
La partizione non può avere come elemento l'insieme vuoto in quanto esso non
appartiene agli elementi dell'insieme A.
\begin{align*}
A = \{ 1,2,3 \}\\
Par(A) = \{ \{ 1 \},\{2,3 \} \}\\
\end{align*}

\begin{align*}
B = \{5,10,\emptyset,23,45\}\\
Par(B) = \{ \{\emptyset,23\}, \{5,10\},\{45\} \\
\end{align*}


%Insieme delle Parti
\section{Funzione Caratteristica}
Sia $U$ un insieme assunto come Universo,si definisce come \emph{funzione caratteristica}
di un sottoinsieme $S \subseteq U$ come:
\begin{equation*}
    car(S,x) = \begin{cases} 1 \quad \text{se} \ x \in S \\
                             0 \quad \text{se} \ x \not \in S\\
               \end{cases}
\end{equation*}

\section{Insieme delle Parti}
L'insieme delle Parti di un insieme $S$, indicato con $\wp S$, è l'insieme formato
da tutti i sottoinsiemi dell'insieme $S$. \newline
$\wp S = \{X | X \subseteq S\} $
\begin{align*}
A = \{\emptyset,3,20 \} \\
\wp A = \{A,\{\emptyset\},\{ 3\},\{ 20 \},\{ \emptyset,3 \},\{\emptyset,20 \},\{3,20\},\emptyset \} \\
\end{align*}

\begin{align*}
S = \{ \emptyset, \{ \emptyset \}, a,b\} \\
\wp S = & \{ \emptyset,\{\emptyset \},\{\{\emptyset\}\},\{a\},\{b\}, \\
        & \ \{ \emptyset,\{\emptyset\}\}, \{ \emptyset,a\},\{ \emptyset,b\} \\
        & \ \{ \emptyset,\{\emptyset\},a\},\{\emptyset,\{\emptyset\},b\}, \\
        & \ \{ \{\emptyset\},a,b\},\{ \emptyset,a,b\},\{\{\emptyset\},a\},\{\{\emptyset\},b\} \} \\
\end{align*}

\begin{defi}
Se $S$ è composto da $n \geq 0$ elementi, il numero di elementi di $\wp S$ è $2 ^ n$.
\end{defi}
%Dimostrazione
\begin{proof}
Supponendo di avere una sequenza binaria di 3 bit,le cui possibili combinazioni
vengono rappresentate da $2 ^ k$ con $k = numBit$.\newline
Prendendo un insieme $A = \{'a','b','c' \}$ e utilizzando la funzione caratteristica,
con la convenzione di indicare il primo elemento dell'insieme $A$ a destra, si nota
che le sequenze di bit sono uguali alla sequenza ottenuta utilizzando la funzione caratteristica.
\end{proof}


%Paragrafo sul Prodotto Cartesiano
\section{Prodotto Cartesiano,Coppie Ordinate e Sequenze}
Le coppie Ordinate sono una collezione di 2 oggetti in cui non si prescinde
dall'ordine e dalla ripetizione infatti $(a,b) \neq (b,a)$ e $(1,2,1,4) \neq (1,2,4)$.

Una \emph{n-upla} ordinata $x_1,\dots,x_n$ è definita come $(x1,\dots,x_n) = ( (x_1,\dots,x_{n-1}),x_n)$
dove $(x_1,\dots,x_{n-1})$ è una $(n-1)$-upla ordinata.

Si definisce come \emph{sequenza}, una sequenza di oggetti in cui non si prescinde
dalla multiplicità degli elementi..

Dati 2 insiemi $A$ e $B$, non necessariamente  distinti, si definisce come \textit{Prodotto Cartesiano},
indicato con $A x B$, l'insieme di tutte le coppie in cui il primo elemento appartiene ad $A$
e il secondo elemento della coppia appartiene ad $B$.\newline
$A \times B = \{(a,b) | a \in A \ \text{e} \ b \in B\} $
\begin{align*}
A = \{1,2,3\} \\
B = \{1,4,5\} \\
A \times B = \{(1,1),(1,4),(1,5),(2,1),(2,4),(2,5),(3,1),(3,4),(3,5)\} \\
S = \{ 4,14,56 \} \\
T = \{ 3,46,12 \} \\
S \times T = \{(4,3),(4,46),(4,12),(14,3),(14,46),(14,12),(56,3),(56,46),(56,12) \} \\
\end{align*}

%MultiInsiemi
\section{Multiinsiemi}
Si definisce come \emph{multiinsiemi} una collezione di elementi in cui si prescinde
dall'ordine ma non dalla multiplicità degli elementi.\newline
Si puo anche definire come una funzione $M:E \mapsto \N$ che associa ad ogni elemento
di un insieme $E$ finito o numerabile, un numero, appartenente ad $\N$ indicante
il numero di occorrenze dell'elemento di $E$ nel multiinsieme $M$.\newline
La cardinalità di un multiinsieme $M$ è definita come
\begin{equation*}
\displaystyle |M| = \sum{_{e_i \in E} ^ {|E|}} M(e_i)
\end{equation*}
\begin{align*}
S = (1,2,1,3,4,4,2,3,3) \\
T = (2,1,3,1,4,4,3,2,3)
\end{align*}
Sono due multiinsiemi uguali con cardinalità 9

\subsection{Operazioni su Multiinsiemi}
\begin{description}
    \item[Intersezione]: $M_1 \cap M_2 = M_3$ dove $M_3(e) = min(M_1(e),M_2(e)) \forall e \in E$
    \item[Unione]: $M_1 \cup M_2 = M_3$ dove $M_3(e) = max(M_1(e),M_2(e)) \forall e \in E$
    \item[Unione Disgiunta]: $M_1 \uplus M_2 = M_3$ dove $M_3(e) = M_1(e) + M_2(e) \forall e \in E$
\end{description}

%Inserire Esempi!!!!

%Capitolo sugli Insiemi
%Capitolo sulle relazioni
\chapter{Relazioni}
Si definisce \textit{relazione n-aria} un sottoinsieme del prodotto cartesiano
rappresentato da tutte le coppie che rispettano la relazione voluta tra gli $n$ insiemi.
Si definisce \textit{arietà} di una relazione il numero e il tipo degli argomente
di una relazione.

%Definizione Dominio e Codominio di una relazione
\textbf{Dominio}:insieme degli elementi $x$ tali che $<x,y> \in R$ per qualsiasi $y$.
\textbf{Codominio}:insieme degli elementi $y$ tali che $<x,y> \in R$ per qualsiasi $x$.

%Definizione Relazione Complementare ed Inversa
Esempio:\newline
$A$ x $B = \{<1,1>,<1,4>,<1,5>,<2,1>,<2,4>,<2,5>,<3,1>,<3,4>,<3,5>\} $ \newline
$R \subseteq A$ x $B$\newline
R = \{<1,1>,<1,4>,<1,5>,<2,4>,<2,5>,<3,4>,<3,5>\}

$R \subseteq B$ x $A$ \newline
R = \{<1,1>,<1,2>,<1,3>\}

%proprietà delle Relazioni
Data una relazione $R$ definita su un dominio $S$ si definiscono le seguenti proprietà:

\begin{itemize}
  \item Riflessiva $\iff \forall x \in S \quad xRx$
  \item Irriflessiva $\iff \forall x \in S \quad x \not R x$
  \item simmetrica $\iff  xRy \rightarrow yRx$
  \item asimmetrica: $\iff xRy \rightarrow y \not R x$
  \item antisimmetrica: $\iff xRy \land yRx \rightarrow x = y$
  \item transitiva: $\iff xRy \land yRz \rightarrow xRz$
\end{itemize}

%Inserire Esempi

Sulle relazioni si possono applicare le usuali operazioni insiemistiche quindi, ad esempio,
date $R_1 \subseteq S x T$ e $R_2 \subseteq S x T$ anche $R_1 \cup R_2$ è una relazione su $S x T$.

%Definizione Relazione Complementare e Relazione Inversa
Data una relazione binaria $R \subseteq S x T$ definiamo \emph{relazione complementare}
$\bar{R} \subseteq S x T$ come $x \bar{R}y \iff <x,y> \not \in R$.
Per definizione si ha $\bar{\bar{R}} = R$ e $R \cup \bar{R} = S x T$.

Data una relazione binaria $R \subseteq S x T$ esiste sempre la \emph{relazione inversa}
$R^-1 = \{<y,x> | <x,y> \in R\} \subseteq T x S$.
Per definizione $(R ^ -1) ^ -1 = R$

%Proprietà relazioni
Proprietà delle relazioni Riflessive
Date due relazioni $R$ e $R'$ definite su $S$ si ha:
\begin{enumerate}
    \item se $R$ è riflessiva anche $R^-1$ è Riflessiva
    \item $R$ è riflessiva se e solo se $\bar{R}$ è riflessiva
    \item se $R$ e $R'$ sono riflessive anche $R \cup R'$ e $R \cap R'$ sono riflessive
\end{enumerate}
%mancano le dimostrazioni

Proprietà relazioni Simmetriche
Date due relazioni $R$ e $R'$ definite su $S$, si ha:
\begin{enumerate}
    \item $R$ è simmetrica se e solo se $R = R^-1$
    \item se $R$ è simmetrica anche $R^-1$ e $\bar{R}$ sono simmetriche
    \item $R$ è antisimmetrico se e solo se $R \cap R^-1 \subseteq \wp S$
    \item $R$ è asimmetrica se e solo se $R \cap R^-1 = \emptyset$
    \item se $R$ e $R'$ sono simmetriche anche $R \cup R'$ e $R \cap R'$ sono simmetriche
\end{enumerate}
%Mancano le dimostrazioni

Proprietà relazioni Transitive
Siano $R$ e $R'$ due relazioni su $S$, se $R$ e $R'$ sono transitive anche $R \cap R'$ è transitiva.
%Manca la dimostrazione

%Modalità di Rappresentazione delle Relazioni
\section{Rappresentazione di Relazioni}
Vi sono diverse modalità di rappresentazione delle relazioni,il cui metodo migliore
dipendono dall'arietà della relazione, che sono:
\begin{description}
    \item[Tabella a $n$ colonne] è una matrice a due dimensioni con righe,rappresentanti
          gli elementi, e colonne, indicanti gli insiemi; è conveniente utilizzare
          quando l'arietà della relazione è $\geq 2$.
    \item[Grafo Bipartito] è un grafo in cui si elencano gli elementi di tutti gli insiemi
         e si usano delle frecce, chiamate archi, per indicare l'associazione tra gli elementi.
         E' meglio utilizzare il grafo bipartito soltanto per le relazioni binarie.
    \item[Matrice Booleana] è una matrice $M_R$ a valori \{0,1\} composta da $n$ righe e $m$ colonn
    \item[Grafi] modalità di rappresentazione di relazioni binarie(spiegate in un paragrafo successivo)
\end{description}

\subsection{Tabelle}
Inserire esempi!!!!!!!!!!!!

\subsection{Grafo Bipartito}
Inserire esempi con risoluzione!!!!

\subsecion{Matrice Booleana}
La \emph{Matrice booleana} è una matrice $M_R$,composta da $n$ righe e $m$ colonne,
i cui elementi sono definiti come $m_{ij} = \{ ^{1 \iff <s_i,t_j \in R} _{0 \text{altrimenti}}$

Inserire esempi!!!!

Da una matrice booleana si possono determinare facilmente le proprietà
di una relazione $R$,definita su $S$, soprattutto la proprietà simmetrica e la riflessiva.

La \emph{Matrice Complementare} $M_{\bar{R}}$ è costituita dai seguenti elementi
$\bar{m}_{ij} = \{ ^ 1 \iff m_{ij} = 0 _ 0 \iff m_{ij} = 1$

La \emph{Matrice inversa} $M_{R ^-1}$ è la trasporta della matrice $M_R$.
%Da fare la dimostrazione

Date due matrici $A$ e $B$,embrambe di $n x m$ elementi, si definiscono 3 operazioni:
\begin{description}
    \item[MEET $A \sqcap B = C$]: è una matrice booleana i cui elementi sono:
    $c_{ij} = \{ ^ 1 \quad a_{ij} = 1 \vand b_{ij} = 1 _ 0 \quad a_{ij} = 0 \vor b_{ij} = 0$
    \item[JOIN $A \sqcup B = C$]: è una matrice booleana i cui elementi sono:
    $c{ij} = \{ ^ 1 \quad a_{ij} = 1 \vor b_{ij} = 1 _ 0 \quad a_{ij} = 0 \vand b_{ij} = 0$
    \item[PRODOTTO BOOLEANO $A \odot B$]: è una matrice booleana $n x p$, i cui elementi sono:
    $c_{ij} = \{1 \quad \text{se per qualche} k(1 \leq k \leq m) \text{si ha} a_{ik} = 1 \vand b_{kj} = 1
    _ 0 \quad \text{altrimenti}$
\end{description}

Inserire Esempi

\section{Grafi}
Si definisce come \emph{Grafo} $G$ una coppia $<V,E>$ in cui $V$ è l'insieme
dei \textbf{vertici} o \textbf{Nodi},indicanti gli elementi, invece $E$
 è l'insieme degli \textbf{archi}, indicanti la relazione esistentre tra i vertici del grafo.\newline
In un grafo orientato è ammesso il cappio ossia archi che escono ed entrano dallo stesso vertice.

Se in grafo tutti gli archi presentano un ordinamento, ossia si definisce una direzione
tra i 2 vertici di un grafo,si definisce \emph{Grafo orientato}
altrimenti si definisce il grafo come \emph{Grafo non orientato}.
%Inserire Esempi

Un arco che congiunge $V_i$ a $V_j$ si dice \emph{uscente} da $V_i$ ed \emph{entrante} in $V_j$.

\subsection{Nomenclatura}
In un grafo si definisce:
\begin{description}
    \item[NODO SORGENTE]: nodi in cui non si hanno archi entranti
    \item[NODO POZZO]:nodi in cui non si hanno archi uscenti
    \item[NODO ISOLATO]: nodi in cui non si hanno archi entranti ed uscenti
    \item[GRADO DI ENTRATA]:è il numero di archi entranti in un nodo
    \item[GRADO DI USCITA]: è il numero degli archi uscenti da un nodo
    \item[CAMMINO da $V_{in}$ a $V_{fin}$]:è una sequenza finita di nodi $<V_1,V_2,\dots,V_n>$
     con $V_1 = V_{in}$ e $V_n = V_{fin}$, dove ciascun nodo è collegato al successivo da un arco orientato
    \item[SEMICAMMINO da $V_{in}$ a $V_{fin}$]: è una sequenza finita di nodi
     $<V_1,V_2,\dots,V_n>$ con $V_1 = V_{in}$ e $V_n = V_{fin}$, dove ciascun nodo
     è collegato al successivo da un arco non orientato.
    \item[CONNESSO]:un grafo in cui dati due nodi $V_a$ e $V_b$, con $V_a \neq V_b$,
                    esiste un semicammino tra di essi.
    \item[CICLO]intorno un nodo $V$ è un cammino in cui $V = V_{in} = V_{fin}$
    \item[SEMICICLO]intorno un nodo $V$ è un semicammino in cui $V = V_{in} = V_{fin}$
    \item[CAPPIO]intorno ad un nodo è un cammino di lunghezza 1 in cui $V_in = V_{fin}$
\end{description}

%Inserire esempi ed esercizi
%Inserire Proprietà Grafi

\section{DAG ed Alberi}
Si definisce come \emph{DAG}(Directed Acyclic Graph), un grafo diretto senza cicli.
%Inserire esempi degli DAG

Si definisce come \emph{Albero}, un DAG connesso con un solo nodo sorgente,detto  radice,
in cui ogni nodo diverso dalla radice ha un solo nodo entrante.\newline
I nodi privi di archi entranti sono detti \emph{foglie} dell'albero

%Inserire esempi e proprietà degli Alberi
%Inserire caratteristiche e Nomenclatura degli Alberi
 %Capitolo sulle relazioni
%Capitolo sulle funzioni
\chapter{Funzioni}
Si definisce \textit{funzione $f:S \mapsto T$} una relazione $f \subseteq S \times T$
tale che $\forall x \in S$ esiste al più un $y \in T$ per cui $(x,y) \in f$.\newline
Se il $dom(f) = S$ la funzione si dice \emph{totale} altrimenti la funzione è \emph{parziale}.

%Tipologie di funzioni
\section{Tipologie di Funzioni}
Una funzione $f:S \mapsto T$ si dice:
\begin{description}
    \item[iniettiva]: $\forall x,y \in S$ con $x \neq y$ si verifica $f(x) \neq f(y)$.
    \item[suriettiva]: $\forall y \in T \exists x \in S$ tale che $f(x) = y$.
    \item[biettiva]: se la funzione è iniettiva e suriettiva
\end{description}
%Determinare se una funzione è iniettiva e suriettiva
Per determinare se una funzione è iniettiva bisogna verificare che $f(x) \neq f(y)$
comporta $x \neq y$.
Per determinare se una funzione è suriettiva bisogna risolvere l'equazione $f(x) = y$
e verificare se $y$ appartiene al codominio della funzione.

%Funzione inversa
Una funzione $f:S \mapsto T$ è detta \emph{invertibile} se la sua relazione inversa
$f ^ -1$ è essa stessa una funzione.\newline
\textbf{Condizione di invertibilità}: una funzione $f:S \mapsto T$ ammette una \emph{funzione inversa}
 $f ^{-1} :T \mapsto S$ se e solo se $f$ è una funzione iniettiva.

%Proprietà funzioni Inverse
\begin{thm}
Sia $f:A \mapsto B$ invertibile, con funzione inversa $f ^ -1$:
\end{thm}
\begin{enumerate}
    \item $f^{-1}$ è totale se e solo se $f$ è suriettiva
    \item $f$ è totale se e solo se $f^{-1}$ è suriettiva
\end{enumerate}

%Esempi
Esempi:\newline
$+:\N x \N \mapsto \N$ è una funzione totale,suriettiva ma non iniettiva \newline
$*:\N x \N \mapsto \N$ è una funzione totale,suriettiva ma non è iniettiva \newline
$successore:\N \mapsto \N$ è una funzione totale ed è iniettiva ma non suriettiva \newline
$successore:\Z \mapsto \Z$ è una funzione totale ed è biettiva \newline
$x^2:\N \mapsto \N$ è una funzione totale,iniettiva,non suriettiva,invertibile \newline
$x^2:\Z \mapsto \Z$ è una funzione totale,non iniettiva,non suriettiva,non invertibile. \newline

%Funzione Composta
Date due funzioni $f:S \mapsto T$ e $g:T \mapsto Q$ si definisce \emph{funzione composta}
$g \circ f:S \mapsto Q$ la funzione tale che $(g \circ f)(x) = g(f(x))$ per ogni $x \in S$.
La funzione composta $(g \circ f)(x)$ è definita se e solo se sono definite entrambe
$g(f(x))$ e $f(x)$.\newline
\textbf{Condizione di componibilità}: codominio della prima coincide col dominio della seconda.

%Esempi funzioni composte
Può capitare a volte di avere una funzione composta non definita in quanto non coincide
il dominio con il codominio ma la funzione risulta calcolabile;in quel caso si dice
che la funzione non è composta ma è calcolabile come ad esempio:

%Inserire esempi di funzioni composte

\begin{thm}
Siano $f:S \mapsto T$ e $g:T \mapsto Q$ invertibili. Allora $g \circ f$ è invertibile
e la sua inversa è $(g \circ f) ^{-1} = f^{-1} \circ g ^{-1}$.
\end{thm}
%Inserire dimostrazione

%Operazione
\section{Operazioni}
Si definisce come \emph{operazione n-aria} su un insieme $S$, una funzione
$f:S^n \mapsto S$ con $n \geq 1$.
Se $f$ è un'operazione binaria su $S$, essa si può rappresentare anche mediante
la notazione infissa $x_1 f x_2$ invece di $f(x_1,x_2)$

%Inserire esempi!!!!

%Definizione Funzioni Monotone
\subsection{Funzioni monotone}
\begin{defi}
    Siano $(S,\leq _S)$ e $(T,\leq _T)$ due poset e sia $f:S \mapsto T$ una funzione allora:
\end{defi}
\begin{enumerate}
    \item $f$ è detta \emph{monotona non decrescente} quando $f(x) \leq_T f(y) \iff x \leq_S y$
    \item $f$ è detta \emph{monotona crescente} quando $f(x) <_T f(y) \iff x <_S y$
    \item $f$ è detta \emph{monotona non crescente} quando $f(x) \leq_T f(y) \iff x \geq_S y$
    \item $f$ è detta \emph{monotona decrescente} quando $f(x) <_T f(y) \iff x >_S y$
\end{enumerate}
%Capitolo sulle Funzioni
%Capitolo sui Numeri ordinali
\section{Ordinali}
I numeri ordinali sono usati per indicare la posizione di un elemento di una sequenza
ordinata, al contrario dei numeri cardinali che indicano la dimensione di un insieme.

Un insieme $\alpha$ è un ordinale se sono verificate le seguenti condizioni:
\begin{enumerate}
    \item $\beta \in \alpha$ implica $\beta \subset \alpha$
    \item $\beta \in \alpha$ e $\gamma \in \alpha$ implica che sia verificata una delle seguenti:
          \begin{itemize}
              \item $\beta = \gamma \land \beta \not \in \gamma \land \gamma \not \in \beta$
              \item $\beta \neq \gamma \land \beta \in \gamma \land \gamma \not \in \beta$
              \item $\beta \neq \gamma \land \beta \not \in \gamma \land \gamma \in \beta$
          \end{itemize}
    \item $\beta \subset \alpha$ e $\beta \neq \emptyset$ implica che esiste un $\gamma \in \alpha$
          tale che $\beta \cap \gamma = \emptyset$
\end{enumerate}

Se $\alpha$ è un ordinale allora il suo successore, indicato con $succ(\alpha)$,
è il numero ordinale $succ(\alpha) = \alpha \cup \{\alpha\}$ e non ci sono altri
numeri ordinali tra $\alpha$ e $succ(\alpha)$.

Esistono degli ordinali che non hanno successori, chiamati \emph{ordinali limite}
ed il più piccolo ordinale limite è $\omega$.
Una definizione formale di ordinale limite è la seguente:
Un ordinale $\lambda$ è un ordinale limite se per ogni $\alpha < \lambda$ si ha $succ(\alpha) < \lambda$.
%Capitolo sui numeri ordinali
\section{Algebra Booleana}
Un \emph{algebra di boole} è un reticolo $(B,R)$ in cui valgono le seguenti proprietà:
\begin{enumerate}
    \item Limitato
    \item Complementato
    \item Distributivo
\end{enumerate}

L'algebra booleana si può definire anche in maniera assiomatica come segue:
\begin{defi}
    Sia $B$ un insieme, un algebra di Boole è una sestupla $(B,\cup,\cap,^{'},0,1)$
dove $\cup$ e $\cap$ sono due operazioni binarie su $B$, $^{'}$ è un operazione unaria su $B$,
$0$ e $1$ sono due elementi distinti di $B$.
\end{defi}

\begin{defi}
    In un algebra di Boole si dice \emph{duale} di un enunciato scambiando $\cup$
    con $\cap$ e $0$ con $1$
\end{defi}

\begin{thm}Nella algebra di Boole valgono le seguenti proprietà, presi qualsiasi $x,y,z \in B$:
\end{thm}
\begin{enumerate}
    \item $x \cap 0 = 0$
    \item $x \cup 1 = 1$
    \item $x \cup 0 = x$
    \item $x \cap 1 = x $
    \item $x \cap (x \cup y) = x$ (Assorbimento)
    \item $x \cup (x \cap y) = x$ (Assorbimento)
    \item $x \cup y = y \cup x $ (Commutativa)
    \item $x \cap y = y \cap x $ (Commutativa)
    \item $x \cup (y \cap z) = (x \cup y) \cap (x \cup z) $ (Distribitiva)
    \item $x \cap (y \cup z) = (x \cap y) \cup (x \cap z) $ (Distribuitiva)
    \item $x \cap (y \cap z) = (x \cap y) \cap z$ (Associatività)
    \item $x \cup (y \cup z) = (x \cup y) \cup z$ (Associatività)
    \item $(x \cap y)' = (x' \cup y')$ (Legge di De Morgan)
    \item $(x \cup y)' = x' \cap y' $ (Legge di de Morgan)
    \item $x \cap y = (x' \cup y')'$
    \item $x \cup y = (x' \cap y')'$
    \item $1' = 0$
    \item $0' = 1$
    \item $x \cup x' = 1$
    \item $x \cap x' = 0$
\end{enumerate}

%Capitolo sulla Algebra Booleana
%File per gli appunti sull'Induzione
\chapter{Induzione}
L'induzione è un importante strumento per la definizione di nuovi insiemi,come
ad esempio l'insieme delle FBF(Formule ben Formate), e la dimostrazione di determinate
proprietà di un insieme.

%Dimostrazione per induzione(Da migliorare)
\section{Principio di Induzione}
Il principio di Induzione si utilizza per dimostrare la correttezza di determinate
proprietà dell'insieme dei numeri Naturali.

Il principio di induzione viene definito nel seguente modo:\newline
Data una proposizione $P(x)$ valida per $\forall x \in N$ bisogna:
\begin{enumerate}
  \item \textbf{Caso Base}: Verificare $P(i)$ con $i$ indicante il primo elemento valido della proposizione
  \item \textbf{Passo Induttivo}: Supposto $P(x)$ vero  bisogna verificare la verità di $P(x+1)$
\end{enumerate}

%Definizione Principio di Induzione Completo
Il principio di Induzione completo è definito nel seguente modo:
\begin{defi}
Sia A(n) una asserzione per ogni elemento $n \geq i \in \N$. Supponendo che:
\begin{itemize}
    \item $A(i)$ è vera (Caso Base)
    \item $\forall m \in \N$, se $A(k)$ è vera $\forall k$,con $0 < k < m$, ne segue
          che è vera $A(m)$
\end{itemize}
Allora $\forall n \in N$,$A(n)$ è vera
\end{defi}

Esempio:\newline
\begin{thm}$\displaystyle \sum_{i = 0} ^ n i = \frac{n(n + 1)}{2}$ \end{thm}

\begin{proof}
Per $n = 0 \quad \displaystyle \sum_{i = 0} ^ 0 i = \frac{0(0 + 1)}{2} \quad 0 = 0$ vero

Se $\displaystyle \sum_{i = 0} ^ n i = \frac{n(n+1)}{2}$ allora
$\displaystyle \sum_{i = 0} ^ {n+1} i = \frac{(n+1)(n+2)}{2}$
\begin{equation*}
\begin{split}
  \sum_{i = 0}^{n+1} i & = \sum_{i = 0} ^ n + (n+1) \\
                     & = \frac{n(n+1)}{2} + (n+1) \\
                     & = \frac{n(n+1) + 2(n+1)}{2} \\
                     & = \frac{(n+1)(n+2)}{2} \\
\end{split}
\end{equation*}
\end{proof}

Esempio:\newline
\begin{thm}
 $\displaystyle \sum_{i = 1}^n 2i-1 = n^2$
\end{thm}

\begin{proof}
Per $n = 1 \quad \displaystyle \sum_{i = 1}^1 2i-1 = 1^2 \quad 1 = 1$ è vero

Se $\displaystyle \sum_{i = 1}^n 2i-1 = n^2$ allora
$\displaystyle \sum_{i = 1}^{n+1} 2i-1 = (n+1)^2$

\begin{equation*}
\begin{split}
  \sum_{i=1}^{n+1} 2i-1 & = \sum_{i=1}^n 2i-1 + 2(n+1) - 1 \\
         & = n^2 + 2n + 1 \\
         & (n+1)^2 \\
\end{split}
\end{equation*}
\end{proof}

\begin{thm}
    $\displaystyle n! \geq 2^{(n-1)} \quad \forall n \in \N$
\end{thm}
%Dimostrazione
\begin{proof}
Per $n = 0$ si ha $0! \geq 2^{0-1}$ ossia $ 1 \geq 1/2$ che è sempre verificato

Se $n! \geq 2^{(n-1)}$ allora $(n+1)! \geq 2^n$
\begin{equation*}
\begin{split}
(n+1)! \geq 2^{n+1-1} & n! (n+1) \geq 2^{n-1} * 2 \\
                      & n! \frac{(n+1)}{2} \geq 2^{n-1} \\
\frac{(n+1)}{2} \geq 0 \forall n \in \N \text{per cui lo si può togliere e si rimane a}\\
                     & n! \geq 2^{n-1} \text{verificato per ipotesi} \\
\end{split}
\end{equation*}
\end{proof}

%Definizione induttiva
\section{Definizione Induttiva}
L'induzione permette anche di definire nuovi insiemi nel seguente modo:
\begin{enumerate}
  \item si definisce un insieme di "oggetti base" appartenenti all'insieme.
  \item si definisce un insieme di operazioni di costruzione che, applicate ad elementi
        dell'insieme, producono nuovi elementi dell'insieme.
  \item nient'altro appartiene all'insieme definito.
\end{enumerate}

%Inserire Esempi
Esempio:Definizione induttiva di numeri naturali\newline
\begin{enumerate}
  \item $0 \in N$
  \item Se $x \in N$ allora $s(x) \in N$
  \item Nient'altro appartiene ai numeri naturali
\end{enumerate}

Esempio:espressione in Java
\begin{enumerate}
    \item le variabili e le costanti sono delle espressioni
    \item se $E_1$ e $E_2$ sono delle espressioni ed $op$ è un operatore binario,
          allora $E_1 op E_2$ è un espressione
    \item se $E_1$ e $E_2$ sono delle espressioni e $op$ è un operatore unario,
          allora $op E_1$ è un espressione
    \item nient'altro è un espressione
\end{enumerate}

%Ricorsione definizione
\section{Ricorsione}
La ricorsione è funzione definitoria che consiste nel definire un insieme
di elementi base e di definire gli altri elementi mediante il richiamo di se stessa,
fino ad arrivare ai casi base.

Esempio:
la definizione ricorsiva del fattoriale è definita come segue:
\begin{equation*}
    n! = \begin{cases} 1 \quad n = 0 \lor n = 1 \\ n * (n-1)! \quad n > 1\\
\end{cases}
\end{equation*}

la definizione del coefficiente binomiale è definita come segue:
\begin{equation*}
    (^n _ k) = \begin{cases} 1 \quad n = k \lor k = 0 \\
                             n \quad k = n-1 \lor k = 1 \\
                             (^{n-1} _{k-1}) + (^{n-1} _k) \quad \text{altrimenti} \\
                \end{cases}
\end{equation*}

la definizione di $somma:Z x Z \mapsto Z$ è la seguente:
\begin{equation*}
    somma(a,b) = \begin{cases} a \quad b = 0 \\
                               somma(b,a) \quad a < b \\
                               1 + somma(b-1) \quad b > 0\\
                               -1 + somma(b+1) \quad b < 0 \\
                  \end{cases}
\end{equation*}

%Stringhe
\section{Stringhe}
La definizione induttiva dell'insieme delle stringhe su un alfabeto è la seguente:
\begin{itemize}
    \item $\epsilon$ è una stringa con $\epsilon = $ stringa vuota
    \item se $X$ è una stringa e $c$ è un carattere allora $X+c$ è una stringa
\end{itemize}

Sulle stringhe possiamo definire ricorsivamente delle funzioni:
\begin{itemize}
    \item $lunghezza:STR \mapsto \N$ indicante la lunghezza delle stringhe definita come:
            \begin{equation*}
                lunghezza(x) = \begin{cases} 0 \quad \text{se} x = \epsilon \\
                                             1 + lunghezza(x') \text{se} x = x' + c \\
                                \end{cases}
            \end{equation*}
    \item
\end{itemize}
%Capitolo Induzione
%File latex per il capitolo sulla logica proposizionale Classica
\chapter{Logica Proposizionale}
%Definizione di Logica
La logica è lo studio del ragionamento e dell’argomentazione e, in particolare,
dei	procedimenti inferenziali, rivolti a chiarire quali	procedimenti di pensiero siano validi e quali no.
Vi sono molteplici tipologie di logiche, come ad esempio la logica classica e le logiche costruttive,
tutte accomunate di essere composte da 3 elementi:

%Elementi di una Logica
\begin{itemize}
  \item \textbf{Linguaggio}:insieme di simboli utilizzati nella Logica per definire le cose
  \item \textbf{Sintassi}:insieme di regole che determina quali elementi appartengono o meno al linguaggio
  \item \textbf{Semantica}:permette di dare un significato alle formule del linguaggio e determinare
        se rappresentano o meno la verità.
\end{itemize}

Noi ci occupiamo della logica Classica che si compone in \textsc{logica proposizionale} e
\textit{logica predicativa}.

La logica proposizionale è un tipo di logica classica che presenta come caratteristica quella
di essere un linguaggio limitato in quanto si possono esprimere soltanto proposizioni senza
avere la possibilità di estenderla a una classe di persone.

%Sintassi proposizionale
\section{Linguaggio e Sintassi}
Un linguaggio predicativo $L$ è composto dai seguenti insiemi di simboli:
\begin{enumerate}
    \item Insieme di variabili individuali(infiniti) $x,y,z,\dots$
    \item Connettivi logici $\land \lor \neg \rightarrow \iff$
    \item Quantificatori esistenziali $\forall \exists$
    \item Simboli ( , )
    \item Costanti proposizionali $T,F$
    \item Simbolo di uguaglianza $=$,eventualmente assente
\end{enumerate}
Questa è la parte del linguaggio tipica di ogni linguaggio del primo ordine poi
ogni linguaggio definisce la propria segnatura ossia definisce in maniera autonomo:
\begin{enumerate}
    \item Insiemi di simboli di costante $a,b,c,\dots$
    \item Simboli di funzione con arieta $f,g,h,\dots$
    \item Simboli di predicato $P,Q,Z,\dots$ con arietà
\end{enumerate}

%Inserire Esempio
Esempio:Linguaggio della teoria degli insiemi \newline
Costante:$\emptyset$\newline
Predicati:$\in(x,y)$, $=(x,y)$

Esempio:Linguaggio della teoria dei Numeri \newline
Costante:$0$ \newline
Predicati:$<(x,y)$,$=(x,y)$ \newline
Funzioni:$succ(x)$,$+(x,y)$,$*(x,y)$

%Definizione di Termini e Formule ben formate
Per definire le formule ben formate della logica Predicativa bisogna prima definire
l'insieme di termini e le formule atomiche.

\begin{defi}
    L'insieme $TERM$ dei termini è definito induttivamente come segue
    \begin{enumerate}
        \item Ogni variabile e costante sono dei Termini
        \item Se $t_1 \dots t_n$ sono dei termini e $f$ è un simbolo di funzione di arietà $n$
              allora $f(t_1,\dots,t_n)$ è un termine
    \end{enumerate}
\end{defi}

\begin{defi}
    L'insieme $ATOM$ delle formule atomiche è definito come:
    \begin{enumerate}
        \item $T$ e $F$ sono degli atomi
        \item Se $t_1$ e $t_2$ sono dei termini, allora $t_1 = t_2$ è un atomo
        \item Se $t_1,\dots,t_n$ sono dei termini e $P$ è un predicato a $n$ argomenti,
              allora $P(t_1,\dots,t_n)$ è un atomo.
    \end{enumerate}
\end{defi}

\begin{defi}
    L'insieme delle formule ben formate($FBF$) di $L$ è definito induttivamente come
    \begin{enumerate}
        \item Ogni atomo è una formula
        \item Se $A,B \in FBF$, allora $\neg A$, $A \land B$,$A \lor B$,$A \rightarrow B$
              e $A \iff B$ appartengono alle formule ben formate
        \item Se $A \in FBF$ e $x$ è una variabile, allora $\forall x A$ e $\exists x A$
              appartengono alle formule ben formate
        \item Nient'altro è una formula
    \end{enumerate}
\end{defi}

%Inserire Esempi

%Precedenza degli Operatori
La precedenza tra gli operatori logici è definita nella logica predicativa come segue
$\forall,\exists,\neg,\land,\lor,\rightarrow,\iff$ e si assume che gli operatori associno a destra.

%Inserire Esempi

%Variabili legate e chiuse
\begin{defi}
    L'insieme $var(t)$ delle variabili di un termine $t$ è definito come segue:
    \begin{itemize}
        \item $var(t) = \{t \}$, se $t$ è una variabile
        \item $var(t) = \emptyset$ se $t$ è una costante
        \item $var(f(t_1,\dots,t_n)) = \bigcup _{i = 1} ^n var(t_i)$
        \item $var(R(t_1,\dots,t_n)) = \bigcup _{i = 1} ^ n var(t_i)$
    \end{itemize}
\end{defi}

%Termini chiusi ed aperti
Si definisce \emph{Termine chiuso}, un termine che non contiene variabili altrimenti
si definisce il termine come \emph{chiuso}.\newline
Le variabili nei termini e nelle formule atomiche possono essere libere
 in quanto gli unici operatori che "legano" le variabili sono i quantificatori.

Il campo di azione dei quantificatori si riferisce soltanto alla parte in cui
si applica il quantificatore per cui una variabile si dice \emph{libera}
se non ricade nel campo di azione di un quantificatore altrimenti la variabile si dice \emph{vincolata}.


%Semantica Logica Proposizionale
\section{Semantica}
La semantica di una logica consente di dare un significato e un interpretazione
 alle formule del Linguaggio.\newline

\begin{defi}
Sia data una formula proposizionale $P$ e sia ${P_1,\dots,P_n}$, l'insieme degli
atomi che compaiono nella formula $A$.Si definisce come \emph{interpretazione} una
funzione $v:\{P_1,\dots,P_n\} \mapsto \{T,F\}$ che attribuisce un valore di verità
a ciascun atomo della formula $A$.
\end{defi}

I connettivi della Logica Proposizionale hanno i seguenti valori di verità:
%Tabella di Verità degli operatori
$\begin{array}{ccccccc}
\toprule
\text{A} & \text{B} & A \land B & A \lor B & \neg A & A \Rightarrow B & A \iff B \\
\midrule
    F & F & F & F & T & T & T \\
    F & T & F & T & T & T & F \\
    T & F & F & T & F & F & F \\
    T & T & T & T & F & T & T \\
\bottomrule
\end{array}$\newline

Essendo ogni formula $A$ definita mediante un unico albero sintattico, l'interpretazione $v$
è ben definito e ciò comporta che data una formula $A$ e un interpretazione $v$,
eseguita la definizione induttiva dei valori di verità, si ottiene un unico $v(A)$.

Una formula nella logica proposizionale può essere di tre diversi tipi:
%Tipologie di formule
\begin{description}
    \item[valida o tautologica]: la formula è soddisfatta da qualsiasi valutazione della Formula
    \item[Soddisfacibile non Tautologica]:la formula è soddisfatta da qualche valutazione
                        della formula ma non da tutte.
    \item[falsibicabile]:la formula non è soddisfatta da qualche valutazione della formula.
    \item[Contraddizione]:la formula non viene mai soddisfatta
\end{description}

\begin{thm}
$A$ è una formula valida se e solo se $\neg A$ è insoddisfacibile.
$A$ è soddisfacibile se e solo se $\neg A$ è falsibicabile
\end{thm}

%Fare la dimostrazione

Esempio:\newline
Formula $A \land \neg A$ \quad contraddizione

%Tabella di Verità
$\begin{array}{ccc}
\toprule A & \neg A & A \land \neg A \\
\midrule
        0 & 1 & 0 \\
        1 & 0 & 0 \\
\bottomrule
\end{array}$\newpage

Formula $Z = (A \land B) \lor C$  soddisfacibile non Tautologica

%Tabella di Verità
$\begin{array}{ccccc}
\toprule A & B & C & A \land B & (A \land B) \lor C \\
\midrule
         0 & 0 & 0 & 0 & 0 \\
         0 & 0 & 1 & 0 & 1 \\
         0 & 1 & 0 & 0 & 0 \\
         0 & 1 & 1 & 0 & 1 \\
         1 & 0 & 0 & 0 & 0 \\
         1 & 0 & 1 & 0 & 1 \\
         1 & 1 & 0 & 1 & 1 \\
         1 & 1 & 1 & 1 & 1 \\
\bottomrule
\end{array}$\newline

Formula $X = (A \land B) \Rightarrow (\neg A \land C)$ \quad Soddisfacibile  non tautologica\newline

%Tabella di Verità della formula X
$\begin{array}{ccccccc}
\toprule A & B & C & \neg A & A \land B & \neg A \land C & X\\
\midrule
         0 & 0 & 0 & 1 & 0 & 0 & 1 \\
         0 & 0 & 1 & 1 & 0 & 1 & 1 \\
         0 & 1 & 0 & 1 & 0 & 0 & 1 \\
         0 & 1 & 1 & 1 & 0 & 1 & 1 \\
         1 & 0 & 0 & 0 & 0 & 0 & 1 \\
         1 & 0 & 1 & 0 & 0 & 0 & 1 \\
         1 & 1 & 0 & 0 & 1 & 0 & 0 \\
         1 & 1 & 1 & 0 & 1 & 0 & 0 \\
\bottomrule
\end{array}$ \newline

Formula $Y = \neg(A \land B) \iff (A \lor B \Rightarrow C)$ soddisfacibile non Tautologica

%Tabella di Verità
$\begin{array}{cccccccc}
\toprule
A & B & C & A \land B & \neg(A \land B) & A \lor B & (A \lor B) \Rightarrow C & Y \\
\midrule
0 & 0 & 0 & 0 & 1 & 0 & 1 & 1 \\
0 & 0 & 1 & 0 & 1 & 0 & 1 & 1 \\
0 & 1 & 0 & 0 & 1 & 1 & 0 & 0 \\
0 & 1 & 1 & 0 & 1 & 1 & 1 & 1 \\
1 & 0 & 0 & 0 & 1 & 1 & 0 & 0 \\
1 & 0 & 1 & 0 & 1 & 1 & 1 & 1 \\
1 & 1 & 0 & 1 & 0 & 1 & 0 & 1 \\
1 & 1 & 1 & 1 & 0 & 1 & 1 & 0 \\
\bottomrule
\end{array}$

\subsection{Modelli e decidibilità}
Si definisce \emph{modello}, indicato con $M \models A$, tutte le valutazioni booleane
che rendono vera la formula $A$.
Si definisce \emph{contromodello}, indicato con , tutte le valutazioni booleane
che rendono falsa la formula $A$.

La logica proposizionale è decidibile (posso sempre verificare il significato di una formula).
Esiste infatti una procedura effettiva che stabilisce la validità o no di una formula, o se questa
ad esempio è una tautologia.
In particolare il verificare se una proposizione è tautologica o meno è l’operazione di decibi-
lità principale che si svolge nel calcolo proposizonale.

\begin{defi}
    Se $M \models A$ per tutti gli $M$, allora $A$ è una tautologia e si indica $\models A$
\end{defi}

\begin{defi}
    Se $M \models A$ per qualche $M$, allora $A$ è soddisfacibile
\end{defi}

\begin{defi}
Se $M \models A$ non è soddisfatta da nessun $M$, allora $A$ è insoddisfacibile
\end{defi}


%Sistema Deduttivo
\section{Sistema Deduttivo}
Il sistema deduttivo è un metodo di calcolo che manipola proposizioni, senza la
necessità di ricorrere ad altri aspetti della logica.\newline
Nella logica proposizionale, tramite i teoremi di completezza e correttezza, esiste
una corrispondenza tra le formule derivanti dal sistema deduttivo e le formule verificabili
tramite la semantica della logica.

I sistemi deduttivi della logica proposizionale sono i seguenti:
\begin{description}
    \item[Sistema deduttivo Hilbertiano]: non viene analizzato
    \item[Metodo dei Tableaux]
    \item [Risoluzione Proposizionale]:non viene analizzato !!!!
\end{description}

\begin{defi}
Una sequenza di formule $A_1,\dots,A_n$ di $\Lambda$ è una \emph{dimostrazione} se
per ogni $i$ compreso tra $1$ e $n$, $A_i$ è un assioma di $\Lambda$ oppure una
conseguenza diretta di una formula precedente.
\end{defi}

\begin{defi}
Una formula $A$ di una logica $\Lambda$ è detta \emph{teorema} di $\Lambda$,indicata
con $\vdash A$ se esiste una dimostrazione di $\Lambda$ che ha $A$ come ultima formula
\end{defi}

Una dimostrazione di una formula di una logica può venire tramite:
\begin{itemize}
  \item  \textbf{Metodo diretto}: Data un'ipotesi, attraverso una serie di passi
          si riesce a dimostrare la correttezza della Tesi
  \item \textbf{Metodo per assurdo}(non sempre accettato in tutte le logiche):
        Si nega la tesi ed attraverso una serie di passi si riesce a dimostrare
        la negazione delle ipotesi.
\end{itemize}

\begin{thm}
    Un apparato deduttivo $R$ è completo se, per ogni formula $A \in Fbf$, $\vdash A$
    implica $\models A$
\end{thm}

\begin{thm}
    Un apparato deduttivo $R$ è corretto se, per ogni formula $A \in Fbf$, $\models A$
    implica $\vdash A$
\end{thm}
\subsection{Tableau Proposizionali}
Il metodo dei Tableau è stato introdotto da Hintikka e Beth alla fine degli anni '50
e poi ripresi successivamente da Smullyan.
Per poter comprendere e capire i Tableaux dobbiamo introdurre una serie di definizioni:
\begin{defi}
Per ogni formula $A$, $\{A,\neg A\}$ è una coppia di formule complementari in cui
$A$ è il complemento di $\neg A$
\end{defi}

\begin{defi}
Un letterale è un atomo o la sua negazione.Se $p$ è un atomo allora $\{p,\neg p\}$
è una coppia di letterali complementari.
\end{defi}

I tableau sono degli alberi,la cui radice è l'enunciato in esame, e gli altri nodi
sono costruiti attraverso l'applicazione di una serie di regole,fino ad arrivare
alle formule atomiche come radici.

I tableaux proposizionali si dividono in due tipologie di formule(e quindi regole):
le $\alpha$ regole e le $\beta$ regole;le $\alpha$ formule sono di tipo congiuntivo
ed è soddisfatta se e soltanto se le sottoformule $\alpha_1$ e $\alpha_2$ sono entrambe soddisfatte
mentre le $\beta$ formule sono di tipo disgiuntivo e sono soddisfatte se e soltanto
se almeno una delle due sottoformule $\beta_1$ e $\beta_2$ è soddisfatta.

Le regole dei Tableau sono le seguenti:
%T AND
\begin{equation*}
%T AND
\begin{prooftree}
\hypo{S,T (A \land B)}
\infer1 {S,TA,TB}
\end{prooftree}
\quad T \land \qquad
%F AND
\begin{prooftree}
\hypo{S,F (A \land B)}
\infer1 {S,FA/S,FB}
\end{prooftree}
F \ \land
\end{equation*}

\begin{equation*}
%T OR
\begin{prooftree}
\hypo{S,T (A \lor B)}
\infer1 {S,TA / S,TB}
\end{prooftree}
\quad T \lor \qquad
%F OR
\begin{prooftree}
\hypo{S,F (A \lor B)}
\infer1 {S,FA,FB}
\end{prooftree}
F \ \lor
\end{equation*}

\begin{equation*}
%T NOT
\begin{prooftree}
\hypo{S,T (\neg A)}
\infer1 {S,FA}
\end{prooftree}
\quad T \neg \qquad
%F NOT
\begin{prooftree}
\hypo{S,F (\neg A)}
\infer1 {S,TA}
\end{prooftree}
F \ \neg
\end{equation*}

\begin{equation*}
%T ->
\begin{prooftree}
\hypo{S,T (A \rightarrow B)}
\infer1 {S,FA / S,TB}
\end{prooftree}
\quad T \rightarrow \qquad
%F ->
\begin{prooftree}
\hypo{S,F (A \rightarrow B)}
\infer1 {S,TA,FB}
\end{prooftree}
F \ \rightarrow
\end{equation*}

\begin{equation*}
%T <-->
\begin{prooftree}
\hypo{S,T (A \iff B)}
\infer1 {S,TA,TB/S,FA,FB}
\end{prooftree}
\quad T \iff \qquad
%F <-->
\begin{prooftree}
\hypo{S,F (A \iff B)}
\infer1{S,TA,FB/S,FA,TB}
\end{prooftree}
F \ \iff
\end{equation*}

Le $\beta$ regole sono quelle che creano due sottoformule indicate nella regola con $/$
mentre, per esclusione, le $\alpha$ regole sono quelle in cui si crea soltanto una sottoformula.

%Definizione induttiva di costruzione di un tableaux
Il tableaux di una formula $A$ inizialmente è composto da un solo nodo, la radice, etichettata
dalla formula $A$.
Il tableaux si costruisce induttivamente come segue:
si sceglie una foglia $l$ non etichettata dell'albero che verrà etichettata da un
insieme di formule $U(l)$ definite come:
\begin{enumerate}
  \item Se $U(l)$ è un insieme di letterali, si controlla se sono presenti una coppia
        di letterali complementati in $U(l)$;
        in caso siano presenti si marca la foglia come chiusa $\times$ altrimenti la foglia è aperta
  \item Se $U(l)$ non è un insieme di letterali si sceglie una formula in $U(l)$ tramite:
        \begin{itemize}
          \item Se la formula è un $\alpha$-formula $B$ si crea un nuovo nodo $l'$,
                figlio di $l$, e lo si etichetta come $U(l') = (U(l) - \{B\}) \cup \{\alpha_1,alpha_2\}$
          \item Se la formula è un $\beta$-formula $C$ si creano due nodi $l'$ e $l''$,
                figli di $l$ con $l'$ etichettato come: $U(l') = (U(l) - \{C\}) \cup \{\beta_1\}$
                mentre $l''$ è etichettata come $U(l'') = (U(l) - \{C\}) \cup \{\beta_2\}$
        \end{itemize}
\end{enumerate}
Il tableaux termina quando tutti i rami sono etichettati come chiusi e/o aperti.

%Definizione di Tableaux completo
\begin{defi}
Si definisce un tableaux \emph{completo} se la sua costruzione è complementata.
Il tableaux si dice \emph{chiuso} se tutte le foglie sono segnate come chiuse
altrimenti il tableaux è \emph{aperto}
\end{defi}

%Dimostrazione di tableau
Il metodo dei Tableau è un metodo dei sistemi deduttivi, che permette attraverso
l'applicazione di una serie di regole, di capire la tipologia della formula.
%Metodi per capire il tipo della Formula
\begin{tabular}{cccc}
\toprule Tipologia & Fare Tableau per & Chiuso? & Aperto? \\
\midrule
         Teorema & $\neg A$ & Si & No \\
         Soddisfacibile & $A$ & No & Si \\
         Contradditoria & $A$ & Si & No \\
\bottomrule
\end{tabular}

Esempio:$C \rightarrow (P \rightarrow ((Q \rightarrow \neg P) \lor (C \rightarrow P)))$
\begin{equation*}
\begin{prooftree}
\hypo{F C \rightarrow (P \rightarrow ((Q \rightarrow \neg P) \lor (C \rightarrow P)))}
\infer1 {TC,F (P \rightarrow ((Q \rightarrow \neg P) \lor (C \rightarrow P)))}
\infer1 {TC,TP,F (Q \rightarrow \neg P) \lor (C \rightarrow P)}
\infer1 {TC,TP,F (Q \rightarrow \neg P),F (C \rightarrow P)}
\infer1{TC,TP,F (Q \rightarrow \neg P),TC,FP}
\end{prooftree}
\end{equation*}
Il tableaux chiude in quanto non può essere contemporaneamente $TP$ e $FP$ per cui
la formula è una tautologia.

Esempio:
%Esempio Tableaux Predicativo
Formula: $(\forall x F(x) \lor \exists G(x)) \rightarrow (\exists x (F(x) \lor G(x)))$
\begin{proof}
\begin{equation*}
\begin{prooftree}
\hypo{F \forall x F(x) \lor \exists G(x) \rightarrow \exists x (F(x) \lor G(x))}
\infer1{T \forall x F(x) \lor \exists G(x),F \exists x (F(x) \lor G(x))}
\infer1{T \forall x F(x),F \exists x (F(x) \lor G(x))/T \exists G(x),F \exists x (F(x) \lor G(x))}
\infer1{T F(a),F \exists x (F(x) \lor G(x)),T \forall x \dots/T G(a),F \exists x (F(x) \lor G(x))}
\infer1{T F(a),F F(a) \lor G(a),T \forall \dots,F \exists \dots/T G(a),F F(a) \lor G(a),F \exists \dots}
\infer1{T F(a),F F(a),F G(a),T \forall \dots,F \exists \dots/T G(a),F F(a),F G(a),F \exists \dots}
\end{prooftree}
\end{equation*}
Il tableaux contraddizione chiude per cui la formula è una tautologia.
\end{proof}


%Completezza dei Tableaux
%Dimostrazione completezza Tableaux

%Dimostrazione correttezza Tableaux


%Equivalenze Logiche
\section{Equivalenze Logiche}
Due proposizioni $A$ e $B$ sono logicamente equivalenti se e solo se A e B hanno
la stessa valutazione booleana.

Nella logica proposizionale sono definite le seguenti equivalenze logiche, indicate con $\equiv$,:
\begin{enumerate}
    \item Idempotenza
            \begin{align*}
                A \lor B & \equiv & A \\
                A \land B & \equiv & A \\
            \end{align*}
    \item Associavità
            \begin{align*}
                A \lor (B \lor C) & \equiv & (A \lor B) \lor C \\
                A \land (B \land C) & \equiv & (A \land B) \land C \\
                A \iff (B \iff C) & \equiv & (A \iff B) \iff C \\
            \end{align*}
    \item Commutatività
            \begin{align*}
                A \lor B & \equiv & B \lor A \\
                A \land B & \equiv & B \land A \\
                A \iff B & \equiv B \iff A \\
            \end{align*}
    \item Distribuitività
            \begin{align*}
                A \lor (B \land C) & \equiv & (A \lor B) \land (A \lor C)\\
                A \land (B \lor C) & \equiv & (A \land B \lor (A \land C) \\
            \end{align*}
    \item Assorbimento
            \begin{align*}
                A \lor (A \land B) & \equiv & A \\
                A \land (A \lor B) & \equiv & A \\
            \end{align*}
    \item Doppia negazione
                \begin{equation}
                    \neg \neg A \equiv A
                \end{equation}
    \item Leggi di De Morgan
            \begin{align*}
                \neg (A \lor B) & \equiv & \neg A \land \neg B \\
                \neg(A \land B) & \equiv & \neg A \lor \neg B \\
            \end{align*}
    \item Terzo escluso
            \begin{equation}
                A \lor \neg A \equiv T
            \end{equation}
    \item Contrapposizione
            \begin{equation}
                A \rightarrow B \equiv \neg B \rightarrow \neg A
            \end{equation}
    \item Contraddizione
            \begin{equation}
                A \land \neg A \equiv F
            \end{equation}
\end{enumerate}


%Completezza dei Connettivi
\section{Completezza di insiemi di Connettivi}
Un insieme di connettivi logici è completo se mediante i suoi connettivi si può
esprimere un qualunque altro connettivo.
Nella logica proposizionale valgono anche le seguenti equivalenze, utili per ridurre il linguaggio,:
\begin{align*}
    (A \rightarrow B) & \equiv & (\neg A \lor B) \\
    (A \lor B) & \equiv & \neg(\neg A \land \neg B) \\
    (A \land B) & \equiv & \neg(\neg A \lor \neg B) \\
    (A \iff B) & \equiv & (A \rightarrow B) \land (B \rightarrow A) \\
\end{align*}

L'insieme dei connettivi $\{ \neg,\lor,\land \}$, $\{ \neg,\land \}$ e $\{ \neg,\lor \}$ sono completi
e ciò è facilmente dimostrabile utilizzando le seguenti equivalenze logiche
%Capitolo Logica Proposizionale
%Capitolo sulla Logica Predicativa
\chapter{Logica Predicativa}

\section{Semantica}
Il connettivo $A \iff B$ equivale per definizione ad $(A \rightarrow B) \land (B \rightarrow A)$ per cui quando si usa il $\iff$ si utilizza l'equivalenza
definita
Tableaux di $F A \iff B$ utilizzare la definizione data 

Struttura di Interpretazione(Semantica) %Cercare meglio la definizione!!!!!!!!!
$S = <D,I> $
D(dominio) è un insieme finito/infinito di costanti
I(interpretazione) è una funzione che associa a simboli e formule del linguaggio un significato a partire dai simboli primitivi del linguaggio
(costanti,predicati e funzioni) cioè l'interpretazione varierà a seconda della signatura del linguaggio

Si può interpretare questa struttura come una funzione che associa ad ogni costante una costante del Linguaggio 
$C_i \mapsto C_d : C_d \in D$

LogicA Aritmetica di Peano
Linguaggio: $0,s(x),+,*,=$ è la segnatura

Dominio: tutti e numeri Naturali
L'unica costante del linguaggio $0$ viene associato la costante del Dominio $0$. Può sembrare strano ma sono dei numeri diversi in quanto 
$0 \in Linguaggio \mapsto 0 \in N$ ossia si associa ad un elemento del linguaggio una sua interpretazione


In Logica Proposizionale abbiamo sviluppato indipendente la parte sintattica e la parte semantica e con il Teorema di correttezza li abbiamo messi assiemi
Anche a livello predicativo è possibile definibile in maniera chiara la semantica ed è definito il teorema di Completezza

In logica chiamiamo Numerale la costante del linguaggio e Numero la sua interpretazione

Ho un simbolo di funzione con n argomenti $f ^ n _L(t1...t_n) \mapsto I(t_k)$ quando $t_k = f(t_1....t_n)$ ossia associa ad ogni elemento della funzione una sua interpretazione

$I(P(t_1....t_n) = <t_1....t_n> $ dove $I(P) \mapsto R ^n$ P è l'insieme delle n-uple che stanno nella relazione definita

Il successore viene interpretato come la funzione successore

Esempio:
$s s(0) \mapsto 2 \in N$ in quanto 0 lo interpreto come 0, il successore di 0 come 1 e il successore di 1 è 2 e questo due è l'interpretazione di s(s(0))

Esempio Interpretazione relazione
Il Predicato = viene interpretato come l'insieme delle coppie che hanno i numeri uguali
$= \mapsto <0,0>,<1,1>,<2,2>,......<n,n>$

Esempio: 
x + y = z    + è una funzione del linguaggio
Fissato il $s(0) e s(s(0))$
$s(0) + s(s(0)) \mapsto 3 \in N$ è l'interpretazione di s(0) e l'interpretazione del s(s(0))

Come si interpretano le formule contenenti delle variabili libere?
$\eta$ che associa alle variabili libere un'interpretazione  allora grazie a questa funzione $\eta$ possiamo valutare la formula come vera o falsa

Es:
$x \mapsto s(0)$
$y \mapsto 0$
$z \mapsto s(0)$
Allora x + y = z con questi valori è vera

Formula aperta ha senso chiedersi 
Si dice che una formula aperta    se esiste un assegnamento alle variabile che la rende vera
Si dice che una formula aperta se non esiste un assegnamento alle variabile che la rende vera

Data una struttura $S $ e $\eta |-- P(t1...t_n)$ si dice che è vera se esiste un interpretazione di P(t1...tn) che la rende vera

Se una formula è chiusa allora non necessita della funzione $\eta$ in quanto l'interpretazione è unica

$S,\eta |-- A \land B$ è vera se $S,\eta |-- A \land S,\eta |-- B$

$S,\eta |-- \exists x A(x) \iff S,\eta |-- A(a) con a \in D$  S contiene il Dominio

$S,\eta |-- \forall x A(x) \iff S,\eta |-- A(a) \forall a \in D$

Dirò che una formula è valida se è sempre valide qualsiasi struttura

Tutte le formule dimostrate con il metodo dei Tableau devono essere valide

Modello è un interpretazione che rende vere le formule di una teoria
Tutte le formule del primo Ordine dimostrabili sono sempre valide per il teorema di Completezza
Teoria è fatta da un insieme di assiomi e da una logica $T = Ax + L$
Ad esempio prendendo la teoria dell'Aritmetica di Peano gli diamo una serie di assiomi e una logica 
i cui modelli sono tutte le formule che soddisfano gli assiomi e la logica

Poi si può modificare la logica e in quel caso mescolando assiomi con logiche diverse otteniamo Teorie diverse

Sono degli assiomi dell'Aritmetica di Peano (da sistemare)
$\forall x \neg S(x) = 0
\forall x x + 0 = x
\forall x,y x + s(y) = s(x + y)
\forall x x * 0 = 0
\forall x,y x * s(y) = (x * y) + x 
\forall x,y (s(x) = s(y)) \rightarrow x = y
P(0)
\neg D(0)
\forall x (P(x) \rightarrow \neg P(s(x))) P indica i numeri Pari
\forall x (D(x) \rightarrow \neg D(S(x))) D indica i numeri Dispari $

%Capitolo Logica Predicativa
\end{document}
