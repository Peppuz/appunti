%Appunti del Corso Metodi Algebrici per l'Informatica
\documentclass[a4paper]{report}
\usepackage[T1]{fontenc}%per rappresentare i font italiani, come le lettere accentate, con la giusta spaziatura
\usepackage[utf8]{inputenc}%per poter inserire nel testo .tex i caratteri unicode8
\usepackage[italian]{babel}%per poter effettuare la giusta sillabazione della lingua italiana
\usepackage{amsmath}%per poter rappresentare ed utilizzare al meglio gli ambienti e le formule matematiche
\usepackage{amssymb}%per rappresentare alcuni simboli particolari matematici
\usepackage{amsthm}%per definire e poter effettuare le dimostrazioni matematiche
%\usepackage{amsfont}%per poter avere i font matematici
\usepackage{booktabs}%per la corretta gestione delle tabelle
\usepackage{graphics}%per effettuare i grafici
\usepackage{pgfplots}%per i grafici
\usepackage{rotating}%per effettuare le rotazioni delle immagini e grafici
\usepackage{microtype}%per effettuare un aggiustamento della spaziatura tra caratteri e del font
\usepackage{url}%per poter rappresentare gli url nel testo latex
%\usepackage{hypertext}%per effettuare un collegamento con una pagina internet
\newtheorem{defi}{Definizione}%Definizione per avere la gestione delle definizioni
\newtheorem{prop}{Proposizione}[chapter]
\newtheorem{lem}{Lemma}
\newtheorem{teorema}{Teorema}[chapter]
\newtheorem{oss}{Oss}[chapter]
\newtheorem{esempio}{Esempio}
\newcommand{\numberset}{\mathbb}
\newcommand{\N}{\numberset{N}}
\newcommand{\Z}{\numberset{Z}}
\newcommand{\Q}{\numberset{Q}}
\newcommand{\R}{\numberset{R}}
\newcommand{\C}{\numberset{C}}

\begin{document}
\title{Appunti del corso Metodi Algebrici per l'Informatica}
\author{Marco Natali}
\date{}
\maketitle

\chapter{Ripasso principio di Induzione e Insieme delle Parti}



\chapter{Divisione con il resto e l'algoritmo di Euclide}
In questo capitolo consideremo la divisione con il resto e l'algoritmo di euclide, importantissimo per calcolare il MCD

\begin{thm}
  Sia $n,m \in \Z$, con $m \neq 0$, allora esistono e sono unici due interi $q$ e $r$ che soddisfano due condizioni:
  \begin{itemize}
        \item $n = mq + r$
        \item $0 \leq r < |m|$
  \end{itemize}
  Gli interi $q$ e $r$ si dicono rispettivamente quoziente e il resto della divisione.
\end{thm}

%Dimostrazione


Se non avessimo definito la seconda condizione in inserireRiferimento non si avrebbe l'unicita dato che preso un qualunque $q$
si avrebbe $r = n - mq$.

\begin{esempio}
  Dato $n = 43$ e $m = 2$ si ha $43 = 21 * 2 + 1$ per cui si ha $q = 21$ e $r = 1$
\end{esempio}

Introduciamo adesso il massimo comun divisore, concetto basilare che verrà utilizzato moltissimo in questo corso per definire altri concetti
per cui la sua definizione è la seguente:
\begin{defi}
  Siano $a,b \in \Z$, diciamo che $b$ divide $a$, indicato con $b | a$, se $a$ è un multiplo intero di $b$, ossia esiste un $c \in \Z$
  tale che $a = bc$
\end{defi}

%Esempi

\begin{oss}
  Se $b | a$ allora $-b | a$
\end{oss}

%Dimostrazione

\section{Massimo Comun Divisore e Algoritmo di Euclide}
Il massimo comun divisore, concetto già visto sin dalle medie, viene definito come:
\begin{defi}
  Siano $a,b \in \Z$, con $a \neq 0, b \neq 0$, si dice che $d$ è il massimo comun divisore tra $a$ e $b$ se:
  \begin{itemize}
  \item $d | a$ e $d | b$
  \item se $c \in Z$ con $c | a$ e $c | b$, allora $c | d$
\end{defi}

Per trovare il MCD, indicato con (a, b) ci sono due modi:
\begin{description}
\item [fattorizzazione]: si scompongono in fattori primi i due numeri $a$ e $b$ e si trovano i fattori comuni, che sono quindi il MCD
  ma questo metodo è molto oneroso dal punto di vista computazionale
\item [algoritmo di euclide]: metodo molto comodo ed implementabile in maniera ottimale su un calcolatore ed è il modo che vediamo
  ed utilizzeremo in questo corso.
\end{description}

\begin{thm}
  Siano $a,b \in \Z$, con $a > 0$ e $b > 0$, allora esiste un massimo comun divisore tra $a$ e $b$.\newline
  Inoltre esistono $s,t \in \Z$ tali che $d = as + bt$(identità di Besout).
\end{thm}

%Dimostrazione


\begin{esempio}
  Dati $a = 654$ e $b = 213$ calcolare $(654,213)$ e mostrare l'equazione di Besout:
  \begin{align}
    654 = 3 * 213 + 15
    213 = 14 * 15 + 3
    15  = 5 * 3 + 0
  \end{align}
  L'M.C.D tra $654$ e $213$ è $3$
\end{esempio}










  \chapter{Numeri in base b}
  In questo capitolo definiamo come sono rappresentati i numeri in una base qualsiasi $b$ e come poter effettuare
  le conversioni da una base ad un'altra.

  Definiamo i numeri in base b attraverso il seguente teorema:
  \begin{thm}
    Sia $b \in \Z$ con $b \geq 2$, ogni numero intero $n \geq 0$ può essere scritto in uno e un solo modo nella forma
    \begin{equation*}
      n = d_kb^k + d_{k-1}b^{k-1} + \dots + d_1b + d_0
    \end{equation*}
    con $0 \leq d_i < b$ per ogni $i = 0 \dots k$, con $d_k \neq 0$ e $k > 0$.
  \end{thm}

  %Dimostrazione

  \begin{esempio}
    Il numero $213$ in base $2$ viene rappresentato come:
    213 = 1 * 2^7 + 1 * 2^6 + 1 * 2^4 + 1 * 2^2 + 1 * 2^1
  \end{esempio}
              
  Data una base $b$ per effettuare la conversione del numero in base $10$ si possono effettuare i seguenti metodi:
  \begin{itemize}
  \item si sfrutta la definizione di numero in base $b$ e si effettua la somma delle $k$ moltiplicazioni usate per definire il numero
    ossia $(12)_5 = 1 * 5^1 + 2 * 1 = (7)_{10}$
  \item si imposta la conversione nel seguente modo $n = (\dots((d_kb + d_{k-1})b + d_{k-2})b + \dots + d_1)b + d_0$ e questo metodo comporta
    l'esecuzione di solo $k$ moltiplicazioni per $b$ e $k$ addizioni.
  \end{itemize}

  \begin{esempio}

  \end{esempio}

  Per passare dalla base $10$ alla base $b$ si osserva che $d_0,d_1,\dots,d_k$ sono i resti delle divisioni
  \begin{align}
    n = bq + d_0, \quad 0 \leq d_0 < b \\
    q = q_1b + d_1 \quad 0 \leq d_1 < b \\
    q_1 = q_2b + d_2 \quad 0 \leq d_2 < b \\
  \end{align}
  e così via finchè non si ottiene quoziente nullo.

  \begin{esempio}


  \end{esempio}

  \begin{oss}
    Il numero di cifre in base b di un intero non negativo $n$ è
    \begin{equation*}
      k + 1 =  \log _b n + 1 = \frac{\ln n}{\log b} + 1
    \end{equation*}
    dato che $b^k \leq n < b^{k+1}$.












\end{document}
