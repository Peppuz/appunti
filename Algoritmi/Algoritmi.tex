\documentclass[a4paper]{report}
\usepackage[T1]{fontenc} %Gestione font in output
\usepackage[utf8]{inputenc} %Permette di inserire caratteri Italiani
\usepackage[italian]{babel} %Gestione sillabazione ed altro italiana
\usepackage{microtype}%Pacchetto per migliorare la gestione del riempimento della pagina
\usepackage{amsmath} %Package gestione Matematica
\usepackage{amsthm}%Package per le dimostrazioni
\usepackage{listings}%Package per introdurre i listati di codice
\usepackage{clrscode3e}%Package Algoritmi stile Cormen
\usepackage{booktabs}%Gestione Tabelle(Guardare Documentazione)
\usepackage{caption}%Package per le Tabelle(guardare Documentazione)
\usepackage{forest}%Package per disegnare Alberi
\usepackage{tkz-graph}%Package per disegnare Grafi
\newtheorem{thm}{Teorema}%Modalità per rappresentare i teoremi
\setlength{\parindent}{0pt}%Evita il rientro dei paragrafi e capoversi

\begin{document}
%Titolo autore e spiegazione
\title{Appunti di Algoritmi e Strutture Dati}
\author{Marco Natali}
\maketitle

\chapter{Introduzione agli Algoritmi}
Il termine Algoritmo proviene da Mohammed ibn-Musa al-Khwarizmi, matematico uzbeco
del IX secolo a.c. da cui proviene la moderna Algebra. \newline%Cercare maggiori informazioni
\textbf{Algoritmo}:sequenza di passi che portano alla risoluzione di un problema

%Da definire come subsection????
\section{Pseudocodice}
Lo pseudocodice è un linguaggio, ispirato ai linguaggi di programmazione, utilizzato
per rappresentare e presentare gli algoritmi in maniera compatta e chiara;
ogni libro e programmatore definisce la propria specifica di Pseudocodice ma comunque
quasi tutti si ispirano alla sintassi del Pascal,C oppure di Java;i costrutti definiti
nel mio pseudocodice sono i seguenti:

\begin{tabular}{lcr}
  \toprule Istruzione & Significato\\
  \midrule
  $x \gets 5$ & assegna ad $x$ il valore $5$\\
  \textbf{integer} nome & definisce una variabile nome di tipo intero\\
  \textbf{real} nome & definisce una variabile nome di tipo reale\\
  \textbf{boolean} nome & definisce una variabile nome di tipo booleano\\
  \textbf{if} condizione & definisce la struttura if\\
  \textbf{For} estrInf \textbf{To} estrSup & definisce la struttura For da estrInf a estrSup \\
  \textbf{For} estrSup \textbf{Downto} estrInf & definisce la struttura For da estrSup a estrInf\\
  \textbf{While} condizione & Definisce l'istruzione While\\
  \textbf{Return} valore & Ritorna valore da una procedura\\
  \bottomrule
\end{tabular}

\section{Analisi di Algoritmi}
In questo paragrafo verranno scritti degli algoritmi e verrà effettuata l'Analisi
dell'Algoritmo, che può avvenire in due modalità:

\begin{itemize}
  \item \textbf{Tempo di Esecuzione}:è il numero di operazioni primitive che vengono eseguite
        da parte di un algoritmo;l'esecuzione di un'istruzione si assume che richiede un tempo costante
        per evitare di rendere la valutazione dipendente dall'hardware e dalla bravura del programmatore.
  \item \textbf{Spazio di Esecuzione}:è il numero di spazio in bit occupato in memoria dall'algoritmo
        ma questa analisi non viene quasi mai eseguita in quanto oramai è superfluo.
\end{itemize}

Dato un Array di interi A[0....length-1] si definisce $\min (A) = a \Leftrightarrow \forall b \in A : a \leq b$
e il pseudocodice della procedura $\proc{min}$ è il seguente:
\begin{algorithm}
\caption{$\proc{min}$(ITEM[] A,\textbf{integer} n)} \qquad \qquad Costo \quad Volte
\end{algorithm}
\begin{codebox}
  \li \id{min} $\gets A[0]$ \>\>\>\>\>\> $c_1$ \>\>\> $1$
  \li \For j $\gets 1$ \To $\id{A.length-1}$ \>\>\>\>\>$c_2$\>\>\>$n$
      \Do
  \li        \If $A[j] \leq \id{min}$ \>\>\>\>\>$c_3$\>\>\>$n-1$
\li             \Then
                    $\id{min} \gets A[j]$\>\>\>\>$c_4$\>\>\>$n-1$
             \End
      \End
  \li \Return min \>\>\>\>\>\>$c_5$\>\>\>$1$
\end{codebox}



Per valutare un Algoritmo bisogna verificare due proprietà:\textbf{CORRETTEZZA} e
\textbf{EFFICENZA}.

Per provare la correttezza di un algoritmo bisogna effettuare una dimostrazione
matematica attraverso l'\textit{invariante di ciclo}, ossia una proprietà che mostra
la correttezza dell'algoritmo.

L'invariante deve essere vera in due casi specifici:
\begin{enumerate}
  \item \textbf{passo base}:l'affermazione è vera all'inizio della prima iterazione del ciclo
  \item \textbf{passo induttivo}:supposta vera all'inizio dell'iterazione deve essere vera
        anche all'inizio dell'iterazione successiva
\end{enumerate}

%Da definire in un file a parte
Es: Verificare la correttezza di min(A)
\textbf{Invariante di ciclo}:all'inizio di ogni iterazione la variabile \textit{min} contiene
il minimo parziale degli elementi A[0....j-1]

\begin{proof}
\textbf{Passo base}:per j = 1 l'array A è composto da un solo elemento \newline
\textbf{Ipotesi induttiva}:$\id{min}$ contiene il minimo degli elementi A[0...j-1] all'inizio dell'iterazione\newline
\textbf{Passo induttivo}:all'esecuzione di un iterazione si possono verificare due casi:\newline
Caso $A[j] < min$\newline \newline
      la variabile $\id{min}$ viene aggiornata con il valore A[j] e ciò verifica la proprietà\newline
Caso $A[j] >= min$\newline \newline
      la variabile $\id{min}$ contiene già il minimo parziale degli elementi A[0...j]
\end{proof}

Per provare l'efficienza di un algoritmo bisogna calcolare e dimostrare il numero di confronti
necessari,in funzione di n, per risolvere un problema computazionale come viene mostrato
nell'esempio.

Es:Calcolo tempo di esecuzione algoritmo min(ITEM[] A,integer n)
\begin{algorithm}
\caption{$\proc{min}$(ITEM[] A,\textbf{integer} n)} \qquad \qquad Costo \quad Volte
\end{algorithm}
\begin{codebox}
  \li \id{min} $\gets A[0]$ \>\>\>\>\>\> $c_1$ \>\>\> $1$
  \li \For j $\gets 1$ \To $\id{A.length-1}$ \>\>\>\>\>$c_2$\>\>\>$n$
      \Do
  \li        \If $A[j] \leq \id{min}$ \>\>\>\>\>$c_3$\>\>\>$n-1$
\li             \Then
                    $\id{min} \gets A[j]$\>\>\>\>$c_4$\>\>\>$n-1$
             \End
      \End
  \li \Return min \>\>\>\>\>\>$c_5$\>\>\>$1$
\end{codebox}


Il costo dell'algoritmo $\proc{min}$ nel caso peggiore è $T(n) = (c_2 + c_3 + c_4)n + (c_1 - c_3 - c_4 + c_5)$. \newline
L'algoritmo $\proc{min}$ nel caso peggiore è una funzione lineare.

%Algoritmo InsertionSort
%Algoritmo di Insertion Sort
L'algoritmo $\proc{Insertion-Sort}$ risolve il problema dell'ordinamento definito come:\newline
\textbf{Input}:una sequenza di $n$ numeri $(a_1,a_2,\dots,a_n)$ \newline
\textbf{Output}:una permutazione $(a'_1,a'_2,\dots,a'_n)$ tale che $a'_1 \leq a'_2 \leq \dots \leq a'_n$

%pseudocodice algoritmo insertionSort
\begin{figure}
    \caption{Algoritmo insertionSort}
    \label{alg:insertion}
    \begin{codebox}
        \Procname{$\proc{Insertion-Sort}(A)$}
        \li \For $j \gets 2$ \To $\attrib{A}{length}$
            \Do
        \li            $\id{key} \gets A[j]$
        \li         $i \gets j-1$
        \li         \While $i > 0$ and $A[i] > \id{key}$
                    \Do
        \li                $A[i+1] \gets A[i]$
        \li                $i \gets i-1$
                    \End
        \li         $A[i+1] \gets \id{key}$
            \End
    \end{codebox}
\end{figure}
L'algoritmo $\proc{Insertion-Sort}$ \ref{alg:insertion} è un algoritmo efficiente per ordinare un ristretto numero di elementi ed opera come farebbe un umano
a riordinare le carte da gioco, ossia prendendo una carta alla volta e facendo il riordinamento delle carte una alla volta.\newline
Per poter affermare che l'algoritmo è corretto, ossia risolve il problema, bisogna dimostrare l'invariante del ciclo $\kw{For}$, attraverso un metodo simile all'induzione matematica.

%Dimostrazione Invariante di Ciclo
L'invariante del ciclo è corretta se si riesce a dimostrare tre cose:
\begin{description}
  \item[Inizializzazione] è corretta prima della prima esecuzione del ciclo.
  \item[Conservazione] se è verificata prima di un iterazione del ciclo lo sarà anche dopo l'esecuzione di quell'iterazione del ciclo.
  \item[Conclusione] alla fine del ciclo è ancora verificato e ciò ci aiuta a determinare la correttezza di un algoritmo.
\end{description}
La terza proprietà è la più importante in quanto assieme alla condizione che è causato la conclusione del ciclo, si riesce a dimostrare la correttezza dell'algoritmo.

L'invariante di ciclo per l'$\proc{Insertion-Sort}$ è:
All'inizio di ogni iterazione del ciclo $\kw{for}$ il sottoarray $A[1 \twodots j-1]$
è ordinato ed è formato dagli stessi elementi che erano originamente in $A[1\twodots j-1]$.
\begin{description}
  \item[Inizializzazione] quando $j = 2$ il sottoarray $A[1\twodots j-1]$ è formato
                           da un solo elemento che è ordinato ed è l'elemento originale $A[1]$.
  \item[Conservazione] all'inizio di ogni esecuzione del ciclo for il sottoarray $A[1 \twodots j-1]$
                        è formato dai primi $j-1$ elementi dell'array ordinati dal
                        più piccolo al più grande.

  \item[Conclusione] Quando $j > \attrib{A}{length}$ il ciclo termina e dato che ogni
                      ciclo aumenta $j$ di $1$ alla fine del ciclo si avrà $j = n + 1$
                      per cui si ha che $A[1\twodots n]$ è ordinato ed è formato
                      dagli elementi ordinati che si trovavano in $A[1 \twodots n]$.
\end{description}
L'analisi di un algoritmo, per poter determinare se un algoritmo è efficiente, può
avvenire in due maniere:
\begin{description}
    \item [Tempo di Esecuzione] è il numero di operazioni primitive che vengono eseguite da parte di un algoritmo;l'esecuzione di un'istruzione si assume che richiede un tempo costante
                                per evitare di rendere la valutazione dipendente dall'hardware e dalla bravura del programmatore.
    \item [Spazio di Esecuzione] è il numero di spazio in bit occupato in memoria dall'algoritmo ma questa analisi non viene quasi mai eseguita in quanto oramai è superfluo.
\end{description}
Il tempo di esecuzione dell'algoritmo è la somma dei tempi di esecuzione per ogni istruzione eseguita quindi il tempo di esecuzione di $\proc{Insertion-Sort}$ è:
\begin{equation*}
    T(n) = c_1n + c_2(n-1) + c_3(n-1) + c_4 \sum_{j=2} ^n t_j + c_5 \sum_{j=2} ^n (t_j -1)
         + c_6 \sum_{j=2} ^n (t_j -1) + c_7(n-1)
\end{equation*}
In caso l'algoritmo sia già ordinato, caso migliore, si avrebbe sempre $A[i] < \id{key}$ quindi $t_j$ è sempre $1$, per cui il tempo di esecuzione sarebbe:
\begin{align*}
    T(n) & = c_1n + c_2(n-1) + c_3(n-1) + c_4(n-1) + c_7(n-1) \\
         & = (c_1 + c_2 + c_3 + c_4 + c_5)n - (c_2 + c_3+ c_4+ c_7) \\
         & = \Omega(n) \\
\end{align*}
Nel caso migliore si ha che l'algoritmo richiede un tempo lineare che è un $\Omega(n)$.\newline
In caso si abbia una sequenza decrescente, corrispondente al caso peggiore, nel ciclo While bisogna
confrontare ogni elemento $A[j]$ con il sottoarray $A[1 \twodots j-1]$ per cui $t_j = j$
per $j=2,3,\dots,n$ e poiche si ha
\begin{align*}
    \sum _{j=2} ^ n j = \frac{n(n+1)}{2} -1 & \quad \sum_{j=2}^n (j-1) = \frac{n(n-1)}{2}
\end{align*}
il tempo dell'algoritmo $\proc{Insertion-Sort}$ nel caso peggiore è il seguente:
\begin{align*}
    T(n) & = c_1n + c_2(n-1) + c_3(n-1) + c_4 (\frac{n(n+1)}{2} -1) + c_5 (\frac{n(n-1)}{2})
         + c_6 (\frac{n(n-1)}{2}) + c_7(n-1) \\
         & = c_1n + c_2(n-1) + c_3(n-1) + c_4(\frac{n^2+n-2}{2}) + c_5(\frac{n^2-n}{2})
           + c_6(\frac{n^2-n}{2}) + c_7(n-1) \\
         & = (\frac{c_4}{2} +\frac{c_5}{2} + \frac{c_6}{2})n^2 + (c_1+c_2+c_3+\frac{c_4}{2} - \frac{c_5}{2}
            -\frac{c_6}{2}+c_7)n -(c_2+c_3+c_4+c_7)\\
         & = O(n^2)\\
\end{align*}
Il tempo dell'algoritmo può essere scritto, nel caso peggiore, come $an^2 + bn + c$ che è una funzione
quadratica che viene indicata, nel caso peggiore come $O(n^2)$.

Nel caso medio mi aspetto, supponendo una distribuzione uniforme della probabilità,
che il vettore sia parzialmente ordinato per cui non avendo modo di rendere il ciclo
while eseguibile solo una volta, si ha bisogno almeno di $n^2$ confronti che è $O(n^2)$.

In sintesi i tempi di esecuzione dell'algoritmo $\proc{Insertion-Sort}$ sono:
\begin{description}
  \item[Caso migliore] $\Omega(n)$
  \item[Caso peggiore] $O(n^2)$
  \item[Caso medio] $O(n^2)$
\end{description}
%Paragrafo sull'algoritmo InsertionSort

%Algoritmo linearSearch
\section{Ricerca di Valori}
\textbf{Input}:una sequenza di valori A[0 \dots length-1] e un valore \textit{key} \newline
\textbf{Output}:\textit{index},indice della sequenza A in cui A[index] = key, altrimenti null
%Esercizio 2.1-3 CLRS
Input: una sequenza di $n$ numeri $A = (a_1,a_2,\dots,a_n)$ e un valore $\id{key}$
Output: un indice $i$ tale che $A[i] = \id{key}$ o il valore speciale $\const{nul}$ se non viene trovato

%Algoritmo linearSearch(A,key)$
\begin{codebox}
    \Procname{$\proc{linearSearch}(A,\id{key})$}
\li \For $j \gets 1$ \To \attrib{A}{length}
    \Do
\li       \If $A[j] \isequal \id{key}$
          \Then
\li               \Return $j$
          \End
    \End
\li \Return \const{nul}
\end{codebox}


%Effettuare controllo correttezza e efficienza Algoritmo
 %Primo Capitolo Introduzione agli Algoritmi
%Capitolo sulla tecnica Divide et Impera
\chapter{Divide et Impera}
Come già visto nel precedente capitolo, \emph{divide et Impera} è una tecnica
di sviluppo di algoritmi ricorsivi, in cui si divide il problema in sottoproblemi
più semplici da risolvere.

Per risolvere l'equazione di ricorrenza, funzione che descrive il tempo di esecuzione
in funzione del tempo di esecuzione dei sottoproblemi, vi sono 3 metodi:
\begin{description}
    \item[Metodo di Sostituzione]:ipotiziamo un tempo di esecuzione e utilizziamo
                l'induzione matematica per dimostrare la correttezza dell'ipotesi
    \item[Metodo dell'albero di Ricorsione]:converte l'equazione in un albero i cui
          nodi rappresentano i costi ai vari livelli della ricorsione
    \item[Metodo dell'Esperto]:fornisce i limiti dell'equazioni di ricorrenza che
          rispettano determinate condizioni(Analizzato nei prossimi paragrafi)
\end{description}

%paragrafo sul metodo di Sostituzione per risolvere le ricorrenze
\section{Il metodo di Sostituzione}
Il metodo di sostituzione è un tecnica di risoluzione delle equazioni di ricorrenza
degli algoritmi divide et impera, che richiede due passi:
\begin{enumerate}
    \item Ipotizzare la forma della risoluzione.
    \item Usare l'induzione matematica per trovare le costanti e dimostrare che
          la soluzione proposta funziona ed è corretta.
\end{enumerate}
Per vedere il funzionamento di questo metodo proviamo ad usarlo per mostrare il seguente teorema
%Dimostrazione
\begin{thm}
    $T(n) = 2T(n/2) + n = \Theta(n \ln n)$
\end{thm}
\begin{proof}
Per dimostrare che $T(n) = \Theta(n \ln n)$ bisogna provare che $T(n) \leq cn \ln n$ per una costante $c > 0$
\begin{equation*}
\begin{split}
T(n) & \leq 2(cn/2 \lg n/2) + n \\
     & \leq cn \lg n - cn \lg 2 + n\\
     & \leq cn \lg n - cn + n \\
     & \leq cn \lg n \text{per} c \geq 1\\
\end{split}
\end{equation*}
L'induzione matematica richiede di verificare i casi base ma solitamente le condizioni di contorno non vengono dimostrate in quanto può essere molto difficile 
in alcuni casi e per risolvere si sfrutta la notazione asintotica, che prevede una costante $n_0$ arbitrariemente scelta e con questo si può rimuovere le condizioni
di contorno difficili da dimostrare.
\end{proof}

%Scegliere una buona ipotesi
Per scegliere una buona ipotesi da verificare mediante il metodo di sostituzione
richiede fantasia, esperienza, cosa che non sempre si dispone soprattutto la fantasia.\newline
Per aiutarci ad ottenere una buona ipotesi si potrebbe utilizzare l'albero di sostituzione
e poi dimostrare l'ipotesi mediante induzione, con il metodo di sostituzione, oppure
ci sono delle euristiche per diventare dei buoni indovini.

Le euristiche per formulare buone ipotesi sono le seguenti:
\begin{itemize}
    \item Se una ricorrenza è simile ad una già analizzata conviene provare a dimostrare
          la stessa soluzione come ad esempio:
          \begin{equation*}
              T(n) = 2T(\frac{n}{2} + 17) + n \text{è simile all'equazione precedente ed è un} \Theta(n \lg n)
          \end{equation*}
    \item Si inizia a dimostrare dei limiti superiori ed inferiori molto larghi e
          poi restringere l'incertezza alzando il limite inferiore ed abbassando
          il limite superiore fino a convergere con il risultato corretto.
\end{itemize}

%esempi

%Errori tipici e togliere termini di ordine inferiore
Ci sono dei casi in cui ipotiziamo correttamente un limite asintotico per la ricorrenza
ma in qualche modo sembra che i calcoli matematici non tornino nell'Induzione.
Per superare questo ostacolo spesso basta correggere l'ipotesi sottraendo un termine
di ordine inferiore per far quadrare i conti, come nell'esempio:
\[    T(n) = T(\lfloor n/2 \rfloor) + T(\lceil n/2 \rceil) + 1 \]
Supponiamo che $T(n) = O(n)$ ossia $T(n) \leq cn$, otteniamo nella ricorrenza:
\[ \begin{split}
    T(n) & \leq c \lfloor n/2 \rfloor + c \lceil n/2 \rceil + 1 \\
         & = cn + 1
 \end{split} \]
Questa equazione non implica che $T(n) \leq cn$ qualunque sia il valore di $c$ però
l'intuizione che $T(n) = O(n)$ è corretta solo che per provarla dobbiamo utilizzare
un ipotesi induttiva più forte, ossia $T(n) \leq cn - d$ con $d \geq 0$ rappresentante una costante.
\[ \begin{split}
    T(n) & \leq (c \lfloor n/2 \rfloor - d) + (c \lceil n/2 \rceil -d) + 1 \\
         & \leq cn - 2d + 1 \\
         & \leq cn - d \quad \text{per ogni} d \geq 1 \\
 \end{split} \]
A volte una piccola manipolazione algebrica può rendere una ricorrenza ignota simile a una che abbiamo già visto ed analizzato per esempio 
\[ T(n) = 2T(\lfloor \sqrt{n} \rfloor) + \log n \]
sembra difficile da risolvere ma, ignorando l'arrotondamento agli interi di valori come $\sqrt{n}$, ponendo $m = \log n$ si ottiente
\[ T(2^m) = 2T(2^{\frac{m}{2}}) + m \]
Ponendo ora $S(m) = T(2^m)$ si ottiene la ricorrenza $S(m) = 2S(\frac{m}{2}) + m$ che è simile alla ricorrenza analizzata nel mergeSort.
%Paragrafo sul metodo di sostituzione

%Paragrafo sull'albero di Ricorsione
\section{Albero di Ricorsione}
In un albero di Ricorsione ogni nodo rappresenta il costo del sottoproblema e noi
sommiamo tutti i(Da vedere la traduzione in Italiano per scriverla meglio!!!!)
Di solito si utilizza l'albero di ricorsione per generare un ipotesi da dimostrare
mediante induzione, con il metodo di sostituzione, per cui si può tollerare un pò
di incertezza nell'analisi, ad esempio togliere le costanti.

Albero di ricorsione dell'equazione di ricorrenza $T(n) = 3T(n/4) + \Theta(n^2)$

%Tradurre e fare grafici degli alberi!!!!!!!

%Scrivere proprietà alberi di avere l'altezza uguale a log n
%Paragrafo sul metodo dell'albero di Ricorsione

%Paragrafo sul Teorema dell'Esperto
\section{Teorema dell'Esperto}
%Paragrafo sul metodo dell'Esperto
%Capitolo 2 Divide et Impera
\chapter{Crescita delle Funzioni}
Nel valutare l'algoritmo $\proc{Insertion-Sort}$ si ottiene una funzione $T(n) = an + b$ che
è una funzione lineare.\newline
Durante la valutazione di un algoritmo difficilmente si riesce a quantificare con
esattezza le costanti coinvolte per cui si analizza il comportamento della funzione
al tendere di n all'infinito.\newline
Al tal fine si utilizzano le notazioni $O,\ \Omega,\ \Theta\ $ definite come segue:

\begin{itemize}
  \item $f(n) \in O(g(n))$ se e solo se $\exists c \geq 0\ \exists m \geq 0 : f(n) \leq cg(n)\ \forall n \geq m$
  \item $f(n) \in \Omega(g(n))$ se e solo se $\exists c, m \geq 0 : f(n) \geq cg(n) \forall n \geq m$
  \item $f(n) \in \Theta(g(n))$ se e solo se $\exists c_1,c_2,m \geq 0 :
        c_1g(n) \leq f(n) \leq cg(n)\ \forall n \geq m$
\end{itemize}
Per maggiore chiarezza si utilizza un'abuso di linguaggio scrivendo $f(n) = O(g(n))$
al posto di $f(n) \in O(g(n))$ e che da queste definizioni si può ricavare che $f(n) = \Theta(g(n))$
se e solo se $f(n) = O(g(n))$ e $f(n) = \Omega(g(n))$.
Per poter definire se una funzione appartiene a una notazione bisogna mostrarlo attraverso
una dimostrazione formale con l'utilizzo dell'induzione matematica ma fortunatamente
c'è il seguente teorema per semplificare il calcolo della notazione:
\begin{thm}
Dato un polinomio del tipo $P(n) = \sum _{i = 0} ^ d a_i n^i$ dove $a_i$ sono i coefficienti
e $a_d > 0$ si ha che $P(n) = \Theta(n^d)$
\end{thm}
Esempio:
\begin{equation*}
  P(n) = 5n^3 + 6n^2 + 3 = \Theta(n^3)
\end{equation*}
%Capitolo 3 Crescita delle funzioni
%Algoritmo QuickSort per risolvere il problema dell'ordinamento
\chapter{QuickSort}
Il $\proc{Quick-Sort}$ è un algoritmo divide et impera in loco ossia senza utilizzare
una struttura di appoggio per effettuare l'ordinamento e funziona in maniera ottimale
nell'implementazione sui calcolatori attuali.\newline
Venne sviluppato ed ideato nel 1959 dall'importantissimo informatico C.H. Hoare
e viene utilizzato come algoritmo di default delle libreria dei maggiori linguaggi.

L'algoritmo $\proc{Quick-Sort}$ esegue i seguenti passi divide et impera:
\begin{description}
  \item[Divide] riarrangia l'array $A[p\twodots r]$ in due sottoarray, eventualmente nulli,
                $A[p \twodots q-1]$ e $A[q+1 \twodots r]$ tali che tutti gli elementi
                del primo sottoarray sono minori o uguali a $A[q]$ e tutti gli elementi
                del secondo sottoarray sono maggiori o uguali a $A[q]$.\newline
                Calcolare l'indice di $q$ viene effettuato nella procedura di riarrangiamento.
  \item[Impera] ordina ricorsivamente i due sottoarray $A[p \twodots q-1]$ e $A[q+1 \twodots r]$.
  \item[Combina] dato che i due sottoarray sono già ordinati per cui non viene eseguito nulla.
\end{description}

L'algoritmo $\proc{Quick-Sort}$ è il seguente
%Pseudocodice Quicksort
\begin{codebox}
\Procname{$\proc{Quick-Sort}(A,p,r)$}
\li \If $p < r$
    \Then
\li           $q = \proc{Partition}(A,p,r)$
\li           $\proc{Quick-Sort}(A,p,q-1)$
\li           $\proc{Quick-Sort}(A,q+1,r)$
    \End
\end{codebox}
Per effettuare l'ordinamento di un array $A$ viene effettuata la chiamata iniziale
$\proc{Quick-Sort}$(A,1,$\attrib{A}{length}$).

La chiave dell'algoritmo è la procedura $\proc{Partition}$ che può essere implementata
in due maniere:la prima, inventata da Hoare e quindi quella originale, prende come elemento
pivot il primo elemento mentre la seconda versione, inventata da Lomuro, utilizza
l'ultimo elemento;lo pseudocodice delle due partition è il seguente:
%Pseudocodice Procedura Partition
\begin{codebox}
\Procname{$\proc{Lomuro-Partition}(A,p,r)$}
\li $\id{pivot} \gets A[r]$
\li $\id{indice} \gets p-1$
\li \For $j \gets p$ \To $r-1$
    \Do
\li               \If $A[j] \leq \id{pivot}$
                  \Then
\li                              $\id{indice} \gets \id{indice} + 1$
\li                              scambia $A[id{indice}]$ con $A[j]$
                  \End
    \End
\li scambia $A[indice+1]$ con $A[r]$
\li \Return $\id{indice} + 1$
\end{codebox}

\begin{codebox}
\Procname{$\proc{Hoare-Partition}(A,p,r)$}
\li $\id{pivot} \gets A[p]$
\li $i \gets p$
\li \For $j \gets p$ \To $r$
    \Do
\li               \If $A[j] < \id{pivot}$
                  \Then
\li                              $i \gets i + 1$
\li                              scambia $A[i]$ con $A[j]$
                  \End
    \End
\li scambia $A[i]$ con $A[r]$
\li \Return $i$
\end{codebox}

%Invariante di Ciclo da fare!!!!

Il tempo di esecuzione della procedura $\proc{Partition}$, sia Lomuro che Hoare, è il seguente:
$T(n) = c_1 + c_2 + c_3(n+1) + c_4n + c_5n(t_{if}) + c_6n(t_{if}) + c_7 + c_8$.

\begin{description}
  \item[Caso migliore]:tutti gli elementi sono maggiori del pivot per cui $t_{if} = 0$
        $T(n) = (c_3 + c_4)n + (c_1+c_2+c_3+c_7+c_8) = \Theta(n)$
  \item[Caso peggiore]:tutti gli elementi sono inferiori del pivot per cui $t_{if} = 1$
        $T(n) = (c_3+c_4+c_5+c_6)n + (c_1+c_2+c_3+c_7+c_8) = \Theta(n)$
\end{description}

\section{Analisi tempo QuickSort}
L'equazione di ricorrenza generale del Quicksort è la seguente:
\begin{equation*}
    T(n) = \begin{cases} 0 \quad \text{se} \ n = 0,1 \\
                         T(n-j) + T(j) + \Theta(n) \quad \text{se} \ n > 1 \\
           \end{cases}
\end{equation*}
L'analisi del tempo di esecuzione del $\proc{Quick-Sort}$ è in base al fatto se
la partizione dell'array è bilanciata o meno ossia se i due sottoproblemi da risolvere
sono più o meno della stessa dimensione.
\begin{description}
  \item[Caso peggiore]:la procedura di partizione produce due sottoproblemi:
        uno di $n-1$ elementi e l'altro di $0$ elementi.
        $T(n) = T(n-1) + T(0) + \Theta(n)
              = T(n-1) + \Theta(n)$
        Attraverso il metodo di sostituzione arrivo a $T(n) = O(n^2)$
  \item[Caso migliore]:la procedura di partizione produce due sottoproblemi di $\lceil n/2 \rceil$
        e $\lfloor n/2 \rfloor$ elementi per cui, ignorando le condizioni di ceil e floor,
        il tempo di esecuzione è $T(n) = 2T(n/2) + \Theta(n) = \Omega(n \log n)$ per il teorema dell'esperto.
  \item[Caso medio]:
\end{description}

%Versione Randomizzata
Una versione randomizzata del quicksort, utile per ottenere una buona prestazione attesa
in tutti gli input, consiste nel prendere come pivot ad ogni iterazione un elemento a caso
tra $p$ e $r$ così l'elemento pivot avrà la stessa possibilità di essere un elemento
del sottoarray per cui ci aspettiamo che la ripartizione dell'array sia ben bilanciata in media.

\begin{codebox}
\Procname{$\proc{Randomized-Partition}(A,p,r)$}
\li $i \gets \proc{RANDOM}(p,r)$
\li scambia $A[i]$ con $A[r]$
\li \Return $\proc{Partition}(A,p,r)$
\end{codebox}
La procedura randomizzata Quicksort chiama $\proc{Randomized-Partition}$
invece che $\proc{Partition}$, per cui il suo pseudocodice è il seguente:
\begin{codebox}
\Procname{$\proc{Randomized-QuickSort}(A,p,r)$}
\li \If $p < r$
    \Then
\li           $q = \proc{Randomized-Partition}(A,p,r)$
\li           $\proc{Randomized-QuickSort}(A,p,q-1)$
\li           $\proc{Randomized-QuickSort}(A,q+1,r)$
    \End
\end{codebox}
Il tempo di esecuzione del Quicksort randomizzato coincide ovviamente a quello normale
solo che il caso medio si verifica molto spesso ed è molto raro il caso peggiore
per cui di solito viene utilizzata la versione randomizzata del Quicksort.
%Capitolo 4 QuickSort
%Capitolo sull'ordinamento in tempo lineare
\chapter{Ordinamento in tempo lineare}
Gli algoritmi di ordinamento analizzati fino ad ora hanno un importante proprietà,
ossia effettuano l'ordinamento soltanto mediante il confronto tra gli elementi in input.
In questo capitolo verrà analizzato il limite minimo di confronti da effettuare e
verranno analizzati altri due algoritmi di Ordinamento.
%Capitolo 5 Ordinamento in Tempo lineare
%%Esercitazione 1 di Guido Zandron
Scrivere un algoritmo che risolva il seguente problema, utilizzando la tecnica divide-et-
impera: dato in input un vettore $V[1..n]$ contenente $n > 0$ valori interi, calcolare e restituire il
valore: $(V[1] + V[2]) * (V[2] + V[3]) + \dots + (V[n-1] + V[n])$.
Scrivere poi l’equazione di ricorrenza che esprime il tempo di calcolo di tale algoritmo, e
risolverla utilizzando un metodo a piacere.

\begin{codebox}
\Procname{$\proc{calcoloParticolare}(V,left,\id{right})$}
\li \If $left \geq \id{right}$
    \Then
\li              \Return $left + \id{right}$
    \End
\li $\id{mid} \gets (left + \id{right}) / 2$
\li $ris_1 \gets \proc{calcoloParticolare}(V,left,\id{mid})$
\li $ris_2 \gets \proc{calcoloParticolare}(V,\id{mid},\id{right})$
\li \Return $ris_1 * ris_2$
\end{codebox}

Il tempo di esecuzione dell'algoritmo è il seguente
\begin{equation*}
  T(n) = \begin{cases} 2c = \Theta(1) \ \text{se} \ n = 1 \\
                       2T(\frac{n}{2}) + 2c \ \text{se} \ n > 1\\
         \end{cases}
\end{equation*}
Essendo $2c = O(n^{\log _ 2 2}) = O(n)$ si applica il primo caso del teorema dell'Esperto
per cui si ottiene $T(n) = \Theta(n)$. Il tempo di esecuzione dell'algoritmo è:
\begin{equation*}
  T(n) = \begin{cases} 2c = \Theta(1) \ \text{se} \ n = 1 \\
                       2T(\frac{n}{2}) + 2c = \Theta(n) \ \text{se} \ n > 1\\
         \end{cases}
\end{equation*}


Scrivere un algoritmo iterativo che, dati in ingresso due vettori $A[1..m]$ e $B[1..n]$ contenenti
rispettivamente m ed n caratteri dell’alfabeto italiano, conti quante volte la stringa
memorizzata in A compare in B. Si assuma che n sia molto maggiore di m.
Valutare poi i tempi di calcolo nei casi migliore e peggiore, indicando quante volte viene
eseguita ogni operazione in pseudocodice. Si assuma che ogni singola operazione elementare
impieghi lo stesso tempo (costante).
Esempio: dati i vettori:
A = [ a, b, a ]
B = [ c, d, c, a, b, a, b, a, c, a, b, a, b, c, d, b, a, b, b, a, b, a, a, b ]
il programma deve stampare il valore 4, dato che la stringa “aba” compare 4 volte in B (a
partire dalle posizioni 4, 6, 10 e 20. Fare attenzione alle eventuali sovrapposizioni!).

\begin{codebox}
\Procname{$\proc{contaRipetizioni}(A,B)$}
\li $n_1 \gets \attrib{A}{length}$
\li $n_2 \gets \attrib{B}{length}$
\li $ripetizioni \gets 0$
\li \For $j \gets 1$ \To $n_2 - n_1$
    \Do
\li                $i \gets 1$
\li                \While $i \leq n_1$ and $B[j+i-1] \isequal A[i]$
                   \Do
\li                               $i \gets i + 1$
                   \End
\li                \If $i > n_1$
                   \Then
\li                               $\id{ripetizioni} = ripetizioni + 1$
    \End
\li \Return ripetizioni
\end{codebox}
Il tempo di esecuzione dell'algoritmo è il seguente:
\begin{equation*}
  T(n) = 4c + (n+1)c + cn + c\sum _{j = 1} ^ n (t_{w} + 1) + c\sum _{j = 1} ^ n t_w
         cn + ct_{if}
\end{equation*}
L'algoritmo ha i seguenti casi:
\begin{description}
  \item[Caso migliore] tutti gli elementi di $B$ sono uguali e diversi da $A[0]$
        per cui risulta $t_w = 0$ e $t_{if} = 0$ per tutte le iterazioni del ciclo for.
        \begin{equation*}
          T(n) = 4c + cn + c + cn + cn = \Omega(n)
        \end{equation*}
  \item[Caso peggiore]: tutti gli elementi di $B$ ed $A$ sono uguali per cui risulta
         $t_w = n_1$ per tutte le iterazioni del ciclo for e $t_{if} = n$.
         Dato che si suppone che $n_2$ sia nettamente maggiore di $n_1$ si arriva
         alla conclusione che $n_1 \leq (n_2-n_1) \leq n$ per cui si ha $\sum _{j = 1} ^ n (n_1 + 1)
         \leq \sum _{j = 1} ^ n ( + 1) \leq (n^2 + n)$.
         \begin{equation*}
           T(n) = 5c + cn + cn^2 + cn + cn^2 + cn + cn = 0(n^2)
         \end{equation*}
\end{description}

%guardare funzione ricorsiva pagina 18 libro The Design and Analysis of Algorithms

%Dato un array di lettere dell'alfabeto effettuare il conto delle sequenze AZAZ e ABBA
%Esercizi Algoritmi
%Capitolo sulle Strutture Dati
\chapter{Strutture Dati Elementari}
In questo capitolo verranno definite le strutture dati elementari, ma prima di poterle
definire bisogna definire il concetto di tipo di dato.\newline
Il tipo di dato è un modello matematico in cui sono definite un certo numero di operazioni
e in ogni linguaggio di programmazione vengono definiti e previsti dei tipi di dato detti
\emph{primitivi}, come ad esempio i numeri interi, i caratteri ed ecc... ma può
essere comodo e conveniente definire altre tipologie di dati per rendere più facile e chiara
la definizione e l'implementazione di un algoritmo.

In generale le proprietà di un tipo di dato devono dipendere soltanto dalla sua specifica
ed essere indipendenti dalla modalità in cui vengono rappresentati per cui si dice
che un tipo di dato è \emph{astratto}, in quanto il dato è astratto rispetto alla sua rappresentazione.

Il vantaggio di avere i tipi di dato astratti consiste nel poter utilizzare il dato
senza conoscere la sua rappresentazione ed eventuali modifiche alla rappresentazione del dato
non comportano alcun cambiamento nell'utilizzo del dato da parte dell'utilizzatore.

%Cercare definizione di Strutture dati

%Strutture Dati Lista
\section{Liste}
Le liste sono una struttura dati elementare che implementa il concetto matematico
di sequenza lineare di oggetti, in cui si possono eventualmente ripetere gli elementi
all'interno della sequenza.\newline
La lista è una struttura dati lineare dinamica in cui l'accesso all'elemento successivo
della sequenza avviene tramite un puntatore all'elemento successivo ed un elemento
è composto da un valore , chiamato $\id{element}$ e da due puntatori $\id{prev}$ e $\id{next}$,
i quali puntano all'elemento precedente o successivo della lista.\newline
In caso $\id{prev} \gets \const{nil}$ l'elemento non ha nessun predecessore ed è la $\id{head}$
della lista mentre il puntatore $\id{next} \gets \const{nil}$ l'elemento non ha
nessun successore per cui è la coda della lista.
Vi sono diversi tipologie di caratteristiche che una lista può possedere, anche in
maniera multipla ossia può possedere più di una proprietà, come si può notare dal seguente elenco:
\begin{itemize}
  \item singolarmente o doppiamente concatenata: in caso una lista sia singolarmente
        concatenata si ha soltanto il collegamento con l'elemento successivo, per
        cui viene omesso il puntatore $\id{prev}$, mentre nella lista doppiamente concatenata
         si ha il collegamento, tramite i puntatori, con l'elemento precedente e l'elemento successivo.
  \item ordinata: una lista si dice ordinata se è previsto un ordinamento tra i valori
        degli elementi presenti in una lista.
  \item circolare: il puntatore $\id{prev}$ della testa della lista punta alla coda
        mentre il puntatore $\id{next}$ della coda della lista punta alla testa
       per cui si può dire che la lista è un anello di elementi.
\end{itemize}

Forniamo ora un implementazione di una lista doppiamente concatenata non ordinata
in cui un elemento della lista è composto da un dato chiamato $\id{element}$ e da due
puntatori, chiamati $\id{prev}$ e $\id{next}$, che puntano all'elemento precedente e successivo.

La prima operazione implementata in una lista è la ricerca di un elemento che opera
secondo l'algoritmo di ricerca Lineare in quanto non essendo ordinati i valori della lista
non è possibile implentare la ricerca tramite la ricerca binaria.
Lo pseudocodice della ricerca di un elemento di una lista è il seguente:
%Pseudocodice della ricerca di una Lista
\begin{codebox}
\Procname{$\proc{List-Search}(L,\id{key})$}
\li $x\gets \attrib{L}{head}$
\li \While $x \neq \const{nil}$ and $\attrib{x}{\id{element}} \neq \id{key}$
    \Do
\li                      $x \gets \attrib{x}{\id{next}}$
    \End
\li \Return $\id{x}$
\end{codebox}
La ricerca di un elemento da una lista richiede il seguente tempo con i diversi casi:
\begin{equation*}
  T(n) = c + c(t_w + 1) + ct_w + c
\end{equation*}
\begin{description}
  \item[Caso migliore]:l'elemento da ricercare viene trovato al primo elemento della lista
        per cui $t_w = 0$ indi il tempo di esecuzione è $T(n) = c + c + c = \Omega(1)$
  \item[Caso peggiore]: l'elemento non è presente nella lista per cui $t_w = n$
        \begin{equation*}
          T(n) = c + cn + c + cn + c = 2cn + 3c = O(n)
        \end{equation*}
\end{description}

%Pseudocodice Inserimento in una lista
Il secondo metodo in una lista è $\proc{List-Insert}$ in cui si suppone che il valore
dell'elemento da inserire sia stato già impostato ossia $\id{element}$ abbia il valore desiderato.
\begin{codebox}
\Procname{$\proc{List-Insert}(L,x)$}
\li $\attrib{x}{\id{next}} \gets \attrib{L}{\id{head}}$
\li \If $\attrib{L}{head} \neq \const{nil}$
    \Then
\li                       $\attrib{L}{\attrib{\id{head}}{\id{prev}}} \gets x$
    \End
\li $\attrib{L}{\id{head}} \gets x$
\li $\attrib{x}{\id{prev}} \gets \const{nil}$
\end{codebox}
Il tempo di esecuzione $T(n) = 5c = \Theta(1)$ e tra il caso migliore e peggiore non
vi è alcuna differenza se non il fatto che non viene eseguita soltanto la terza istruzione.

La rimozione di un elemento da una lista, implementata tramite $\proc{List-Delete}$,
prevede di darne il puntatore all'elemento alla procedura per cui per effettuare
la rimozione di un elemento qualsiasi della lista bisogna effettuare la chiamata
a $\proc{List-Search}$ prima per ottenere l'elemento da eliminare.
Lo pseudocodice per la rimozione di un elemento è il seguente:
%Pseudocodice Rimozione da una Lista
\begin{codebox}
\Procname{$\proc{List-Delete}(L,x)$}
\li \If $\attrib{x}{prev} \neq \const{nil}$
    \Then
\li              $\attrib{x}{prev} \gets \attrib{x}{next}$
    \End
\li \Else        $\attrib{L}{head} \gets \attrib{x}{next}$

\li \If $\attrib{x}{next} \neq \const{nil}$
    \Then
\li              $\attribb{x}{\id{next}}{\id{prev}} \gets \attrib{x}{\id{prev}}$
    \End
\end{codebox}
Il tempo di esecuzione è $T(n) = 4c = \Theta(1)$ e tra il caso peggiore e migliore non cambia
nulla però va detto che se si volesse implementare la rimozione da un elemento qualsiasi
della lista si avrebbe un tempo di esecuzione $\Theta(n)$ in quanto si avrebbe la chiamata
alla procedura $\proc{List-Search}$ per stabilire l'elemento da rimuovere.

Le altre implementazioni delle diverse tipologie di liste sono similari soltanto
che implementano o meno l'ordinamento tra gli elementi, considerano o meno il puntatore prev
ed altri considerazioni fatte in base alla tipologia di lista.

Un implementazione alternativa della sequenza, anche se meno intutiva e naturale
di quella presentata fino ad ora, è quella tramite la memorizzazione degli elementi
in un vettore, in cui la posizione di un elemento corrisponde all'indice del vettore.\newline
Questa implementazione permette di passare in maniera costante da un elemento ad un altro,
di accorgersi se si supera un estremo della sequenza, di modificare o leggere il valore
di un elemento anche tramite un accesso diretto tramite indice, ma sfortunatamente
richiede di conoscere la dimensione massima della sequenza per evitare sprechi di memoria
e il tempo di inserimento e cancellazione richiede la scansione della sequenza per cui a tempo $\Theta(n)$.
%Paragrafo sulle liste

%Paragrafo sulle code
\section{Code}
La \emph{Queue}, in italiano \emph{coda}, è una struttura dati di tipo FIFO(First in First out) per memorizzare una sequenza di elementi,
in cui l'inserimento di un elemento avviene in coda alla sequenza mentre la rimozione avviene in testa alla sequenza.\newline
Una coda può essere implementata attraverso array oppure delle liste a seconda della scelta implementativa e della capacità di stabilire un limite massimo di elementi utilizzati.

Essendo una coda una particolare tipologia di sequenza può essere implementata facilmente utilizzando una lista e le sue operazioni però,
a differenza dello stack, la scelta della lista utilizzata per l'implementazione cambia il tempo di esecuzione delle operazioni,
infatti soltanto con una lista bidirezionale si ottiene un tempo $\Theta(1)$ in tutte le operazioni per cui noi utilizziamo questa tipologia di lista per implementare una coda.

Le code prevedono la definizione di 3 operazioni:$\proc{isEmpty}$ \ref{alg:isEmpty} indica se la coda contiene o meno degli elementi, $\proc{dequeue()}$ \ref{alg:dequeue}
estrae il primo elemento della coda ed infine si ha $\proc{enqueue(Q, x)}$ \ref{alg:enqueue} che inserisce un elemento nella coda e si assume che $x$ sia un elemento già definito.

%Pseudocodice delle operazioni di una Coda
\begin{codebox}
\Procname{$\proc{isEmpty}()$}
\li \Return ($\attrib{Q}{head} \isequal \const{nil}$)
\end{codebox}
\begin{codebox}
\Procname{$\proc{Enqueue}(Q,x)$}
\li $\attribb{Q}{tail}{next} \gets x$
\li $\attrib{Q}{tail} \gets x$
\end{codebox}
L'esecuzione del metodo $\proc{enqueue}$ richiede $2c$ ossia $\Theta(1)$ per ciò
l'inserimento in una coda richiede un tempo costante.

Il metodo $\proc{dequeue()}$ estra il primo elemento della coda ed è implementato come
\begin{codebox}
\Procname{$\proc{Dequeue}()$}
\li \If $\proc{isEmpty}()$
    \Then
\li           \Return $\const{nil}$
\li $\id{temp} \gets \attrib{Q}{head}{element}$
\li $\attrib{Q}{head} \gets \attribb{Q}{head}{next}$
\li \Return temp
\end{codebox}
L'operazione $\proc{dequeue}$ richede un tempo costante $\Theta(1)$ per effettuare la rimozione.

Tutte le operazioni presentate impiegano tempo costante $\Theta(1)$, cosa che le rende molto efficiente per rappresentare dati cui si vuole una politica FIFO.
%Coda tramite Vettore
Dopo aver visto come viene implementata una coda tramite puntatori, consideriamo l'implementazione tramite un vettore $Q[1 \twodots n]$ 
e ai campi $\attrib{Q}{head}$ e $\attrib{Q}{tail}$ per accedere all'elemento in testa e in coda ai vettore della coda;lo pseudocodice della coda tramite un vettore è il seguente:
\begin{codebox}
\Procname{$\proc{enqueue}(Q,x)$}
\li $Q[\attrib{Q}{tail}] \gets x$
\li \If $\attrib{Q}{tail} \isequal \attrib{Q}{length}$
\li    \Then $\attrib{Q}{tail} \gets 1$
\li \Else $\attrib{Q}{tail} \gets \attrib{Q}{tail} + 1$
\end{codebox}

\begin{codebox}
\Procname{$\proc{dequeue}(Q)$}
\li $x \gets Q[\attrib{Q}{head}]$
\li \If $\attrib{Q}{head} \isequal \attrib{Q}{length}$
\li \Then $\attrib{Q}{head} \gets 1$
\li \Else $\attrib{Q}{head} \gets \attrib{Q}{head} + 1$
\li \Return x
\end{codebox}

\begin{codebox}
\Procname{$\proc{queue-Empty}(Q)$}
\li \Return ($\attrib{Q}{head} \leq 0$)
\end{codebox}

Il tempo di esecuzione delle seguente procedure è sempre costante $\Theta(1)$
come anche la coda tramite sequenza ma ha senso implementare tramite vettore se e soltanto se
si sa determinare il numero degli elementi per evitare uno spreco di memoria.
%Paragrafo sulle code

%Paragrafo sulle stack
\section{Stack}
Lo \emph{stack}, è una struttura dati di tipo \emph{LIFO}(Last in first out), utilizzata in tutti i linguaggi di programmazione per effettuare 
la memorizzazione di tutti i dati di tipo statico e dei record di attivazione, che può essere vista come un caso particolare di sequenza 
in cui l'inserimento avviene alla fine della sequenza e la rimozione avviene sempre in fondo.\newline
Uno stack può essere implementato attraverso array oppure delle liste a seconda della scelta implementatica e della capacità di stabilire un limite massimo di elementi utilizzati.

Essendo lo stack un particolare tipo di sequenza, essa può essere simulata tramite le operazioni di una lista, in particolare quella singolarmente concatenata,
anche se  è prassi comune utilizzare nomi diversi per indicarne le operazioni per migliore chiarezza.

%specifica operazioni su uno stack
\begin{minted}{c}
 void push(S,x) %inserisce un elemento in testa allo stack 
 Item* pop(S) %rimuove l'elemento in testa allo stack  
 Boolean stackEmpty() %indica se lo stack contiene elementi.
\end{minted}
%pseudocodice delle operazioni dello stack
Lo pseudocodice delle operazioni di uno stack implementate tramite liste sono:
\begin{codebox}
\Procname{$\proc{push}(S,x)$}
\li $\attrib{x}{next} \gets \attrib{S}{top}$
\li $\attrib{S}{top} \gets x$
\end{codebox}

La procedura $\proc{pop}$ rimuove l'ultimo elemento inserito nello stack e lo ritorna
come valore; in caso lo stack è vuoto genera underflow come errore.
\begin{codebox}
\Procname{$\proc{pop}(S)$}
\li $\id{temp} \gets \attrib{S}{top}$
\li \If $\proc{stackEmpty}(S)$
    \Then
\li        \Error "Underflow"
    \End
\li $\attrib{S}{top} \gets \attribb{S}{top}{next}$
\li \Return $\id{temp}$
\end{codebox}

La procedura $\proc{stackEmpty}$ indica se lo stack è vuoto e viene utilizzato per
assicurarsi di non provare ad accedere allo stack vuoto per la rimozione
\begin{codebox}
\Procname{$\proc{stackEmpty}(S)$}
\li \Return $ \attrib{S}{top} \isequal \const{nil}$
\end{codebox}

Utilizzando le liste per implementare lo stack otteniamo in tutte le operazioni l'
impiego di tempo costante $\Theta(1)$.

L'implementazione dello stack tramite un vettore ha lo stesso impiego di tempo $\Theta(1)$
in tutte le operazioni però per evitare uno spreco di memoria bisogna sapere il numero
di elementi necessari e soprattutto non è possibile superare il numero di elementi massimo
stabilito alla creazione dello stack.

%Pseudocodice delle operazioni dello Stack tramite vettore
La realizzazione delle operazioni dello stack tramite un vettore sono le seguenti:
\begin{codebox}
\Procname{$\proc{push}(S,x)$}
\li $\attrib{S}{top} \gets \attrib{S}{top} + 1$
\li $S[\attrib{S}{top}] \gets x$
\end{codebox}

\begin{codebox}
\Procname{$\proc{pop}(S)$}
\li \If $\proc{stackEmpty}(S)$
\li \Then \Error "Underflow"
\li \Else $\attrib{S}{top} \gets \attrib{S}{top}-1$
\li \Return $S[\attrib{S}{top}+1]$
\end{codebox}

\begin{codebox}
\Procname{$\proc{stackEmpty}(S)$}
\li \Return $\attrib{S}{top} \leq 0$
\end{codebox}

%Paragrafo sulle pile/Stack
%Capitolo 5 Strutture Dati elementari
%Capitolo sul Heap
\chapter{Heap}
Lo \emph{heap} è una struttura dati rappresentata da un array $A$ che può essere vista come un albero binario
in cui ogni nodo dell'albero è un elemento dell'array, chiamato \emph{chiave}.\newline
L'array $A$ ha 2 attributi: $\attrib{A}{length}$ per rappresentare la lunghezza dell'array
e $\attrib{A}{heap-size}$ che indica il numero degli elementi dell'heap memorizzati
in cui $0 \leq \attrib{A}{heap-size} \leq \attrib{A}{length}$.
Ci sono due tipologie di Heap:\emph{max-heap}, utilizzato nel heapSort e \emph{min-heap},
utilizzato principalmente per implementare code prioritarie;queste due tipologie verranno analizzate entrambe
in seguito in questo paragrafo.

%Fare disegno albero e array del heap

%Fare pseudocodice immediato di left,right e parent
Per poter effettuare l'accesso ai nodi left,right e parent si utilizzano le seguenti 3 procedure

In un $\emph{max-heap}$ è soddisfatta per ogni nodo $i$ la seguente proprietà:
$A[\proc{Parent}(i)] \geq A[i]$ per cui il massimo valore della array si trova nella radice
dell'heap mentre in un $\emph{min-heap}$ avviene il contrario ossia in ogni nodo si ha
$A[\proc{Parent}(i)] \leq A[i]$ e la radice rappresenta l'elemento minimo dell'array.
Essendo lo heap definito tramite un albero binario la sua altezza è $\Theta(\log n)$.




%Strutture dati aggiornate
%suffix-tree
%FM-index
%suffix-array
%wavelet-tree
%Bloom filters
%Sequence Bloom trees
%Tabelle Hash
 Capitolo 6 Heap
%Paragrafo sugli Alberi
\section{Alberi}
Si definisce come \emph{Albero libero}, un DAG connesso con un solo nodo sorgente, detto \emph{radice},
in cui ogni nodo diverso dalla radice ha un solo nodo entrante.\newline
I nodi privi di archi entranti sono detti \emph{foglie} dell'albero.

%Inserire esempi e proprietà degli Alberi
%%Paragrafo sugli Alberi
\section{Alberi}
Si definisce come \emph{Albero libero}, un DAG connesso con un solo nodo sorgente, detto \emph{radice},
in cui ogni nodo diverso dalla radice ha un solo nodo entrante.\newline
I nodi privi di archi entranti sono detti \emph{foglie} dell'albero.

%Inserire esempi e proprietà degli Alberi
%%Paragrafo sugli Alberi
\section{Alberi}
Si definisce come \emph{Albero libero}, un DAG connesso con un solo nodo sorgente, detto \emph{radice},
in cui ogni nodo diverso dalla radice ha un solo nodo entrante.\newline
I nodi privi di archi entranti sono detti \emph{foglie} dell'albero.

%Inserire esempi e proprietà degli Alberi
%\input{Esempi/alberi}Esempi Alberi!!!!

L'albero è una struttura matematica importantissima in informatica utilizzata per
rappresentare una serie di situazioni, come ad esempio organizzazioni gerarchiche di dati,
procedimenti enumerativi o decisionali, e ve ne esistono un'infinita di implementazioni di alberi
però iniziamo ad analizzare per prima gli alberi liberi binari.

%Inserire immagine albero binario

%Metodi di visita di un albero
I metodi di visita di un albero binario sono 3:
\begin{itemize}
  \item inorder: si visiona prima il sottoalbero sinistro poi il nodo e infine il sottoalbero destro
  \item preorder: si visiona prima il nodo poi i suoi sottoalberi
  \item postorder: si visionano prima i sottoalberi ed infine il nodo
\end{itemize}
Il primo metodo di visita viene usato soprattutto negli alberi binari di ricerca per
stampare gli elementi dell'albero in maniera crescente mentre in un albero binario
normale la scelta di quale metodo di visita utilizzare è inifluente e ogni programmatore
sceglie nell'utilizzo quale metodo di visita utilizzare per stampare l'albero.
Il cammino dalla radice ad un elemento foglia dell'albero richiede al massimo $O(h)$,
in cui $h$ è l'altezza dell'albero, in quanto richiede di scendere di livello fino
ad arrivare alle foglie, che si trovano al livello $h$.

La specifica di un albero binario, in cui ogni implementazione per essere valida deve prevedere:\newline
\textbf{Item} search(Tree T,Item k);\newline
\textbf{void} insert(Tree T,Item x);\newline
\textbf{Item} delete(Tree T,Item x);\newline
\textbf{Item} mininum(Tree T);\newline
\textbf{Item} maxinum(Tree T);\newline
\textbf{Item} predecessor(Tree T,Item x);\newline
\textbf{Item} successor(Tree T,Item x);\newline
Esempi Alberi!!!!

L'albero è una struttura matematica importantissima in informatica utilizzata per
rappresentare una serie di situazioni, come ad esempio organizzazioni gerarchiche di dati,
procedimenti enumerativi o decisionali, e ve ne esistono un'infinita di implementazioni di alberi
però iniziamo ad analizzare per prima gli alberi liberi binari.

%Inserire immagine albero binario

%Metodi di visita di un albero
I metodi di visita di un albero binario sono 3:
\begin{itemize}
  \item inorder: si visiona prima il sottoalbero sinistro poi il nodo e infine il sottoalbero destro
  \item preorder: si visiona prima il nodo poi i suoi sottoalberi
  \item postorder: si visionano prima i sottoalberi ed infine il nodo
\end{itemize}
Il primo metodo di visita viene usato soprattutto negli alberi binari di ricerca per
stampare gli elementi dell'albero in maniera crescente mentre in un albero binario
normale la scelta di quale metodo di visita utilizzare è inifluente e ogni programmatore
sceglie nell'utilizzo quale metodo di visita utilizzare per stampare l'albero.
Il cammino dalla radice ad un elemento foglia dell'albero richiede al massimo $O(h)$,
in cui $h$ è l'altezza dell'albero, in quanto richiede di scendere di livello fino
ad arrivare alle foglie, che si trovano al livello $h$.

La specifica di un albero binario, in cui ogni implementazione per essere valida deve prevedere:\newline
\textbf{Item} search(Tree T,Item k);\newline
\textbf{void} insert(Tree T,Item x);\newline
\textbf{Item} delete(Tree T,Item x);\newline
\textbf{Item} mininum(Tree T);\newline
\textbf{Item} maxinum(Tree T);\newline
\textbf{Item} predecessor(Tree T,Item x);\newline
\textbf{Item} successor(Tree T,Item x);\newline
Esempi Alberi!!!!

L'albero è una struttura matematica importantissima in informatica utilizzata per
rappresentare una serie di situazioni, come ad esempio organizzazioni gerarchiche di dati,
procedimenti enumerativi o decisionali, e ve ne esistono un'infinita di implementazioni di alberi
però iniziamo ad analizzare per prima gli alberi liberi binari.

%Inserire immagine albero binario

%Metodi di visita di un albero
I metodi di visita di un albero binario sono 3:
\begin{itemize}
  \item inorder: si visiona prima il sottoalbero sinistro poi il nodo e infine il sottoalbero destro
  \item preorder: si visiona prima il nodo poi i suoi sottoalberi
  \item postorder: si visionano prima i sottoalberi ed infine il nodo
\end{itemize}
Il primo metodo di visita viene usato soprattutto negli alberi binari di ricerca per
stampare gli elementi dell'albero in maniera crescente mentre in un albero binario
normale la scelta di quale metodo di visita utilizzare è inifluente e ogni programmatore
sceglie nell'utilizzo quale metodo di visita utilizzare per stampare l'albero.
Il cammino dalla radice ad un elemento foglia dell'albero richiede al massimo $O(h)$,
in cui $h$ è l'altezza dell'albero, in quanto richiede di scendere di livello fino
ad arrivare alle foglie, che si trovano al livello $h$.

La specifica di un albero binario, in cui ogni implementazione per essere valida deve prevedere:\newline
\textbf{Item} search(Tree T,Item k);\newline
\textbf{void} insert(Tree T,Item x);\newline
\textbf{Item} delete(Tree T,Item x);\newline
\textbf{Item} mininum(Tree T);\newline
\textbf{Item} maxinum(Tree T);\newline
\textbf{Item} predecessor(Tree T,Item x);\newline
\textbf{Item} successor(Tree T,Item x);\newline
%Capitolo 7 Alberi
%Capitolo sugli Alberi Binari di Ricerca
\chapter{Alberi Binari di Ricerca}
Dato un insieme di nodi in cui è definita una relazione d'ordine, si definisce come
\emph{albero di ricerca} un albero in cui tutti i nodi della radice sinistra sono
minori della radice e tutti i nodi a destra della radice sono maggiori e ogni sottoalbero
è anch'esso un albero di ricerca.

%FARE DISEGNO ALBERO BINARIO DI RICERCA

I metodi di visita di un albero visti nel capitolo precedente sono valide anche
per gli alberi binari di ricerca in particolare la visita $\proc{Inorder}$ consente
di mostrare i valori dell'albero ordinati in maniera corretta.

Le operazioni definite di solito in un albero binario di ricerca sono le seguenti:
\textbf{Item} search(Tree T,Item k);\newline
\textbf{void} insert(Tree T,Item z);\newline
\textbf{void} delete(Tree T,Item z);\newline
\textbf{Item} min(Tree T);\newline
\textbf{Item} maxinum(Tree T);\newline
\textbf{Item} predecessor(Tree T,Item x);\newline
\textbf{Item} successor(Tree T,Item x);

Tutte le seguenti operazioni richiedono un tempo $O(h)$ con $h$ indicante l'altezza
dell'albero per cui è meglio riuscire a mantenere il più possibile l'albero bilanciato
per ottenere l'altezza $h = \log n$.

L'operazione $\proc{search}$ permette di ricercare un elemento all'interno di un albero
binario di ricerca;per la proprietà di avere un ordinamento tra gli elementi dell'albero
la ricerca di un elemento ricalca quella della ricerca binaria per cui il pseudocodice è:
\begin{codebox}
\Procname{$\proc{BST-search}(Tree T,Item k)$}
\li $x \gets \attrib{T}{root}$
\li \If $x \isequal \const{nil}$ or $\attrib{x}{key} \isequal \attrib{k}{key}$
    \Then
\li                    \Return x
\li \If $\attrib{x}{key} < \attrib{k}{key}$
    \Then
\li                     \Return $\proc{Tree-search}(\attrib{T}{right},k)$
\li \Else \Return $\proc{Tree-search}(\attrib{T}{left},k)$
\end{codebox}

La ricerca di un elemento richiede la scansione dell'albero dalla radice alle foglie
che sappiamo essere pari a $O(h)$ con $h$ indicante l'altezza dell'albero.
La procedura $\proc{Tree-min}$ calcola il minimo dell'albero binario di ricerca sfruttando
il fatto che l'elemento minimo dell'albero si trova nel sottoalbero più a sinistra presente nell'albero
per cui lo pseudocodice è facilmente implementabile come:
%Pseudocodice minimo di un albero binario di ricerca
\begin{codebox}
\Procname{$\proc{BST-min}(Tree T)$}
\li \If $\attrib{T}{left} \isequal \const{nil}$
    \Then
\li                       \Return $T$
\li \Else \Return $\proc{Tree-min}(\attrib{T}{left})$
\end{codebox}

La scansione dell'albero per trovare il sottoalbero più a sinistra richiede
$O(h)$ con $h$ rappresentante l'altezza  dell'albero;in caso l'albero sia bilanciato il tempo sarebbe $O(\log n)$.
Simmetricamente e con il tempo di esecuzione uguale, è l'algoritmo di ricerca del massimo di un
albero di ricerca, in cui il massimo si trova nel sottoalbero più a destra:
%Pseudocodice Massimo di un albero binario di ricerca
\begin{codebox}
\Procname{$\proc{BST-max}(Tree T)$}
\li \If $\attrib{T}{right} \isequal \const{nil}$
    \Then
\li                        \Return T
\li \Else \Return $\proc{Tree-max}(\attrib{T}{right})$
\end{codebox}
Scandendo l'array andando sempre nel sottoalbero destro richiede un tempo $O(h)$,
come anche già visto per la ricerca del minimo, con $h$ indicante l'altezza dell'albero.

%Pseudocodice Successore di un albero binario di ricerca
\begin{codebox}
\Procname{$\proc{BST-successor}(Tree T,Item x)$}
\li \If $\attrib{x}{right} \neq \const{nil}$
    \Then
\li             \Return $\proc{Tree-min}(\attrib{T}{right})$
\li $y \gets \attrib{x}{p}$
\li \While $y \neq \const{nil}$ and $x \isequal \attrib{y}{right}$
    \Do
\li                $x \gets y$
\li                $y \gets \attrib{y}{p}$
    \End
\Return $y$
\end{codebox}

%Pseudocodice Predecessore di un albero binario di ricerca
\begin{codebox}
\Procname{$\proc{BST-predecessor}(Tree T,Item x)$}
\li \If $\attrib{x}{left} \neq \const{nil}$
    \Then
\li                     \Return $\proc{Tree-maxinum}(\attrib{T}{left})$
    \End
\li y $\gets \attrib{x}{p}$
\li \While $y \neq \const{nil}$ and $x \isequal \attrib{y}{left}$
    \Do
\li                $x \gets y$
\li                $y \gets \attrib{y}{p}$
    \End
\li \Return $y$
\end{codebox}


%Pseudocodice inserimento di un albero binario di ricerca
L'inserimento di un'elemento in un albero binario di ricerca prevede di effettuare
un cammino, considerando il valore dell'elemento, per trovare la prima posizione
nulla, che rispetta la proprietà di albero di ricerca, dove inserire l'elemento.\newline
Analogamente alle altre operazioni su alberi l'inserimento richiede $O(h)$ in un albero
di altezza $h$ e il suo pseudocodice è il seguente:
\begin{codebox}
\Procname{$\proc{BST-Insert}(\textbf{Tree} \ T,\textbf{Item} \ z)$}
\li $ y \gets \const{nil}$
\li $ x \gets \attrib{T}{root}$
\li \While $ x \neq \const{nil}$
    \Do
\li            $ y \gets x $
\li            \If $\attrib{x}{value} < \attrib{z}{value}$
               \Then
\li                           $ x \gets \attrib{x}{right}$
\li            \Else $x \gets \attrib{x}{left}$
    \End
\li $\attrib{z}{p} \gets y$
\li \If $y \isequal \const{nil}$
    \Then
\li        $\attrib{T}{root} \gets z$
\li \ElseIf $\attrib{y}{value} < \attrib{z}{value}$
    \Then
\li                  $\attrib{y}{right} \gets z$
\li \Else $\attrib{y}{left} \gets z$
\end{codebox}



%Pseudocodice Rimozione elemento di un albero binario di ricerca
%Capitolo 8 Alberi di Ricerca
%\input{Capitolo/dynamicprogramming}%Capitolo 7 Programmazione Dinamica
%\input{Capitolo/programmazionegreedy} %Capitolo 8 Programmazione Greedy
%Terzo Capitolo Strutture Dati Elementari


\end{document}
