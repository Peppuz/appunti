%Capitolo sulla tecnica Divide et Impera
\chapter{Divide et Impera}
Nello sviluppo dell'algoritmo $\proc{Insertion-Sort}$ abbiamo utilizzato una
struttura incrementale, detta anche iterativa, ma in informatica per sviluppare
gli algoritmi si può utilizzare una forma alternativa, chiamata \emph{Divide et Impera},
come verrà specificato in questo paragrafo.

Molti utili algoritmi sono ricorsivi, ossia per risolvere un particolare problema
chiamano se stessi su sottoproblemi dello stesso tipo.
Generalmente gli algoritmi ricorsivi adottano un approccio \textbf{divide et impera},
il quale prevede tre passi a ogni livello di ricorsione:
\begin{description}
    \item[Divide]il problema viene diviso in un certo numero di sottoproblemi, istanze
                  più piccole del problema.
    \item[Impera]i sottoproblemi vengono risolti in maniera ricorsiva
    \item[Combina]le soluzioni dei sottoproblemi vengono combinate per generare
                   la soluzione del problema originario
\end{description}

%Paragrafo sull'algoritmo mergeSort ed introduzione alla ricorsione
\section{MergeSort}
Nello sviluppo dell'algoritmo $\proc{Insertion-Sort}$ abbiamo utilizzato una
struttura incrementale, detta anche iterativa, ma in Informatica per sviluppare
gli algoritmi si può utilizzare una forma alternativa, chiamata \emph{Divide et Impera},
come specificato in questo paragrafo.

Molti utili algoritmi sono ricorsivi, ossia per risolvere un particolare problema
questi algoritmi chiamano se stessi per risolvere sottoproblemi dello stesso tipo.
Generalmente gli algoritmi ricorsivi adottano un approccio \textbf{divide et impera},
il quale prevede tre passi a ogni livello di ricorsione:
\begin{description}
    \item[Divide]:il problema viene diviso in un certo numero di sottoproblemi, istanze
                  più piccole del problema.
    \item[Impera]:i sottoproblemi vengono risolti in maniera ricorsiva
    \item[Combina]:le soluzioni dei sottoproblemi vengono combinate per generare
                   la soluzione del problema originario
\end{description}
Un esempio del paradigma divide et impera viene dato dall'algoritmo $\proc{Merge-Sort}$,
algoritmo che risolve il problema dell'ordinamento

L'algoritmo $\proc{Merge-Sort}$ opera seguendo il paradigma divide et impera:
\begin{description}
    \item[Divide]:divide la sequenza di $n$ elementi in due sottosequenze di $n/2$ elementi ciascuna.
    \item[Impera]:ordina le due sottosequenze in maniera ricorsiva mediante l'algoritmo $\proc{Merge-Sort}$.
    \item[Combina]:fonde le due sottosequenze ordinate per generare la sequenza ordinata.
\end{description}
Per effettuare la fusione utilizzo una procedura ausiliaria $\proc{Merge}(A,p,q,r)$,
dove $A$ è un array e $p,q,r$ sono degli indici tali che $p \leq q \leq r$ e la
procedura assume che i sottoarray $A[p \twodots q]$ e $A[q+1 \twodots r]$ siano ordinati
e li fonde per formare il sottoarray $A[p \twodots r]$ ordinato.
Utilizzo un elemento sentinella, elemento con un valore speciale tipo $\infty$,
per semplificare il nostro pseudocodice.
%Procedura Merge(A,p,q,r)
\begin{codebox}
    \Procname{$\proc{Merge}(A,p,q,r)$}
\li $n_1 \gets q - p + 1$
\li $n_2 \gets r - q$
\li crea due nuovi array $L[1 \twodots n_1]$ e $R[1 \twodots n_2]$
\li \For $i \gets 1$ \To $n_1$
         \Do
\li      $L[i] \gets A[p+i-1]$
        \End
\li \For $j \gets 1$ \To $n_2$
        \Do
\li      $R[j] \gets A[q+j]$
        \End
\li $L[n_1+1] = \infty$
\li $R[n_2+1] = \infty$
\li $i \gets 1$
\li $j \gets 1$
\li \For $k = p$ \To $r$
    \Do
\li      \If $L[i] \leq R[j]$
         \Then
\li              $A[k] = L[i]$
\li              $i = i+1$
         \End
\li      \Else $A[k] \gets R[j]$
\li            $j \gets j+1$
    \End
\end{codebox}

%Invariante di Ciclo con dimostrazione

%Costo procedura
la procedura $\proc{Merge}$ ha un costo $\omega(n)$, con $n = r-q+1$, in quanto
i tre cicli for presenti nell'algoritmo richiedono nel caso peggiore $n$ esecuzione
e non essendo annidati richiedono soltanto un tempo lineare di esecuzione.
Ora possiamo utilizzare le procedura merge nell'algoritmo $\proc{Merge-Sort}$, il quale
ordina gli elementi nel sottoarray $A[p \twodots r]$.
In caso $p \geq r$, il sottoarray ha al massimo un elemento e, quindi, è già ordinato;
altrimenti il passo "Divide" calcola semplicemente un indice $q$ che separa il sottoarray
in due sottoarray di $n/2$ elementi, come mostrato dal pseudocodice:
%Procedura MergeSort(A,p,r)$
\begin{codebox}
    \Procname{$\proc{Merge-Sort}(A,p,r)$}
\li \If $p < r$
    \Then
\li     $q \gets (p+r)/ 2$
\li     $\proc{Merge-Sort}(A,p,q)$
\li     $\proc{Merge-Sort}(A,q+1,r)$
\li     $\proc{Merge}(A,p,q,r)$
    \End
\end{codebox}

Per ordinare l'intera sequenza $A = (A[1],A[2],\dots,A[n])$ effettuiamo la chiamata
iniziale $\proc{Merge-Sort}(A,1,\attrib{A}{length})$

%Analisi algoritmo Divide et Impera
\section{Analisi algoritmi divide et impera}
In caso un algoritmo contiene una chiamata a se stesso, il suo tempo di esecuzione
spesso può essere espresso mediante un'\textbf{equazione di ricorrenza}, in cui
si esprime il tempo di esecuzione totale in funzione del tempo di esecuzione dei sottoproblemi.

Una ricorrenza per il tempo di esecuzione di un algoritmo divide et impera si basa
sui 3 passi del paradigma di base; se la dimensione del problema è sufficientemente piccola,
per esempio $n \leq c$ per qualche costante $c$, allora il tempo di esecuzione è
costante, indicato con $\omega(1)$.
In caso contrario serve un tempo $aT(n/b)$ per risolvere i sottoproblemi, con $a$
indicante il numero di sottoproblemi generati e $b$ indicante il rapporto di grandezza
tra il problema e i sottoproblemi, ed indicando con $D(n)$ il tempo per dividere
i sottoproblemi e indicando con $C(n)$ il tempo per combinare le soluzioni dei
sottoproblemi si ottiene la seguente ricorrenza
\begin{equation*}
    T(n) = \begin{cases} \omega(1) \text{se} n \leq c \\
                         aT(n/b) + D(n) + C(n) \text{negli altri casi}\\
            \end{cases}
\end{equation*}

\subsection{Analisi di Merge Sort}
Lo pseudocodice di $\proc{Merge-Sort}$ funziona per qualsiasi dimensione ma, al fine
di agevolare i calcoli, supponiamo che la dimensione del problema originale sia una
potenza di 2, per cui ogni passo divide genera due sottosequenze di dimensione pari a $n/2$.

L'algoritmo merge sort se applicato a un solo elemento impiega un tempo costante $\omega(1)$,
altrimenti suddividiamo il tempo di esecuzione nel seguente modo:
\begin{description}
    \item[Divide]:questo passo semplicemente calcola il centro del sottoarray quindi
                  richiede un tempo costante $\omega(1)$
    \item[Impera]:risolviamo in maniera ricorsiva i due sottoproblemi di dimensione $n/2$
                  e ciò richiede $2T(n/2)$ per la risoluzione
    \item[Combina]:la procedura $\proc{Merge}$ richiede un tempo $\omega(n)$
\end{description}
Quando sommiamo $\omega(1)$ e $\omega(n)$ otteniamo una funzione lineare che è $\omega(n)$
e per cui l'equazione di ricorrenza di $\proc{Merge-Sort}$ è:
\begin{equation*}
    T(n) = \begin{cases} \Theta(1) \text{se} n = 1 \\
                         2T(n/2) + \Theta(n) \text{se} n > 1 \\
           \end{cases}
\end{equation*}
Per risolvere codesta equazione di ricorrenza vi sono 3 modalità, come verrà spiegato
nel seguente capitolo.
%Algoritmo Merge Sort

%%Algoritmo per trovare il massimo e il minimo in maniera simultanea
 Algoritmo per trovare il massimo e minimo simultaneo con Divide et Impera
Per risolvere l'equazione di ricorrenza, funzione che descrive il tempo di esecuzione
in funzione del tempo di esecuzione dei sottoproblemi, vi sono 4 metodi:
\begin{description}
    \item[Metodo di Sostituzione]:ipotiziamo un tempo di esecuzione e utilizziamo
                l'induzione matematica per dimostrare la correttezza dell'ipotesi
    \item[Metodo dell'Esperto]:fornisce i limiti dell'equazioni di ricorrenza che
          rispettano determinate condizioni(Analizzato nei prossimi paragrafi)
    \item[Metodo di espansione] si espande l'equazione di ricorrenza fino ad arrivare ai casi base
          ad esempio $T(n) = T(n-1) + 3$ si espande $T(n-1),T(n-2)$ fino ad arrivare a $T(1)$.
\end{description}

%paragrafo sul metodo di Sostituzione per risolvere le ricorrenze
\section{Il metodo di Sostituzione}
Il metodo di sostituzione è un tecnica di risoluzione delle equazioni di ricorrenza
degli algoritmi divide et impera, che richiede due passi:
\begin{enumerate}
    \item Ipotizzare la forma della risoluzione.
    \item Usare l'induzione matematica per trovare le costanti e dimostrare che
          la soluzione proposta funziona ed è corretta.
\end{enumerate}
Per vedere il funzionamento di questo metodo proviamo ad usarlo per mostrare il seguente teorema
%Dimostrazione
\begin{thm}
    $T(n) = 2T(n/2) + n = \Theta(n \ln n)$
\end{thm}
\begin{proof}
Per dimostrare che $T(n) = \Theta(n \ln n)$ bisogna provare che $T(n) \leq cn \ln n$ per una costante $c > 0$
\begin{equation*}
\begin{split}
T(n) & \leq 2(cn/2 \lg n/2) + n \\
     & \leq cn \lg n - cn \lg 2 + n\\
     & \leq cn \lg n - cn + n \\
     & \leq cn \lg n \text{per} c \geq 1\\
\end{split}
\end{equation*}
L'induzione matematica richiede di verificare i casi base ma solitamente le condizioni di contorno non vengono dimostrate in quanto può essere molto difficile 
in alcuni casi e per risolvere si sfrutta la notazione asintotica, che prevede una costante $n_0$ arbitrariemente scelta e con questo si può rimuovere le condizioni
di contorno difficili da dimostrare.
\end{proof}

%Scegliere una buona ipotesi
Per scegliere una buona ipotesi da verificare mediante il metodo di sostituzione
richiede fantasia, esperienza, cosa che non sempre si dispone soprattutto la fantasia.\newline
Per aiutarci ad ottenere una buona ipotesi si potrebbe utilizzare l'albero di sostituzione
e poi dimostrare l'ipotesi mediante induzione, con il metodo di sostituzione, oppure
ci sono delle euristiche per diventare dei buoni indovini.

Le euristiche per formulare buone ipotesi sono le seguenti:
\begin{itemize}
    \item Se una ricorrenza è simile ad una già analizzata conviene provare a dimostrare
          la stessa soluzione come ad esempio:
          \begin{equation*}
              T(n) = 2T(\frac{n}{2} + 17) + n \text{è simile all'equazione precedente ed è un} \Theta(n \lg n)
          \end{equation*}
    \item Si inizia a dimostrare dei limiti superiori ed inferiori molto larghi e
          poi restringere l'incertezza alzando il limite inferiore ed abbassando
          il limite superiore fino a convergere con il risultato corretto.
\end{itemize}

%esempi

%Errori tipici e togliere termini di ordine inferiore
Ci sono dei casi in cui ipotiziamo correttamente un limite asintotico per la ricorrenza
ma in qualche modo sembra che i calcoli matematici non tornino nell'Induzione.
Per superare questo ostacolo spesso basta correggere l'ipotesi sottraendo un termine
di ordine inferiore per far quadrare i conti, come nell'esempio:
\[    T(n) = T(\lfloor n/2 \rfloor) + T(\lceil n/2 \rceil) + 1 \]
Supponiamo che $T(n) = O(n)$ ossia $T(n) \leq cn$, otteniamo nella ricorrenza:
\[ \begin{split}
    T(n) & \leq c \lfloor n/2 \rfloor + c \lceil n/2 \rceil + 1 \\
         & = cn + 1
 \end{split} \]
Questa equazione non implica che $T(n) \leq cn$ qualunque sia il valore di $c$ però
l'intuizione che $T(n) = O(n)$ è corretta solo che per provarla dobbiamo utilizzare
un ipotesi induttiva più forte, ossia $T(n) \leq cn - d$ con $d \geq 0$ rappresentante una costante.
\[ \begin{split}
    T(n) & \leq (c \lfloor n/2 \rfloor - d) + (c \lceil n/2 \rceil -d) + 1 \\
         & \leq cn - 2d + 1 \\
         & \leq cn - d \quad \text{per ogni} d \geq 1 \\
 \end{split} \]
A volte una piccola manipolazione algebrica può rendere una ricorrenza ignota simile a una che abbiamo già visto ed analizzato per esempio 
\[ T(n) = 2T(\lfloor \sqrt{n} \rfloor) + \log n \]
sembra difficile da risolvere ma, ignorando l'arrotondamento agli interi di valori come $\sqrt{n}$, ponendo $m = \log n$ si ottiente
\[ T(2^m) = 2T(2^{\frac{m}{2}}) + m \]
Ponendo ora $S(m) = T(2^m)$ si ottiene la ricorrenza $S(m) = 2S(\frac{m}{2}) + m$ che è simile alla ricorrenza analizzata nel mergeSort.
%Paragrafo sul metodo di sostituzione

%Paragrafo sul Teorema dell'Esperto
\section{Teorema dell'Esperto}
%Paragrafo sul metodo dell'Esperto
