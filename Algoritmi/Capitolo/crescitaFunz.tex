\chapter{Crescita delle Funzioni}
Nel valutare l'algoritmo $\proc{min}$ si ottiene una funzione $T(n) = an + b$ che
è una funzione lineare.\newline
Durante la valutazione di un algoritmo difficilmente si riesce a quantificare con
esattezza le costanti coinvolte per cui si analizza il comportamento della funzione
al tendere di n all'infinito.\newline
Al tal fine si utilizzano le notazioni $O,\ \Omega,\ \Theta\ $ definite come segue:

\begin{itemize}
  \item $f(n) \in O(g(n)) \Leftrightarrow\ \exists c \geq 0\ \exists m \geq 0 : f(n) \leq cg(n)\ \forall n \geq m$
  \item $f(n) \in \Omega(g(n)) \Leftrightarrow \exists c, m \geq 0 : f(n) \geq cg(n) \forall n \geq m$
  \item $f(n) \in \Theta(g(n)) \Leftrightarrow \exists c_1,c_2,m \geq 0 :
        c_1g(n) \leq f(n) \leq cg(n)\ \forall n \geq m$
\end{itemize}
Per maggiore chiarezza si utilizza un'abuso di linguaggio scrivendo $f(n) = O(g(n))$
al posto di $f(n) \in O(g(n))$\newline
Per poter definire se una funzione appartiene a una notazione bisogna mostrarlo attraverso
una dimostrazione formale con l'utilizzo dell'induzione matematica.\newline

Esempio: la funzione $f(n) = 4n^2 + 4n - 1 \in O(n^2)$ va dimostrata attraverso induzione\newline
\textbf{Caso Base:} $n = 1$\newline
Per $n = 1$ si ha $4 + 4 -1 \leq c * 1$ che è vera per $0 \leq c \leq 7$\newline

\textbf{Ipotesi Induttiva}:
%FINIRE dimostrazione

\section{Proprietà Notazioni $O$}
La notazione $O$ possiede le seguenti proprietà:
\begin{enumerate}
  \item Riflessività: $\forall c\ cf(n) = O(f(n))$(lo stesso vale per $\Theta ed \Omega)$
  \item Transività:$f(n) = O(g(n)) \land g(n) = O(h(n)) \Leftarrow f(n) = O(h(n))$
  \item Simmetrià trasposta:$f(n) = O(g(n)) \Leftrightarrow g(n) = \Omega f(n)$
  \item Somma: $f(n) + g(n) = O(\max\{f(n),g(n)\})$
  \item Prodotto: $f(n) = O(h(n)) \land g(n) = O(q(n)) \Leftrightarrow f(n)g(n) = O(h(n)q(n))$
\end{enumerate}

%Da fare le dimostrazioni

\section{Proprietà Notazione $\Omega$}
La notazione $\Omega$ possiede le seguenti proprietà:
\begin{enumerate}
  \item Riflessività: $\forall c\ cf(n) = \Omega(f(n))$
  \item Transività: $f(n) = \Omega(g(n)) \land g(n)= \Omega(h(n)) \Leftarrow f(n) = \Omega(h(n))$
  \item Simmetrià trasposta: $f(n) = \Omega g(n) \Leftrightarrow g(n) = O(f(n))$
  \item Somma: $f(n) + g(n) = \Omega(\max\{f(n),g(n))\})$
  \item Prodotto: $f(n) = \Omega(h(n)) \land g(n) = \Omega(q(n)) \Leftrightarrow f(n)g(n) = \Omega(h(n)q(n))$
\end{enumerate}

%Da fare le dimostrazioni

\section{Proprietà Notazione $\Theta$}
La notazione $\Theta$ possiede le seguenti proprietà:
\begin{enumerate}
  \item Riflessività: $\forall c\ cf(n) = \Theta(f(n))$
  \item Transività: $f(n) = \Theta(g(n)) \land g(n) = \Theta(h(n)) \Leftarrow f(n)= \Theta(h(n))$
  \item Simmetria: $f(n) = \Theta(g(n)) \Leftrightarrow g(n) = \Theta(f(n))$
  \item Somma: $f(n) + g(n) = \Theta(\max\{f(n),g(n))\})$
  \item Prodotto: $f(n) = \Theta(h(n)) \land g(n) = \Theta(q(n)) \Leftrightarrow f(n)g(n) = \Theta(h(n)q(n))$
\end{enumerate}

$\Theta$ implementa una relazione di equivalenza sulle funzioni in quanto
sono rispettate la Riflessività,Transività e la Simmetria.

%Da fare le dimostrazioni
