%Algoritmo LinearSearch
Il primo algoritmo di ricerca analizzato è il $\proc{linearSearch}$ che risolve:\newline
\begin{description}
  \item[Input:] una sequenza di valori interi $A[1 \twodots n]$ e un valore intero $\id{key}$
  \item[Output:] un indice $i$ tale che $A[i] \gets \id{key}$ altrimenti ritorna $\const{nil}$
\end{description}

%pseudocodice
\begin{codebox}
\Procname{$\proc{linearSearch}(A,\id{key})$}
\li \For $j \gets 1$ \To $\attrib{A}{length}$
    \Do
\li               \If $A[j] \isequal \id{key}$
                  \Then
\li                         \Return $j$
                  \End
    \End
\li \Return $\const{nil}$
\end{codebox}
Il tempo di esecuzione del $\proc{linearSearch}$ nei diversi casi è il seguente:
\begin{description}
  \item[Caso migliore:] l'elemento $\id{key}$ viene trovato direttamente nel primo elemento
        per cui viene risolto in un tempo costante $\Omega(1)$
  \item[Caso peggiore:] l'elemento $\id{key}$ non viene trovato nella sequenza quindi
        \begin{equation*}
           T(n) = c(n+1) + cn + c = O(n)
        \end{equation*}
\end{description}
