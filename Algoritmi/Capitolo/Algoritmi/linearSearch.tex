%Algoritmo LinearSearch
Il primo algoritmo di ricerca analizzato è il $\proc{linearSearch}$ che risolve:\newline
\begin{description}
  \item[Input:] una sequenza di valori interi $A[1 \twodots n]$ e un valore intero $\id{key}$
  \item[Output:] un indice $i$ tale che $A[i] \gets \id{key}$ altrimenti ritorna $\const{nil}$
\end{description}
La ricerca lineare analizza tutti gli elementi fino a quando non trova l'elemento da
ricercare per cui lo pseudocodice è il seguente:
%pseudocodice
\begin{codebox}
\Procname{$\proc{linearSearch}(A,\id{key})$}
\li \For $j \gets 1$ \To $\attrib{A}{length}$
    \Do
\li               \If $A[j] \isequal \id{key}$
                  \Then
\li                         \Return $j$
                  \End
    \End
\li \Return $\const{nil}$
\end{codebox}

%Dimostrazione correttezza dell'algoritmo
L'invariante di ciclo della ricerca lineare afferma che al termine del ciclo for
si abbia il valore dell'indice nella sequenza del valore ricercato:
\begin{description}
  \item[Inizializzazione]:nella prima iterazione del ciclo for, per $j = 1$ si ha
        $A$ formato da un solo elemento che è ovviamente ordinato.
  \item[Conservazione]:ad ogni iterazione del ciclo si ha che il valore dell'indice
                       dell'elemento da trovare è $\const{nil}$ a meno che nell'esecuzione
                       del ciclo venga trovato l'elemento e sia l'indice uguale a $j$.
  \item[Conclusione]:al termine del ciclo, con $j = n$ si ha che l'indice del valore
                     ricercato è $j$ in caso sia stato trovato l'elemento altrimenti $\const{nil}$.
\end{description}

Il tempo di esecuzione del $\proc{linearSearch}$ nei diversi casi è il seguente:
\begin{description}
  \item[Caso migliore:] l'elemento $\id{key}$ viene trovato direttamente nel primo elemento
        per cui viene risolto in un tempo costante $\Omega(1)$
  \item[Caso peggiore:] l'elemento $\id{key}$ non viene trovato nella sequenza quindi
        \begin{equation*}
           T(n) = c(n+1) + cn + c = O(n)
        \end{equation*}
\end{description}
