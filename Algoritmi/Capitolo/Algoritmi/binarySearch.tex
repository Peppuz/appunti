%Analisi Algoritmo binarySearch
Il secondo algoritmo di ricerca è il $\proc{binarySearch}$, algoritmo divide et impera
in cui viene già previsto che la sequenza di valori sia ordinata e sfruttando ciò
ad ogni passo viene eliminata una parte della sequenza.\newline
La ricerca binaria ad ogni passo controlla l'elemento mediano per vedere se è il valore
ricercato altrimenti richiama l'algoritmo sulla sottosequenza sinistra in caso il valore
da trovare sia minore oppure lo effettua sulla sottosequenza destra.

%Pseudocodice binarySearch
\begin{codebox}
\Procname{$\proc{binarySearch}(A,\id{left},\id{right},\id{key})$}
\li \If $\id{left} \isequal \id{right}$
    \Then
\li                \If $A[\id{left}] \isequal \id{key}$
                   \Then
\li                            \Return $\id{left}$
\li                \Else \Return $\const{nil}$
    \End
\li \Else
\li                $\id{mid} \gets (\id{left} + \id{right}) / 2$
\li                \If $A[\id{mid}] \isequal \id{key}$
\li                   \Then \Return $\id{mid}$
                   \End
\li                \If $A[\id{mid}] > \id{key}$
                   \Then
\li                          \Return $\proc{binarySearch}(A,\id{left},\id{mid}-1,\id{key})$
\li                \Else \Return $\proc{binarySearch}(A,\id{mid}+1,\id{right},\id{key})$
    \End
\end{codebox}

Il tempo di esecuzione del $\proc{binarySearch}$ nei diversi casi è il seguente:
\begin{description}
  \item[Caso migliore:] l'elemento $\id{key}$ viene trovato direttamente nell'elemento mediano
        per cui viene risolto in un tempo costante $\Omega(1)$
  \item[Caso peggiore:] l'elemento $\id{key}$ non viene trovato nella sequenza quindi
        \begin{equation*}
           T(n) = \begin{cases} 3c = \Theta(1) \quad \text{se} \ n = 1 \\
                                T(\frac{n}{2}) + 4c \quad \text{se} \ n > 1\\
                  \end{cases}
        \end{equation*}
        Applicando il secondo caso del teorema dell'esperto in quanto $4c = \Theta(n^{\log _2 1}) = \Theta(1)$
        si ha che $T(n) = \Theta(\log n)$
\end{description}
