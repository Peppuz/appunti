%Paragrafo sull'algoritmo mergeSort ed introduzione alla ricorsione
\section{MergeSort}
L'algoritmo $\proc{Merge-Sort}$, che risolve il problema dell'ordinamento, opera seguendo il paradigma divide et impera:
\begin{description}
    \item[Divide]divide la sequenza di $n$ elementi in due sottosequenze di $n/2$ elementi ciascuna.
    \item[Impera]ordina le due sottosequenze in maniera ricorsiva mediante l'algoritmo $\proc{Merge-Sort}$.
    \item[Combina]fonde le due sottosequenze ordinate per generare la sequenza ordinata.
\end{description}
Per effettuare la fusione utilizzo una procedura ausiliaria $\proc{Merge}(A,\id{left},\id{mid},\id{right})$,
dove $A$ è un array e $\id{left},\id{mid},\id{right}$ sono degli indici tali che
$\id{left} \leq \id{mid} \leq \id{right}$ e la procedura assume che i sottoarray
$A[\id{left} \twodots \id{mid}]$ e $A[\id{mid}+1 \twodots \id{right}]$ siano ordinati
e li fonde per formare il sottoarray $A[\id{left} \twodots \id{right}]$ ordinato.
Utilizziamo un elemento sentinella, elemento con un valore speciale tipo $\infty$,
per semplificare il nostro pseudocodice.
%Procedura Merge(A,p,q,r)
\begin{codebox}
    \Procname{$\proc{Merge}(A,\id{left},\id{mid},\id{right})$}
\li $n_1 \gets q - p + 1$
\li $n_2 \gets r - q$
\li crea due nuovi array $L[1 \twodots n_1]$ e $R[1 \twodots n_2]$
\li \For $i \gets 1$ \To $n_1$
         \Do
\li      $L[i] \gets A[p+i-1]$
        \End
\li \For $j \gets 1$ \To $n_2$
        \Do
\li      $R[j] \gets A[q+j]$
        \End
\li $L[n_1+1] = \infty$
\li $R[n_2+1] = \infty$
\li $i \gets 1$
\li $j \gets 1$
\li \For $k = \id{left}$ \To $\id{right}$
    \Do
\li      \If $L[i] \leq R[j]$
         \Then
\li              $A[k] = L[i]$
\li              $i = i+1$
         \End
\li      \Else $A[k] \gets R[j]$
\li            $j \gets j+1$
    \End
\end{codebox}

%Invariante di Ciclo con dimostrazione

%Costo procedura
la procedura $\proc{Merge}$ ha un costo $\Theta(n)$, con $n = \id{right} - \id{mid} + 1$, in quanto
i tre cicli for presenti nell'algoritmo richiedono nel caso peggiore $n$ iterazioni
e non essendo annidati richiedono soltanto un tempo lineare di esecuzione.
Ora possiamo utilizzare le procedura merge nell'algoritmo $\proc{Merge-Sort}$, il quale
ordina gli elementi nel sottoarray $A[\id{left}\twodots \id{right}]$.
In caso $\id{left} \geq \id{right}$, il sottoarray ha al massimo un elemento e, quindi, è già ordinato;
altrimenti il passo "Divide" calcola semplicemente un indice $q$ che separa il sottoarray
in due sottoarray di $n/2$ elementi, come mostrato dal pseudocodice:
%Procedura MergeSort(A,p,r)$
\begin{codebox}
    \Procname{$\proc{Merge-Sort}(A,\id{left},\id{right})$}
\li \If $\id{left} < \id{right}$
    \Then
\li     $\id{mid} \gets (\id{left} + \id{right})/ 2$
\li     $\proc{Merge-Sort}(A,\id{left},\id{mid})$
\li     $\proc{Merge-Sort}(A,\id{mid} + 1,\id{right})$
\li     $\proc{Merge}(A,\id{left},\id{mid},\id{right})$
    \End
\end{codebox}

Per ordinare l'intera sequenza $A = (A[1],A[2],\dots,A[n])$ effettuiamo la chiamata
iniziale $\proc{Merge-Sort}(A,1,\attrib{A}{length})$

%Analisi algoritmo Divide et Impera
\section{Analisi algoritmi divide et impera}
In caso un algoritmo contiene una chiamata a se stesso, il suo tempo di esecuzione
spesso può essere espresso mediante un'\textbf{equazione di ricorrenza}, in cui
si esprime il tempo di esecuzione totale in funzione del tempo di esecuzione dei sottoproblemi.

Una ricorrenza per un algoritmo divide et impera si basa sui 3 passi del paradigma di base;
se la dimensione del problema è sufficientemente piccola,
per esempio $n \leq c$ per qualche costante $c$, allora il tempo di esecuzione è
costante, indicato con $\Theta(1)$.
In caso contrario serve un tempo $aT(n/b)$ per risolvere i sottoproblemi, con $a$
indicante il numero di sottoproblemi generati e $b$ indicante il rapporto di grandezza
tra il problema e i sottoproblemi, ed indicando con $D(n)$ il tempo per dividere
i sottoproblemi e indicando con $C(n)$ il tempo per combinare le soluzioni dei
sottoproblemi si ottiene la seguente ricorrenza
\begin{equation*}
    T(n) = \begin{cases} \Theta(1) \quad \text{se} \ n \leq c \\
                         aT(n/b) + D(n) + C(n) \quad \text{se} \ n > c\\
            \end{cases}
\end{equation*}

\subsection{Analisi di Merge Sort}
Lo pseudocodice di $\proc{Merge-Sort}$ funziona per qualsiasi dimensione ma, al fine
di agevolare i calcoli, supponiamo che la dimensione del problema originale sia una
potenza di 2, per cui ogni passo divide genera due sottosequenze di dimensione pari a $n/2$.

L'algoritmo merge sort se applicato a un solo elemento impiega un tempo costante $\Theta(1)$,
altrimenti suddividiamo il tempo di esecuzione nel seguente modo:
\begin{description}
    \item[Divide]:questo passo semplicemente calcola il centro del sottoarray quindi
                  richiede un tempo costante $\Theta(1)$
    \item[Impera]:risolviamo in maniera ricorsiva i due sottoproblemi di dimensione $n/2$
                  e ciò richiede $2T(n/2)$ per la risoluzione
    \item[Combina]:la procedura $\proc{Merge}$ richiede un tempo $\Theta(n)$
\end{description}
Quando sommiamo $\Theta(1)$ e $\Theta(n)$ otteniamo una funzione lineare che è $\Theta(n)$
e per cui l'equazione di ricorrenza di $\proc{Merge-Sort}$ è:
\begin{equation*}
    T(n) = \begin{cases} \Theta(1) \quad \text{se} \ n = 1 \\
                         2T(\frac{n}{2}) + \Theta(n) \quad \text{se} \ n > 1 \\
           \end{cases}
\end{equation*}
Per risolvere codesta equazione di ricorrenza vi sono 3 modalità , che saranno poi analizzate,
però comunque il tempo di esecuzione nel caso peggiore è $O(n \log n)$.
