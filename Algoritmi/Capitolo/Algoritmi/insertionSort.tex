%Paragrafo sull'algoritmo dell'Insertion Sort
\section{InsertionSort}
L'algoritmo $\proc{Insertion-Sort}$ risolve il problema dell'ordinamento definito come:
\textbf{Input}:una sequenza di $n$ numeri $(a_1,a_2,\dots,a_n)$ \newline
\textbf{Output}:una permutazione $(a'_1,a'_2,\dots,a'_n)$ tale che $a'_1,a'_2,\dots,a'_n$

%algoritmo insertionSort
\begin{codebox}
    \Procname{$\proc{Insertion-Sort}(A)$}
\li \For $j \gets 2$ \To $\attrib{A}{length}$
    \Do
\li            $\id{key} \gets A[j]$
\li         $i \gets j-1$
\li         \While $i > 0$ and $A[i] > id{key}$
            \Do
\li                $A[i+1] \gets A[i]$
\li                $i \gets i-1$
            \End
\li         $A[i+1] \gets \id{key}$
    \End
\end{codebox}

L'algoritmo $\proc{Insertion-Sort}$ è un algoritmo efficiente per ordinare un ristretto
numero di elementi.\newline
L'algoritmo $\proc{Insertion-Sort}$ prende una carta alla volta ed effettua l'ordinamento
cosicchè ogni volta che si ha un sequenza di oggetti la si ha ordinata ed aggiungendo
un nuovo elemento si effettua ancora l'ordinamento.\newline
Questo algoritmo $\proc{Insertion-Sort}$ è la soluzione più naturale ed utilizzata
dalle persone umane per effettuare l'ordinamento di una sequenza, ad esempio di Numeri e di Carte.
Per poter affermare che l'algoritmo è corretto e risolve il problema dell'ordinamento
bisogna dimostrare l'invariante del ciclo $\kw{For}$.

%Dimostrazione Invariante di Ciclo
Invariante di Ciclo: All'inizio di ogni iterazione del ciclo $\For$, linee $1-8$,
il sottoarray $A[1\twodots j-1]$ è ordinato.

%Effettuare Dimostrazione

L'analisi del tempo di esecuzione dell'algoritmo $\proc{Insertion-Sort}$ è il seguente:
\begin{codebox}
    \Procname{$\proc{Insertion-Sort}(A)$} \>\>\> costo \quad numero di volte
\li \For $j \gets 2$ \To $\attrib{A}{length}$\>\>\>$c_1$ \quad $n$
    \Do
\li            $\id{key} \gets A[j]$\>\>$c_2$ \quad $n-1$
\li            $i \gets j-1$\>\>\>$c_3$ \quad $n-1$
\li         \While $i > 0$ and $A[i] > id{key}$\>\>$c_4$ $\sum_{j=2}^n (t_j)$
            \Do
\li                $A[i+1] \gets A[i]$\>\>$c_5$ $\sum_{j=2}^n (t_j -1)$
\li                $i \gets i-1$\>\>$c_6$ $\sum_{j=2}^n (t_j -1)$
            \End
\li         $A[i+1] \gets \id{key}$\>\>\>$c_7$ \quad $n-1$
    \End
\end{codebox}

Il tempo di esecuzione dell'algoritmo è la somma dei tempi di esecuzione per ogni
istruzione eseguita quindi il tempo di esecuzione di $\proc{Insertion-Sort}$ è:
\begin{equation*}
    T(n) = c_1n + c_2(n-1) + c_3(n-1) + c_4 \sum_{j=2} ^n t_j + c_5 \sum_{j=2} ^n (t_j -1)
         + c_6 \sum_{j=2} ^n (t_j -1) + c_7(n-1)
\end{equation*}
In caso l'algoritmo sia già ordinato, caso migliore, si avrebbe sempre $A[i] < \id{key}$
quindi $t_j$ è sempre $1$, per cui il tempo di esecuzione sarebbe:
\begin{align}
    T(n) = c_1n + c_2(n-1) + c_3(n-1) + c_4(n-1) + c_7(n-1) \\
         = (c_1 + c_2 + c_3 + c_4 + c_5)n - (c_2 + c_3+ c_4+ c_7) \\
\end{align}
In caso si abbia una sequenza decrescente, caso peggiore, nel ciclo While bisogna
confrontare ogni elemento $A[j]$ con il sottoarray $A[1 \twodots j-1]$ per cui $t_j = j$
per $j=2,3,\dots,n$ e poiche si ha
\begin{align}
    \sum _{j=2} ^ n j = \frac{n(n+1)}{2} -1 & \sum_{j=2}^n (j-1) = \frac{n(n-1)}{2}
\end{align}
il tempo dell'algoritmo $\proc{Insertion-Sort}$ nel caso peggiore è il seguente:
\begin{align}
    T(n) & = c_1n + c_2(n-1) + c_3(n-1) + c_4 (\frac{n(n+1)}{2} -1) + c_5 (\frac{n(n-1)}{2})
         + c_6 (\frac{n(n-1)}{2}) + c_7(n-1) \\
         & = c_1n + c_2(n-1) + c_3(n-1) + c_4(\frac{n^2+n-2}{2}) + c_5(\frac{n^2-n}{2})
           + c_6(\frac{n^2-n}{2}) + c_7(n-1) \\
         & = (c_4/2 +c_5/2 + c_6/2)n^2 + (c_1+c_2+c_3+c_4/2-c_5/2-c_6/2+c_7)n
           -(c_2+c_3+c_4+c_7)
\end{align}
Il tempo dell'algoritmo può essere scritto come $an^2 + bn + c$ che è una funzione
quadratica che viene indicata, nel caso peggiore come $\omega(n^2)$.
