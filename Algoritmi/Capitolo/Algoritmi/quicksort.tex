%Algoritmo QuickSort per risolvere il problema dell'ordinamento
\chapter{QuickSort}
Il $\proc{Quick-Sort}$ è un algoritmo divide et impera in loco ossia senza utilizzare
una struttura di appoggio per effettuare l'ordinamento e funziona in maniera ottimale
nell'implementazione sui calcolatori attuali.

L'algoritmo $\proc{Quick-Sort}$ esegue i seguenti passi divide et impera:
\begin{description}
  \item[Divide]:riarrangia l'array $A[p\twodots r]$ in due sottoarray, eventualmente nulli,
                $A[p \twodots q-1]$ e $A[q+1 \twodots r]$ tali che tutti gli elementi
                del primo sottoarray sono minori o uguali a $A[q]$ e tutti gli elementi
                del secondo sottoarray sono maggiori o uguali a $A[q]$.\newline
                Calcolare l'indice di $q$ viene effettuato nella procedura di riarrangiamento.
  \item[Impera]:ordina ricorsivamente i due sottoarray $A[p \twodots q-1]$ e $A[q+1 \twodots r]$.
  \item[Combina]:dato che i due sottoarray sono già ordinati per cui non viene eseguito nulla.
\end{description}

L'algoritmo $\proc{Quick-Sort}$ è il seguente
%Pseudocodice Quicksort
\begin{codebox}
\Procname{$\proc{Quick-Sort}(A,p,r)$}
\li \If $p < r$
    \Then
\li           $q = \proc{Partition}(A,p,r)$
\li           $\proc{Quick-Sort}(A,p,q-1)$
\li           $\proc{Quick-Sort}(A,q+1,r)$
    \End
\end{codebox}
Per effettuare l'ordinamento di un array $A$ viene effettuata la chiamata iniziale
$\proc{Quick-Sort}$(A,1,$\attrib{A}{length}$).\newline
La chiave dell'algoritmo è la procedura $\proc{Partition}$ implementato come:
%Pseudocodice Procedura Partition
\begin{codebox}
\Procname{$\proc{Partition}(A,p,r)$}
\li $\id{pivot} \gets A[r]$
\li $\id{indice} \gets p-1$
\li \For $j \gets p$ \To $r$
    \Do
\li               \If $A[j] \leq \id{pivot}$
                  \Then
\li                              $\id{indice} \gets id{indice} + 1$
\li                              scambia $A[id{indice}]$ con $A[j]$
                  \End
    \End
\li scambia $A[indice+1]$ con $A[r]$
\li \Return $\id{indice} + 1$
\end{codebox}

%Invariante di Ciclo da fare!!!!

Il tempo di esecuzione della procedura $\proc{Partition}$ è il seguente:
$T(n) = c_1 + c_2 + c_3(n+1) + c_4 + c_5(t_{if}) + c_6(t_{if}) + c_7 + c_8$.

\begin{description}
  \item[Caso migliore]:tutti gli elementi sono maggiori del pivot per cui $t_{if} = 0$
        $T(n) = c_3n + (c_1+c_2+c_3+c_4+c_7+c_8) = \Theta(n)$
  \item[Caso peggiore]:tutti gli elementi sono inferiori del pivot per cui $t_{if} = 1$
        $T(n) = c_3n + (c_1+c_2+c_3+c_4+c_5+c_6+c_7+c_8) = \Theta(n)$
\end{description}

\section{Analisi tempo QuickSort}
L'analisi del tempo di esecuzione del $\proc{Quick-Sort}$ è in base al fatto se
la partizione dell'array è bilanciata o meno.
\begin{description}
  \item[Caso peggiore]:la procedura di partizione produce due sottoproblemi:
        uno di $n-1$ elementi e l'altro di $0$ elementi.
        $T(n) = T(n-1) + T(0) + \Theta(n)
              = T(n-1) + \Theta(n)$
        Attraverso il metodo di sostituzione arrivo a $T(n) = O(n^2)$
  \item[Caso migliore]:la procedura di partizione produce due sottoproblemi di $\lceil n/2 \rceil$
        e $\lfloor n/2 \rfloor$ elementi per cui, ignorando le condizioni di ceil e floor,
        il tempo di esecuzione è $T(n) = 2T(n/2) + \Theta(n) = \Omega(n \log n)$ per il teorema dell'esperto.

\end{description}


%Algoritmo QuickSort randomizzato
