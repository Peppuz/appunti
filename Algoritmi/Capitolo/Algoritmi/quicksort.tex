%Algoritmo QuickSort per risolvere il problema dell'ordinamento
\chapter{QuickSort}
Il $\proc{Quick-Sort}$ è un algoritmo divide et impera in loco ossia senza utilizzare
una struttura di appoggio per effettuare l'ordinamento e funziona in maniera ottimale
nell'implementazione sui calcolatori attuali.\newline
Venne sviluppato ed ideato nel 1959 dall'importantissimo informatico C.H. Hoare
e viene utilizzato come algoritmo di default delle libreria dei maggiori linguaggi.

L'algoritmo $\proc{Quick-Sort}$ esegue i seguenti passi divide et impera:
\begin{description}
  \item[Divide] riarrangia l'array $A[p\twodots r]$ in due sottoarray, eventualmente nulli,
                $A[p \twodots q-1]$ e $A[q+1 \twodots r]$ tali che tutti gli elementi
                del primo sottoarray sono minori o uguali a $A[q]$ e tutti gli elementi
                del secondo sottoarray sono maggiori o uguali a $A[q]$.\newline
                Calcolare l'indice di $q$ viene effettuato nella procedura di riarrangiamento.
  \item[Impera] ordina ricorsivamente i due sottoarray $A[p \twodots q-1]$ e $A[q+1 \twodots r]$.
  \item[Combina] dato che i due sottoarray sono già ordinati per cui non viene eseguito nulla.
\end{description}

L'algoritmo $\proc{Quick-Sort}$ è il seguente
%Pseudocodice Quicksort
\begin{codebox}
\Procname{$\proc{Quick-Sort}(A,p,r)$}
\li \If $p < r$
    \Then
\li           $q = \proc{Partition}(A,p,r)$
\li           $\proc{Quick-Sort}(A,p,q-1)$
\li           $\proc{Quick-Sort}(A,q+1,r)$
    \End
\end{codebox}
Per effettuare l'ordinamento di un array $A$ viene effettuata la chiamata iniziale
$\proc{Quick-Sort}$(A,1,$\attrib{A}{length}$).

La chiave dell'algoritmo è la procedura $\proc{Partition}$ che può essere implementata
in due maniere:la prima, inventata da Hoare e quindi quella originale, prende come elemento
pivot il primo elemento mentre la seconda versione, inventata da Lomuro, utilizza
l'ultimo elemento;lo pseudocodice delle due partition è il seguente:
%Pseudocodice Procedura Partition
\begin{codebox}
\Procname{$\proc{Lomuro-Partition}(A,p,r)$}
\li $\id{pivot} \gets A[r]$
\li $\id{indice} \gets p-1$
\li \For $j \gets p$ \To $r-1$
    \Do
\li               \If $A[j] \leq \id{pivot}$
                  \Then
\li                              $\id{indice} \gets \id{indice} + 1$
\li                              scambia $A[id{indice}]$ con $A[j]$
                  \End
    \End
\li scambia $A[indice+1]$ con $A[r]$
\li \Return $\id{indice} + 1$
\end{codebox}

\begin{codebox}
\Procname{$\proc{Hoare-Partition}(A,p,r)$}
\li $\id{pivot} \gets A[p]$
\li $i \gets p$
\li \For $j \gets p$ \To $r$
    \Do
\li               \If $A[j] < \id{pivot}$
                  \Then
\li                              $i \gets i + 1$
\li                              scambia $A[i]$ con $A[j]$
                  \End
    \End
\li scambia $A[i]$ con $A[r]$
\li \Return $i$
\end{codebox}

%Invariante di Ciclo da fare!!!!

Il tempo di esecuzione della procedura $\proc{Partition}$, sia Lomuro che Hoare, è il seguente:
$T(n) = c_1 + c_2 + c_3(n+1) + c_4n + c_5n(t_{if}) + c_6n(t_{if}) + c_7 + c_8$.

\begin{description}
  \item[Caso migliore]:tutti gli elementi sono maggiori del pivot per cui $t_{if} = 0$
        $T(n) = (c_3 + c_4)n + (c_1+c_2+c_3+c_7+c_8) = \Theta(n)$
  \item[Caso peggiore]:tutti gli elementi sono inferiori del pivot per cui $t_{if} = 1$
        $T(n) = (c_3+c_4+c_5+c_6)n + (c_1+c_2+c_3+c_7+c_8) = \Theta(n)$
\end{description}

\section{Analisi tempo QuickSort}
L'equazione di ricorrenza generale del Quicksort è la seguente:
\begin{equation*}
    T(n) = \begin{cases} 0 \quad \text{se} \ n = 0,1 \\
                         T(n-j) + T(j) + \Theta(n) \quad \text{se} \ n > 1 \\
           \end{cases}
\end{equation*}
L'analisi del tempo di esecuzione del $\proc{Quick-Sort}$ è in base al fatto se
la partizione dell'array è bilanciata o meno ossia se i due sottoproblemi da risolvere
sono più o meno della stessa dimensione.
\begin{description}
  \item[Caso peggiore]:la procedura di partizione produce due sottoproblemi:
        uno di $n-1$ elementi e l'altro di $0$ elementi.
        $T(n) = T(n-1) + T(0) + \Theta(n)
              = T(n-1) + \Theta(n)$
        Attraverso il metodo di sostituzione arrivo a $T(n) = O(n^2)$
  \item[Caso migliore]:la procedura di partizione produce due sottoproblemi di $\lceil n/2 \rceil$
        e $\lfloor n/2 \rfloor$ elementi per cui, ignorando le condizioni di ceil e floor,
        il tempo di esecuzione è $T(n) = 2T(n/2) + \Theta(n) = \Omega(n \log n)$ per il teorema dell'esperto.
  \item[Caso medio]:
\end{description}

%Versione Randomizzata
Una versione randomizzata del quicksort, utile per ottenere una buona prestazione attesa
in tutti gli input, consiste nel prendere come pivot ad ogni iterazione un elemento a caso
tra $p$ e $r$ così l'elemento pivot avrà la stessa possibilità di essere un elemento
del sottoarray per cui ci aspettiamo che la ripartizione dell'array sia ben bilanciata in media.

\begin{codebox}
\Procname{$\proc{Randomized-Partition}(A,p,r)$}
\li $i \gets \proc{RANDOM}(p,r)$
\li scambia $A[i]$ con $A[r]$
\li \Return $\proc{Partition}(A,p,r)$
\end{codebox}
La procedura randomizzata Quicksort chiama $\proc{Randomized-Partition}$
invece che $\proc{Partition}$, per cui il suo pseudocodice è il seguente:
\begin{codebox}
\Procname{$\proc{Randomized-QuickSort}(A,p,r)$}
\li \If $p < r$
    \Then
\li           $q = \proc{Randomized-Partition}(A,p,r)$
\li           $\proc{Randomized-QuickSort}(A,p,q-1)$
\li           $\proc{Randomized-QuickSort}(A,q+1,r)$
    \End
\end{codebox}
Il tempo di esecuzione del Quicksort randomizzato coincide ovviamente a quello normale
solo che il caso medio si verifica molto spesso ed è molto raro il caso peggiore
per cui di solito viene utilizzata la versione randomizzata del Quicksort.
