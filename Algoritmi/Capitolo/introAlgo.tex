\chapter{Introduzione agli Algoritmi}
Il termine Algoritmo proviene da Mohammed ibn-Musa al-Khwarizmi, matematico uzbeco
del IX secolo a.c. da cui proviene la moderna Algebra.%Cercare maggiori informazioni

\textbf{Algoritmo}:sequenza di passi che portano alla risoluzione di un problema

Per potere scrivere un algoritmo il più formale ed indipendente dal linguaggio di programmazione
si utilizza il Pseudocodice, come specificato nel prossimo paragrafo.

%Da definire come subsection????
\section{Pseudocodice}
Lo pseudocodice è un linguaggio, ispirato ai linguaggi di programmazione, utilizzato
per rappresentare e presentare gli algoritmi in maniera compatta e chiara;
ogni libro e programmatore definisce la propria specifica di Pseudocodice ma comunque
quasi tutti si ispirano alla sintassi del Pascal,C oppure di Java;i costrutti definiti
nel mio pseudocodice sono i seguenti:

\begin{tabular}{lcr}
  \toprule Istruzione & Significato\\
  \midrule
  $x \gets 5$ & assegna ad $x$ il valore $5$\\
  \textbf{integer} nome & definisce una variabile nome di tipo intero\\
  \textbf{real} nome & definisce una variabile nome di tipo reale\\
  \textbf{boolean} nome & definisce una variabile nome di tipo booleano\\
  \textbf{if} condizione & definisce la struttura if\\
  \textbf{For} estrInf \textbf{To} estrSup & definisce la struttura For da estrInf a estrSup \\
  \textbf{For} estrSup \textbf{Downto} estrInf & definisce la struttura For da estrSup a estrInf\\
  \textbf{While} condizione & Definisce l'istruzione While\\
  \textbf{Return} valore & Ritorna valore da una procedura\\
  \bottomrule
\end{tabular}

\section{Analisi di Algoritmi}
In questo paragrafo verranno scritti degli algoritmi e verrà effettuata l'Analisi
dell'Algoritmo, che può avvenire in due modalità:

\begin{itemize}
  \item \textbf{Tempo di Esecuzione}:è il numero di operazioni primitive che vengono eseguite
        da parte di un algoritmo;l'esecuzione di un'istruzione si assume che richiede un tempo costante
        per evitare di rendere la valutazione dipendente dall'hardware e dalla bravura del programmatore.
  \item \textbf{Spazio di Esecuzione}:è il numero di spazio in bit occupato in memoria dall'algoritmo
        ma questa analisi non viene quasi mai eseguita in quanto oramai è superfluo.
\end{itemize}

Dato un Array di interi A[0....length-1] si definisce $\min (A) = a \Leftrightarrow \forall b \in A : a \leq b$
e il pseudocodice della procedura $\proc{min}$ è il seguente:
\begin{algorithm}
\caption{$\proc{min}$(ITEM[] A,\textbf{integer} n)} \qquad \qquad Costo \quad Volte
\end{algorithm}
\begin{codebox}
  \li \id{min} $\gets A[0]$ \>\>\>\>\>\> $c_1$ \>\>\> $1$
  \li \For j $\gets 1$ \To $\id{A.length-1}$ \>\>\>\>\>$c_2$\>\>\>$n$
      \Do
  \li        \If $A[j] \leq \id{min}$ \>\>\>\>\>$c_3$\>\>\>$n-1$
\li             \Then
                    $\id{min} \gets A[j]$\>\>\>\>$c_4$\>\>\>$n-1$
             \End
      \End
  \li \Return min \>\>\>\>\>\>$c_5$\>\>\>$1$
\end{codebox}



Per valutare un Algoritmo bisogna verificare due proprietà:\textbf{CORRETTEZZA} e
\textbf{EFFICENZA}.

Per provare la correttezza di un algoritmo bisogna effettuare una dimostrazione
matematica attraverso l'\textit{invariante di ciclo}, ossia una proprietà che mostra
la correttezza dell'algoritmo.

L'invariante deve essere vera in due casi specifici:
\begin{enumerate}
  \item \textbf{passo base}:l'affermazione è vera all'inizio della prima iterazione del ciclo
  \item \textbf{passo induttivo}:supposta vera all'inizio dell'iterazione deve essere vera
        anche all'inizio dell'iterazione successiva
\end{enumerate}

Es: Verificare la correttezza di min(A)
\textbf{Invariante di ciclo}:all'inizio di ogni iterazione la variabile \textit{min} contiene
il minimo parziale degli elementi A[0....j-1]

\textbf{Passo base}:per j = 1 l'array A è composto da un solo elemento \newline
\textbf{Ipotesi induttiva}:$\id{min}$ contiene il minimo degli elementi A[0...j-1] all'inizio dell'iterazione\newline
\textbf{Passo induttivo}:all'esecuzione di un iterazione si possono verificare due casi:\newline
Caso $A[j] < min$\newline \newline
      la variabile $\id{min}$ viene aggiornata con il valore A[j] e ciò verifica la proprietà\newline
Caso $A[j] >= min$\newline \newline
      la variabile $\id{min}$ contiene già il minimo parziale degli elementi A[0...j]

Per provare l'efficienza di un algoritmo bisogna calcolare e dimostrare il numero di confronti
necessari,in funzione di n, per risolvere un problema computazionale come viene mostrato
nell'esempio.

Es:Calcolo tempo di esecuzione algoritmo min(ITEM[] A,integer n)
\begin{algorithm}
\caption{$\proc{min}$(ITEM[] A,\textbf{integer} n)} \qquad \qquad Costo \quad Volte
\end{algorithm}
\begin{codebox}
  \li \id{min} $\gets A[0]$ \>\>\>\>\>\> $c_1$ \>\>\> $1$
  \li \For j $\gets 1$ \To $\id{A.length-1}$ \>\>\>\>\>$c_2$\>\>\>$n$
      \Do
  \li        \If $A[j] \leq \id{min}$ \>\>\>\>\>$c_3$\>\>\>$n-1$
\li             \Then
                    $\id{min} \gets A[j]$\>\>\>\>$c_4$\>\>\>$n-1$
             \End
      \End
  \li \Return min \>\>\>\>\>\>$c_5$\>\>\>$1$
\end{codebox}


Il costo dell'algoritmo $\proc{min}$ nel caso peggiore è $T(n) = (c_2 + c_3 + c_4)n + (c_1 - c_3 - c_4 + c_5)$. \newline
L'algoritmo $\proc{min}$ nel caso peggiore è una funzione lineare.

%Algoritmo linearSearch
\section{Ricerca di Valori}
\textbf{Input}:una sequenza di valori A[0 \dots length-1] e un valore \textit{key} \newline
\textbf{Output}:\textit{index},indice della sequenza A in cui A[index] = key, altrimenti null
%Esercizio 2.1-3 CLRS
Input: una sequenza di $n$ numeri $A = (a_1,a_2,\dots,a_n)$ e un valore $\id{key}$
Output: un indice $i$ tale che $A[i] = \id{key}$ o il valore speciale $\const{nul}$ se non viene trovato

%Algoritmo linearSearch(A,key)$
\begin{codebox}
    \Procname{$\proc{linearSearch}(A,\id{key})$}
\li \For $j \gets 1$ \To \attrib{A}{length}
    \Do
\li       \If $A[j] \isequal \id{key}$
          \Then
\li               \Return $j$
          \End
    \End
\li \Return \const{nul}
\end{codebox}


%Effettuare controllo correttezza e efficienza Algoritmo
