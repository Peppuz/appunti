\chapter{Introduzione agli Algoritmi}
Il termine Algoritmo proviene da Mohammed ibn-Musa al-Khwarizmi, matematico uzbeco
del IX secolo a.c. da cui proviene la moderna Algebra. \newline%Cercare maggiori informazioni
\textbf{Algoritmo}:sequenza di passi che portano alla risoluzione di un problema

%Da definire come subsection????
\section{Pseudocodice}
Lo pseudocodice è un linguaggio, ispirato ai linguaggi di programmazione, utilizzato
per rappresentare e presentare gli algoritmi in maniera compatta e chiara;
ogni libro e programmatore definisce la propria specifica di Pseudocodice ma comunque
quasi tutti si ispirano alla sintassi del Pascal,C oppure di Java;i costrutti definiti
nel mio pseudocodice sono i seguenti:

\begin{tabular}{lcr}
  \toprule Istruzione & Significato\\
  \midrule
  $x \gets 5$ & assegna ad $x$ il valore $5$\\
  \textbf{integer} nome & definisce una variabile nome di tipo intero\\
  \textbf{real} nome & definisce una variabile nome di tipo reale\\
  \textbf{boolean} nome & definisce una variabile nome di tipo booleano\\
  \textbf{if} condizione & definisce la struttura if\\
  \textbf{For} estrInf \textbf{To} estrSup & definisce la struttura For da estrInf a estrSup \\
  \textbf{For} estrSup \textbf{Downto} estrInf & definisce la struttura For da estrSup a estrInf\\
  \textbf{While} condizione & Definisce l'istruzione While\\
  \textbf{Return} valore & Ritorna valore da una procedura\\
  \bottomrule
\end{tabular}

\section{Analisi di Algoritmi}
In questo paragrafo verranno scritti degli algoritmi e verrà effettuata l'Analisi
dell'Algoritmo, che può avvenire in due modalità:

\begin{itemize}
  \item \textbf{Tempo di Esecuzione}:è il numero di operazioni primitive che vengono eseguite
        da parte di un algoritmo;l'esecuzione di un'istruzione si assume che richiede un tempo costante
        per evitare di rendere la valutazione dipendente dall'hardware e dalla bravura del programmatore.
  \item \textbf{Spazio di Esecuzione}:è il numero di spazio in bit occupato in memoria dall'algoritmo
        ma questa analisi non viene quasi mai eseguita in quanto oramai è superfluo.
\end{itemize}

Dato un Array di interi A[0....length-1] si definisce $\min (A) = a \Leftrightarrow \forall b \in A : a \leq b$
e il pseudocodice della procedura $\proc{min}$ è il seguente:
\begin{algorithm}
\caption{$\proc{min}$(ITEM[] A,\textbf{integer} n)} \qquad \qquad Costo \quad Volte
\end{algorithm}
\begin{codebox}
  \li \id{min} $\gets A[0]$ \>\>\>\>\>\> $c_1$ \>\>\> $1$
  \li \For j $\gets 1$ \To $\id{A.length-1}$ \>\>\>\>\>$c_2$\>\>\>$n$
      \Do
  \li        \If $A[j] \leq \id{min}$ \>\>\>\>\>$c_3$\>\>\>$n-1$
\li             \Then
                    $\id{min} \gets A[j]$\>\>\>\>$c_4$\>\>\>$n-1$
             \End
      \End
  \li \Return min \>\>\>\>\>\>$c_5$\>\>\>$1$
\end{codebox}



Per valutare un Algoritmo bisogna verificare due proprietà:\textbf{CORRETTEZZA} e
\textbf{EFFICENZA}.

Per provare la correttezza di un algoritmo bisogna effettuare una dimostrazione
matematica attraverso l'\textit{invariante di ciclo}, ossia una proprietà che mostra
la correttezza dell'algoritmo.

L'invariante deve essere vera in due casi specifici:
\begin{enumerate}
  \item \textbf{passo base}:l'affermazione è vera all'inizio della prima iterazione del ciclo
  \item \textbf{passo induttivo}:supposta vera all'inizio dell'iterazione deve essere vera
        anche all'inizio dell'iterazione successiva
\end{enumerate}

%Da definire in un file a parte
Es: Verificare la correttezza di min(A)
\textbf{Invariante di ciclo}:all'inizio di ogni iterazione la variabile \textit{min} contiene
il minimo parziale degli elementi A[0....j-1]

\begin{proof}
\textbf{Passo base}:per j = 1 l'array A è composto da un solo elemento \newline
\textbf{Ipotesi induttiva}:$\id{min}$ contiene il minimo degli elementi A[0...j-1] all'inizio dell'iterazione\newline
\textbf{Passo induttivo}:all'esecuzione di un iterazione si possono verificare due casi:\newline
Caso $A[j] < min$\newline \newline
      la variabile $\id{min}$ viene aggiornata con il valore A[j] e ciò verifica la proprietà\newline
Caso $A[j] >= min$\newline \newline
      la variabile $\id{min}$ contiene già il minimo parziale degli elementi A[0...j]
\end{proof}

Per provare l'efficienza di un algoritmo bisogna calcolare e dimostrare il numero di confronti
necessari,in funzione di n, per risolvere un problema computazionale come viene mostrato
nell'esempio.

Es:Calcolo tempo di esecuzione algoritmo min(ITEM[] A,integer n)
\begin{algorithm}
\caption{$\proc{min}$(ITEM[] A,\textbf{integer} n)} \qquad \qquad Costo \quad Volte
\end{algorithm}
\begin{codebox}
  \li \id{min} $\gets A[0]$ \>\>\>\>\>\> $c_1$ \>\>\> $1$
  \li \For j $\gets 1$ \To $\id{A.length-1}$ \>\>\>\>\>$c_2$\>\>\>$n$
      \Do
  \li        \If $A[j] \leq \id{min}$ \>\>\>\>\>$c_3$\>\>\>$n-1$
\li             \Then
                    $\id{min} \gets A[j]$\>\>\>\>$c_4$\>\>\>$n-1$
             \End
      \End
  \li \Return min \>\>\>\>\>\>$c_5$\>\>\>$1$
\end{codebox}


Il costo dell'algoritmo $\proc{min}$ nel caso peggiore è $T(n) = (c_2 + c_3 + c_4)n + (c_1 - c_3 - c_4 + c_5)$. \newline
L'algoritmo $\proc{min}$ nel caso peggiore è una funzione lineare.

%Algoritmo InsertionSort
%Algoritmo di Insertion Sort
L'algoritmo $\proc{Insertion-Sort}$ risolve il problema dell'ordinamento definito come:\newline
\textbf{Input}:una sequenza di $n$ numeri $(a_1,a_2,\dots,a_n)$ \newline
\textbf{Output}:una permutazione $(a'_1,a'_2,\dots,a'_n)$ tale che $a'_1 \leq a'_2 \leq \dots \leq a'_n$

%pseudocodice algoritmo insertionSort
\begin{figure}
    \caption{Algoritmo insertionSort}
    \label{alg:insertion}
    \begin{codebox}
        \Procname{$\proc{Insertion-Sort}(A)$}
        \li \For $j \gets 2$ \To $\attrib{A}{length}$
            \Do
        \li            $\id{key} \gets A[j]$
        \li         $i \gets j-1$
        \li         \While $i > 0$ and $A[i] > \id{key}$
                    \Do
        \li                $A[i+1] \gets A[i]$
        \li                $i \gets i-1$
                    \End
        \li         $A[i+1] \gets \id{key}$
            \End
    \end{codebox}
\end{figure}
L'algoritmo $\proc{Insertion-Sort}$ \ref{alg:insertion} è un algoritmo efficiente per ordinare un ristretto numero di elementi ed opera come farebbe un umano
a riordinare le carte da gioco, ossia prendendo una carta alla volta e facendo il riordinamento delle carte una alla volta.\newline
Per poter affermare che l'algoritmo è corretto, ossia risolve il problema, bisogna dimostrare l'invariante del ciclo $\kw{For}$, attraverso un metodo simile all'induzione matematica.

%Dimostrazione Invariante di Ciclo
L'invariante del ciclo è corretta se si riesce a dimostrare tre cose:
\begin{description}
  \item[Inizializzazione] è corretta prima della prima esecuzione del ciclo.
  \item[Conservazione] se è verificata prima di un iterazione del ciclo lo sarà anche dopo l'esecuzione di quell'iterazione del ciclo.
  \item[Conclusione] alla fine del ciclo è ancora verificato e ciò ci aiuta a determinare la correttezza di un algoritmo.
\end{description}
La terza proprietà è la più importante in quanto assieme alla condizione che è causato la conclusione del ciclo, si riesce a dimostrare la correttezza dell'algoritmo.

L'invariante di ciclo per l'$\proc{Insertion-Sort}$ è:
All'inizio di ogni iterazione del ciclo $\kw{for}$ il sottoarray $A[1 \twodots j-1]$
è ordinato ed è formato dagli stessi elementi che erano originamente in $A[1\twodots j-1]$.
\begin{description}
  \item[Inizializzazione] quando $j = 2$ il sottoarray $A[1\twodots j-1]$ è formato
                           da un solo elemento che è ordinato ed è l'elemento originale $A[1]$.
  \item[Conservazione] all'inizio di ogni esecuzione del ciclo for il sottoarray $A[1 \twodots j-1]$
                        è formato dai primi $j-1$ elementi dell'array ordinati dal
                        più piccolo al più grande.

  \item[Conclusione] Quando $j > \attrib{A}{length}$ il ciclo termina e dato che ogni
                      ciclo aumenta $j$ di $1$ alla fine del ciclo si avrà $j = n + 1$
                      per cui si ha che $A[1\twodots n]$ è ordinato ed è formato
                      dagli elementi ordinati che si trovavano in $A[1 \twodots n]$.
\end{description}
L'analisi di un algoritmo, per poter determinare se un algoritmo è efficiente, può
avvenire in due maniere:
\begin{description}
    \item [Tempo di Esecuzione] è il numero di operazioni primitive che vengono eseguite da parte di un algoritmo;l'esecuzione di un'istruzione si assume che richiede un tempo costante
                                per evitare di rendere la valutazione dipendente dall'hardware e dalla bravura del programmatore.
    \item [Spazio di Esecuzione] è il numero di spazio in bit occupato in memoria dall'algoritmo ma questa analisi non viene quasi mai eseguita in quanto oramai è superfluo.
\end{description}
Il tempo di esecuzione dell'algoritmo è la somma dei tempi di esecuzione per ogni istruzione eseguita quindi il tempo di esecuzione di $\proc{Insertion-Sort}$ è:
\begin{equation*}
    T(n) = c_1n + c_2(n-1) + c_3(n-1) + c_4 \sum_{j=2} ^n t_j + c_5 \sum_{j=2} ^n (t_j -1)
         + c_6 \sum_{j=2} ^n (t_j -1) + c_7(n-1)
\end{equation*}
In caso l'algoritmo sia già ordinato, caso migliore, si avrebbe sempre $A[i] < \id{key}$ quindi $t_j$ è sempre $1$, per cui il tempo di esecuzione sarebbe:
\begin{align*}
    T(n) & = c_1n + c_2(n-1) + c_3(n-1) + c_4(n-1) + c_7(n-1) \\
         & = (c_1 + c_2 + c_3 + c_4 + c_5)n - (c_2 + c_3+ c_4+ c_7) \\
         & = \Omega(n) \\
\end{align*}
Nel caso migliore si ha che l'algoritmo richiede un tempo lineare che è un $\Omega(n)$.\newline
In caso si abbia una sequenza decrescente, corrispondente al caso peggiore, nel ciclo While bisogna
confrontare ogni elemento $A[j]$ con il sottoarray $A[1 \twodots j-1]$ per cui $t_j = j$
per $j=2,3,\dots,n$ e poiche si ha
\begin{align*}
    \sum _{j=2} ^ n j = \frac{n(n+1)}{2} -1 & \quad \sum_{j=2}^n (j-1) = \frac{n(n-1)}{2}
\end{align*}
il tempo dell'algoritmo $\proc{Insertion-Sort}$ nel caso peggiore è il seguente:
\begin{align*}
    T(n) & = c_1n + c_2(n-1) + c_3(n-1) + c_4 (\frac{n(n+1)}{2} -1) + c_5 (\frac{n(n-1)}{2})
         + c_6 (\frac{n(n-1)}{2}) + c_7(n-1) \\
         & = c_1n + c_2(n-1) + c_3(n-1) + c_4(\frac{n^2+n-2}{2}) + c_5(\frac{n^2-n}{2})
           + c_6(\frac{n^2-n}{2}) + c_7(n-1) \\
         & = (\frac{c_4}{2} +\frac{c_5}{2} + \frac{c_6}{2})n^2 + (c_1+c_2+c_3+\frac{c_4}{2} - \frac{c_5}{2}
            -\frac{c_6}{2}+c_7)n -(c_2+c_3+c_4+c_7)\\
         & = O(n^2)\\
\end{align*}
Il tempo dell'algoritmo può essere scritto, nel caso peggiore, come $an^2 + bn + c$ che è una funzione
quadratica che viene indicata, nel caso peggiore come $O(n^2)$.

Nel caso medio mi aspetto, supponendo una distribuzione uniforme della probabilità,
che il vettore sia parzialmente ordinato per cui non avendo modo di rendere il ciclo
while eseguibile solo una volta, si ha bisogno almeno di $n^2$ confronti che è $O(n^2)$.

In sintesi i tempi di esecuzione dell'algoritmo $\proc{Insertion-Sort}$ sono:
\begin{description}
  \item[Caso migliore] $\Omega(n)$
  \item[Caso peggiore] $O(n^2)$
  \item[Caso medio] $O(n^2)$
\end{description}
%Paragrafo sull'algoritmo InsertionSort

%Algoritmo linearSearch
\section{Ricerca di Valori}
\textbf{Input}:una sequenza di valori A[0 \dots length-1] e un valore \textit{key} \newline
\textbf{Output}:\textit{index},indice della sequenza A in cui A[index] = key, altrimenti null
%Esercizio 2.1-3 CLRS
Input: una sequenza di $n$ numeri $A = (a_1,a_2,\dots,a_n)$ e un valore $\id{key}$
Output: un indice $i$ tale che $A[i] = \id{key}$ o il valore speciale $\const{nul}$ se non viene trovato

%Algoritmo linearSearch(A,key)$
\begin{codebox}
    \Procname{$\proc{linearSearch}(A,\id{key})$}
\li \For $j \gets 1$ \To \attrib{A}{length}
    \Do
\li       \If $A[j] \isequal \id{key}$
          \Then
\li               \Return $j$
          \End
    \End
\li \Return \const{nul}
\end{codebox}


%Effettuare controllo correttezza e efficienza Algoritmo
