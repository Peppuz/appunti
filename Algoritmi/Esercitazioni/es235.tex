%Esercitazione del 23/5/2018 Algoritmi
Implementare la procedura extract per estrarre l'elemento k-esimo  di una coda Q
senza utilizzare code d'appoggio e mantenendo i valori della coda.

int extract(Queue Q,int k){
    Q.enqueue(-1);
    c = 1;
    r = 0;
    while(c \leq k and r != -1){
        r = Q.dequeue()
        if(r != -1){
          Q.enqueue(r);
        }
        c++
    }
    if(r == -1)
        return "underflow"
    Q.dequeue()//Toglie l'elemento k-esimo

    %Sistema la coda agli elementi iniziali
    r = Q.dequeue()
    while(r != -1){
        Q.enqueue(r);
        r = Q.dequeue();
    }
}

SI usa un valore sentinella per capire che si ha finito di iterare la coda;il valore sentinella
varia in base ai valori che utilizziamo infatti la coda Q usata nell'esercizio è
composta solo da numeri interi positivi per cui usiamo il -1.


Implementare la rimozione di tutte le occorrenze di un elemento da una coda utilizzando
soltanto come struttura dati le code



while(!Q.emptyQueue()){
    crea una coda d'appoggio QA = new Queue();
    r = Q.dequeue();
    if(r != a){
      Q.enqueue(QA,r);
    }
}

while(!Q2.emptyQueue()){
    r = Q2.dequeue();
    Q2.enqueue(r);
}

Tempo di esecuzione = cn + cn + cn + ct_if + ctw_2 + c + ctw_2 + ctw_2

Caso migliore: T(n) = 3cn = \Omega(n) (tutta la coda è composta dall'elemento a)

Caso peggiore: non vi è alcun elemento a nella coda
   T(n) = 3cn + cn + 3cn + c = O(n)

Tempo dell'algoritmo T(n) = \Theta(n)


Implementare l'ordinamento tra due code Q1 e Q2, già ordinate.

Queue* ordina(Queue Q1,Queue Q2){
    controllare che Q1 e Q2 contengono almeno un elemento

    r1 = Q1.dequeue();
    r2 = Q2.dequeue();
    while(!Q1.empty() and !Q2.empty()){
          if(r1 <= r2){
              Q.enqueue(r1);
              r1 = Q1.dequeue()
          }else{
              Q.enqueue(r2);
              r2 = Q2.dequeue();
          }
    }
    %Ipotiziamo che manchino da inserire solo gli elementi di Q1
    while(!Q1.empty()){

    }
}

Tempo di esecuzione T(n) circa n+n = \Theta(n)%FARE ANALISI NEL TEMPO
