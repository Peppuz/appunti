%Esercitazione 1 di Guido Zandron
Scrivere un algoritmo che risolva il seguente problema, utilizzando la tecnica divide-et-
impera: dato in input un vettore $V[1..n]$ contenente $n > 0$ valori interi, calcolare e restituire il
valore: $(V[1] + V[2]) * (V[2] + V[3]) + \dots + (V[n-1] + V[n])$.
Scrivere poi l’equazione di ricorrenza che esprime il tempo di calcolo di tale algoritmo, e
risolverla utilizzando un metodo a piacere.

\begin{codebox}
\Procname{$\proc{calcoloParticolare}(V,left,\id{right})$}
\li \If $left \geq \id{right}$
    \Then
\li              \Return $left + \id{right}$
    \End
\li $\id{mid} \gets (left + \id{right}) / 2$
\li $ris_1 \gets \proc{calcoloParticolare}(V,left,\id{mid})$
\li $ris_2 \gets \proc{calcoloParticolare}(V,\id{mid},\id{right})$
\li \Return $ris_1 * ris_2$
\end{codebox}

Il tempo di esecuzione dell'algoritmo è il seguente
\begin{equation*}
  T(n) = \begin{cases} 2c = \Theta(1) \ \text{se} \ n = 1 \\
                       2T(\frac{n}{2}) + 2c \ \text{se} \ n > 1\\
         \end{cases}
\end{equation*}
Essendo $2c = O(n^{\log _ 2 2}) = O(n)$ si applica il primo caso del teorema dell'Esperto
per cui si ottiene $T(n) = \Theta(n)$. Il tempo di esecuzione dell'algoritmo è:
\begin{equation*}
  T(n) = \begin{cases} 2c = \Theta(1) \ \text{se} \ n = 1 \\
                       2T(\frac{n}{2}) + 2c = \Theta(n) \ \text{se} \ n > 1\\
         \end{cases}
\end{equation*}


Scrivere un algoritmo iterativo che, dati in ingresso due vettori $A[1..m]$ e $B[1..n]$ contenenti
rispettivamente m ed n caratteri dell’alfabeto italiano, conti quante volte la stringa
memorizzata in A compare in B. Si assuma che n sia molto maggiore di m.
Valutare poi i tempi di calcolo nei casi migliore e peggiore, indicando quante volte viene
eseguita ogni operazione in pseudocodice. Si assuma che ogni singola operazione elementare
impieghi lo stesso tempo (costante).
Esempio: dati i vettori:
A = [ a, b, a ]
B = [ c, d, c, a, b, a, b, a, c, a, b, a, b, c, d, b, a, b, b, a, b, a, a, b ]
il programma deve stampare il valore 4, dato che la stringa “aba” compare 4 volte in B (a
partire dalle posizioni 4, 6, 10 e 20. Fare attenzione alle eventuali sovrapposizioni!).

\begin{codebox}
\Procname{$\proc{contaRipetizioni}(A,B)$}
\li $n_1 \gets \attrib{A}{length}$
\li $n_2 \gets \attrib{B}{length}$
\li $ripetizioni \gets 0$
\li \For $j \gets 1$ \To $n_2 - n_1$
    \Do
\li                $i \gets 1$
\li                \While $i \leq n_1$ and $B[j+i-1] \isequal A[i]$
                   \Do
\li                               $i \gets i + 1$
                   \End
\li                \If $i > n_1$
                   \Then
\li                               $\id{ripetizioni} = ripetizioni + 1$
    \End
\li \Return ripetizioni
\end{codebox}
Il tempo di esecuzione dell'algoritmo è il seguente:
\begin{equation*}
  T(n) = 4c + (n+1)c + cn + c\sum _{j = 1} ^ n (t_{w} + 1) + c\sum _{j = 1} ^ n t_w
         cn + ct_{if}
\end{equation*}
L'algoritmo ha i seguenti casi:
\begin{description}
  \item[Caso migliore] tutti gli elementi di $B$ sono uguali e diversi da $A[0]$
        per cui risulta $t_w = 0$ e $t_{if} = 0$ per tutte le iterazioni del ciclo for.
        \begin{equation*}
          T(n) = 4c + cn + c + cn + cn = \Omega(n)
        \end{equation*}
  \item[Caso peggiore]: tutti gli elementi di $B$ ed $A$ sono uguali per cui risulta
         $t_w = n_1$ per tutte le iterazioni del ciclo for e $t_{if} = n$.
         Dato che si suppone che $n_2$ sia nettamente maggiore di $n_1$ si arriva
         alla conclusione che $n_1 \leq (n_2-n_1) \leq n$ per cui si ha $\sum _{j = 1} ^ n (n_1 + 1)
         \leq \sum _{j = 1} ^ n ( + 1) \leq (n^2 + n)$.
         \begin{equation*}
           T(n) = 5c + cn + cn^2 + cn + cn^2 + cn + cn = 0(n^2)
         \end{equation*}
\end{description}
