%Paragrafo sull'albero di Ricorsione
\section{Albero di Ricorsione}
In un albero di Ricorsione ogni nodo rappresenta il costo del sottoproblema nella
chiamata ricorsiva di risoluzione dell'algoritmo fino ad arrivare al caso base.
Dato un albero l'altezza è  al massimo il $\log n$ dove $n$ indica il numero dei nodi
per cui l'ultimo livello della ricorsione ci aspettiamo che intervenga per il $\log n$
al computo del tempo di esecuzione dell'Algoritmo.
Di solito si utilizza l'albero di ricorsione per generare un ipotesi da dimostrare
mediante induzione, con il metodo di sostituzione, per cui si può tollerare un pò
di incertezza nell'analisi, ad esempio togliere le costanti ed evitare di considerare
i floor e ceil quando si effettua la divisione dell'input in due sottoproblemi.

Albero di ricorsione dell'equazione di ricorrenza $T(n) = 3T(n/4) + \Theta(n^2)$

%Tradurre e fare grafici degli alberi!!!!!!!
