%Paragrafo sul Teorema dell'Esperto
\section{Teorema dell'Esperto}
Il teorema dell'Esperto è una per risolvere in maniera semplice ed immediata
le equazioni di ricorrenza della forma $T(n) = aT(\frac{n}{b}) + f(n)$
dove $a \geq 1 , b > 1$ e $f(n)$ una funzione asintoticamente positiva.
Utilizzare il metodo dell'esperto richiede di memorizzare tre diversi casi e grazie
a quelli posso risolvere una grande quantità di equazione di ricorrenza in maniera semplice e veloce.

\begin{thm}[Master Theorem]
Sia $a \geq 1, b > 1$ e $f(n)$ una funzione asintoticamente positiva e sia abbia
un equazione di ricorrenza nella forma $T(n) = aT(\frac{n}{b}) + f(n)$ allora:
\begin{enumerate}
  \item se $f(n) = O(n^{\log _b  a - \epsilon})$ per $\epsilon > 0$, allora $T(n) = \Theta(n^{\log _b  a})$
  \item se $f(n) = \Theta(n^{\log _b a})$, allora $T(n) = \Theta(n^{\log _b  a} \log n)$
  \item se $f(n) = \Omega(n^{\log _b  a + \epsilon}$) per $\epsilon > 0$ e se
        $af(\frac{n}{b}) \leq cf(n)$ per una costante $c < 1$ e $n$ sufficientemente grande,
        allora $T(n) = \Theta(f(n))$
\end{enumerate}
\end{thm}
Intuitivamente confrontiamo in tutti e tre i casi $f(n)$ con $n ^{\log _ b  a}$
e il più grande tra di essi determina la soluzione della ricorrenza ma bisogna stare attenti
che vi deve essere una differenza polinomiale, controllata tramite $\epsilon$ tra $f(n)$ e $n^{\log _b  a}$.

%Inserire esempi Metodo dell'Esperto
Esempio:\newline
Data un equazione $T(n) = 2T(\frac{n}{2}) + cn$ determinare il tempo tramite il metodo dell'Esperto.
$f(n) = \Theta(n^{\log _ 2 2}) = \Theta(n)$ per cui si applica il secondo caso del Teorema dell'esperto
quindi $T(n) = \Theta(n^{\log _2 2} \log n) = \Theta(n \log n)$.
