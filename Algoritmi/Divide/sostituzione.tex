%paragrafo sul metodo di Sostituzione per risolvere le ricorrenze
\section{Il metodo di Sostituzione}
Il metodo di sostituzione è un tecnica di risoluzione delle equazioni di ricorrenza
degli algoritmi divide et impera, che richiede due passi:
\begin{enumerate}
    \item Ipotizzare la forma della risoluzione.
    \item Usare l'induzione matematica per trovare le costanti e dimostrare che
          la soluzione proposta funziona ed è corretta.
\end{enumerate}

%Dimostrazione
\begin{thm}
    $T(n) = 2T(n/2) + n = \Theta(n \ln n)$
\end{thm}
\begin{proof}
Per dimostrare che $T(n) = \Theta(n \ln n)$ bisogna provare che $T(n) \leq cn \ln n$ per una costante $c > 0$
\begin{equation*}
\begin{split}
T(n) & \leq 2(cn/2 \lg n/2) + n \\
     & \leq cn \lg n - cn \lg 2 + n\\
     & \leq cn \lg n - cn + n \\
     & \leq cn \lg n \text{per} c \geq 1\\
\end{split}
\end{equation*}
L'induzione matematica richiede di verificare i casi base ora
\end{proof}

%Scegliere una buona ipotesi
Per scegliere una buona ipotesi da verificare mediante il metodo di sostituzione
richiede fantasia, esperienza.
Per aiutarci ad ottenere una buona ipotesi si potrebbe utilizzare l'albero di sostituzione
e poi dimostrare l'ipotesi mediante induzione, con il metodo di sostituzione, oppure
ci sono delle euristiche per diventare dei buoni indovini.

Le euristiche per formulare buone ipotesi sono le seguenti:
\begin{itemize}
    \item Se una ricorrenza è simile ad una già analizzata conviene provare a dimostrare
          la stessa soluzione come ad esempio:
          \begin{equation*}
              T(n) = 2T(\frac{n}{2} + 17) + n \text{è simile all'equazione precedente ed è un} \Theta(n \lg n)
          \end{equation*}
    \item Si inizia a dimostrare dei limiti superiori ed inferiori molto larghi e
          poi restringere l'incertezza alzando il limite inferiore ed abbassando
          il limite superiore fino a convergere con il risultato corretto.
\end{itemize}

%esempi

%Errori tipici e togliere termini di ordine inferiore
Ci sono dei casi in cui ipotiziamo correttamente un limite asintotico per la ricorrenza
ma in qualche modo sembra che i calcoli matematici non tornino nell'Induzione.
Per superare questo ostacolo spesso basta correggere l'ipotesi sottraendo un termine
di ordine inferiore per far quadrare i conti, come nell'esempio:
\begin{equation*}
    T(n) = T(\lfloor n/2 \rfloor) + T(\lceil n/2 \rceil) + 1
\end{equation*}
Supponiamo che $T(n) = O(n)$ ossia $T(n) \leq cn$, otteniamo nella ricorrenza:
\begin{equation*}
\begin{split}
    T(n) & \leq c \lfloor n/2 \rfloor + c \lceil n/2 \rceil + 1 \\
         & = cn + 1
\end{split}
\end{equation*}
Questa equazione non implica che $T(n) \leq cn$ qualunque sia il valore di $c$ però
l'intuizione che $T(n) = O(n)$ è corretta solo che per provarla dobbiamo utilizzare
un ipotesi induttiva più forte, ossia $T(n) \leq cn - d$ con $d \geq 0$ rappresentante una costante.
\begin{equation*}
\begin{split}
    T(n) & \leq (c \lfloor n/2 \rfloor - d) + (c \lceil n/2 \rceil -d) + 1 \\
         & \leq cn - 2d + 1 \\
         & \leq cn - d \quad \text{per ogni} d \geq 1 \\
\end{split}
\end{equation*}
