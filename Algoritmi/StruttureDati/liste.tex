%Strutture Dati Lista
\section{Liste}
Le liste sono una struttura dati elementare che implementa il concetto matematico
di sequenza lineare di oggetti, in cui si possono eventualmente ripetere gli elementi
all'interno della sequenza.\newline
La lista è una struttura dati lineare dinamica in cui l'accesso all'elemento successivo
della sequenza avviene tramite un puntatore all'elemento successivo ed un elemento
è composto da un valore , chiamato $\id{element}$ e da due puntatori $\id{prev}$ e $\id{next}$,
i quali puntano all'elemento precedente o successivo della lista.\newline
In caso $\id{prev} \gets \const{nil}$ l'elemento non ha nessun predecessore ed è la $\id{head}$
della lista mentre il puntatore $\id{next} \gets \const{nil}$ l'elemento non ha
nessun successore per cui è la coda della lista.
Vi sono diversi tipologie di caratteristiche che una lista può possedere, anche in
maniera multipla ossia può possedere più di una proprietà, come si può notare dal seguente elenco:
\begin{itemize}
  \item singolarmente o doppiamente concatenata: in caso una lista sia singolarmente
        concatenata si ha soltanto il collegamento con l'elemento successivo, per
        cui viene omesso il puntatore $\id{prev}$, mentre nella lista doppiamente concatenata
         si ha il collegamento, tramite i puntatori, con l'elemento precedente e l'elemento successivo.
  \item ordinata: una lista si dice ordinata se è previsto un ordinamento tra i valori
        degli elementi presenti in una lista.
  \item circolare: il puntatore $\id{prev}$ della testa della lista punta alla coda
        mentre il puntatore $\id{next}$ della coda della lista punta alla testa
       per cui si può dire che la lista è un anello di elementi.
\end{itemize}

Forniamo ora un implementazione di una lista doppiamente concatenata non ordinata
in cui un elemento della lista è composto da un dato chiamato $\id{element}$ e da due
puntatori, chiamati $\id{prev}$ e $\id{next}$, che puntano all'elemento precedente e successivo.

La prima operazione implementata in una lista è la ricerca di un elemento che opera
secondo l'algoritmo di ricerca Lineare in quanto non essendo ordinati i valori della lista
non è possibile implentare la ricerca tramite la ricerca binaria.
Lo pseudocodice della ricerca di un elemento di una lista è il seguente:
%Pseudocodice della ricerca di una Lista
\begin{codebox}
\Procname{$\proc{List-Search}(L,\id{key})$}
\li $x\gets \attrib{L}{head}$
\li \While $x \neq \const{nil}$ and $\attrib{x}{\id{element}} \neq \id{key}$
    \Do
\li                      $x \gets \attrib{x}{\id{next}}$
    \End
\li \Return $\id{x}$
\end{codebox}
La ricerca di un elemento da una lista richiede il seguente tempo con i diversi casi:
\begin{equation*}
  T(n) = c + c(t_w + 1) + ct_w + c
\end{equation*}
\begin{description}
  \item[Caso migliore]:l'elemento da ricercare viene trovato al primo elemento della lista
        per cui $t_w = 0$ indi il tempo di esecuzione è $T(n) = c + c + c = \Omega(1)$
  \item[Caso peggiore]: l'elemento non è presente nella lista per cui $t_w = n$
        \begin{equation*}
          T(n) = c + cn + c + cn + c = 2cn + 3c = O(n)
        \end{equation*}
\end{description}

%Pseudocodice Inserimento in una lista
Il secondo metodo in una lista è $\proc{List-Insert}$ in cui si suppone che il valore
dell'elemento da inserire sia stato già impostato ossia $\id{element}$ abbia il valore desiderato.
\begin{codebox}
\Procname{$\proc{List-Insert}(L,x)$}
\li $\attrib{x}{\id{next}} \gets \attrib{L}{\id{head}}$
\li \If $\attrib{L}{head} \neq \const{nil}$
    \Then
\li                       $\attrib{L}{\attrib{\id{head}}{\id{prev}}} \gets x$
    \End
\li $\attrib{L}{\id{head}} \gets x$
\li $\attrib{x}{\id{prev}} \gets \const{nil}$
\end{codebox}
Il tempo di esecuzione $T(n) = 5c = \Theta(1)$ e tra il caso migliore e peggiore non
vi è alcuna differenza se non il fatto che non viene eseguita soltanto la terza istruzione.

La rimozione di un elemento da una lista, implementata tramite $\proc{List-Delete}$,
prevede di darne il puntatore all'elemento alla procedura per cui per effettuare
la rimozione di un elemento qualsiasi della lista bisogna effettuare la chiamata
a $\proc{List-Search}$ prima per ottenere l'elemento da eliminare.
Lo pseudocodice per la rimozione di un elemento è il seguente:
%Pseudocodice Rimozione da una Lista
\begin{codebox}
\Procname{$\proc{List-Delete}(L,x)$}
\li \If $\attrib{x}{prev} \neq \const{nil}$
    \Then
\li              $\attrib{x}{prev} \gets \attrib{x}{next}$
    \End
\li \Else        $\attrib{L}{head} \gets \attrib{x}{next}$

\li \If $\attrib{x}{next} \neq \const{nil}$
    \Then
\li              $\attribb{x}{\id{next}}{\id{prev}} \gets \attrib{x}{\id{prev}}$
    \End
\end{codebox}
Il tempo di esecuzione è $T(n) = 4c = \Theta(1)$ e tra il caso peggiore e migliore non cambia
nulla però va detto che se si volesse implementare la rimozione da un elemento qualsiasi
della lista si avrebbe un tempo di esecuzione $\Theta(n)$ in quanto si avrebbe la chiamata
alla procedura $\proc{List-Search}$ per stabilire l'elemento da rimuovere.

Le altre implementazioni delle diverse tipologie di liste sono similari soltanto
che implementano o meno l'ordinamento tra gli elementi, considerano o meno il puntatore prev
ed altri considerazioni fatte in base alla tipologia di lista.

Un implementazione alternativa della sequenza, anche se meno intutiva e naturale
di quella presentata fino ad ora, è quella tramite la memorizzazione degli elementi
in un vettore, in cui la posizione di un elemento corrisponde all'indice del vettore.\newline
Questa implementazione permette di passare in maniera costante da un elemento ad un altro,
di accorgersi se si supera un estremo della sequenza, di modificare o leggere il valore
di un elemento anche tramite un accesso diretto tramite indice, ma sfortunatamente
richiede di conoscere la dimensione massima della sequenza per evitare sprechi di memoria
e il tempo di inserimento e cancellazione richiede la scansione della sequenza per cui a tempo $\Theta(n)$.
