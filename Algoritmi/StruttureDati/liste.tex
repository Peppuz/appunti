%Strutture Dati Lista
\section{Liste}
Le liste vengono solitamente usate per implementare il concetto matematico di sequenza lineare di oggetti, in cui si possono eventualmente ripetere gli elementi
ma questo non significa che le liste sono le sequenze dato che quest'ultima rappresenta la specifica del tipo del dato anche se solitamente è naturale usare per 
implementare una sequenza la realizzazione di una lista.\newline
La lista è una struttura dati lineare dinamica in cui l'accesso all'elemento successivo della sequenza avviene tramite un puntatore all'elemento successivo
ed un elemento è composto da un valore , chiamato $\id{element}$ e da due puntatori $\id{prev}$ e $\id{next}$,
i quali puntano all'elemento precedente o successivo della lista.\newline
In caso $\id{prev} \gets \const{nil}$ l'elemento non ha nessun predecessore ed è la $\id{head}$ della lista mentre in caso il puntatore $\id{next} \gets \const{nil}$
l'elemento non ha nessun successore per cui è la coda della lista.
Vi sono diversi tipologie di caratteristiche che una lista può possedere, anche in maniera multipla ossia può possedere più di una proprietà, come si può notare dal seguente elenco:
\begin{itemize}
    \item \emph{singolarmente o doppiamente concatenata}: in caso una lista sia singolarmente
        concatenata si ha soltanto il collegamento con l'elemento successivo, per
        cui viene omesso il puntatore $\id{prev}$, mentre nella lista doppiamente concatenata
         si ha il collegamento, tramite i puntatori, con l'elemento precedente e l'elemento successivo.
     \item \emph{ordinata}: una lista si dice ordinata se è previsto un ordinamento tra i valori
        degli elementi presenti in una lista.
    \item \emph{circolare}: il puntatore $\id{prev}$ della testa della lista punta alla coda
        mentre il puntatore $\id{next}$ della coda della lista punta alla testa
       per cui si può dire che la lista è un anello di elementi.
\end{itemize}
La specifica di una sequenza, qualsiasi implementazione abbia, è la seguente:
\begin{minted}{c}
 void insert(List L, Item x)%inserisce un elemento 
 List* delete(List L)%elimina il primo elemento
 List* search(List L, Item key)%cerca l'elemento key
 Item min(List L)%calcola l'elemento minimo
 Item max(List L)%calcola l'elemento massimo 
 Item successor(List L, Item  key)%calcola l'elemento precedente come valore
 Item predecessor(List L, Item key)%calcola l'elemento successivo come valore 
\end{minted}
Forniamo ora un implementazione di una lista doppiamente concatenata non ordinata
in cui un elemento della lista è composto da un dato chiamato $\id{element}$ e da due
puntatori, chiamati $\id{prev}$ e $\id{next}$, che puntano all'elemento precedente e successivo.

La prima operazione implementata in una lista è la ricerca di un elemento che opera secondo l'algoritmo di ricerca Lineare in quanto non essendo ordinati i valori della lista
non è possibile implentare la ricerca tramite la ricerca binaria.\newline
Lo pseudocodice della ricerca di un elemento di una lista è il seguente:
%Pseudocodice della ricerca di una Lista
\begin{codebox}
\Procname{$\proc{List-Search}(L,\id{key})$}
\li $x\gets \attrib{L}{head}$
\li \While $x \neq \const{nil}$ and $\attrib{x}{\id{element}} \neq \id{key}$
    \Do
\li                      $x \gets \attrib{x}{\id{next}}$
    \End
\li \Return $x$
\end{codebox}
La ricerca di un elemento da una lista richiede il seguente tempo con i diversi casi:
\begin{equation*}
  T(n) = c + c(t_w + 1) + ct_w + c
\end{equation*}
\begin{description}
  \item[Caso migliore]:l'elemento da ricercare viene trovato al primo elemento della lista
        per cui $t_w = 0$ indi il tempo di esecuzione è $T(n) = c + c + c = \Omega(1)$
  \item[Caso peggiore]: l'elemento non è presente nella lista per cui $t_w = n$
        \begin{equation*}
          T(n) = c + cn + c + cn + c = 2cn + 3c = O(n)
        \end{equation*}
\end{description}

%Pseudocodice Inserimento in una lista
Il secondo metodo in una lista è $\proc{List-Insert}$ in cui si suppone che il valore
dell'elemento da inserire sia stato già impostato ossia $\id{element}$ abbia già il valore desiderato.
\begin{codebox}
\Procname{$\proc{List-Insert}(L, x)$}
\li $\attrib{x}{\id{next}} \gets \attrib{L}{\id{head}}$
\li \If $\attrib{L}{head} \neq \const{nil}$
    \Then
\li                       $\attrib{L}{\attrib{\id{head}}{\id{prev}}} \gets x$
    \End
\li $\attrib{L}{\id{head}} \gets x$
\li $\attrib{x}{\id{prev}} \gets \const{nil}$
\end{codebox}
Il tempo di esecuzione $T(n) = 5c = \Theta(1)$ e tra il caso migliore e peggiore non
vi è alcuna differenza se non il fatto che non viene eseguita soltanto la terza istruzione.

La rimozione di un elemento da una lista, implementata tramite $\proc{List-Delete}$, prevede di darne il puntatore all'elemento alla procedura per cui per effettuare
la rimozione di un elemento qualsiasi della lista bisogna effettuare la chiamata a $\proc{List-Search}$ prima per ottenere l'elemento da eliminare.\newline
Lo pseudocodice per la rimozione di un elemento è il seguente:
%Pseudocodice Rimozione da una Lista
\begin{codebox}
\Procname{$\proc{List-Delete}(L,x)$}
\li \If $\attrib{x}{prev} \neq \const{nil}$
    \Then
\li              $\attrib{x}{prev} \gets \attrib{x}{next}$
    \End
\li \Else        $\attrib{L}{head} \gets \attrib{x}{next}$

\li \If $\attrib{x}{next} \neq \const{nil}$
    \Then
\li              $\attribb{x}{\id{next}}{\id{prev}} \gets \attrib{x}{\id{prev}}$
    \End
\end{codebox}
Il tempo di esecuzione è $T(n) = 4c = \Theta(1)$ e tra il caso peggiore e migliore non cambia
nulla però va detto che se si volesse implementare la rimozione da un elemento qualsiasi
della lista si avrebbe un tempo di esecuzione $\Theta(n)$ in quanto si avrebbe la chiamata
alla procedura $\proc{List-Search}$ per stabilire l'elemento da rimuovere.

%Pseudocodice Massimo e minimo della lista
Per trovare il massimo e/o il minimo di una lista bisogna scansionare tutta la lista
e scegliere l'elemento con valore massimo/minimo come si farebbe anche se si dovesse
trovare il massimo di un array; lo pseudocodice del massimo e del minimo sono i seguenti:
\begin{codebox}
\Procname{$\proc{List-Max}(L)$}
\li \If $\attrib{L}{head} \isequal \const{nil}$
    \Then
\li                \Return $\const{nil}$
\li $\id{max} \gets \attrib{L}{head}$
\li $\id{elem} \gets \attribb{L}{head}{next}$
\li \While $\id{elem} \neq \const{nil}$
    \Do
\li                 \If $\attrib{\id{elem}}{\id{value}} > \attrib{max}{value}$
                    \Then
\li                              $\id{max} \gets \id{elem}$
                    \End
\li                 $\id{elem} \gets \attrib{\id{elem}}{\id{next}}$
    \End
\li \Return $\id{max}$
\end{codebox}
Senza effettuare un'analisi linea per linea si può facilmente notare che il calcolo
del massimo di una lista di $n$ elementi richieda $\Theta(n)$ in quanto bisogna
scansionare tuttti gli elementi per poterne determinare il massimo.

\begin{codebox}
\Procname{$\proc{List-Min}(L)$}
\li \If $\attrib{L}{head} \isequal \const{nil}$
    \Then
\li                \Return $\const{nil}$
\li $\id{min} \gets \attrib{L}{\id{head}}$
\li $\id{elem} \gets \attribb{L}{\id{head}}{\id{next}}$
\li \While $\id{elem} \neq \const{nil}$
    \Do
\li                 \If $\attrib{\id{elem}}{\id{value}} < \attrib{\id{min}}{\id{value}}$
                    \Then
\li                              $\id{min} \gets \id{elem}$
                    \End
\li                 $\id{elem} \gets \attrib{\id{elem}}{\id{next}}$
    \End
\li \Return $\id{min}$
\end{codebox}
Ovviamente, come anche già notato nel calcolo del massimo di una lista, il tempo di
esecuzione per il calcolo del minimo è $\Theta(n)$ in quanto bisogna sempre scansionare la lista.

%Pseudocodice Predecessore e successore della Lista
Le ultime due procedure da implementare sono quello di trovare l'elemento successore
e predecessore come valore di un elemento ossia viene definito come successore il più
piccolo elemento più grande dell'elemento dato mentre il predecessore il più grande
elemento più piccolo come valore dell'elemento dato;ovviamente, come anche già visto
per il minimo e il massimo, bisogna scandirre tutta la lista per stabilire il prede/successore
e lo pseudocodice è il seguente:
\begin{codebox}
    \Procname{$\proc{List-Successor}(L, \id{key})$}
\li \If $\attrib{L}{head} \isequal \const{nil}$
    \Then
\li                \Return $\const{nil}$
\li $\id{succ} \gets \attrib{L}{head}$
\li \While $\id{succ} \neq \const{nil}$ and $\id{succ}{value} \leq \id{key}$
    \Do
\li           $\id{succ} \gets \attrib{\id{succ}}{\id{next}}$
    \End
\li \If $\id{succ} \isequal \const{nil}$ \Return $\const{nil}$
\li $\id{elem} \gets \attrib{\id{succ}}{\id{next}}$
\li \While $\id{elem} \neq \const{nil}$
    \Do
\li      \If $\attrib{\id{elem}}{\id{value}} > \id{key}$ and $\attrib{\id{elem}}{\id{value}} < \attrib{succ}{\id{value}}$
         \Then
\li                              $\id{succ} \gets \id{elem}$
         \End
\li      $\id{elem} \gets \attrib{\id{elem}}{\id{next}}$
    \End
\li \Return $\id{succ}$
\end{codebox}
Il tempo di esecuzione è $\Theta(n)$ in quanto sia nel caso migliore che in quello
peggiore bisogna analizzare tutti gli elementi della lista.

\begin{codebox}
    \Procname{$\proc{List-Predeccessor}(L, \id{key})$}
\li \If $\attrib{L}{head} \isequal \const{nil}$
    \Then
\li                \Return $\const{nil}$
\li $\id{prev} \gets \attrib{L}{head}$
\li \While $\id{prev} \neq \const{nil}$ and $\id{prev}{value} \geq \id{key}$
    \Do
\li           $\id{prev} \gets \attrib{\id{prev}}{\id{next}}$
    \End
\li \If $\id{prev} \isequal \const{nil}$ \Return $\const{nil}$
\li $\id{elem} \gets \attrib{\id{prev}}{\id{next}}$
\li \While $\id{elem} \neq \const{nil}$
    \Do
\li      \If $\attrib{\id{elem}}{\id{value}} < \id{key}$ and $\attrib{\id{elem}}{\id{value}} > \attrib{prev}{\id{value}}$
         \Then
\li                              $\id{prev} \gets \id{elem}$
         \End
\li      $\id{elem} \gets \attrib{\id{elem}}{\id{next}}$
    \End
\li \Return $\id{prev}$
\end{codebox}
Come anche per il successore, il tempo di esecuzione per trovare il predecessore di
un elemento è $\Theta(n)$ in quanto bisogna analizzare tutti gli elementi di una lista.

Le altre implementazioni delle diverse tipologie di liste sono similari soltanto che implementano o meno l'ordinamento tra gli elementi,
considerano o meno il puntatore $\id{prev}$ ed altre considerazioni fatte in base alla tipologia di lista.

Un implementazione alternativa della sequenza, anche se meno intutiva e naturale di quella presentata fino ad ora,
è quella tramite la memorizzazione degli elementi in un vettore, in cui la posizione di un elemento corrisponde all'indice del vettore.\newline
Questa implementazione permette di passare in maniera costante da un elemento ad un altro, di accorgersi se si supera un estremo della sequenza,
di modificare o leggere il valore di un elemento anche tramite un accesso diretto tramite indice, ma sfortunatamente richiede di conoscere 
la dimensione massima della sequenza per evitare sprechi di memoria e il tempo di inserimento e cancellazione richiede la scansione della sequenza quindi a tempo $\Theta(n)$.

Si evita di mostrare lo pseudocodice di questa implementazione alternativa perchè viene usata raramente e perchè conoscendo il funzionamento della sequenza è banale
implementare le sue operazioni mediante un vettore di elementi.
