\documentclass[a4paper, 11pt]{book}
\usepackage[T1]{fontenc} %Gestione font in output
\usepackage[utf8]{inputenc} %Permette di inserire caratteri Italiani
\usepackage[italian]{babel} %Gestione sillabazione ed altro italiana
\usepackage{microtype}%Pacchetto per migliorare la gestione del riempimento della pagina
\usepackage{amsmath} %Package gestione Matematica
\usepackage{amsthm}%Package per le dimostrazioni
\usepackage{classicthesis}
\usepackage{arsclassica}
\usepackage{listings}%Package per introdurre i listati di codice
\usepackage{clrscode3e}%Package Algoritmi stile Cormen
\usepackage{booktabs}%Gestione Tabelle(Guardare Documentazione)
\usepackage{caption}%Package per le Tabelle(guardare Documentazione)
\usepackage{forest}%Package per disegnare Alberi
\usepackage{tkz-graph}%Package per disegnare Grafi
\usepackage{minted}%Code Highlighting package
\newtheorem{thm}{Teorema}%Modalità per rappresentare i teoremi 
\setlength{\parindent}{0pt}%Evita il rientro dei paragrafi e capoversi

\begin{document}
%Titolo autore e spiegazione
\title{Appunti di Algoritmi e Strutture Dati}
\author{Marco Natali}
\maketitle
\tableofcontents
`
\chapter{Introduzione agli Algoritmi}
Il termine Algoritmo proviene da Mohammed ibn-Musa al-Khwarizmi, matematico uzbeco
del IX secolo a.c. da cui proviene la moderna Algebra. \newline
\textbf{Algoritmo}:sequenza di passi che portano alla risoluzione di un problema

\textbf{Pseudocodice}: un linguaggio utilizzato per rappresentare e presentare gli algoritmi
 in maniera compatta e chiara;ogni libro e programmatore definisce la propria specifica
di Pseudocodice ma comunque quasi tutti si ispirano alla sintassi del Pascal,C e Java.

Gli algoritmi permettono di poter migliorare e rendere il più efficiente e veloce
la risoluzione di un problema e l'esecuzione di un programma.

Il primo algoritmo che è affrontiamo è $\proc{Insertion-Sort}$ che risolve il problema
dell'ordinamento di sequenze di dati.
%Algoritmo di Insertion Sort
L'algoritmo $\proc{Insertion-Sort}$ risolve il problema dell'ordinamento definito come:\newline
\textbf{Input}:una sequenza di $n$ numeri $(a_1,a_2,\dots,a_n)$ \newline
\textbf{Output}:una permutazione $(a'_1,a'_2,\dots,a'_n)$ tale che $a'_1 \leq a'_2 \leq \dots \leq a'_n$

%pseudocodice algoritmo insertionSort
\begin{figure}
    \caption{Algoritmo insertionSort}
    \label{alg:insertion}
    \begin{codebox}
        \Procname{$\proc{Insertion-Sort}(A)$}
        \li \For $j \gets 2$ \To $\attrib{A}{length}$
            \Do
        \li            $\id{key} \gets A[j]$
        \li         $i \gets j-1$
        \li         \While $i > 0$ and $A[i] > \id{key}$
                    \Do
        \li                $A[i+1] \gets A[i]$
        \li                $i \gets i-1$
                    \End
        \li         $A[i+1] \gets \id{key}$
            \End
    \end{codebox}
\end{figure}
L'algoritmo $\proc{Insertion-Sort}$ \ref{alg:insertion} è un algoritmo efficiente per ordinare un ristretto numero di elementi ed opera come farebbe un umano
a riordinare le carte da gioco, ossia prendendo una carta alla volta e facendo il riordinamento delle carte una alla volta.\newline
Per poter affermare che l'algoritmo è corretto, ossia risolve il problema, bisogna dimostrare l'invariante del ciclo $\kw{For}$, attraverso un metodo simile all'induzione matematica.

%Dimostrazione Invariante di Ciclo
L'invariante del ciclo è corretta se si riesce a dimostrare tre cose:
\begin{description}
  \item[Inizializzazione] è corretta prima della prima esecuzione del ciclo.
  \item[Conservazione] se è verificata prima di un iterazione del ciclo lo sarà anche dopo l'esecuzione di quell'iterazione del ciclo.
  \item[Conclusione] alla fine del ciclo è ancora verificato e ciò ci aiuta a determinare la correttezza di un algoritmo.
\end{description}
La terza proprietà è la più importante in quanto assieme alla condizione che è causato la conclusione del ciclo, si riesce a dimostrare la correttezza dell'algoritmo.

L'invariante di ciclo per l'$\proc{Insertion-Sort}$ è:
All'inizio di ogni iterazione del ciclo $\kw{for}$ il sottoarray $A[1 \twodots j-1]$
è ordinato ed è formato dagli stessi elementi che erano originamente in $A[1\twodots j-1]$.
\begin{description}
  \item[Inizializzazione] quando $j = 2$ il sottoarray $A[1\twodots j-1]$ è formato
                           da un solo elemento che è ordinato ed è l'elemento originale $A[1]$.
  \item[Conservazione] all'inizio di ogni esecuzione del ciclo for il sottoarray $A[1 \twodots j-1]$
                        è formato dai primi $j-1$ elementi dell'array ordinati dal
                        più piccolo al più grande.

  \item[Conclusione] Quando $j > \attrib{A}{length}$ il ciclo termina e dato che ogni
                      ciclo aumenta $j$ di $1$ alla fine del ciclo si avrà $j = n + 1$
                      per cui si ha che $A[1\twodots n]$ è ordinato ed è formato
                      dagli elementi ordinati che si trovavano in $A[1 \twodots n]$.
\end{description}
L'analisi di un algoritmo, per poter determinare se un algoritmo è efficiente, può
avvenire in due maniere:
\begin{description}
    \item [Tempo di Esecuzione] è il numero di operazioni primitive che vengono eseguite da parte di un algoritmo;l'esecuzione di un'istruzione si assume che richiede un tempo costante
                                per evitare di rendere la valutazione dipendente dall'hardware e dalla bravura del programmatore.
    \item [Spazio di Esecuzione] è il numero di spazio in bit occupato in memoria dall'algoritmo ma questa analisi non viene quasi mai eseguita in quanto oramai è superfluo.
\end{description}
Il tempo di esecuzione dell'algoritmo è la somma dei tempi di esecuzione per ogni istruzione eseguita quindi il tempo di esecuzione di $\proc{Insertion-Sort}$ è:
\begin{equation*}
    T(n) = c_1n + c_2(n-1) + c_3(n-1) + c_4 \sum_{j=2} ^n t_j + c_5 \sum_{j=2} ^n (t_j -1)
         + c_6 \sum_{j=2} ^n (t_j -1) + c_7(n-1)
\end{equation*}
In caso l'algoritmo sia già ordinato, caso migliore, si avrebbe sempre $A[i] < \id{key}$ quindi $t_j$ è sempre $1$, per cui il tempo di esecuzione sarebbe:
\begin{align*}
    T(n) & = c_1n + c_2(n-1) + c_3(n-1) + c_4(n-1) + c_7(n-1) \\
         & = (c_1 + c_2 + c_3 + c_4 + c_5)n - (c_2 + c_3+ c_4+ c_7) \\
         & = \Omega(n) \\
\end{align*}
Nel caso migliore si ha che l'algoritmo richiede un tempo lineare che è un $\Omega(n)$.\newline
In caso si abbia una sequenza decrescente, corrispondente al caso peggiore, nel ciclo While bisogna
confrontare ogni elemento $A[j]$ con il sottoarray $A[1 \twodots j-1]$ per cui $t_j = j$
per $j=2,3,\dots,n$ e poiche si ha
\begin{align*}
    \sum _{j=2} ^ n j = \frac{n(n+1)}{2} -1 & \quad \sum_{j=2}^n (j-1) = \frac{n(n-1)}{2}
\end{align*}
il tempo dell'algoritmo $\proc{Insertion-Sort}$ nel caso peggiore è il seguente:
\begin{align*}
    T(n) & = c_1n + c_2(n-1) + c_3(n-1) + c_4 (\frac{n(n+1)}{2} -1) + c_5 (\frac{n(n-1)}{2})
         + c_6 (\frac{n(n-1)}{2}) + c_7(n-1) \\
         & = c_1n + c_2(n-1) + c_3(n-1) + c_4(\frac{n^2+n-2}{2}) + c_5(\frac{n^2-n}{2})
           + c_6(\frac{n^2-n}{2}) + c_7(n-1) \\
         & = (\frac{c_4}{2} +\frac{c_5}{2} + \frac{c_6}{2})n^2 + (c_1+c_2+c_3+\frac{c_4}{2} - \frac{c_5}{2}
            -\frac{c_6}{2}+c_7)n -(c_2+c_3+c_4+c_7)\\
         & = O(n^2)\\
\end{align*}
Il tempo dell'algoritmo può essere scritto, nel caso peggiore, come $an^2 + bn + c$ che è una funzione
quadratica che viene indicata, nel caso peggiore come $O(n^2)$.

Nel caso medio mi aspetto, supponendo una distribuzione uniforme della probabilità,
che il vettore sia parzialmente ordinato per cui non avendo modo di rendere il ciclo
while eseguibile solo una volta, si ha bisogno almeno di $n^2$ confronti che è $O(n^2)$.

In sintesi i tempi di esecuzione dell'algoritmo $\proc{Insertion-Sort}$ sono:
\begin{description}
  \item[Caso migliore] $\Omega(n)$
  \item[Caso peggiore] $O(n^2)$
  \item[Caso medio] $O(n^2)$
\end{description}
%Algoritmo di Insertion Sort

Un altro algoritmo per risolvere il problema dell'ordinamento è il $\proc{Selection-Sort}$,
che trovando via a via i più piccoli elementi della sequenza e li mette negli elementi più a sinistra.
%Esercizio 2.2-2 ordinamento mediante selectionSort
\textbf{Input}:una sequenza di $n$ numeri $(a_1,a_2,\dots,a_n)$ \newline
\textbf{Output}:una permutazione $(a'_1,a'_2,\dots,a'_n)$ tale che $a'_1 \leq a'_2 \leq \dots \leq a'_n$

%algoritmo selectionSort(A)
\begin{codebox}
    \Procname{$\proc{Selection-Sort}(A)$}
\li \For $j = 1$ \To $\attrib{A}{length-1}$
    \Do
\li           $\id{min} \gets A[j]$
\li           \For $i = j+1$ \To $\attrib{A}{length}$
              \Do
\li                  \If $A[j] < \id{min}$
                     \Then
\li                         $\id{index} \gets i$
\li                         $\id{min} \gets A[i]$
                     \End
              \End
\li           $\id{temp} \gets A[j]$
\li           $A[j] \gets \id{min}$
\li           $A[\id{index}] \gets \id{temp}$
\end{codebox}
%Algoritmo di Selection Sort

\section{Ricerca di Valori}
Un altro importante problema da risolvere è la \emph{ricerca} di valori all'interno
di una sequenza di dati in quanto di solito quando si effettua un qualsiasi Algoritmo
può capitare di dover effettuare una ricerca all'interno dei dati.
Vi sono due algoritmi per effettuare la ricerca di valori all'interno di una sequenza:
il primo è $\proc{linearSearch}$ che effettua una scansione della sequenza mentre
l'altro è il $\proc{binarySearch}$, algoritmo divide et impera che opera su sequenze ordinate.
Il secondo è utile quando si sa che i dati saranno e sono ordinati altrimenti effettuare
l'ordinamento rende vano il guadagno dell'algoritmo rispetto alla ricerca Lineare.

\begin{algorithm}
\caption{\proc{linearSearch}(ITEM[] A,\textbf{integer} \id{key})}
\begin{codebox}
\li  \For $j = 0$ \To A.length-1
     \Do
\li           \If A[j] = key
              \Do
\li                    \Return A[j]
              \End
      \End
\li \Return \null
\end{codebox}
\end{algorithm}
%Algoritmo linearSearch


%Analisi Algoritmo binarySearch
Il secondo algoritmo di ricerca è il $\proc{binarySearch}$, algoritmo divide et impera
in cui viene già previsto che la sequenza di valori sia ordinata e sfruttando ciò
ad ogni passo viene eliminata una parte della sequenza.\newline
La ricerca binaria ad ogni passo controlla l'elemento mediano per vedere se è il valore
ricercato altrimenti richiama l'algoritmo sulla sottosequenza sinistra in caso il valore
da trovare sia minore oppure lo effettua sulla sottosequenza destra.

%Pseudocodice binarySearch
\begin{codebox}
\Procname{$\proc{binarySearch}(A,\id{left},\id{right},\id{key})$}
\li \If $\id{left} \isequal \id{right}$
    \Then
\li                \If $A[\id{left}] \isequal \id{key}$
                   \Then
\li                            \Return $\id{left}$
\li                \Else \Return $\const{nil}$
    \End
\li \Else
\li                $\id{mid} \gets (\id{left} + \id{right}) / 2$
\li                \If $A[\id{mid}] \isequal \id{key}$
\li                   \Then \Return $\id{mid}$
                   \End
\li                \If $A[\id{mid}] > \id{key}$
                   \Then
\li                          \Return $\proc{binarySearch}(A,\id{left},\id{mid}-1,\id{key})$
\li                \Else \Return $\proc{binarySearch}(A,\id{mid}+1,\id{right},\id{key})$
    \End
\end{codebox}

Il tempo di esecuzione del $\proc{binarySearch}$ nei diversi casi è il seguente:
\begin{description}
  \item[Caso migliore:] l'elemento $\id{key}$ viene trovato direttamente nell'elemento mediano
        per cui viene risolto in un tempo costante $\Omega(1)$
  \item[Caso peggiore:] l'elemento $\id{key}$ non viene trovato nella sequenza quindi
        \begin{equation*}
           T(n) = \begin{cases} 3c = \Theta(1) \quad \text{se} \ n = 1 \\
                                T(\frac{n}{2}) + 4c \quad \text{se} \ n > 1\\
                  \end{cases}
        \end{equation*}
        Applicando il secondo caso del teorema dell'esperto in quanto $4c = \Theta(n^{\log _2 1}) = \Theta(1)$
        si ha che $T(n) = \Theta(\log n)$
\end{description}
%Algoritmo binarySearch
 %Primo Capitolo Introduzione agli Algoritmi
%Capitolo sulla tecnica Divide et Impera
\chapter{Divide et Impera}
Nello sviluppo dell'algoritmo $\proc{Insertion-Sort}$ abbiamo utilizzato una
struttura incrementale, detta anche iterativa, ma in informatica per sviluppare
gli algoritmi si può utilizzare una forma alternativa, chiamata \emph{Divide et Impera},
come specificato in questo paragrafo.

Molti utili algoritmi sono ricorsivi, ossia per risolvere un particolare problema
questi algoritmi chiamano se stessi per risolvere sottoproblemi dello stesso tipo.
Generalmente gli algoritmi ricorsivi adottano un approccio \textbf{divide et impera},
il quale prevede tre passi a ogni livello di ricorsione:
\begin{description}
    \item[Divide]il problema viene diviso in un certo numero di sottoproblemi, istanze
                  più piccole del problema.
    \item[Impera]i sottoproblemi vengono risolti in maniera ricorsiva
    \item[Combina]le soluzioni dei sottoproblemi vengono combinate per generare
                   la soluzione del problema originario
\end{description}

%Paragrafo sull'algoritmo mergeSort ed introduzione alla ricorsione
\section{MergeSort}
L'algoritmo $\proc{Merge-Sort}$, che risolve il problema dell'ordinamento, opera seguendo il paradigma divide et impera:
\begin{description}
    \item[Divide] divide la sequenza di $n$ elementi in due sottosequenze di $n/2$ elementi ciascuna.
    \item[Impera] ordina le due sottosequenze in maniera ricorsiva mediante l'algoritmo $\proc{Merge-Sort}$.
    \item[Combina] fonde le due sottosequenze ordinate per generare la sequenza ordinata.
\end{description}
Per effettuare la fusione utilizzo una procedura ausiliaria $\proc{Merge}(A,\id{left},\id{mid},\id{right})$,
dove $A$ è un array e $\id{left},\id{mid},\id{right}$ sono degli indici tali che
$\id{left} \leq \id{mid} \leq \id{right}$ e la procedura assume che i sottoarray
$A[\id{left} \twodots \id{mid}]$ e $A[\id{mid}+1 \twodots \id{right}]$ siano ordinati
e li fonde per formare il sottoarray $A[\id{left} \twodots \id{right}]$ ordinato.
Utilizziamo un elemento sentinella, elemento con un valore speciale tipo $\infty$,
per semplificare il nostro pseudocodice.
%Procedura Merge(A,p,q,r)
\begin{codebox}
    \Procname{$\proc{Merge}(A,\id{left},\id{mid},\id{right})$}
\li $n_1 \gets q - p + 1$
\li $n_2 \gets r - q$
\li crea due nuovi array $L[1 \twodots n_1]$ e $R[1 \twodots n_2]$
\li \For $i \gets 1$ \To $n_1$
         \Do
\li      $L[i] \gets A[p+i-1]$
        \End
\li \For $j \gets 1$ \To $n_2$
        \Do
\li      $R[j] \gets A[q+j]$
        \End
\li $L[n_1+1] = \infty$
\li $R[n_2+1] = \infty$
\li $i \gets 1$
\li $j \gets 1$
\li \For $k = \id{left}$ \To $\id{right}$
    \Do
\li      \If $L[i] \leq R[j]$
         \Then
\li              $A[k] = L[i]$
\li              $i = i+1$
         \End
\li      \Else $A[k] \gets R[j]$
\li            $j \gets j+1$
    \End
\end{codebox}

%Invariante di Ciclo con dimostrazione

%Costo procedura
la procedura $\proc{Merge}$ ha un costo $\Theta(n)$, con $n = \id{right} - \id{mid} + 1$, in quanto
i tre cicli for presenti nell'algoritmo richiedono nel caso peggiore $n$ iterazioni
e non essendo annidati richiedono soltanto un tempo lineare di esecuzione.
Ora possiamo utilizzare le procedura merge nell'algoritmo $\proc{Merge-Sort}$, il quale
ordina gli elementi nel sottoarray $A[\id{left}\twodots \id{right}]$.
In caso $\id{left} \geq \id{right}$, il sottoarray ha al massimo un elemento e, quindi, è già ordinato;
altrimenti il passo "Divide" calcola semplicemente un indice $q$ che separa il sottoarray
in due sottoarray di $n/2$ elementi, come mostrato dal pseudocodice:
%Procedura MergeSort(A,p,r)$
\begin{codebox}
    \Procname{$\proc{Merge-Sort}(A,\id{left},\id{right})$}
\li \If $\id{left} < \id{right}$
    \Then
\li     $\id{mid} \gets (\id{left} + \id{right})/ 2$
\li     $\proc{Merge-Sort}(A,\id{left},\id{mid})$
\li     $\proc{Merge-Sort}(A,\id{mid} + 1,\id{right})$
\li     $\proc{Merge}(A,\id{left},\id{mid},\id{right})$
    \End
\end{codebox}

Per ordinare l'intera sequenza $A = (A[1],A[2],\dots,A[n])$ effettuiamo la chiamata
iniziale $\proc{Merge-Sort}(A,1,\attrib{A}{length})$

%Analisi algoritmo Divide et Impera
\section{Analisi algoritmi divide et impera}
In caso un algoritmo contiene una chiamata a se stesso, il suo tempo di esecuzione
spesso può essere espresso mediante un'\textbf{equazione di ricorrenza}, in cui
si esprime il tempo di esecuzione totale in funzione del tempo di esecuzione dei sottoproblemi.

Una ricorrenza per un algoritmo divide et impera si basa sui 3 passi del paradigma di base;
se la dimensione del problema è sufficientemente piccola,
per esempio $n \leq c$ per qualche costante $c$, allora il tempo di esecuzione è
costante, indicato con $\Theta(1)$.
In caso contrario serve un tempo $aT(n/b)$ per risolvere i sottoproblemi, con $a$
indicante il numero di sottoproblemi generati e $b$ indicante il rapporto di grandezza
tra il problema e i sottoproblemi, ed indicando con $D(n)$ il tempo per dividere
i sottoproblemi e indicando con $C(n)$ il tempo per combinare le soluzioni dei
sottoproblemi si ottiene la seguente ricorrenza
\begin{equation*}
    T(n) = \begin{cases} \Theta(1) \quad \text{se} \ n \leq c \\
                         aT(n/b) + D(n) + C(n) \quad \text{se} \ n > c\\
            \end{cases}
\end{equation*}

\subsection{Analisi di Merge Sort}
Lo pseudocodice di $\proc{Merge-Sort}$ funziona per qualsiasi dimensione ma, al fine
di agevolare i calcoli, supponiamo che la dimensione del problema originale sia una
potenza di 2, per cui ogni passo divide genera due sottosequenze di dimensione pari a $n/2$.

L'algoritmo merge sort se applicato a un solo elemento impiega un tempo costante $\Theta(1)$,
altrimenti suddividiamo il tempo di esecuzione nel seguente modo:
\begin{description}
    \item[Divide] questo passo semplicemente calcola il centro del sottoarray quindi
                  richiede un tempo costante $\Theta(1)$
    \item[Impera] risolviamo in maniera ricorsiva i due sottoproblemi di dimensione $n/2$
                  e ciò richiede $2T(n/2)$ per la risoluzione
    \item[Combina] la procedura $\proc{Merge}$ richiede un tempo $\Theta(n)$
\end{description}
Quando sommiamo $\Theta(1)$ e $\Theta(n)$ otteniamo una funzione lineare che è $\Theta(n)$
e per cui l'equazione di ricorrenza di $\proc{Merge-Sort}$ è:
\begin{equation*}
    T(n) = \begin{cases} \Theta(1) \quad \text{se} \ n = 1 \\
                         2T(\frac{n}{2}) + \Theta(n) \quad \text{se} \ n > 1 \\
           \end{cases}
\end{equation*}
Per risolvere codesta equazione di ricorrenza vi sono 3 modalità , che saranno poi analizzate,
però comunque il tempo di esecuzione nel caso peggiore è $O(n \log n)$.
%Algoritmo Merge Sort

%%Algoritmo per trovare il massimo e il minimo in maniera simultanea
 Algoritmo per trovare il massimo e minimo simultaneo con Divide et Impera
Per risolvere l'equazione di ricorrenza, funzione che descrive il tempo di esecuzione
in funzione del tempo di esecuzione dei sottoproblemi, vi sono 4 metodi:
\begin{description}
    \item[Metodo di Sostituzione]:ipotiziamo un tempo di esecuzione e utilizziamo
                l'induzione matematica per dimostrare la correttezza dell'ipotesi
    \item[Metodo dell'albero di Ricorsione]:converte l'equazione in un albero i cui
          nodi rappresentano i costi ai vari livelli della ricorsione
    \item[Metodo dell'Esperto]:fornisce i limiti dell'equazioni di ricorrenza che
          rispettano determinate condizioni(Analizzato nei prossimi paragrafi)
    \item[Metodo di espansione] si espande l'equazione di ricorrenza fino ad arrivare ai casi base
          ad esempio $T(n) = T(n-1) + 3$ si espande $T(n-1),T(n-2)$ fino ad arrivare al caso base.
\end{description}

%paragrafo sul metodo di Sostituzione per risolvere le ricorrenze
\section{Il metodo di Sostituzione}
Il metodo di sostituzione è un tecnica di risoluzione delle equazioni di ricorrenza
degli algoritmi divide et impera, che richiede due passi:
\begin{enumerate}
    \item Ipotizzare la forma della risoluzione.
    \item Usare l'induzione matematica per trovare le costanti e dimostrare che
          la soluzione proposta funziona ed è corretta.
\end{enumerate}
Per vedere il funzionamento di questo metodo proviamo ad usarlo per mostrare il seguente teorema
%Dimostrazione
\begin{thm}
    $T(n) = 2T(n/2) + n = \Theta(n \ln n)$
\end{thm}
\begin{proof}
Per dimostrare che $T(n) = \Theta(n \ln n)$ bisogna provare che $T(n) \leq cn \ln n$ per una costante $c > 0$
\begin{equation*}
\begin{split}
T(n) & \leq 2(cn/2 \lg n/2) + n \\
     & \leq cn \lg n - cn \lg 2 + n\\
     & \leq cn \lg n - cn + n \\
     & \leq cn \lg n \text{per} c \geq 1\\
\end{split}
\end{equation*}
L'induzione matematica richiede di verificare i casi base ma solitamente le condizioni di contorno non vengono dimostrate in quanto può essere molto difficile 
in alcuni casi e per risolvere si sfrutta la notazione asintotica, che prevede una costante $n_0$ arbitrariemente scelta e con questo si può rimuovere le condizioni
di contorno difficili da dimostrare.
\end{proof}

%Scegliere una buona ipotesi
Per scegliere una buona ipotesi da verificare mediante il metodo di sostituzione
richiede fantasia, esperienza, cosa che non sempre si dispone soprattutto la fantasia.\newline
Per aiutarci ad ottenere una buona ipotesi si potrebbe utilizzare l'albero di sostituzione
e poi dimostrare l'ipotesi mediante induzione, con il metodo di sostituzione, oppure
ci sono delle euristiche per diventare dei buoni indovini.

Le euristiche per formulare buone ipotesi sono le seguenti:
\begin{itemize}
    \item Se una ricorrenza è simile ad una già analizzata conviene provare a dimostrare
          la stessa soluzione come ad esempio:
          \begin{equation*}
              T(n) = 2T(\frac{n}{2} + 17) + n \text{è simile all'equazione precedente ed è un} \Theta(n \lg n)
          \end{equation*}
    \item Si inizia a dimostrare dei limiti superiori ed inferiori molto larghi e
          poi restringere l'incertezza alzando il limite inferiore ed abbassando
          il limite superiore fino a convergere con il risultato corretto.
\end{itemize}

%esempi

%Errori tipici e togliere termini di ordine inferiore
Ci sono dei casi in cui ipotiziamo correttamente un limite asintotico per la ricorrenza
ma in qualche modo sembra che i calcoli matematici non tornino nell'Induzione.
Per superare questo ostacolo spesso basta correggere l'ipotesi sottraendo un termine
di ordine inferiore per far quadrare i conti, come nell'esempio:
\[    T(n) = T(\lfloor n/2 \rfloor) + T(\lceil n/2 \rceil) + 1 \]
Supponiamo che $T(n) = O(n)$ ossia $T(n) \leq cn$, otteniamo nella ricorrenza:
\[ \begin{split}
    T(n) & \leq c \lfloor n/2 \rfloor + c \lceil n/2 \rceil + 1 \\
         & = cn + 1
 \end{split} \]
Questa equazione non implica che $T(n) \leq cn$ qualunque sia il valore di $c$ però
l'intuizione che $T(n) = O(n)$ è corretta solo che per provarla dobbiamo utilizzare
un ipotesi induttiva più forte, ossia $T(n) \leq cn - d$ con $d \geq 0$ rappresentante una costante.
\[ \begin{split}
    T(n) & \leq (c \lfloor n/2 \rfloor - d) + (c \lceil n/2 \rceil -d) + 1 \\
         & \leq cn - 2d + 1 \\
         & \leq cn - d \quad \text{per ogni} d \geq 1 \\
 \end{split} \]
A volte una piccola manipolazione algebrica può rendere una ricorrenza ignota simile a una che abbiamo già visto ed analizzato per esempio 
\[ T(n) = 2T(\lfloor \sqrt{n} \rfloor) + \log n \]
sembra difficile da risolvere ma, ignorando l'arrotondamento agli interi di valori come $\sqrt{n}$, ponendo $m = \log n$ si ottiente
\[ T(2^m) = 2T(2^{\frac{m}{2}}) + m \]
Ponendo ora $S(m) = T(2^m)$ si ottiene la ricorrenza $S(m) = 2S(\frac{m}{2}) + m$ che è simile alla ricorrenza analizzata nel mergeSort.
%Paragrafo sul metodo di sostituzione

%Paragrafo sull'albero di Ricorsione
\section{Albero di Ricorsione}
In un albero di Ricorsione ogni nodo rappresenta il costo del sottoproblema e noi
sommiamo tutti i(Da vedere la traduzione in Italiano per scriverla meglio!!!!)
Di solito si utilizza l'albero di ricorsione per generare un ipotesi da dimostrare
mediante induzione, con il metodo di sostituzione, per cui si può tollerare un pò
di incertezza nell'analisi, ad esempio togliere le costanti.

Albero di ricorsione dell'equazione di ricorrenza $T(n) = 3T(n/4) + \Theta(n^2)$

%Tradurre e fare grafici degli alberi!!!!!!!

%Scrivere proprietà alberi di avere l'altezza uguale a log n
%Paragrafo sul metodo dell'albero di Ricorsione

%Paragrafo sul Teorema dell'Esperto
\section{Teorema dell'Esperto}
%Paragrafo sul metodo dell'Esperto
%Capitolo 2 Divide et Impera
\chapter{Crescita delle Funzioni}
Nel valutare l'algoritmo $\proc{min}$ si ottiene una funzione $T(n) = an + b$ che
è una funzione lineare.\newline
Durante la valutazione di un algoritmo difficilmente si riesce a quantificare con
esattezza le costanti coinvolte per cui si analizza il comportamento della funzione
al tendere di n all'infinito.\newline
Al tal fine si utilizzano le notazioni $O,\ \Omega,\ \Theta\ $ definite come segue:

\begin{itemize}
  \item $f(n) \in O(g(n)) \Leftrightarrow\ \exists c \geq 0\ \exists m \geq 0 : f(n) \leq cg(n)\ \forall n \geq m$
  \item $f(n) \in \Omega(g(n)) \Leftrightarrow \exists c, m \geq 0 : f(n) \geq cg(n) \forall n \geq m$
  \item $f(n) \in \Theta(g(n)) \Leftrightarrow \exists c_1,c_2,m \geq 0 :
        c_1g(n) \leq f(n) \leq cg(n)\ \forall n \geq m$
\end{itemize}
Per maggiore chiarezza si utilizza un'abuso di linguaggio scrivendo $f(n) = O(g(n))$
al posto di $f(n) \in O(g(n))$\newline
Per poter definire se una funzione appartiene a una notazione bisogna mostrarlo attraverso
una dimostrazione formale con l'utilizzo dell'induzione matematica.\newline

Esempio: la funzione $f(n) = 4n^2 + 4n - 1 \in O(n^2)$ va dimostrata attraverso induzione\newline
\textbf{Caso Base:} $n = 1$\newline
Per $n = 1$ si ha $4 + 4 -1 \leq c * 1$ che è vera per $0 \leq c \leq 7$\newline

\textbf{Ipotesi Induttiva}:
%FINIRE dimostrazione

\section{Proprietà Notazioni $O$}
La notazione $O$ possiede le seguenti proprietà:
\begin{enumerate}
  \item Riflessività: $\forall c\ cf(n) = O(f(n))$(lo stesso vale per $\Theta ed \Omega)$
  \item Transività:$f(n) = O(g(n)) \land g(n) = O(h(n)) \Leftarrow f(n) = O(h(n))$
  \item Simmetrià trasposta:$f(n) = O(g(n)) \Leftrightarrow g(n) = \Omega f(n)$
  \item Somma: $f(n) + g(n) = O(\max\{f(n),g(n)\})$
  \item Prodotto: $f(n) = O(h(n)) \land g(n) = O(q(n)) \Leftrightarrow f(n)g(n) = O(h(n)q(n))$
\end{enumerate}

%Da fare le dimostrazioni

\section{Proprietà Notazione $\Omega$}
La notazione $\Omega$ possiede le seguenti proprietà:
\begin{enumerate}
  \item Riflessività: $\forall c\ cf(n) = \Omega(f(n))$
  \item Transività: $f(n) = \Omega(g(n)) \land g(n)= \Omega(h(n)) \Leftarrow f(n) = \Omega(h(n))$
  \item Simmetrià trasposta: $f(n) = \Omega g(n) \Leftrightarrow g(n) = O(f(n))$
  \item Somma: $f(n) + g(n) = \Omega(\max\{f(n),g(n))\})$
  \item Prodotto: $f(n) = \Omega(h(n)) \land g(n) = \Omega(q(n)) \Leftrightarrow f(n)g(n) = \Omega(h(n)q(n))$
\end{enumerate}

%Da fare le dimostrazioni

\section{Proprietà Notazione $\Theta$}
La notazione $\Theta$ possiede le seguenti proprietà:
\begin{enumerate}
  \item Riflessività: $\forall c\ cf(n) = \Theta(f(n))$
  \item Transività: $f(n) = \Theta(g(n)) \land g(n) = \Theta(h(n)) \Leftarrow f(n)= \Theta(h(n))$
  \item Simmetria: $f(n) = \Theta(g(n)) \Leftrightarrow g(n) = \Theta(f(n))$
  \item Somma: $f(n) + g(n) = \Theta(\max\{f(n),g(n))\})$
  \item Prodotto: $f(n) = \Theta(h(n)) \land g(n) = \Theta(q(n)) \Leftrightarrow f(n)g(n) = \Theta(h(n)q(n))$
\end{enumerate}

$\Theta$ implementa una relazione di equivalenza sulle funzioni in quanto
sono rispettate la Riflessività,Transività e la Simmetria.

%Da fare le dimostrazioni
%Capitolo 3 Crescita delle funzioni
%Algoritmo QuickSort per risolvere il problema dell'ordinamento
\chapter{QuickSort}
Il $\proc{Quick-Sort}$ è un algoritmo divide et impera in loco ossia senza utilizzare
una struttura di appoggio per effettuare l'ordinamento e funziona in maniera ottimale
nell'implementazione sui calcolatori attuali.

L'algoritmo $\proc{Quick-Sort}$ esegue i seguenti passi divide et impera:
\begin{description}
  \item[Divide]:riarrangia l'array $A[p\twodots r]$ in due sottoarray, eventualmente nulli,
                $A[p \twodots q-1]$ e $A[q+1 \twodots r]$ tali che tutti gli elementi
                del primo sottoarray sono minori o uguali a $A[q]$ e tutti gli elementi
                del secondo sottoarray sono maggiori o uguali a $A[q]$.\newline
                Calcolare l'indice di $q$ viene effettuato nella procedura di riarrangiamento.
  \item[Impera]:ordina ricorsivamente i due sottoarray $A[p \twodots q-1]$ e $A[q+1 \twodots r]$.
  \item[Combina]:dato che i due sottoarray sono già ordinati per cui non viene eseguito nulla.
\end{description}

L'algoritmo $\proc{Quick-Sort}$ è il seguente
%Pseudocodice Quicksort
\begin{codebox}
\Procname{$\proc{Quick-Sort}(A,p,r)$}
\li \If $p < r$
    \Then
\li           $q = \proc{Partition}(A,p,r)$
\li           $\proc{Quick-Sort}(A,p,q-1)$
\li           $\proc{Quick-Sort}(A,q+1,r)$
    \End
\end{codebox}
Per effettuare l'ordinamento di un array $A$ viene effettuata la chiamata iniziale
$\proc{Quick-Sort}$(A,1,$\attrib{A}{length}$).\newline
La chiave dell'algoritmo è la procedura $\proc{Partition}$ implementato come:
%Pseudocodice Procedura Partition
\begin{codebox}
\Procname{$\proc{Partition}(A,p,r)$}
\li $\id{pivot} \gets A[r]$
\li $\id{indice} \gets p-1$
\li \For $j \gets p$ \To $r$
    \Do
\li               \If $A[j] \leq \id{pivot}$
                  \Then
\li                              $\id{indice} \gets id{indice} + 1$
\li                              scambia $A[id{indice}]$ con $A[j]$
                  \End
    \End
\li scambia $A[indice+1]$ con $A[r]$
\li \Return $\id{indice} + 1$
\end{codebox}

%Invariante di Ciclo da fare!!!!

Il tempo di esecuzione della procedura $\proc{Partition}$ è il seguente:
$T(n) = c_1 + c_2 + c_3(n+1) + c_4 + c_5(t_{if}) + c_6(t_{if}) + c_7 + c_8$.

\begin{description}
  \item[Caso migliore]:tutti gli elementi sono maggiori del pivot per cui $t_{if} = 0$
        $T(n) = c_3n + (c_1+c_2+c_3+c_4+c_7+c_8) = \Theta(n)$
  \item[Caso peggiore]:tutti gli elementi sono inferiori del pivot per cui $t_{if} = 1$
        $T(n) = c_3n + (c_1+c_2+c_3+c_4+c_5+c_6+c_7+c_8) = \Theta(n)$
\end{description}

\section{Analisi tempo QuickSort}
L'analisi del tempo di esecuzione del $\proc{Quick-Sort}$ è in base al fatto se
la partizione dell'array è bilanciata o meno.
\begin{description}
  \item[Caso peggiore]:la procedura di partizione produce due sottoproblemi:
        uno di $n-1$ elementi e l'altro di $0$ elementi.
        $T(n) = T(n-1) + T(0) + \Theta(n)
              = T(n-1) + \Theta(n)$
        Attraverso il metodo di sostituzione arrivo a $T(n) = O(n^2)$
  \item[Caso migliore]:la procedura di partizione produce due sottoproblemi di $\lceil n/2 \rceil$
        e $\lfloor n/2 \rfloor$ elementi per cui, ignorando le condizioni di ceil e floor,
        il tempo di esecuzione è $T(n) = 2T(n/2) + \Theta(n) = \Omega(n \log n)$ per il teorema dell'esperto.

\end{description}


%Algoritmo QuickSort randomizzato
%Capitolo 4 QuickSort
%Capitolo sull'ordinamento in tempo lineare
\chapter{Ordinamento in tempo lineare}
Gli algoritmi di ordinamento analizzati fino ad ora hanno un importante proprietà,
ossia effettuano l'ordinamento soltanto mediante il confronto tra gli elementi in input.
In questo capitolo verrà analizzato il limite minimo di confronti da effettuare e
verranno analizzati altri due algoritmi di Ordinamento.
%Capitolo 5 Ordinamento in Tempo lineare
%\documentclass[a4paper,11pt]{report}
\usepackage[T1]{fontenc}
\usepackage[utf8]{inputenc}
\usepackage[italian]{babel}
\usepackage{amsmath}%Package per la matematica
\usepackage{amsthm}%Package per gestione dei teoremi
\usepackage{amsfonts}%Package per i font matematici
\usepackage{booktabs}%Package per le tabelle
\usepackage{caption}%Package per la gestione delle Tabelle
\setlength{\parindent}{0pt}%Toglie il rientro dei capoversi
\newtheorem{thm}{Teorema}%Definisce i teoremi
\newcommand{\numberset}{\mathbb}
\newcommand{\N}{\numberset{N}}
\newcommand{\Z}{\numberset{Z}}
\newcommand{\Q}{\numberset{Q}}
\newcommand{\R}{\numberset{R}}

\begin{document}
\input{Induzione/esInduzione}%Esercizi su Induzione
%\input{Logica/tableauProposizionali}%Esercizi su Tableau Proposizionali
%\input{Logica/semanticaProposizionale}%Esercizi sulla semantica Proposizionale
\section{Traduzione linguaggio Naturale in linguaggio predicativo}

%Frase Tutti i docenti hanno un età maggiore di 24 anni
Esercizio: Tutti i docenti hanno un età maggiore di 24 anni

Costanti: $24$
Predicati:$Docente(x)$,$>(x,y)$
Funzioni:$eta(x)$

$\forall x (Docente(x) \rightarrow eta(x) > 24)$

Esercizi:Tutti i docenti hanno una chiave d'accesso all'edificio U6

Costanti:$U6$
Predicati:$Docente(x)$,$Avere(y)$,$Chiave(x,y)$
Funzioni:non presente

\begin{equation*}
    \forall x (Docente(x) \land \exists y (Chiave(y,U6) \land Avere(y)))
\end{equation*}

Esercizio:Tutti i canali televisivi con una share maggiore del 10\% sono
          considerati canali principali

Costanti:$10\%$
Predicati:$Canale(x)$,$>(x,y)$,$CanalePrincipale(x)$
Funzioni:$share(x)$
\begin{equation*}
    \forall x (Canale(x) \land share(x) > 10\% \rightarrow CanalePrincipale(x))
\end{equation*}

Esercizio:il fratello di Marco ha copiato il compito ed è stato respinto

Costanti:$Marco$,$compito$
Predicati:$Copiare(x,y)$,$Uomo(x)$,$Bocciato(x)$,$Fratello(x,y)$
Funzioni: non presenti
\begin{equation*}
    \exists x (Uomo(x) \land Fratello(x,Marco) \land Copiare(x,compito) \rightarrow Bocciato(x))
\end{equation*}

Esercizio: Tutte le sere gli studenti ascoltano musica Uzbeka e bevono caffè

Costanti:$musicaUzbeka$,$caffè$
Predicati:$Studenti(x)$,$Ascoltare(x,y)$,$Bere(x,y)$,$Sera(y)$
Funzioni:non presenti
\begin{equation*}
\forall x,y (Studente(x) \land Sera(y) \rightarrow (Ascoltare(x,musicaUzbeka) \land Bere(x,caffè)))
\end{equation*}

Esercizio:Gli studenti che non si iscrivono all'appello di Fondamenti non possono svolgere l'esame

Costanti:$Fondamenti$
Predicati:$Studente(x)$,$Iscrivere(x,y)$,$Svolgere(x,y)$,$Esame(y)$
Funzioni:
\begin{equation*}
\forall x (Studente(x) \land \neg Iscrivere(x,Fondamenti) \rightarrow
\exists y (Esame(y) \land \neg Svolgere(x,y)))
\end{equation*}

Esercizio:Tutti i professori fanno esami

Costanti: non presenti \newline
Predicati:$Professore(x)$,$Fare(x,y)$,$Esame(y)$ \newline
Funzioni: non presenti
\begin{equation*}
    \forall x (Professore(x) \rightarrow \exists y(Esame(y) \land Fare(x,y)))
\end{equation*}

Esercizio: Se uno studente non è iscritto via Sifa ad un appello non può fare l'esame

Costanti: non presenti \newline
Predicati:$Studente(x)$,$Iscritto(x,y)$,$Appello(y)$,$Esame(x)$ \newline
Funzioni: non presenti
\begin{equation*}
    \forall x (Studenti(x) \land \exists y(Appello(y) \land \neg Iscritto(x,y)) \rightarrow \neg Esame(x))
\end{equation*}

Esercizio: il voto di un esame universitario va da 0 a 30 e lode

Costanti:$0$ e $30L$ \newline
Predicati:$Esame(x)$,$>=(x,y)$,$<=(x,y)$ \newline
Funzioni:$voto(x)$
\begin{equation*}
    \forall x (Esame(x) \rightarrow voto(x) >= 0 \land voto(x) <= 30L)
\end{equation*}

Esercizio:Tutti i docenti sono sposati con una donna antipatica

Costanti: non presenti \newline
Predicati:$Docente(x)$,$Donna(y)$,$Sposati(x,y)$,$Antipatica(y)$ \newline
Funzioni: non presenti
\begin{equation*}
    \forall x (Docente(x) \rightarrow \exists y(Donna(y) \land Antipatica(y) \land Sposati(x,y)))
\end{equation*}

Esercizio:Marco ha un capo magnanimo

Costanti:$Marco$ \newline
Predicati:$Capo(x,y)$,$Magnanimo(x)$ \newline
Funzioni: non presenti \newline
\begin{equation*}
    Capo(x,Marco) \land Magnanimo(x)
\end{equation*}

Esercizio:L'everest è la montagna più alta al mondo

Costanti:$Everest$ \newline
Predicati:$Montagna(x)$,$<(x,y)$ \newline
Funzioni:$altezza(x)$
\begin{equation*}
    \forall x (Montagna(x) \rightarrow altezza(x) < altezza(Everest))
\end{equation*}

Esercizio:Se ogni amico di Mario è amico di Diego e Pietro non è amico di
          Mario, allora Pietro non è amico di Diego

Costanti:$Mario$,$Diego$,$Pietro$ \newline
Predicati:$Amico(x,y)$ \newline
Funzioni:non presenti
\begin{equation*}
    \forall x (((Amico(x,Mario) \rightarrow Amico(x,Diego)) \land \neg Amico(Pietro,Mario))
                \rightarrow \neg Amico(Pietro,Diego))
\end{equation*}

Esercizio:Se tutti i filosofi intelligenti sono curiosi, e solo i tedeschi sono
filosofi intelligenti, allora, se ci sono filosofi intelligenti, qualche
tedesco è curioso.

Esercizio:Se tutti gli studenti sono persone serie, tutti gli studenti sono
studiosi e tutte le persone serie e studiose non fanno tardi la
sera, allora se esiste qualcuno che fa tardi la sera, non tutti sono
studenti

Esercizio:Tutti i ragazzi e le ragazze iscritti all’università sanno usare il
          computer o conoscono qualcuno che lo sa usare
%Esercizi sulla traduzione da italiano a logica predicativa
\end{document}
%Esercizi Algoritmi
%Capitolo sulle Strutture Dati
\chapter{Strutture Dati Elementari}
In questo capitolo verranno definite le strutture dati elementari, ma prima di poterle
definire bisogna definire il concetto di tipo di dato.\newline
Il tipo di dato è un modello matematico in cui sono definite un certo numero di operazioni
e in ogni linguaggio di programmazione vengono definiti e previsti dei tipi di dato detti
\emph{primitivi}, come ad esempio i numeri interi, i caratteri ed ecc... ma può
essere comodo e conveniente definire altre tipologie di dati per rendere più facile e chiara
la definizione e l'implementazione di un algoritmo.

In generale le proprietà di un tipo di dato devono dipendere soltanto dalla sua specifica
ed essere indipendenti dalla modalità in cui vengono rappresentati per cui si dice
che un tipo di dato è \emph{astratto}, in quanto il dato è astratto rispetto alla sua rappresentazione.

Il vantaggio di avere i tipi di dato astratti consiste nel poter utilizzare il dato
senza conoscere la sua rappresentazione ed eventuali modifiche alla rappresentazione del dato
non comportano alcun cambiamento nell'utilizzo del dato da parte dell'utilizzatore.

%Cercare definizione di Strutture dati

%Strutture Dati Lista
\section{Liste}
Le liste vengono solitamente usate per implementare il concetto matematico di sequenza lineare di oggetti, in cui si possono eventualmente ripetere gli elementi
ma questo non significa che le liste sono le sequenze dato che quest'ultima rappresenta la specifica del tipo del dato anche se solitamente è naturale usare per 
implementare una sequenza la realizzazione di una lista.\newline
La lista è una struttura dati lineare dinamica in cui l'accesso all'elemento successivo della sequenza avviene tramite un puntatore all'elemento successivo
ed un elemento è composto da un valore , chiamato $\id{element}$ e da due puntatori $\id{prev}$ e $\id{next}$,
i quali puntano all'elemento precedente o successivo della lista.\newline
In caso $\id{prev} \gets \const{nil}$ l'elemento non ha nessun predecessore ed è la $\id{head}$ della lista mentre in caso il puntatore $\id{next} \gets \const{nil}$
l'elemento non ha nessun successore per cui è la coda della lista.
Vi sono diversi tipologie di caratteristiche che una lista può possedere, anche in maniera multipla ossia può possedere più di una proprietà, come si può notare dal seguente elenco:
\begin{itemize}
    \item \emph{singolarmente o doppiamente concatenata}: in caso una lista sia singolarmente
        concatenata si ha soltanto il collegamento con l'elemento successivo, per
        cui viene omesso il puntatore $\id{prev}$, mentre nella lista doppiamente concatenata
         si ha il collegamento, tramite i puntatori, con l'elemento precedente e l'elemento successivo.
     \item \emph{ordinata}: una lista si dice ordinata se è previsto un ordinamento tra i valori
        degli elementi presenti in una lista.
    \item \emph{circolare}: il puntatore $\id{prev}$ della testa della lista punta alla coda
        mentre il puntatore $\id{next}$ della coda della lista punta alla testa
       per cui si può dire che la lista è un anello di elementi.
\end{itemize}
La specifica di una sequenza, qualsiasi implementazione abbia, è la seguente:
\begin{minted}{c}
 void insert(List L, Item x)%inserisce un elemento 
 List* delete(List L)%elimina il primo elemento
 List* search(List L, Item key)%cerca l'elemento key
 Item min(List L)%calcola l'elemento minimo
 Item max(List L)%calcola l'elemento massimo 
 Item successor(List L, Item  key)%calcola l'elemento precedente come valore
 Item predecessor(List L, Item key)%calcola l'elemento successivo come valore 
\end{minted}
Forniamo ora un implementazione di una lista doppiamente concatenata non ordinata
in cui un elemento della lista è composto da un dato chiamato $\id{element}$ e da due
puntatori, chiamati $\id{prev}$ e $\id{next}$, che puntano all'elemento precedente e successivo.

La prima operazione implementata in una lista è la ricerca di un elemento che opera secondo l'algoritmo di ricerca Lineare in quanto non essendo ordinati i valori della lista
non è possibile implentare la ricerca tramite la ricerca binaria.\newline
Lo pseudocodice della ricerca di un elemento di una lista è il seguente:
%Pseudocodice della ricerca di una Lista
\begin{codebox}
\Procname{$\proc{List-Search}(L,\id{key})$}
\li $x\gets \attrib{L}{head}$
\li \While $x \neq \const{nil}$ and $\attrib{x}{\id{element}} \neq \id{key}$
    \Do
\li                      $x \gets \attrib{x}{\id{next}}$
    \End
\li \Return $x$
\end{codebox}
La ricerca di un elemento da una lista richiede il seguente tempo con i diversi casi:
\begin{equation*}
  T(n) = c + c(t_w + 1) + ct_w + c
\end{equation*}
\begin{description}
  \item[Caso migliore]:l'elemento da ricercare viene trovato al primo elemento della lista
        per cui $t_w = 0$ indi il tempo di esecuzione è $T(n) = c + c + c = \Omega(1)$
  \item[Caso peggiore]: l'elemento non è presente nella lista per cui $t_w = n$
        \begin{equation*}
          T(n) = c + cn + c + cn + c = 2cn + 3c = O(n)
        \end{equation*}
\end{description}

%Pseudocodice Inserimento in una lista
Il secondo metodo in una lista è $\proc{List-Insert}$ in cui si suppone che il valore
dell'elemento da inserire sia stato già impostato ossia $\id{element}$ abbia già il valore desiderato.
\begin{codebox}
\Procname{$\proc{List-Insert}(L, x)$}
\li $\attrib{x}{\id{next}} \gets \attrib{L}{\id{head}}$
\li \If $\attrib{L}{head} \neq \const{nil}$
    \Then
\li                       $\attrib{L}{\attrib{\id{head}}{\id{prev}}} \gets x$
    \End
\li $\attrib{L}{\id{head}} \gets x$
\li $\attrib{x}{\id{prev}} \gets \const{nil}$
\end{codebox}
Il tempo di esecuzione $T(n) = 5c = \Theta(1)$ e tra il caso migliore e peggiore non
vi è alcuna differenza se non il fatto che non viene eseguita soltanto la terza istruzione.

La rimozione di un elemento da una lista, implementata tramite $\proc{List-Delete}$, prevede di darne il puntatore all'elemento alla procedura per cui per effettuare
la rimozione di un elemento qualsiasi della lista bisogna effettuare la chiamata a $\proc{List-Search}$ prima per ottenere l'elemento da eliminare.\newline
Lo pseudocodice per la rimozione di un elemento è il seguente:
%Pseudocodice Rimozione da una Lista
\begin{codebox}
\Procname{$\proc{List-Delete}(L,x)$}
\li \If $\attrib{x}{prev} \neq \const{nil}$
    \Then
\li              $\attrib{x}{prev} \gets \attrib{x}{next}$
    \End
\li \Else        $\attrib{L}{head} \gets \attrib{x}{next}$

\li \If $\attrib{x}{next} \neq \const{nil}$
    \Then
\li              $\attribb{x}{\id{next}}{\id{prev}} \gets \attrib{x}{\id{prev}}$
    \End
\end{codebox}
Il tempo di esecuzione è $T(n) = 4c = \Theta(1)$ e tra il caso peggiore e migliore non cambia
nulla però va detto che se si volesse implementare la rimozione da un elemento qualsiasi
della lista si avrebbe un tempo di esecuzione $\Theta(n)$ in quanto si avrebbe la chiamata
alla procedura $\proc{List-Search}$ per stabilire l'elemento da rimuovere.

%Pseudocodice Massimo e minimo della lista
Per trovare il massimo e/o il minimo di una lista bisogna scansionare tutta la lista
e scegliere l'elemento con valore massimo/minimo come si farebbe anche se si dovesse
trovare il massimo di un array; lo pseudocodice del massimo e del minimo sono i seguenti:
\begin{codebox}
\Procname{$\proc{List-Max}(L)$}
\li \If $\attrib{L}{head} \isequal \const{nil}$
    \Then
\li                \Return $\const{nil}$
\li $\id{max} \gets \attrib{L}{head}$
\li $\id{elem} \gets \attribb{L}{head}{next}$
\li \While $\id{elem} \neq \const{nil}$
    \Do
\li                 \If $\attrib{\id{elem}}{\id{value}} > \attrib{max}{value}$
                    \Then
\li                              $\id{max} \gets \id{elem}$
                    \End
\li                 $\id{elem} \gets \attrib{\id{elem}}{\id{next}}$
    \End
\li \Return $\id{max}$
\end{codebox}
Senza effettuare un'analisi linea per linea si può facilmente notare che il calcolo
del massimo di una lista di $n$ elementi richieda $\Theta(n)$ in quanto bisogna
scansionare tuttti gli elementi per poterne determinare il massimo.

\begin{codebox}
\Procname{$\proc{List-Min}(L)$}
\li \If $\attrib{L}{head} \isequal \const{nil}$
    \Then
\li                \Return $\const{nil}$
\li $\id{min} \gets \attrib{L}{\id{head}}$
\li $\id{elem} \gets \attribb{L}{\id{head}}{\id{next}}$
\li \While $\id{elem} \neq \const{nil}$
    \Do
\li                 \If $\attrib{\id{elem}}{\id{value}} < \attrib{\id{min}}{\id{value}}$
                    \Then
\li                              $\id{min} \gets \id{elem}$
                    \End
\li                 $\id{elem} \gets \attrib{\id{elem}}{\id{next}}$
    \End
\li \Return $\id{min}$
\end{codebox}
Ovviamente, come anche già notato nel calcolo del massimo di una lista, il tempo di
esecuzione per il calcolo del minimo è $\Theta(n)$ in quanto bisogna sempre scansionare la lista.

%Pseudocodice Predecessore e successore della Lista
Le ultime due procedure da implementare sono quello di trovare l'elemento successore
e predecessore come valore di un elemento ossia viene definito come successore il più
piccolo elemento più grande dell'elemento dato mentre il predecessore il più grande
elemento più piccolo come valore dell'elemento dato;ovviamente, come anche già visto
per il minimo e il massimo, bisogna scandirre tutta la lista per stabilire il prede/successore
e lo pseudocodice è il seguente:
\begin{codebox}
    \Procname{$\proc{List-Successor}(L, \id{key})$}
\li \If $\attrib{L}{head} \isequal \const{nil}$
    \Then
\li                \Return $\const{nil}$
\li $\id{succ} \gets \attrib{L}{head}$
\li \While $\id{succ} \neq \const{nil}$ and $\id{succ}{value} \leq \id{key}$
    \Do
\li           $\id{succ} \gets \attrib{\id{succ}}{\id{next}}$
    \End
\li \If $\id{succ} \isequal \const{nil}$ \Return $\const{nil}$
\li $\id{elem} \gets \attrib{\id{succ}}{\id{next}}$
\li \While $\id{elem} \neq \const{nil}$
    \Do
\li      \If $\attrib{\id{elem}}{\id{value}} > \id{key}$ and $\attrib{\id{elem}}{\id{value}} < \attrib{succ}{\id{value}}$
         \Then
\li                              $\id{succ} \gets \id{elem}$
         \End
\li      $\id{elem} \gets \attrib{\id{elem}}{\id{next}}$
    \End
\li \Return $\id{succ}$
\end{codebox}
Il tempo di esecuzione è $\Theta(n)$ in quanto sia nel caso migliore che in quello
peggiore bisogna analizzare tutti gli elementi della lista.

\begin{codebox}
    \Procname{$\proc{List-Predeccessor}(L, \id{key})$}
\li \If $\attrib{L}{head} \isequal \const{nil}$
    \Then
\li                \Return $\const{nil}$
\li $\id{prev} \gets \attrib{L}{head}$
\li \While $\id{prev} \neq \const{nil}$ and $\id{prev}{value} \geq \id{key}$
    \Do
\li           $\id{prev} \gets \attrib{\id{prev}}{\id{next}}$
    \End
\li \If $\id{prev} \isequal \const{nil}$ \Return $\const{nil}$
\li $\id{elem} \gets \attrib{\id{prev}}{\id{next}}$
\li \While $\id{elem} \neq \const{nil}$
    \Do
\li      \If $\attrib{\id{elem}}{\id{value}} < \id{key}$ and $\attrib{\id{elem}}{\id{value}} > \attrib{prev}{\id{value}}$
         \Then
\li                              $\id{prev} \gets \id{elem}$
         \End
\li      $\id{elem} \gets \attrib{\id{elem}}{\id{next}}$
    \End
\li \Return $\id{prev}$
\end{codebox}
Come anche per il successore, il tempo di esecuzione per trovare il predecessore di
un elemento è $\Theta(n)$ in quanto bisogna analizzare tutti gli elementi di una lista.

Le altre implementazioni delle diverse tipologie di liste sono similari soltanto che implementano o meno l'ordinamento tra gli elementi,
considerano o meno il puntatore $\id{prev}$ ed altre considerazioni fatte in base alla tipologia di lista.

Un implementazione alternativa della sequenza, anche se meno intutiva e naturale di quella presentata fino ad ora,
è quella tramite la memorizzazione degli elementi in un vettore, in cui la posizione di un elemento corrisponde all'indice del vettore.\newline
Questa implementazione permette di passare in maniera costante da un elemento ad un altro, di accorgersi se si supera un estremo della sequenza,
di modificare o leggere il valore di un elemento anche tramite un accesso diretto tramite indice, ma sfortunatamente richiede di conoscere 
la dimensione massima della sequenza per evitare sprechi di memoria e il tempo di inserimento e cancellazione richiede la scansione della sequenza quindi a tempo $\Theta(n)$.

Si evita di mostrare lo pseudocodice di questa implementazione alternativa perchè viene usata raramente e perchè conoscendo il funzionamento della sequenza è banale
implementare le sue operazioni mediante un vettore di elementi.
%Paragrafo sulle liste

%Paragrafo sulle code
\section{Code}
La \emph{Queue}, in italiano \emph{coda}, è una struttura dati di tipo FIFO(First in First out)
per memorizzare una sequenza di elementi, in cui l'inserimento di un elemento
avviene in coda alla sequenza mentre la rimozione avviene in testa alla sequenza.

Una coda può essere implementata attraverso array oppure delle liste a seconda della
scelta implementatica e della capacità di stabilire un limite massimo di elementi utilizzati.

Essendo una coda una particolare tipologia di sequenza può essere implementata facilmente
utilizzando una lista e le sue operazioni però, a differenza dello stack, la scelta
della lista utilizzata per l'implementazione cambia il tempo di esecuzione delle operazioni,
infatti soltanto con una lista bidirezionale si ottiene un tempo $\Theta(1)$ in tutte le operazioni
per cui noi utilizziamo una lista bidirezionale per implementare una coda.

%Pseudocodice delle operazioni di una Coda
Il metodo $\proc{isEmpty}$ indica se la coda contiene o meno degli elementi
\begin{codebox}
\Procname{$\proc{isEmpty}()$}
\li \Return 
\end{codebox}
Il metodo $\proc{enqueue}(Q,x)$ inserisce un elemento nella coda e si assume che $x$ sia un elemento
definito prima della chiamata al metodo.
\begin{codebox}
\Procname{$\proc{Enqueue}(Q,x)$}
\li $\attribb{Q}{tail}{next} \gets x$
\li $\attrib{Q}{tail} \gets x$
\end{codebox}
L'esecuzione del metodo $\proc{enqueue}$ richiede $2c$ ossia $\Theta(1)$ per ciò
l'inserimento in una coda richiede un tempo costante.

Il metodo $\proc{dequeue()}$ estra il primo elemento della coda ed è implementato come
\begin{codebox}
\Procname{$\proc{Dequeue()}$}
\li \If $\proc{isEmpty()}$
    \Then
\li           \Return $\const{nil}$
\li $\id{temp} \gets \attrib{Q}{head}{element}$
\li $\attrib{Q}{head} \gets \attribb{Q}{head}{next}$
\li \Return temp
\end{codebox}
L'operazione $\proc{dequeue}$ richede un tempo costante $\Theta(1)$ per effettuare la rimozione.

%Coda tramite Vettore(guardare il Cormen)
%Paragrafo sulle code

%Paragrafo sulle stack
\section{Stack}
Lo \emph{stack}, è una struttura dati di tipo \emph{LIFO}(Last in first out), utilizzata
in tutti i linguaggi di programmazione per effettuare la memorizzazione di tutti i dati
di tipo statico e dei record di attivazione, che può essere vista come un caso particolare
di sequenza in cui l'inserimento avviene alla fine della sequenza e la rimozione avviene sempre in fondo.

Uno stack può essere implementato attraverso array oppure delle liste a seconda della
scelta implementatica e della capacità di stabilire un limite massimo di elementi utilizzati.

Essendo lo stack un particolare tipo di sequenza, essa può essere simulata tramite le operazioni di una lista,
in particolare quella singolarmente concatenata, però è prassi comune utilizzare
nomi diversi per indicarne le operazioni per migliore chiarezza.

%specifica operazioni su uno stack
\textbf{void} $\proc{push}(S,x)$ inserisce un elemento in testa allo stack \newline
\textbf{Item*} $\proc{pop}(S)$ rimuove l'elemento in testa allo stack \newline
\textbf{boolean} $\proc{stackEmpty}$ indica se lo stack è vuoto o meno.

%pseudocodice delle operazioni dello stack
Lo pseudocodice delle operazioni di uno stack implementate tramite liste sono:
\begin{codebox}
\Procname{$\proc{push}(S,x)$}
\li $\attrib{x}{next} \gets \attrib{S}{top}$
\li $\attrib{S}{top} \gets x$
\end{codebox}

La procedura $\proc{pop}$ rimuove l'ultimo elemento inserito nello stack e lo ritorna
come valore; in caso lo stack è vuoto genera underflow come errore.
\begin{codebox}
\Procname{$\proc{pop}(S)$}
\li $\id{temp} \gets \attrib{S}{top}$
\li \If $\proc{stackEmpty}(S)$
    \Then
\li        \Error "Underflow"
    \End
\li $\attrib{S}{top} \gets \attribb{S}{top}{next}$
\li \Return $\id{temp}$
\end{codebox}

La procedura $\proc{stackEmpty}$ indica se lo stack è vuoto e viene utilizzato per
assicurarsi di non provare ad accedere allo stack vuoto per la rimozione
\begin{codebox}
\Procname{$\proc{stackEmpty}(S)$}
\li \Return $ \attrib{S}{top} \isequal \const{nil}$
\end{codebox}

Utilizzando le liste per implementare lo stack otteniamo in tutte le operazioni l'
impiego di tempo costante $\Theta(1)$.

L'implementazione dello stack tramite un vettore ha lo stesso impiego di tempo $\Theta(1)$
in tutte le operazioni però per evitare uno spreco di memoria bisogna sapere il numero
di elementi necessari e soprattutto non è possibile superare il numero di elementi massimo
stabilito alla creazione dello stack.

%Pseudocodice delle operazioni dello Stack tramite vettore
La realizzazione delle operazioni dello stack tramite un vettore sono le seguenti:
\begin{codebox}
\Procname{$\proc{push}(S,x)$}
\li $\attrib{S}{top} \gets \attrib{S}{top} + 1$
\li $S[\attrib{S}{top}] \gets x$
\end{codebox}

\begin{codebox}
\Procname{$\proc{pop}(S)$}
\li \If $\proc{stackEmpty}(S)$
\li \Then \Error "Underflow"
\li \Else $\attrib{S}{top} \gets \attrib{S}{top}-1$
\li \Return $S[\attrib{S}{top}+1]$
\end{codebox}

\begin{codebox}
\Procname{$\proc{stackEmpty}(S)$}
\li \Return $\attrib{S}{top} \leq 0$
\end{codebox}

Un famoso esempio di utilizzo dello stack è quello di valutare delle espressioni
algebriche scritte in input tramite stringhe come cercheremo di fare ora:
%Paragrafo sulle pile/Stack
%Capitolo 5 Strutture Dati elementari
%Capitolo sul Heap
\chapter{Heap}
L'$\proc{heapSort}$ è un algoritmo di ordinamento che ordina sul posto, ossia solo un numero costante di elementi della sequenza
sono salvati fuori dalla sequenza di input ed inoltre utilizza la struttura dati \emph{heap} per gestire le informazioni.\newline
Lo \emph{heap} è una struttura dati rappresentata da un array $A$ che può essere vista come un albero binario
in cui ogni nodo dell'albero è un elemento dell'array, chiamato \emph{chiave}.\newline
L'array $A$ ha 2 attributi: $\attrib{A}{length}$ per rappresentare la lunghezza dell'array
e $\attrib{A}{heap-size}$ che indica il numero degli elementi dell'heap memorizzati in cui $0 \leq \attrib{A}{heap-size} \leq \attrib{A}{length}$.\newline 
Ci sono due tipologie di Heap:\emph{max-heap}, utilizzato nel heapSort e \emph{min-heap}, utilizzato principalmente per implementare code prioritarie;
queste due tipologie verranno analizzate entrambe in seguito in questo paragrafo.

%Fare disegno albero e array del heap

%Fare pseudocodice immediato di left,right e parent
Per poter effettuare l'accesso ai nodi left,right e parent si utilizzano le seguenti 3 procedure
\begin{codebox}
    \Procname{$\proc{left(i)}$}
    \li \Return 2i 
\end{codebox}
\begin{codebox}
    \Procname{$\proc{right(i)}$}
    \li \Return 2i + 1
\end{codebox}
\begin{codebox}
    \Procname{$\proc{parent(i)}$}
    \li \Return \cfloor \frac{i}{2} \rfloor 
\end{codebox}

Nei moderni computer le procedure $\proc{left}, \proc{right}$ e $\proc{parent}$ effettuano il loro lavoro mediante una solo istruzione, attraverso uno shift di un solo bit
verso sinistra in caso della procedura left e right mentre in caso della parent viene effettuato un shift di un solo bit verso destra.\newline
Una buona implementazione del heapsort consiste nell'implementare queste procedure come delle macro.

In un $\proc{max-heap}$ è soddisfatta per ogni nodo $i$ la seguente proprietà:
$A[\proc{Parent}(i)] \geq A[i]$ per cui il massimo valore della array si trova nella radice dell'heap mentre in un $\proc{min-heap}$ avviene il contrario 
ossia in ogni nodo si ha $A[\proc{Parent}(i)] \leq A[i]$ e la radice rappresenta l'elemento minimo dell'array.\newline
Essendo lo heap definito tramite un albero binario la sua altezza è $\Theta(\log n)$, definita come il più lungo percorso per raggiungere i nodi foglia a partire dalla radice.

\begin{figure}
    \caption{Max-Heapify implementation}
    \label{alg:maxHeapify}
    \begin{codebox}
        \li $l \gets \proc{Left(i)}$
        \li $r \gets \proc{Right(i)}$
        \li \If $l \leq \attrib{A}{heap-size}$ \And $A[l] > A[i]$
        \li             $\id{largest} \gets l$
        \li \Else $\id{largest} \gets i$
        \li \If $r \leq \attrib{A}{heap-size}$ \And $A[r] > A[\id{largest}]$
        \li             $\id{largest} \gets r$
            \EndIf
        \li \If $\id{largest} \neq i$
        \li      $\proc{Swap(A[i], A[\id{largest}])}$
        \li      $\proc{Max-Heapify(A, largest)}$
    \end{codebox}
\end{figure}
La procedura $\proc{maxHeapify}$ \ref{alg:maxHeapify} controlla se l'elemento $i$-esimo rispetta la proprietà del max-heap ossia che il figlio sinistro e destro di $i$ sia 
inferiore al valore dell'elemento $i$-esimo e in caso non sia rispettato viene sostituito il più grande tra $A[\proc{Left(i)}], A[\proc{Right(i)}], A[i]$ e lo sostituisce 
al posto di $A[i]$ poi si applica l'algoritmo Max-Heapify al sottoalbero dell'elemento maggiore, che non è detto che rispetta ancora
la proprietà di max heap dopo averlo sostituito con $A[i]$.

Il tempo di esecuzione di $\proc{Max-Heapify}$ prevede di avere un tempo $\Theta(1)$ per trovare il valore maggiore e scambiarlo con $A[i]$ mentre il tempo per 
risolvere l'algoritmo al sottoalbero prevede di avere al massimo $\frac{2n}{3}$  nodi per cui l'equazione di ricorrenza è la seguente
\[ T(n) \leq T(\frac{2n}{3}) + \Theta(1) \]
che per il teorema dell'esperto è $T(n) = O(\log n)$ o in maniera generale dato un heap di altezza $h$ il tempo di esecuzione di $\proc{Max-Heapify}$ è di $O(h)$.




%Strutture dati aggiornate
%suffix-tree
%FM-index
%suffix-array
%wavelet-tree
%Bloom filters
%Sequence Bloom trees
%Tabelle Hash
 Capitolo 6 Heap
%Paragrafo sugli Alberi
\section{Alberi}
Si definisce come \emph{Albero libero}, un DAG connesso con un solo nodo sorgente, detto \emph{radice},
in cui ogni nodo diverso dalla radice ha un solo nodo entrante.\newline
I nodi privi di archi entranti sono detti \emph{foglie} dell'albero.

%Inserire esempi e proprietà degli Alberi
%%Paragrafo sugli Alberi
\section{Alberi}
Si definisce come \emph{Albero libero}, un DAG connesso con un solo nodo sorgente, detto \emph{radice},
in cui ogni nodo diverso dalla radice ha un solo nodo entrante.\newline
I nodi privi di archi entranti sono detti \emph{foglie} dell'albero.

%Inserire esempi e proprietà degli Alberi
%%Paragrafo sugli Alberi
\section{Alberi}
Si definisce come \emph{Albero libero}, un DAG connesso con un solo nodo sorgente, detto \emph{radice},
in cui ogni nodo diverso dalla radice ha un solo nodo entrante.\newline
I nodi privi di archi entranti sono detti \emph{foglie} dell'albero.

%Inserire esempi e proprietà degli Alberi
%\input{Esempi/alberi}Esempi Alberi!!!!

L'albero è una struttura matematica importantissima in informatica utilizzata per
rappresentare una serie di situazioni, come ad esempio organizzazioni gerarchiche di dati,
procedimenti enumerativi o decisionali, e ve ne esistono un'infinita di implementazioni di alberi
però iniziamo ad analizzare per prima gli alberi liberi.

%Inserire immagine albero binario
Esempi Alberi!!!!

L'albero è una struttura matematica importantissima in informatica utilizzata per
rappresentare una serie di situazioni, come ad esempio organizzazioni gerarchiche di dati,
procedimenti enumerativi o decisionali, e ve ne esistono un'infinita di implementazioni di alberi
però iniziamo ad analizzare per prima gli alberi liberi.

%Inserire immagine albero binario
Esempi Alberi!!!!

L'albero è una struttura matematica importantissima in informatica utilizzata per
rappresentare una serie di situazioni, come ad esempio organizzazioni gerarchiche di dati,
procedimenti enumerativi o decisionali, e ve ne esistono un'infinita di implementazioni di alberi
però iniziamo ad analizzare per prima gli alberi liberi.

%Inserire immagine albero binario
%Capitolo 7 Alberi
%Capitolo sugli Alberi Binari di Ricerca
%Capitolo 8 Alberi di Ricerca
%\input{Capitolo/dynamicprogramming}%Capitolo 7 Programmazione Dinamica
%\input{Capitolo/programmazionegreedy} %Capitolo 8 Programmazione Greedy
%Terzo Capitolo Strutture Dati Elementari


\end{document}
