%Capitolo sul Heap
\chapter{Heap}
Lo \emph{heap} è una struttura dati rappresentata da un array $A$ che può essere vista come un albero binario
in cui ogni nodo dell'albero è un elemento dell'array, chiamato \emph{chiave}.\newline
L'array $A$ ha 2 attributi: $\attrib{A}{length}$ per rappresentare la lunghezza dell'array
e $\attrib{A}{heap-size}$ che indica il numero degli elementi dell'heap memorizzati
in cui $0 \leq \attrib{A}{heap-size} \leq \attrib{A}{length}$.
Ci sono due tipologie di Heap:\emph{max-heap}, utilizzato nel heapSort e \emph{min-heap},
utilizzato principalmente per implementare code prioritarie;queste due tipologie verranno analizzate entrambe
in seguito in questo paragrafo.

%Fare disegno albero e array del heap

%Fare pseudocodice immediato di left,right e parent
Per poter effettuare l'accesso ai nodi left,right e parent si utilizzano le seguenti 3 procedure

In un $\emph{max-heap}$ è soddisfatta per ogni nodo $i$ la seguente proprietà:
$A[\proc{Parent}(i)] \geq A[i]$ per cui il massimo valore della array si trova nella radice
dell'heap mentre in un $\emph{min-heap}$ avviene il contrario ossia in ogni nodo si ha
$A[\proc{Parent}(i)] \leq A[i]$ e la radice rappresenta l'elemento minimo dell'array.
Essendo lo heap definito tramite un albero binario la sua altezza è $\Theta(\log n)$.




%Strutture dati aggiornate
%suffix-tree
%FM-index
%suffix-array
%wavelet-tree
%Bloom filters
%Sequence Bloom trees
%Tabelle Hash
