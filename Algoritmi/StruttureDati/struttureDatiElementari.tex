%Capitolo sulle Strutture Dati
\chapter{Strutture Dati Elementari}
In questo capitolo verranno definite le strutture dati elementari, ma prima di poterle
definire bisogna definire il concetto di tipo di dato.\newline
Il tipo di dato è un modello matematico in cui sono definite un certo numero di operazioni
e in ogni linguaggio di programmazione vengono definiti e previsti dei tipi di dato detti
\emph{primitivi}, come ad esempio i numeri interi, i caratteri ed ecc... ma può
essere comodo e conveniente definire altre tipologie di dati per rendere più facile e chiara
la definizione e l'implementazione di un algoritmo.

In generale le proprietà di un tipo di dato devono dipendere soltanto dalla sua specifica
ed essere indipendenti dalla modalità in cui vengono rappresentati per cui si dice
che un tipo di dato è \emph{astratto}, in quanto il dato è astratto rispetto alla sua rappresentazione.

Il vantaggio di avere i tipi di dato astratti consiste nel poter utilizzare il dato
senza conoscere la sua rappresentazione ed eventuali modifiche alla rappresentazione del dato
non comportano alcun cambiamento nell'utilizzo del dato da parte dell'utilizzatore.

%Cercare definizione di Strutture dati

%Strutture Dati Lista
\section{Liste}
Le liste sono una struttura dati elementare che implementa il concetto matematico
di sequenza lineare di oggetti, in cui si possono eventualmente ripetere gli elementi
all'interno della sequenza.\newline
La lista è una struttura dati lineare dinamica in cui l'accesso all'elemento successivo
della sequenza avviene tramite un puntatore all'elemento successivo ed un elemento
è composto da un valore , chiamato $\id{element}$ e da due puntatori $\id{prev}$ e $\id{next}$,
i quali puntano all'elemento precedente o successivo della lista.\newline
In caso $\id{prev} \gets \const{nil}$ l'elemento non ha nessun predecessore ed è la $\id{head}$
della lista mentre il puntatore $\id{next} \gets \const{nil}$ l'elemento non ha
nessun successore per cui è la coda della lista.
Vi sono diversi tipologie di caratteristiche che una lista può possedere, anche in
maniera multipla ossia può possedere più di una proprietà, come si può notare dal seguente elenco:
\begin{itemize}
  \item singolarmente o doppiamente concatenata: in caso una lista sia singolarmente
        concatenata si ha soltanto il collegamento con l'elemento successivo, per
        cui viene omesso il puntatore $\id{prev}$, mentre nella lista doppiamente concatenata
         si ha il collegamento, tramite i puntatori, con l'elemento precedente e l'elemento successivo.
  \item ordinata: una lista si dice ordinata se è previsto un ordinamento tra i valori
        degli elementi presenti in una lista.
  \item circolare: il puntatore $\id{prev}$ della testa della lista punta alla coda
        mentre il puntatore $\id{next}$ della coda della lista punta alla testa
       per cui si può dire che la lista è un anello di elementi.
\end{itemize}

Forniamo ora un implementazione di una lista doppiamente concatenata non ordinata
in cui un elemento della lista è composto da un dato chiamato $\id{element}$ e da due
puntatori, chiamati $\id{prev}$ e $\id{next}$, che puntano all'elemento precedente e successivo.

La prima operazione implementata in una lista è la ricerca di un elemento che opera
secondo l'algoritmo di ricerca Lineare in quanto non essendo ordinati i valori della lista
non è possibile implentare la ricerca tramite la ricerca binaria.
Lo pseudocodice della ricerca di un elemento di una lista è il seguente:
%Pseudocodice della ricerca di una Lista
\begin{codebox}
\Procname{$\proc{List-Search}(L,\id{key})$}
\li $x\gets \attrib{L}{head}$
\li \While $x \neq \const{nil}$ and $\attrib{x}{\id{element}} \neq \id{key}$
    \Do
\li                      $x \gets \attrib{x}{\id{next}}$
    \End
\li \Return $\id{x}$
\end{codebox}
La ricerca di un elemento da una lista richiede il seguente tempo con i diversi casi:
\begin{equation*}
  T(n) = c + c(t_w + 1) + ct_w + c
\end{equation*}
\begin{description}
  \item[Caso migliore]:l'elemento da ricercare viene trovato al primo elemento della lista
        per cui $t_w = 0$ indi il tempo di esecuzione è $T(n) = c + c + c = \Omega(1)$
  \item[Caso peggiore]: l'elemento non è presente nella lista per cui $t_w = n$
        \begin{equation*}
          T(n) = c + cn + c + cn + c = 2cn + 3c = O(n)
        \end{equation*}
\end{description}

%Pseudocodice Inserimento in una lista
Il secondo metodo in una lista è $\proc{List-Insert}$ in cui si suppone che il valore
dell'elemento da inserire sia stato già impostato ossia $\id{element}$ abbia il valore desiderato.
\begin{codebox}
\Procname{$\proc{List-Insert}(L,x)$}
\li $\attrib{x}{\id{next}} \gets \attrib{L}{\id{head}}$
\li \If $\attrib{L}{head} \neq \const{nil}$
    \Then
\li                       $\attrib{L}{\attrib{\id{head}}{\id{prev}}} \gets x$
    \End
\li $\attrib{L}{\id{head}} \gets x$
\li $\attrib{x}{\id{prev}} \gets \const{nil}$
\end{codebox}
Il tempo di esecuzione $T(n) = 5c = \Theta(1)$ e tra il caso migliore e peggiore non
vi è alcuna differenza se non il fatto che non viene eseguita soltanto la terza istruzione.

La rimozione di un elemento da una lista, implementata tramite $\proc{List-Delete}$,
prevede di darne il puntatore all'elemento alla procedura per cui per effettuare
la rimozione di un elemento qualsiasi della lista bisogna effettuare la chiamata
a $\proc{List-Search}$ prima per ottenere l'elemento da eliminare.
Lo pseudocodice per la rimozione di un elemento è il seguente:
%Pseudocodice Rimozione da una Lista
\begin{codebox}
\Procname{$\proc{List-Delete}(L,x)$}
\li \If $\attrib{x}{prev} \neq \const{nil}$
    \Then
\li              $\attrib{x}{prev} \gets \attrib{x}{next}$
    \End
\li \Else        $\attrib{L}{head} \gets \attrib{x}{next}$

\li \If $\attrib{x}{next} \neq \const{nil}$
    \Then
\li              $\attribb{x}{\id{next}}{\id{prev}} \gets \attrib{x}{\id{prev}}$
    \End
\end{codebox}
Il tempo di esecuzione è $T(n) = 4c = \Theta(1)$ e tra il caso peggiore e migliore non cambia
nulla però va detto che se si volesse implementare la rimozione da un elemento qualsiasi
della lista si avrebbe un tempo di esecuzione $\Theta(n)$ in quanto si avrebbe la chiamata
alla procedura $\proc{List-Search}$ per stabilire l'elemento da rimuovere.

Le altre implementazioni delle diverse tipologie di liste sono similari soltanto
che implementano o meno l'ordinamento tra gli elementi, considerano o meno il puntatore prev
ed altri considerazioni fatte in base alla tipologia di lista.

Un implementazione alternativa della sequenza, anche se meno intutiva e naturale
di quella presentata fino ad ora, è quella tramite la memorizzazione degli elementi
in un vettore, in cui la posizione di un elemento corrisponde all'indice del vettore.\newline
Questa implementazione permette di passare in maniera costante da un elemento ad un altro,
di accorgersi se si supera un estremo della sequenza, di modificare o leggere il valore
di un elemento anche tramite un accesso diretto tramite indice, ma sfortunatamente
richiede di conoscere la dimensione massima della sequenza per evitare sprechi di memoria
e il tempo di inserimento e cancellazione richiede la scansione della sequenza per cui a tempo $\Theta(n)$.
%Paragrafo sulle liste

%Paragrafo sulle code
\section{Code}
La \emph{Queue}, in italiano \emph{coda}, è una struttura dati di tipo FIFO(First in First out) per memorizzare una sequenza di elementi,
in cui l'inserimento di un elemento avviene in coda alla sequenza mentre la rimozione avviene in testa alla sequenza.\newline
Una coda può essere implementata attraverso array oppure delle liste a seconda della scelta implementativa e della capacità di stabilire un limite massimo di elementi utilizzati.

Essendo una coda una particolare tipologia di sequenza può essere implementata facilmente utilizzando una lista e le sue operazioni però,
a differenza dello stack, la scelta della lista utilizzata per l'implementazione cambia il tempo di esecuzione delle operazioni,
infatti soltanto con una lista bidirezionale si ottiene un tempo $\Theta(1)$ in tutte le operazioni per cui noi utilizziamo questa tipologia di lista per implementare una coda.

Le code prevedono la definizione di 3 operazioni:$\proc{isEmpty}$ \ref{alg:isEmpty} indica se la coda contiene o meno degli elementi, $\proc{dequeue()}$ \ref{alg:dequeue}
estrae il primo elemento della coda ed infine si ha $\proc{enqueue(Q, x)}$ \ref{alg:enqueue} che inserisce un elemento nella coda e si assume che $x$ sia un elemento già definito.

%Pseudocodice delle operazioni di una Coda
\begin{codebox}
\Procname{$\proc{isEmpty}()$}
\li \Return ($\attrib{Q}{head} \isequal \const{nil}$)
\end{codebox}
\begin{codebox}
\Procname{$\proc{Enqueue}(Q,x)$}
\li $\attribb{Q}{tail}{next} \gets x$
\li $\attrib{Q}{tail} \gets x$
\end{codebox}
L'esecuzione del metodo $\proc{enqueue}$ richiede $2c$ ossia $\Theta(1)$ per ciò
l'inserimento in una coda richiede un tempo costante.

Il metodo $\proc{dequeue()}$ estra il primo elemento della coda ed è implementato come
\begin{codebox}
\Procname{$\proc{Dequeue}()$}
\li \If $\proc{isEmpty}()$
    \Then
\li           \Return $\const{nil}$
\li $\id{temp} \gets \attrib{Q}{head}{element}$
\li $\attrib{Q}{head} \gets \attribb{Q}{head}{next}$
\li \Return temp
\end{codebox}
L'operazione $\proc{dequeue}$ richede un tempo costante $\Theta(1)$ per effettuare la rimozione.

Tutte le operazioni presentate impiegano tempo costante $\Theta(1)$, cosa che le rende molto efficiente per rappresentare dati cui si vuole una politica FIFO.
%Coda tramite Vettore
Dopo aver visto come viene implementata una coda tramite puntatori, consideriamo l'implementazione tramite un vettore $Q[1 \twodots n]$ 
e ai campi $\attrib{Q}{head}$ e $\attrib{Q}{tail}$ per accedere all'elemento in testa e in coda ai vettore della coda;lo pseudocodice della coda tramite un vettore è il seguente:
\begin{codebox}
\Procname{$\proc{enqueue}(Q,x)$}
\li $Q[\attrib{Q}{tail}] \gets x$
\li \If $\attrib{Q}{tail} \isequal \attrib{Q}{length}$
\li    \Then $\attrib{Q}{tail} \gets 1$
\li \Else $\attrib{Q}{tail} \gets \attrib{Q}{tail} + 1$
\end{codebox}

\begin{codebox}
\Procname{$\proc{dequeue}(Q)$}
\li $x \gets Q[\attrib{Q}{head}]$
\li \If $\attrib{Q}{head} \isequal \attrib{Q}{length}$
\li \Then $\attrib{Q}{head} \gets 1$
\li \Else $\attrib{Q}{head} \gets \attrib{Q}{head} + 1$
\li \Return x
\end{codebox}

\begin{codebox}
\Procname{$\proc{queue-Empty}(Q)$}
\li \Return ($\attrib{Q}{head} \leq 0$)
\end{codebox}

Il tempo di esecuzione delle seguente procedure è sempre costante $\Theta(1)$
come anche la coda tramite sequenza ma ha senso implementare tramite vettore se e soltanto se
si sa determinare il numero degli elementi per evitare uno spreco di memoria.
%Paragrafo sulle code

%Paragrafo sulle stack
\section{Stack}
Lo \emph{stack}, è una struttura dati di tipo \emph{LIFO}(Last in first out), utilizzata in tutti i linguaggi di programmazione per effettuare 
la memorizzazione di tutti i dati di tipo statico e dei record di attivazione, che può essere vista come un caso particolare di sequenza 
in cui l'inserimento avviene alla fine della sequenza e la rimozione avviene sempre in fondo.\newline
Uno stack può essere implementato attraverso array oppure delle liste a seconda della scelta implementatica e della capacità di stabilire un limite massimo di elementi utilizzati.

Essendo lo stack un particolare tipo di sequenza, essa può essere simulata tramite le operazioni di una lista, in particolare quella singolarmente concatenata,
anche se  è prassi comune utilizzare nomi diversi per indicarne le operazioni per migliore chiarezza.

%specifica operazioni su uno stack
\begin{minted}{c}
 void push(S,x) %inserisce un elemento in testa allo stack 
 Item* pop(S) %rimuove l'elemento in testa allo stack  
 Boolean stackEmpty() %indica se lo stack contiene elementi.
\end{minted}
%pseudocodice delle operazioni dello stack
Lo pseudocodice delle operazioni di uno stack implementate tramite liste sono:
\begin{codebox}
\Procname{$\proc{push}(S,x)$}
\li $\attrib{x}{next} \gets \attrib{S}{top}$
\li $\attrib{S}{top} \gets x$
\end{codebox}

La procedura $\proc{pop}$ rimuove l'ultimo elemento inserito nello stack e lo ritorna
come valore; in caso lo stack è vuoto genera underflow come errore.
\begin{codebox}
\Procname{$\proc{pop}(S)$}
\li $\id{temp} \gets \attrib{S}{top}$
\li \If $\proc{stackEmpty}(S)$
    \Then
\li        \Error "Underflow"
    \End
\li $\attrib{S}{top} \gets \attribb{S}{top}{next}$
\li \Return $\id{temp}$
\end{codebox}

La procedura $\proc{stackEmpty}$ indica se lo stack è vuoto e viene utilizzato per
assicurarsi di non provare ad accedere allo stack vuoto per la rimozione
\begin{codebox}
\Procname{$\proc{stackEmpty}(S)$}
\li \Return $ \attrib{S}{top} \isequal \const{nil}$
\end{codebox}

Utilizzando le liste per implementare lo stack otteniamo in tutte le operazioni l'
impiego di tempo costante $\Theta(1)$.

L'implementazione dello stack tramite un vettore ha lo stesso impiego di tempo $\Theta(1)$
in tutte le operazioni però per evitare uno spreco di memoria bisogna sapere il numero
di elementi necessari e soprattutto non è possibile superare il numero di elementi massimo
stabilito alla creazione dello stack.

%Pseudocodice delle operazioni dello Stack tramite vettore
La realizzazione delle operazioni dello stack tramite un vettore sono le seguenti:
\begin{codebox}
\Procname{$\proc{push}(S,x)$}
\li $\attrib{S}{top} \gets \attrib{S}{top} + 1$
\li $S[\attrib{S}{top}] \gets x$
\end{codebox}

\begin{codebox}
\Procname{$\proc{pop}(S)$}
\li \If $\proc{stackEmpty}(S)$
\li \Then \Error "Underflow"
\li \Else $\attrib{S}{top} \gets \attrib{S}{top}-1$
\li \Return $S[\attrib{S}{top}+1]$
\end{codebox}

\begin{codebox}
\Procname{$\proc{stackEmpty}(S)$}
\li \Return $\attrib{S}{top} \leq 0$
\end{codebox}

%Paragrafo sulle pile/Stack
