%Capitolo sugli Alberi Binari di Ricerca
\chapter{Alberi Binari di Ricerca}
Dato un insieme di nodi in cui è definita una relazione d'ordine, si definisce come
\emph{albero di ricerca} un albero in cui tutti i nodi della radice sinistra sono
minori della radice e tutti i nodi a destra della radice sono maggiori e ogni sottoalbero
è anch'esso un albero di ricerca.

%FARE DISEGNO ALBERO BINARIO DI RICERCA

I metodi di visita di un albero visti nel capitolo precedente sono valide anche
per gli alberi binari di ricerca in particolare la visita $\proc{Inorder}$ consente
di mostrare i valori dell'albero ordinati in maniera corretta.

Le operazioni definite di solito in un albero binario di ricerca sono le seguenti:
\textbf{Item} search(Tree T,Item k);\newline
\textbf{void} insert(Tree T,Item z);\newline
\textbf{void} delete(Tree T,Item z);\newline
\textbf{Item} min(Tree T);\newline
\textbf{Item} maxinum(Tree T);\newline
\textbf{Item} predecessor(Tree T,Item x);\newline
\textbf{Item} successor(Tree T,Item x);

Tutte le seguenti operazioni richiedono un tempo $O(h)$ con $h$ indicante l'altezza
dell'albero per cui è meglio riuscire a mantenere il più possibile l'albero bilanciato
per ottenere l'altezza $h = \log n$.

L'operazione $\proc{search}$ permette di ricercare un elemento all'interno di un albero
binario di ricerca;per la proprietà di avere un ordinamento tra gli elementi dell'albero
la ricerca di un elemento ricalca quella della ricerca binaria per cui il pseudocodice è:
\begin{codebox}
\Procname{$\proc{BST-search}(Tree T,Item k)$}
\li $x \gets \attrib{T}{root}$
\li \If $x \isequal \const{nil}$ or $\attrib{x}{key} \isequal \attrib{k}{key}$
    \Then
\li                    \Return x
\li \If $\attrib{x}{key} < \attrib{k}{key}$
    \Then
\li                     \Return $\proc{Tree-search}(\attrib{T}{right},k)$
\li \Else \Return $\proc{Tree-search}(\attrib{T}{left},k)$
\end{codebox}

La ricerca di un elemento richiede la scansione dell'albero dalla radice alle foglie
che sappiamo essere pari a $O(h)$ con $h$ indicante l'altezza dell'albero.
La procedura $\proc{Tree-min}$ calcola il minimo dell'albero binario di ricerca sfruttando
il fatto che l'elemento minimo dell'albero si trova nel sottoalbero più a sinistra presente nell'albero
per cui lo pseudocodice è facilmente implementabile come:
%Pseudocodice minimo di un albero binario di ricerca
\begin{codebox}
\Procname{$\proc{BST-min}(Tree T)$}
\li \If $\attrib{T}{left} \isequal \const{nil}$
    \Then
\li                       \Return $T$
\li \Else \Return $\proc{Tree-min}(\attrib{T}{left})$
\end{codebox}

La scansione dell'albero per trovare il sottoalbero più a sinistra richiede
$O(h)$ con $h$ rappresentante l'altezza  dell'albero;in caso l'albero sia bilanciato il tempo sarebbe $O(\log n)$.
Simmetricamente e con il tempo di esecuzione uguale, è l'algoritmo di ricerca del massimo di un
albero di ricerca, in cui il massimo si trova nel sottoalbero più a destra:
%Pseudocodice Massimo di un albero binario di ricerca
\begin{codebox}
\Procname{$\proc{BST-max}(Tree T)$}
\li \If $\attrib{T}{right} \isequal \const{nil}$
    \Then
\li                        \Return T
\li \Else \Return $\proc{Tree-max}(\attrib{T}{right})$
\end{codebox}
Scandendo l'array andando sempre nel sottoalbero destro richiede un tempo $O(h)$,
come anche già visto per la ricerca del minimo, con $h$ indicante l'altezza dell'albero.

%Pseudocodice Successore di un albero binario di ricerca
\begin{codebox}
\Procname{$\proc{BST-successor}(Tree T,Item x)$}
\li \If $\attrib{x}{right} \neq \const{nil}$
    \Then
\li             \Return $\proc{Tree-min}(\attrib{T}{right})$
\li $y \gets \attrib{x}{p}$
\li \While $y \neq \const{nil}$ and $x \isequal \attrib{y}{right}$
    \Do
\li                $x \gets y$
\li                $y \gets \attrib{y}{p}$
    \End
\Return $y$
\end{codebox}

%Pseudocodice Predecessore di un albero binario di ricerca
\begin{codebox}
\Procname{$\proc{BST-predecessor}(Tree T,Item x)$}
\li \If $\attrib{x}{left} \neq \const{nil}$
    \Then
\li                     \Return $\proc{Tree-maxinum}(\attrib{T}{left})$
    \End
\li y $\gets \attrib{x}{p}$
\li \While $y \neq \const{nil}$ and $x \isequal \attrib{y}{left}$
    \Do
\li                $x \gets y$
\li                $y \gets \attrib{y}{p}$
    \End
\li \Return $y$
\end{codebox}


%Pseudocodice inserimento di un albero binario di ricerca
L'inserimento di un'elemento in un albero binario di ricerca prevede di effettuare
un cammino, considerando il valore dell'elemento, per trovare la prima posizione
nulla, che rispetta la proprietà di albero di ricerca, dove inserire l'elemento.\newline
Analogamente alle altre operazioni su alberi l'inserimento richiede $O(h)$ in un albero
di altezza $h$ e il suo pseudocodice è il seguente:
\begin{codebox}
\Procname{$\proc{BST-Insert}(\textbf{Tree} \ T,\textbf{Item} \ z)$}
\li $ y \gets \const{nil}$
\li $ x \gets \attrib{T}{root}$
\li \While $ x \neq \const{nil}$
    \Do
\li            $ y \gets x $
\li            \If $\attrib{x}{value} < \attrib{z}{value}$
               \Then
\li                           $ x \gets \attrib{x}{right}$
\li            \Else $x \gets \attrib{x}{left}$
    \End
\li $\attrib{z}{p} \gets y$
\li \If $y \isequal \const{nil}$
    \Then
\li        $\attrib{T}{root} \gets z$
\li \ElseIf $\attrib{y}{value} < \attrib{z}{value}$
    \Then
\li                  $\attrib{y}{right} \gets z$
\li \Else $\attrib{y}{left} \gets z$
\end{codebox}



%Pseudocodice Rimozione elemento di un albero binario di ricerca
