%Capitolo sui grafi
\chapter{Grafi}
I grafi sono un importante struttura dati usata in molti ambiti ed applicazioni nel mondo informatico, ad esempio viene usato per trovare il migliore percorso per arrivare 
da un posto all'altro, infatti moltissimi ed interessanti problemi computazionali sono definiti in termini di grafi.\newline
Prima di analizzare come sono rappresentati su un calcolatore e i relativi algoritmi per poter ricavare informazioni preziose, andiamo a definire formalmente la 
nomenclatura rigurdo ai grafi, anche se è stata già affrontata nel corso di Fondamenti dell'Informatica.

Si definisce come \emph{grafo orientato} $G$ una coppia $(V, E)$, dove $V$ è un insieme finito di vertici ed $E$ è una relazione binaria su $V$,
i cui elementi sono chiamati \emph{archi}.\newline
Definiamo invece come \emph{grafo non orientato} $G$ una coppia $(V, E)$, dove $V$ è un insieme finito di vertici mentre l'insieme degli archi $E$ è composto da 
una coppia di vertici non ordinate, ossia se esiste un arco $(u, v)$ che collega il vertice $u$ a $v$ esiste anche l'arco simmetrico $(v, u)$, cosa non sempre verificata
nel caso di grafo orientato.\newline
Molte definizioni per i grafi orientati sono equivalenti anche per quelli non orientati, sebbene alcuni termini hanno significati leggermente differenti nei due contesti.

Se $(u, v)$ è un arco in un grafo orientato $G = (V, E)$ diciamo che $(u, v)$ esce nel vertice $u$ ed entra nel vertice $v$ mentre se $(u, v)$ è un arco in un grafo orientato
diciamo che $(u, v)$ è incidente nei vertici $u$ e $v$ ed inoltre $v$ è adiacente al vertice $u$, questa definizione di adiacenza è estendibile anche per i grafi orientati soltanto
che solo in quelli non orientati si ha che è simmetrica.\newline
Il \emph{grado} di un vertice in un grafo non orientato è il numero di archi che incidono nel vertice mentre in un grafo orientato si definisce come \emph{grado entrante}
il numero di archi che entrano in un vertice, come \emph{grado uscente} il numero di archi che escono da un vertice ed infine il \emph{grado} di un vertice in un grafo orientato
è la somma del suo grado uscente ed entrante.

\section{Rappresentazione nei calcolatori}
Per rappresentare un grafo in un calcolatore ci sono due maniere standard:una mediante una collezione di \emph{liste di adiacenza} ed una mediante una \emph{matrice di adiacenza}
ed entrambi possono essere usati sia per i grafi orientiati che per quelli non orientati.\newline
Di solito si preferisce utilizzare la rappresentazione mediante liste di adiacenza, perchè permette di rappresentare in modo compatto i grafi \emph{sparsi}, ossia quelli in cui 
$|E|$ è molto più piccolo di $|V|^2$ ed inoltre per la maggior parte degli algoritmi presentati si suppone la rappresentazione mediante liste.\newline
È preferibile usare la matrice di adiacenza quando si ha un grafo \emph{denso}, in cui $|E|$ è prossimo a $|V|^2$, o quando si ha la necessità di dire rapidamente se 
c'è un arco che collega due vertici particolari.

La rappresentazione con liste di adiacenza di un grafo $G = (V, E)$ consiste in una sequnza $Adj$ di $|V|$ liste, una per ogni vertice in $V$ in cui per ogni $u \in V$
la lista di adiacenza $Adj[u]$ contiene tutti i vertici $v$ tali che esiste un arco $(u, v) \in E$ per cui si può dire che le liste di adiacenza rappresentano
tutti gli archi di un grafo, per cui nello pseudocodice useremo la notazione $\attrib{G}{Adj[u]}$.\newline
Una rappresentazione di un grafo orientato e di uno non orientato si può notare nelle figure \ref{img:22.1} e \ref{img:22.2}.\newline
Se $G$ è un grafo orientato la somma delle lunghezze di tutte le liste di adiacenza è $|E|$, perchè un arco della forma $(u, v)$ è rappresentato inserendo $v$ in $Adj[u]$
mentre se $G$ è un grafo non orientato la somma è pari a $2|E|$, in quando con un arco $(u, v)$ si ha che $u$ appare nella lista di adiacenza di $v$ e viceversa.\newline
Le liste di adiacenza possono essere facilmente adattate per rappresentare i \emph{grafi pesati}, cioè i grafi per i quali ogni arco ha un \emph{peso} associato, 
tipicamente dato da una \emph{funzione peso} $w:E \to \R$, in cui il peso $w(u, v)$ dell'arco $(u, v) \in E$ viene memorizzato semplicemente assieme al vertice $v$
nella lista di adiacenza di $u$.

La rappresentazione mediante matrice di adiacenza presuppone che i vertici siano numerati $1, 2, \dots, |V|$ in modo arbitrario e la sua rappresentazione consiste 
in una matrice $A = (a_{ij})$ di dimensioni $|V| \times |V|$ tale che 
\[ 
    a_{ij} = \begin{cases}
                1 \quad (i, j) \in E \\
                0 \quad \text{negli altri casi} \\
             \end{cases} \]
Per comprendere al meglio la matrice di adiacenza si può notare le figure \ref{img:22.1} e \ref{img:22.2} ed inoltre richiede una memoria $\Theta(V^2)$,
indipendentemente dal numero di archi in un grafo, in quanto la matrice è di dimensione $V \times V$.

In un grafo non orientato la matrice di adiacenza è uguale alla sua trasposta, ossia $A = A^T$ ed in alcune applicazioni conviene memorizzare soltanto gli elementi
che si trovano sopra e lungo la diagonale, riducendo così la memoria per memorizzare il grafo quasi della metà.\newline
Anche la rappresentazione mediante matrice può essere utilizzata per i grafi pesati, in cui il peso $w(u, v)$ di un arco $(u, v) \in E$ viene semplicemente memorizzato
come l'elemento nella riga $u$ e nella colonna $v$ della matrice di adiacenza ed in caso un arco non esiste si può memorizzare il valore $\nil$, anche se per 
molti problemi sia preferibile usare il valore $0$ o $\infty$.\newline
Le due rappresentazioni sono equivalenti dal punto di vista asintotico, anche se quella mediante matrice può essere preferita in caso di grafi abbastanza piccoli
per la sua semplicità ed inoltre in caso di grafi non pesati con la rappresentazione con matrice si ha il vantaggio in memoria, in cui invece di rappresentare una 
parola della memoria del calcolatore per ogni elemento della matrice, basta utilizzare un solo bit per ogni elemento.

La maggior parte degli algoritmi che operano sui grafi devono mantenere degli attributi dei vertici e/o degli archi ed li indichiamo mediante $\attrib{v}{d}$,
per un attributo $d$ di un vertice $v$ e l'implemetazione degli attributi dei vertici ed archi nei programmi reali può essere completamente differente, in quanto
non esiste un modo migliore e la decisione su quale modo usare può dipendere dall'algoritmo che si sta implementando, dal linguaggio di programmazione usato ed ecc...

\section{Visita in ampiezza}
La \emph{visita in ampiezza}(breadth-first search) è uno dei più semplici algoritmi di ricerca nei grafi ed è la base per molti algoritmi importanti che operano
con i grafi, come ad esempio l'algoritmo di \emph{Prim} per l'albero di connessione minimo e l'algoritmo di \emph{Dijkstra} per i cammini minimi da sorgente unica.

Dato un grafo $G = (V, E)$ e un vertice distinto $s$, detto \emph{sorgente}, la visita in ampiezza ispeziona sistematicamente gli archi di $G$ per "scoprire"
tutti i vertici raggiungibili da $s$ e calcola la distanza, il minimo numero di archi, da $s$ a ciascun vertice raggiungibile.\newline
Per ogni vertice $v$ raggiungibile da $s$, il cammino semplice dell'albero BF che va da $s$ a $v$ corrisponde a un cammino minimo da $s$ a $v$, ossia un cammino
che contiene il minor numero di archi.

La visita in ampiezza è chiamata così perchè espande la frontiera fra i vertici scoperti e quelli da scoprire in maniera uniforme lungo l'ampiezza della frontiera,
ossia l'algortimo scopre tutti i vertici che si trovano a distanza $k$ da $s$ prima di scoprire i vertici a distanza $k + 1$ e per tenere traccia del lavoro svolto
vengono colorati i vertici di bianco, di grigio o di nero.\newline
Inizialmente tutti i vertici sono bianchi e quando un vertice viene \emph{scoperto}, ossia quando viene incontrato per la prima volta durante la visita, 
smette di essere definitavamente di colore bianco.

La visita in ampiezza costruisce l'albero BF che inizialmente contiene soltanto la sua radice, ossia il vertice sorgente $s$ e quando un vertice bianco $v$
viene scoperto durante l'ispezione della lista di adiacenza di un vertice $u$ già scoperto, il vertice $v$ e l'arco $(u, v)$ vengono aggiunti all'albero.\newline
Il vertice $u$ viene detto \emph{precedessore} o \emph{padre} di $v$ nell'albero BF ed inoltre poichè un vertice può venire scoperto una sola volta esso ha al più un padre.

La seguente procedura $\proc{BFS}$ \ref{alg:bfs} per la visita in ampiezza suppone che il grafo di input $G = (V, E)$ sia rappresentato mediante lista di adiacenza e
il colore di ogni vertice $u \in V$ è memorizzato nell'attributo $\id{color}$ mentre il precedessore di un vertice è memorizzato nel suo attributo $\pi$, che in caso 
non possiede un predecessore si ha $\attrib{v}{\pi} = \nil$.\newline
La distanza dalla sorgente $s$ al vertice $v$ calcolata dall'algoritmo è memorizzata nell'attributo $\attrib{v}{d}$ ed inoltre l'algoritmo usa anche la coda $Q$, 
con schema FIFO per gestire l'insieme dei vertici grigi.

Per calcolare il tempo di esecuzione dell'algoritmo $\proc{BFS}$ usiamo il metodo dell'aggregazione ed dopo l'inizializzazione nessun vertice sarà colorato di bianco, 
quindi il test $\attrib{v}{\id{color}} \equiv \id{WHITE}$ garantisce che ciascun vertice venga accodato al più una volta e di conseguenza venga eliminato dalla coda 
al più una volta.\newline
Le operazioni di inserimento e cancellazione dalla coda richiedono un tempo $O(1)$ quindi il tempo totale dedicato alle operazioni con la coda è $O(V)$ e 
poichè la lista di adiacenza di ciascun vertice viene ispezionata soltanto quando viene rimosso dalla coda, ogni lista di adiacenza viene ispezionata al più una volta
ed essendo la somma delle lunghezze di tutte le liste di adiacenza pari a $O(E)$, il costo di inizializzazione pari a $O(V)$, si ha che il tempo di esecuzione totale
di $\proc{BFS}$ è pari a $O(V + E)$

\subsection{Cammini minimi}
Come detto in precedenza, la visita in ampiezza trova la distanza di ciascun vertice raggiungibile in un grafo $G = (V, E)$ da un dato vertice sorgente $s \in V$ 
e definiamo ora la \emph{distanza di cammino minimo} $\gamma(s, v)$ da $s$ a $v$ come il numero minimo di archi in un cammino qualsiasi che va dal vertice $s$
al vertice $v$ e se non esiste un cammino tra i due vertici si ha $\gamma(s, v) = \infty$.\newline
Iniziamo ora a mostrare un importante lemma riguardo alla distanza di cammino minimo, che stabilisce un importante proprietà:
\begin{lem}
Sia $G = (V, E)$ un grafo qualsiasi e sia $s \in V$ un vertice qualsiasi del grafo allora per ogni arco $(u, v) \in E$ si ha
\[ \gamma(s, v) \leq \gamma(s, u) + 1 \]
\end{lem}
\begin{proof}
Se u è raggiungibile da s, allora anche v è raggiungibile dato che esiste, per ipotesi, l'arco $(u, v) \in E$.\newline
In questo caso il cammino minimo tra s e v non può essere più grande del cammino minimo tra s e u seguita dall'arco $(u, v)$ e questo che la disuglianza stabilisce
mentre in caso il cammino minimo tra s e v è pari a $\infty$ allora anche in quel caso la disuglianza è soddisfatta.
\end{proof}
Ora proviamo a mostrare che $v.d = \gamma(s, v)$ per ogni vertice $v \in V$, incominciando dal seguente lemma:
\begin{lem}
Sia $G = (V, E)$ un grafo qualsiasi e supponiamo che si è eseguita la BFS a partire da un vertice sorgente $s \in V$, allora per ogni vertice $v \in V$
alla fine dell'algoritmo BFS risulta $v.d \geq \gamma(s, v)$
\end{lem}
\begin{proof}
Per la dimostrazione usiamo l'induzione sul numero di operazioni $\proc{ENQUEUE}$ usate nell'algoritmo $\proc{BFS}$
\end{proof}

\section{Visita in profondità}
La \emph{Visita in profondità}(depth-first search) prevede di andare in profondità nel grafo per trovare gli elementi ossia di andare alla ricerca dei vertici ancora inesplorati
adiacenti all'ultimo vertice scoperto $u$
