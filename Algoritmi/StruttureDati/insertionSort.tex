%Algoritmo di Insertion Sort
L'algoritmo $\proc{Insertion-Sort}$ risolve il problema dell'ordinamento definito come:\newline
\textbf{Input}:una sequenza di $n$ numeri $(a_1,a_2,\dots,a_n)$ \newline
\textbf{Output}:una permutazione $(a'_1,a'_2,\dots,a'_n)$ tale che $a'_1 \leq a'_2 \leq \dots \leq a'_n$

%pseudocodice algoritmo insertionSort
\begin{figure}
    \caption{Algoritmo insertionSort}
    \label{alg:insertion}
    \begin{codebox}
        \Procname{$\proc{Insertion-Sort}(A)$}
        \li \For $j \gets 2$ \To $\attrib{A}{length}$
            \Do
        \li            $\id{key} \gets A[j]$
        \li         $i \gets j-1$
        \li         \While $i > 0$ and $A[i] > \id{key}$
                    \Do
        \li                $A[i+1] \gets A[i]$
        \li                $i \gets i-1$
                    \End
        \li         $A[i+1] \gets \id{key}$
            \End
    \end{codebox}
\end{figure}
L'algoritmo $\proc{Insertion-Sort}$ \ref{alg:insertion} è un algoritmo efficiente per ordinare un ristretto numero di elementi ed opera come farebbe un umano
a riordinare le carte da gioco, ossia prendendo una carta alla volta e facendo il riordinamento delle carte una alla volta.\newline
Per poter affermare che l'algoritmo è corretto, ossia risolve il problema, bisogna dimostrare l'invariante del ciclo $\kw{For}$, attraverso un metodo simile all'induzione matematica.

%Dimostrazione Invariante di Ciclo
L'invariante del ciclo è corretta se si riesce a dimostrare tre cose:
\begin{description}
  \item[Inizializzazione] è corretta prima della prima esecuzione del ciclo.
  \item[Conservazione] se è verificata prima di un iterazione del ciclo lo sarà anche dopo l'esecuzione di quell'iterazione del ciclo.
  \item[Conclusione] alla fine del ciclo è ancora verificato e ciò ci aiuta a determinare la correttezza di un algoritmo.
\end{description}
La terza proprietà è la più importante in quanto assieme alla condizione che è causato la conclusione del ciclo, si riesce a dimostrare la correttezza dell'algoritmo.

L'invariante di ciclo per l'$\proc{Insertion-Sort}$ è:
All'inizio di ogni iterazione del ciclo $\kw{for}$ il sottoarray $A[1 \twodots j-1]$
è ordinato ed è formato dagli stessi elementi che erano originamente in $A[1\twodots j-1]$.
\begin{description}
  \item[Inizializzazione] quando $j = 2$ il sottoarray $A[1\twodots j-1]$ è formato
                           da un solo elemento che è ordinato ed è l'elemento originale $A[1]$.
  \item[Conservazione] all'inizio di ogni esecuzione del ciclo for il sottoarray $A[1 \twodots j-1]$
                        è formato dai primi $j-1$ elementi dell'array ordinati dal
                        più piccolo al più grande.

  \item[Conclusione] Quando $j > \attrib{A}{length}$ il ciclo termina e dato che ogni
                      ciclo aumenta $j$ di $1$ alla fine del ciclo si avrà $j = n + 1$
                      per cui si ha che $A[1\twodots n]$ è ordinato ed è formato
                      dagli elementi ordinati che si trovavano in $A[1 \twodots n]$.
\end{description}
L'analisi di un algoritmo, per poter determinare se un algoritmo è efficiente, può
avvenire in due maniere:
\begin{description}
    \item [Tempo di Esecuzione] è il numero di operazioni primitive che vengono eseguite da parte di un algoritmo;l'esecuzione di un'istruzione si assume che richiede un tempo costante
                                per evitare di rendere la valutazione dipendente dall'hardware e dalla bravura del programmatore.
    \item [Spazio di Esecuzione] è il numero di spazio in bit occupato in memoria dall'algoritmo ma questa analisi non viene quasi mai eseguita in quanto oramai è superfluo.
\end{description}
Il tempo di esecuzione dell'algoritmo è la somma dei tempi di esecuzione per ogni istruzione eseguita quindi il tempo di esecuzione di $\proc{Insertion-Sort}$ è:
\begin{equation*}
    T(n) = c_1n + c_2(n-1) + c_3(n-1) + c_4 \sum_{j=2} ^n t_j + c_5 \sum_{j=2} ^n (t_j -1)
         + c_6 \sum_{j=2} ^n (t_j -1) + c_7(n-1)
\end{equation*}
In caso l'algoritmo sia già ordinato, caso migliore, si avrebbe sempre $A[i] < \id{key}$ quindi $t_j$ è sempre $1$, per cui il tempo di esecuzione sarebbe:
\begin{align*}
    T(n) & = c_1n + c_2(n-1) + c_3(n-1) + c_4(n-1) + c_7(n-1) \\
         & = (c_1 + c_2 + c_3 + c_4 + c_5)n - (c_2 + c_3+ c_4+ c_7) \\
         & = \Omega(n) \\
\end{align*}
Nel caso migliore si ha che l'algoritmo richiede un tempo lineare che è un $\Omega(n)$.\newline
In caso si abbia una sequenza decrescente, corrispondente al caso peggiore, nel ciclo While bisogna
confrontare ogni elemento $A[j]$ con il sottoarray $A[1 \twodots j-1]$ per cui $t_j = j$
per $j=2,3,\dots,n$ e poiche si ha
\begin{align*}
    \sum _{j=2} ^ n j = \frac{n(n+1)}{2} -1 & \quad \sum_{j=2}^n (j-1) = \frac{n(n-1)}{2}
\end{align*}
il tempo dell'algoritmo $\proc{Insertion-Sort}$ nel caso peggiore è il seguente:
\begin{align*}
    T(n) & = c_1n + c_2(n-1) + c_3(n-1) + c_4 (\frac{n(n+1)}{2} -1) + c_5 (\frac{n(n-1)}{2})
         + c_6 (\frac{n(n-1)}{2}) + c_7(n-1) \\
         & = c_1n + c_2(n-1) + c_3(n-1) + c_4(\frac{n^2+n-2}{2}) + c_5(\frac{n^2-n}{2})
           + c_6(\frac{n^2-n}{2}) + c_7(n-1) \\
         & = (\frac{c_4}{2} +\frac{c_5}{2} + \frac{c_6}{2})n^2 + (c_1+c_2+c_3+\frac{c_4}{2} - \frac{c_5}{2}
            -\frac{c_6}{2}+c_7)n -(c_2+c_3+c_4+c_7)\\
         & = O(n^2)\\
\end{align*}
Il tempo dell'algoritmo può essere scritto, nel caso peggiore, come $an^2 + bn + c$ che è una funzione
quadratica che viene indicata, nel caso peggiore come $O(n^2)$.

Nel caso medio mi aspetto, supponendo una distribuzione uniforme della probabilità,
che il vettore sia parzialmente ordinato per cui non avendo modo di rendere il ciclo
while eseguibile solo una volta, si ha bisogno almeno di $n^2$ confronti che è $O(n^2)$.

In sintesi i tempi di esecuzione dell'algoritmo $\proc{Insertion-Sort}$ sono:
\begin{description}
  \item[Caso migliore] $\Omega(n)$
  \item[Caso peggiore] $O(n^2)$
  \item[Caso medio] $O(n^2)$
\end{description}
