%Paragrafo sulle stack
\section{Stack}
Lo \emph{stack}, è una struttura dati di tipo \emph{LIFO}(Last in first out) utilizzata
in tutti i linguaggi di programmazione per effettuare la memorizzazione di tutti i dati
di tipo statico, che può essere vista come un caso particolare di sequenza in cui
l'inserimento avviene alla fine della sequenza e la rimozione avviene sempre in fondo.

Uno stack può essere implementato attraverso array oppure delle liste a seconda della
scelta implementatica e della capacità di stabilire un limite massimo di elementi utilizzati.

Essendo lo stack un particolare tipo di sequenza, essa può essere simulata tramite le operazioni di una lista
però è prassi comune utilizzare nomi diversi per indicarne le operazioni per migliore chiarezza.
%Inserire specifica operazioni su uno stack

%Fare pseudocodice delle operazioni dello stack

Utilizzando le liste per implementare lo stack otteniamo in tutte le operazioni l'
impiego di tempo costante $\Theta(1)$.

L'implementazione dello stack tramite un vettore ha lo stesso impiego di tempo $\Theta(1)$
in tutte le operazioni però per evitare uno spreco di memoria bisogna sapere il numero
di elementi necessari e soprattutto non è possibile superare il numero di elementi massimo
stabilito alla creazione dello stack.

La realizzazione delle operazioni dello stack tramite un vettore sono le seguenti:

%Pseudocodice delle operazioni dello Stack tramite vettore

































































.


Un alternativa implementazione dello stack avviene tramite
