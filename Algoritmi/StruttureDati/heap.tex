%Capitolo sul Heap
\chapter{Heap}
L'$\proc{heapSort}$ è un algoritmo di ordinamento che ordina sul posto, ossia solo un numero costante di elementi della sequenza
sono salvati fuori dalla sequenza di input ed inoltre utilizza la struttura dati \emph{heap} per gestire le informazioni.\newline
Lo \emph{heap} è una struttura dati rappresentata da un array $A$ che può essere vista come un albero binario
in cui ogni nodo dell'albero è un elemento dell'array, chiamato \emph{chiave}.\newline
L'array $A$ ha 2 attributi: $\attrib{A}{length}$ per rappresentare la lunghezza dell'array
e $\attrib{A}{heap-size}$ che indica il numero degli elementi dell'heap memorizzati in cui $0 \leq \attrib{A}{heap-size} \leq \attrib{A}{length}$.\newline 
Ci sono due tipologie di Heap:\emph{max-heap}, utilizzato nel heapSort e \emph{min-heap}, utilizzato principalmente per implementare code prioritarie;
queste due tipologie verranno analizzate entrambe in seguito in questo paragrafo.

%Fare disegno albero e array del heap

%Fare pseudocodice immediato di left,right e parent
Per poter effettuare l'accesso ai nodi left,right e parent si utilizzano le seguenti 3 procedure
\begin{codebox}
    \Procname{$\proc{left(i)}$}
    \li \Return 2i 
\end{codebox}
\begin{codebox}
    \Procname{$\proc{right(i)}$}
    \li \Return 2i + 1
\end{codebox}
\begin{codebox}
    \Procname{$\proc{parent(i)}$}
    \li \Return \cfloor \frac{i}{2} \rfloor 
\end{codebox}

Nei moderni computer le procedure $\proc{left}, \proc{right}$ e $\proc{parent}$ effettuano il loro lavoro mediante una solo istruzione, attraverso uno shift di un solo bit
verso sinistra in caso della procedura left e right mentre in caso della parent viene effettuato un shift di un solo bit verso destra.\newline
Una buona implementazione del heapsort consiste nell'implementare queste procedure come delle macro.

In un $\proc{max-heap}$ è soddisfatta per ogni nodo $i$ la seguente proprietà:
$A[\proc{Parent}(i)] \geq A[i]$ per cui il massimo valore della array si trova nella radice dell'heap mentre in un $\proc{min-heap}$ avviene il contrario 
ossia in ogni nodo si ha $A[\proc{Parent}(i)] \leq A[i]$ e la radice rappresenta l'elemento minimo dell'array.\newline
Essendo lo heap definito tramite un albero binario la sua altezza è $\Theta(\log n)$, definita come il più lungo percorso per raggiungere i nodi foglia a partire dalla radice.

\begin{figure}
    \caption{Max-Heapify implementation}
    \label{alg:maxHeapify}
    \begin{codebox}
        \li $l \gets \proc{Left(i)}$
        \li $r \gets \proc{Right(i)}$
        \li \If $l \leq \attrib{A}{heap-size}$ \And $A[l] > A[i]$
        \li             $\id{largest} \gets l$
        \li \Else $\id{largest} \gets i$
        \li \If $r \leq \attrib{A}{heap-size}$ \And $A[r] > A[\id{largest}]$
        \li             $\id{largest} \gets r$
            \EndIf
        \li \If $\id{largest} \neq i$
        \li      $\proc{Swap(A[i], A[\id{largest}])}$
        \li      $\proc{Max-Heapify(A, largest)}$
    \end{codebox}
\end{figure}
La procedura $\proc{maxHeapify}$ \ref{alg:maxHeapify} controlla se l'elemento $i$-esimo rispetta la proprietà del max-heap ossia che il figlio sinistro e destro di $i$ sia 
inferiore al valore dell'elemento $i$-esimo e in caso non sia rispettato viene sostituito il più grande tra $A[\proc{Left(i)}], A[\proc{Right(i)}], A[i]$ e lo sostituisce 
al posto di $A[i]$ poi si applica l'algoritmo Max-Heapify al sottoalbero dell'elemento maggiore, che non è detto che rispetta ancora
la proprietà di max heap dopo averlo sostituito con $A[i]$.

Il tempo di esecuzione di $\proc{Max-Heapify}$ prevede di avere un tempo $\Theta(1)$ per trovare il valore maggiore e scambiarlo con $A[i]$ mentre il tempo per 
risolvere l'algoritmo al sottoalbero prevede di avere al massimo $\frac{2n}{3}$  nodi per cui l'equazione di ricorrenza è la seguente
\[ T(n) \leq T(\frac{2n}{3}) + \Theta(1) \]
che per il teorema dell'esperto è $T(n) = O(\log n)$ o in maniera generale dato un heap di altezza $h$ il tempo di esecuzione di $\proc{Max-Heapify}$ è di $O(h)$.




%Strutture dati aggiornate
%suffix-tree
%FM-index
%suffix-array
%wavelet-tree
%Bloom filters
%Sequence Bloom trees
%Tabelle Hash
