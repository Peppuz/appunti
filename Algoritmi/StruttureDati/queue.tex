%Paragrafo sulle code
\section{Code}
La \emph{Queue}, in italiano \emph{coda}, è una struttura dati di tipo FIFO(First in First out)
per memorizzare una sequenza di elementi, in cui l'inserimento di un elemento
avviene in coda alla sequenza mentre la rimozione avviene in testa alla sequenza.

Una coda può essere implementata attraverso array oppure delle liste a seconda della
scelta implementatica e della capacità di stabilire un limite massimo di elementi utilizzati.

Essendo una coda una particolare tipologia di sequenza può essere implementata facilmente
utilizzando una lista e le sue operazioni però, a differenza dello stack, la scelta
della lista utilizzata per l'implementazione cambia il tempo di esecuzione delle operazioni,
infatti soltanto con una lista bidirezionale si ottiene un tempo $\Theta(1)$ in tutte le operazioni
per cui noi utilizziamo una lista bidirezionale per implementare una coda.

%Pseudocodice delle operazioni di una Coda
Il metodo $\proc{isEmpty}$ indica se la coda contiene o meno degli elementi
\begin{codebox}
\Procname{$\proc{isEmpty}()$}
\li \Return 
\end{codebox}
Il metodo $\proc{enqueue}(Q,x)$ inserisce un elemento nella coda e si assume che $x$ sia un elemento
definito prima della chiamata al metodo.
\begin{codebox}
\Procname{$\proc{Enqueue}(Q,x)$}
\li $\attribb{Q}{tail}{next} \gets x$
\li $\attrib{Q}{tail} \gets x$
\end{codebox}
L'esecuzione del metodo $\proc{enqueue}$ richiede $2c$ ossia $\Theta(1)$ per ciò
l'inserimento in una coda richiede un tempo costante.

Il metodo $\proc{dequeue()}$ estra il primo elemento della coda ed è implementato come
\begin{codebox}
\Procname{$\proc{Dequeue()}$}
\li \If $\proc{isEmpty()}$
    \Then
\li           \Return $\const{nil}$
\li $\id{temp} \gets \attrib{Q}{head}{element}$
\li $\attrib{Q}{head} \gets \attribb{Q}{head}{next}$
\li \Return temp
\end{codebox}
L'operazione $\proc{dequeue}$ richede un tempo costante $\Theta(1)$ per effettuare la rimozione.

%Coda tramite Vettore(guardare il Cormen)
