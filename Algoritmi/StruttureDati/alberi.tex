%Paragrafo sugli Alberi
\section{Alberi}
Si definisce come \emph{Albero libero}, un DAG connesso con un solo nodo sorgente, detto \emph{radice}, in cui ogni nodo diverso dalla radice ha un solo nodo entrante
mentre i nodi privi di archi entranti sono detti \emph{foglie} dell'albero.\newline


%Inserire esempi e proprietà degli Alberi
%%Paragrafo sugli Alberi
\section{Alberi}
Si definisce come \emph{Albero libero}, un DAG connesso con un solo nodo sorgente, detto \emph{radice},
in cui ogni nodo diverso dalla radice ha un solo nodo entrante.\newline
I nodi privi di archi entranti sono detti \emph{foglie} dell'albero.

%Inserire esempi e proprietà degli Alberi
%%Paragrafo sugli Alberi
\section{Alberi}
Si definisce come \emph{Albero libero}, un DAG connesso con un solo nodo sorgente, detto \emph{radice},
in cui ogni nodo diverso dalla radice ha un solo nodo entrante.\newline
I nodi privi di archi entranti sono detti \emph{foglie} dell'albero.

%Inserire esempi e proprietà degli Alberi
%%Paragrafo sugli Alberi
\section{Alberi}
Si definisce come \emph{Albero libero}, un DAG connesso con un solo nodo sorgente, detto \emph{radice},
in cui ogni nodo diverso dalla radice ha un solo nodo entrante.\newline
I nodi privi di archi entranti sono detti \emph{foglie} dell'albero.

%Inserire esempi e proprietà degli Alberi
%\input{Esempi/alberi}Esempi Alberi!!!!

L'albero è una struttura matematica importantissima in informatica utilizzata per
rappresentare una serie di situazioni, come ad esempio organizzazioni gerarchiche di dati,
procedimenti enumerativi o decisionali, e ve ne esistono un'infinita di implementazioni di alberi
però iniziamo ad analizzare per prima gli alberi liberi binari.

%Inserire immagine albero binario

%Metodi di visita di un albero
I metodi di visita di un albero binario sono 3:
\begin{itemize}
  \item inorder: si visiona prima il sottoalbero sinistro poi il nodo e infine il sottoalbero destro
  \item preorder: si visiona prima il nodo poi i suoi sottoalberi
  \item postorder: si visionano prima i sottoalberi ed infine il nodo
\end{itemize}
Il primo metodo di visita viene usato soprattutto negli alberi binari di ricerca per
stampare gli elementi dell'albero in maniera crescente mentre in un albero binario
normale la scelta di quale metodo di visita utilizzare è inifluente e ogni programmatore
sceglie nell'utilizzo quale metodo di visita utilizzare per stampare l'albero.
Il cammino dalla radice ad un elemento foglia dell'albero richiede al massimo $O(h)$,
in cui $h$ è l'altezza dell'albero, in quanto richiede di scendere di livello fino
ad arrivare alle foglie, che si trovano al livello $h$.

La specifica di un albero binario, in cui ogni implementazione per essere valida deve prevedere:\newline
\textbf{Item} search(Tree T,Item k);\newline
\textbf{void} insert(Tree T,Item x);\newline
\textbf{Item} delete(Tree T,Item x);\newline
\textbf{Item} mininum(Tree T);\newline
\textbf{Item} maxinum(Tree T);\newline
\textbf{Item} predecessor(Tree T,Item x);\newline
\textbf{Item} successor(Tree T,Item x);\newline
Esempi Alberi!!!!

L'albero è una struttura matematica importantissima in informatica utilizzata per
rappresentare una serie di situazioni, come ad esempio organizzazioni gerarchiche di dati,
procedimenti enumerativi o decisionali, e ve ne esistono un'infinita di implementazioni di alberi
però iniziamo ad analizzare per prima gli alberi liberi binari.

%Inserire immagine albero binario

%Metodi di visita di un albero
I metodi di visita di un albero binario sono 3:
\begin{itemize}
  \item inorder: si visiona prima il sottoalbero sinistro poi il nodo e infine il sottoalbero destro
  \item preorder: si visiona prima il nodo poi i suoi sottoalberi
  \item postorder: si visionano prima i sottoalberi ed infine il nodo
\end{itemize}
Il primo metodo di visita viene usato soprattutto negli alberi binari di ricerca per
stampare gli elementi dell'albero in maniera crescente mentre in un albero binario
normale la scelta di quale metodo di visita utilizzare è inifluente e ogni programmatore
sceglie nell'utilizzo quale metodo di visita utilizzare per stampare l'albero.
Il cammino dalla radice ad un elemento foglia dell'albero richiede al massimo $O(h)$,
in cui $h$ è l'altezza dell'albero, in quanto richiede di scendere di livello fino
ad arrivare alle foglie, che si trovano al livello $h$.

La specifica di un albero binario, in cui ogni implementazione per essere valida deve prevedere:\newline
\textbf{Item} search(Tree T,Item k);\newline
\textbf{void} insert(Tree T,Item x);\newline
\textbf{Item} delete(Tree T,Item x);\newline
\textbf{Item} mininum(Tree T);\newline
\textbf{Item} maxinum(Tree T);\newline
\textbf{Item} predecessor(Tree T,Item x);\newline
\textbf{Item} successor(Tree T,Item x);\newline
Esempi Alberi!!!!

L'albero è una struttura matematica importantissima in informatica utilizzata per
rappresentare una serie di situazioni, come ad esempio organizzazioni gerarchiche di dati,
procedimenti enumerativi o decisionali, e ve ne esistono un'infinita di implementazioni di alberi
però iniziamo ad analizzare per prima gli alberi liberi binari.

%Inserire immagine albero binario

%Metodi di visita di un albero
I metodi di visita di un albero binario sono 3:
\begin{itemize}
  \item inorder: si visiona prima il sottoalbero sinistro poi il nodo e infine il sottoalbero destro
  \item preorder: si visiona prima il nodo poi i suoi sottoalberi
  \item postorder: si visionano prima i sottoalberi ed infine il nodo
\end{itemize}
Il primo metodo di visita viene usato soprattutto negli alberi binari di ricerca per
stampare gli elementi dell'albero in maniera crescente mentre in un albero binario
normale la scelta di quale metodo di visita utilizzare è inifluente e ogni programmatore
sceglie nell'utilizzo quale metodo di visita utilizzare per stampare l'albero.
Il cammino dalla radice ad un elemento foglia dell'albero richiede al massimo $O(h)$,
in cui $h$ è l'altezza dell'albero, in quanto richiede di scendere di livello fino
ad arrivare alle foglie, che si trovano al livello $h$.

La specifica di un albero binario, in cui ogni implementazione per essere valida deve prevedere:\newline
\textbf{Item} search(Tree T,Item k);\newline
\textbf{void} insert(Tree T,Item x);\newline
\textbf{Item} delete(Tree T,Item x);\newline
\textbf{Item} mininum(Tree T);\newline
\textbf{Item} maxinum(Tree T);\newline
\textbf{Item} predecessor(Tree T,Item x);\newline
\textbf{Item} successor(Tree T,Item x);\newline
Esempi Alberi!!!!

L'albero è una struttura matematica importantissima in informatica utilizzata per rappresentare una serie di situazioni, come ad esempio organizzazioni gerarchiche di dati,
procedimenti enumerativi o decisionali, e ve ne esistono un'infinita di implementazioni di alberi.\newline
Solitamente si implementano ed utilizzano di solito gli alberi binari, in cui ogni nodo può avere al massimo due nodi uscenti, ma anche gli alberi liberi
non avendo questa limitazione possono essere usati ed utili per cui forniremo un implementazione in pseudocodice sia degli alberi liberi che di quelli binari.

%Inserire immagine albero binario

%Metodi di visita di un albero
I metodi di visita di un albero binario sono 3:
\begin{itemize}
  \item inorder: si visiona prima il sottoalbero sinistro poi il nodo e infine il sottoalbero destro
  \item preorder: si visiona prima il nodo poi i suoi sottoalberi
  \item postorder: si visionano prima i sottoalberi ed infine il nodo
\end{itemize}
Il primo metodo di visita viene usato soprattutto negli alberi binari di ricerca per
stampare gli elementi dell'albero in maniera crescente mentre in un albero binario
normale la scelta di quale metodo di visita utilizzare è inifluente e ogni programmatore
sceglie nell'utilizzo quale metodo di visita utilizzare per stampare l'albero.
Il cammino dalla radice ad un elemento foglia dell'albero richiede al massimo $O(h)$,
in cui $h$ è l'altezza dell'albero, in quanto richiede di scendere di livello fino
ad arrivare alle foglie, che si trovano al livello $h$.

La specifica di un albero binario, in cui ogni implementazione per essere valida deve prevedere:\newline
\begin{minted}{python}
    Item search(Tree T,Item k);
    void insertLeft(Tree T,Item x);
    void insertyRight(Tree T, Item x);
    Item deleteLeft(Tree T);
    Item deleteRight(Tree T);
    Item parent(Tree T);
    Item left(Tree T);
    Item right(Tree T);
\end{minted}
L'implementazione di un albero binario avviene solitamente mediante l'uso di puntatori, usando un puntatore per memorizzare il padre di un nodo, una per memorizzare
il figlio sinistro ed infine uno per poter riconoscere il figlio destro di un nodo.\newline
Lo pseudocodice delle operazioni di un albero binario si può trovare in seguito nel listato \ref{alg:binaryTree}, con i costi computazionale delle operazioni si 
può trovare in \ref{img:binaryTreeCost}.


