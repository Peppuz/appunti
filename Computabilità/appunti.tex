%Appunti del Corso Computabilità
\documentclass[a4paper]{report}
\usepackage[T1]{fontenc}%per rappresentare i font italiani, come le lettere accentate, con la giusta spaziatura
\usepackage[utf8]{inputenc}%per poter inserire nel testo .tex i caratteri unicode8
\usepackage[italian]{babel}%per poter effettuare la giusta sillabazione della lingua italiana
\usepackage{amsmath}%per poter rappresentare ed utilizzare al meglio gli ambienti e le formule matematiche
\usepackage{amssymb}%per rappresentare alcuni simboli particolari matematici
\usepackage{amsthm}%per definire e poter effettuare le dimostrazioni matematiche
%\usepackage{amsfont}%per poter avere i font matematici
\usepackage{booktabs}%per la corretta gestione delle tabelle
\usepackage{graphics}%per effettuare i grafici
\usepackage{pgfplots}%per i grafici
\usepackage{rotating}%per effettuare le rotazioni delle immagini e grafici
\usepackage{microtype}%per effettuare un aggiustamento della spaziatura tra caratteri e del font
%\usepackage{minted}
\usepackage{url}%per poter rappresentare gli url nel testo latex
%\usepackage{hypertext}%per effettuare un collegamento con una pagina internet
\theoremstyle{definition}%per avere lo stile tondo quando uso un ambiente definito da newtheorem
\newtheorem*{defi}{Def}%Definizione per avere la gestione delle definizioni
\newtheorem{prop}{Prop}[chapter]
\newtheorem{lem}{Lemma:}
\newtheorem{thm}{Thm:}[chapter]
\newtheorem{corol}{Corol}[chapter]
\newtheorem*{esempio}{Esempio}
\newcommand{\numberset}{\mathbb}
\newcommand{\N}{\numberset{N}}
\newcommand{\Z}{\numberset{Z}}
\newcommand{\Q}{\numberset{Q}}
\newcommand{\R}{\numberset{R}}
\newcommand{\C}{\numberset{C}}

\begin{document}
\title{Appunti del corso Linguaggi e Computabilità}
\author{Marco Natali}
\date{}
\maketitle

Il corso di Linguaggi e Computabilità riguarda l'informatica teorica e si occupa di definire la calcolabilità di un problema,
di definire le grammatiche e i linguaggi formali con l'ausilio di anche di automi e macchine di Turing.

Incominciamo con la definizione dei componenti basilari attraverso cui svilupperemo poi i concetti del corso
\begin{defi}
  Si definisce come \emph{alfabeto}, indicato con $\Sigma$, una sequenza di simboli, attraverso cui possiamo stabilire un alfabeto.
\end{defi}
\begin{esempio}
  $\Sigma = \{0, 1\}$ e $\Sigma = \{a, b, c\}$
\end{esempio}

\begin{defi}
  Si definisce come stringa, indicata solitamente con $w$, una sequenza di simboli appartenenti ad un alfabeto $\Sigma$
\end{defi}
Esempio: su un alfabeto $\Sigma = \{0, 1\}$ definiamo le seguenti stringhe:
  $w = 10110 \,\,\, z = 10111111$

Considerando un qualsiasi alfabeto $\Sigma$, esisterà sempre la stringa $\epsilon$, rappresentante la stringa vuota per cui
attraverso questa considerazione si definisce una stringa in maniera induttiva come:
\begin{defi}
 Una stringa viene definita induttiva come:
    \begin{description}
      \item [caso base]: $\epsilon$ è una stringa vuota
      \item [caso passo]: se $w$ è una stringa, allora anche $a \circ w $ è una stringa
    \end{description}
\end{defi}
Dopo aver definito le stringhe, definiamo le seguenti operazioni definite su le stringhe:
  \begin{itemize}
  \item lunghezza $|w|: \Sigma ^ * \to \N$: rappresenta il numero di caratteri presenti in una stringa, con $\Sigma ^ *$ indicante
        una qualsiasi stringa e la definizione di lunghezza avviene induttivamente come segue:
        \begin{description}
        \item [base]: la lunghezza di $|\epsilon| = 0$
        \item [passo]: se $|w| = n$ con $n \in \N$ e sia $a \in \Sigma$ allora $|a \circ w| = 1 + |w| = n + 1$.
        \end{description}
        \begin{esempio}
          $w = abcdec  \quad |w| = 6$
        \end{esempio}
  \item insieme di stringhe: definiamo come $\Sigma ^ k$ l'insieme di stringhe su $\Sigma$ con $k$ caratteri come segue:
        \begin{align}
          \Sigma ^ 0 & = \{ \epsilon\} \\
          \Sigma ^ 1 & = \Sigma \\
          \Sigma ^ 2 & = insieme di stringhe di due caratteri \\
          \dots \dots \\
          \Sigma ^ k & = insieme di stringhe di k caratteri \\
        \end{align}
        Le due più importanti insiemi di stringhe, usate per rappresentare l'insieme di stringhe di qualsiasi lunghezza, sono:
        \begin{align}
          \Sigma ^ * & = \cup _{i = 0} \Sigma ^ i \\
          \Sigma ^ + & = \Sigma ^ * - \{\epsilon\} \\
        \end{align}
        
  \item concatenazione $\circ: \sum ^ * \times \sum ^ * \to \sum ^ *$: rappresenta l'aggiunta dei caratteri della seconda stringa
        al termine della prima stringa ma vediamo ora una definizione più formale:
        \begin{defi}
          Date due stringhe $x = a_1 a_2 \dots a_n$ e $y = b_1 b_2 \dots b_n$ si definisce $x \circ y = a_1 a_2 \dots a_n b_1 b_2 \dots b_n$
          con $|x \circ y| = |x| + |y| = n + m$.
        \end{defi}
        La concatenazione possiede le seguenti proprietà:
        \begin{itemize}
        \item è associativa: $\forall x,y,z \in \sum^* x \circ (y \circ z) = (x \circ y) \circ z$.
        \item non è commutativa infatti presi due stringhe $x,y$ diverse risulta $x \circ y \neq y \circ x$.
        \item possiede l'elemento neutro $\epsilon$ per cui $x \circ \epsilon = x = \epsilon \circ x$.
        \end{itemize}
        \begin{esempio}
          \begin{equation*}
          \begin{split}
            w = 1011100 \quad z = 1011110  \\
          w \circ z = 10111001011110 \\
          z \circ w = 10111101011100 \\
          \end{split}
          \end{equation*}
        \end{esempio}
  \end{itemize}
  Attraverso le seguenti operazioni si può stabilire che $(\Sigma ^ *,\circ,\epsilon)$ è un monoide libero su $\Sigma$.

  \begin{defi}
    Definiamo come \emph{Linguaggio} un'insieme di stringhe scelte su $\Sigma ^ *$ scelte per far parte del linguaggio
    ossia $L \subseteq \Sigma ^ *$.
    Le componenti di un linguaggio sono:
    \begin{description}
    \item [alfabeto] : insieme di simboli su cui definiamo poi il lessico e la sintassi
    \item [lessico]: definisce il vocabolario del linguaggio e viene definito tramite una grammatica di tipo 3
    \item [sintassi]: definisce come le varie frasi del linguaggio devono essere disposte nel linguaggio e ciò
      viene definito da una grammatica di tipo 2.
    \item [semantica]: il significato attribuito alle frasi del linguaggio però nei linguaggi formali deriva dalla sintassi
      anche se in questo corso la semantica non verrà affrontata.
    \end{description}
  \end{defi}

  \begin{defi}
    Si definisce come \emph{Grammatica} un'insieme di regole che delineano le stringhe ammissibili del linguaggio e
    possono essere di due tipologie:
    \begin{description}
      \item [grammatica generativa] : insieme di regole che permettono di generare tutte le stringhe di un lunguaggio partendo dalle stringhe base 
      \item [grammatica analitica] : analizza le stringhe passate in input e stabilisce l'appartenenza o meno al linguaggio
    \end{description}
    \end{defi}
  Nel 1956 il linguista Choumsky introdusse e definì la gerarchia delle grammatiche, che ha avuto una notevole importanza
  nell'informatica teorica anche se il suo intento era quello di catalogare le varie tipologie di linguaggi naturali:
  \begin{description}
  \item [grammatiche di tipo 3]: sono i linguaggi regolari, generati da grammatiche regolari e riconosciuti da automi a stati finiti.
    Una spiegazione dettagliata avverrà nel seguito dei paragrafi/capitoli.
    
  \item [grammatiche di tipo 2]: grammatiche libere dal contesto, riconosciute da automi a pila; una spiegazione più dettagliata
    avverrà nei paragrafi successivi.
    
  \item [grammatiche di tipo 1]: grammatica dipendente dal contesto in cui per definizione i lati destri delle produzioni delle
    grammatiche non posso essere più lunghi dei rispetti lati sinistri, ossia $\alpha \to \beta |\alpha| \geq |\beta|$.
    Per rappresentarli e stabilire se una stringa appartiene al linguaggio si usano gli \emph{automi lineari}, che non sono oggetto del corso.
    
  \item [grammatiche di tipo 0]: grammatiche in cui non vi è alcun vincolo per le produzioni della grammatica e può essere
         rappresentato tramite una macchina di turing nondeterministica però questa tipologia di grammatica non verrà affrontata.
  \end{description}
  
\chapter{Grammatiche di tipo 2}
Iniziamo ora a considerare le grammatiche di tipo 2 incominciando da un esempio e poi definendola in maniera formale.
\begin{esempio}
  Dato il linguaggio delle stringhe palindrome $L _{pal} = \{w \in \Sigma ^ * : w = w ^ R\}$ dove $w^R$ rappresenta la stringa
  reversa per cui ad esempio $''OTTO'' \in L _ {pal}$ mentre $''PAPA''$ non appartiene al linguaggio.
 
  Definiamo in maniera più formale e meno ambigua i componenti di $L_{pal}$:
  \begin{description}
  \item [caso base]: $\epsilon, 0, 1 \in L_{pal}$
  \item [caso induttivo]: se $w \in L_{pal}$ allora $0w0$ e $1w1$ appartengono a $L_{pal}$ e nient'altro appartiene al linguaggio.
  \end{description}
  Le regole di derivazione per $L_{pal}$ sono le seguenti, e derivano dalla definizione dei componenti:
  \begin{itemize}
  \item $P \to \epsilon$
  \item $P \to 0$
  \item $P \to 1$
  \item $P \to 0P0$
  \item $P \to 1P1$
  \end{itemize}
  Dopo aver dato le regole di derivazione dobbiamo capire se le seguente regole definiscono tutte e sole le stringhe del linguaggio per
  cui ci chiediamo per esempio se $010 \in L_{pal}$ e $10001 \in L_{pal}$?
  \begin{align}
    P \Rightarrow 0P0 \Rightarrow 010  \quad \text{la stringa appartiene correttamente al linguaggio}\\
    P \Rightarrow 1P1 \Rightarrow 10P01 \Rightarrow 10001 \quad \text{la stringa appartiene correttamente al linguaggio}\\
  \end{align}
  
  Attraverso l'applicazione di una serie di regole di derivazione otteniamo una stringa $w \in \Sigma ^*$ se e solo $w \in L$
  della grammatica definita.
\end{esempio}

%Definizione di Grammatica
\begin{defi}
  Si definisce una grammatica context free come $G = (V, T, P_g, S)$, i cui componenti sono:
  \begin{itemize}
  \item $V$ rappresenta l'insieme delle variabili usate per rappresentare un linguaggio
  \item $T$ rappresenta l'insieme dei simboli terminali, ossia l'insieme dei simboli attraverso cui sono definite le stringhe
    del linguaggio per questo di solito coincide con l'alfabeto del linguaggio.

  \item $S$ rappresenta la variabile di inizio della grammatica, ossia la variabile attraverso cui si definisce la grammatica
    per cui le altre variabili sono classi ausiliari di stringhe che aiutano a definire le stringhe del linguaggio.

  \item $P_g$ indica l'insieme di regole, che rappresentano la definizione ricorsiva del linguaggio, della seguente forma:
    \begin{equation*}
        P_g = \{ X \to \beta | \beta \in (V \cup  T)^* \ \text{e} \ X \in V\}
    \end{equation*}
    la variabile $X$ rappresenta la testa della produzione mentre $\beta$ indica il corpo
  \end{itemize}
\end{defi}

  Nell'esempio del linguaggio palidromo la grammatica che lo genera è
  \begin{equation*}
    G = (\{P\},\{0, 1\}, \{P \to \epsilon,P \to 0,P \to 1,P \to 0P0,P \to 1P1\}, S\}
  \end{equation*}

  %Concetto di derivazione
  Dopo aver definito in maniera formale di grammatica introduciamo il concetto di derivazione, per stabilire se una stringa appartiene
  o meno al linguaggio.
  
  \begin{defi}
    Sia $G = (V, T,P_g,S)$ una grammatica context-free, sia $\alpha A \beta$ una stringa di terminali e variabili con $A \in V$ e
    sia infine $A \to \gamma$ una regola di derivazione allora $\alpha A \beta \Rightarrow _g \alpha \gamma \beta$.
  \end{defi}
  \begin{defi}
    Si indica $\Rightarrow _g ^*$, il simbolo di applicazione di zero,uno o più step di derivazione, definiti nel seguente modo:
    \begin{description}
    \item [caso base]: per ogni stringa $\alpha$ di terminali e variabili, si ha $\alpha \Rightarrow _g ^* \alpha$
    \item [caso passo]: se $\alpha \Rightarrow _g ^* \beta$ e $\beta \Rightarrow _ g \gamma$ allora $\alpha \Rightarrow _g ^* \gamma$
    \end{description}
  \end{defi}
  %Esempio

  Le stringhe che otteniamo sono delle forme sentenziali, ossia stringhe appartenenti a $(V \cup T)^*$ e un particolare
  sottoinsieme, in cui le stringhe sono composte da letterali, definisce le stringhe del linguaggio.

  Sempre considerando l'esempio del linguaggio delle stringhe palindrome la derivazione di $10011001$ è la seguente:
  \begin{equation*}
    P \Rightarrow 1P1 \Rightarrow 10P01 \Rightarrow 100P001 \Rightarrow 1001P1001 \Rightarrow 10011001
  \end{equation*}
  
  Al fine di ridurre il numero di scelte nella derivazione di una stringa introduciamo ora:
  \begin{description}
  \item [left derivation]: sostituiamo la variabile più a sinistra nell'applicazione di una regola di derivazione
                           e ciò viene rappresentato con il simbolo $\Rightarrow _{lm}$.

  \item [right derivation]: sostituiamo la variabile più a destra nell'applicazione di una regola di derivazione
                            ed essoviene rappresentato con il simbolo $\Rightarrow _{rm}$.
  \end{description}

  \begin{defi}
    Data una grammatica context-free $G = (V, T, P_g, S)$ si definisce un linguaggio context-free $L$ come:
    \begin{equation*}
      L_g = \{ w \in T^* | S \Rightarrow _g ^* w \}
    \end{equation*}
  \end{defi}

  Nei linguaggi context-free si può soltanto effettuare la concatenazione e l'annidamento dei sottolinguaggi come vediamo
  nei seguenti esempi:
  
  %Esempio di linguaggio Context-free
  \begin{esempio}
  Mostriamo ora un esempio di linguaggio definito come la concatenazione di due linguaggi
  \begin{equation*}
     L_g = \{ w \in \{0,1\}^* | w = 0^m 1^{m+1}01^n001^n \ n\geq 0\}
  \end{equation*}
  Per sapere come costruire la grammatica del linguaggio dobbiamo visualizzare la struttura della stringa
  \begin{equation*}
    0^m11^m 0 1^n001^n  \quad \text{lo spazio rappresenta i blocchi su cui viene formata la stringa}
  \end{equation*}
  Si può vedere che i 3 blocchi sono 3 linguaggi che concatenati generano il linguaggio $L_g$ per cui le regole
  di derivazione di $L_g$ sono $P_g = \{S \to X0Y, X \to 1 | 0X1, Y \to 1001 | 1Y1\}$.
  \end{esempio}
  
  \begin{esempio}
    Mostriamo ora un esempio di linguaggio definito come l'annidamento di linguaggi:
    \begin{equation*}
      L = \{ w = \{a,b,c,d\}^* | w = a^nb^mc^md^n \, m > 0, n \geq 0\}
    \end{equation*}
    Le regole di produzione del seguente linguaggio sono $P_g = \{S \to aSd | Y, Y \to bYc | bc\}$
    e come si nota il linguaggio viene definito come una sequenza di a e d intrammezzate da un blocco Y, formato da b e c,
    e ciò è l'annidamento tra diversi blocchi di una stringa che formano le stringhe del linguaggio.
  \end{esempio}

  %Inferenza Ricorsiva
  Un'altra modalità per stabilire l'appartenza di una stringa di un linguaggio è \emph{inferenza ricorsiva}, il quale a differenza delle altre
  applica le regole dal corpo alla testa, ossia concateniamo  ogni terminale che appare nel corpo e inferiamo che la stringa trovata
  è nel linguaggio delle variabili, presenti in testa alle regole.

  Viene poco utilizzato in quanto è più naturale e chiaro pensare secondo la derivazione per cui ne vediamo soltanto un esempio
  e lo usiamo in un importante teorema sull'equivalenza delle modalità di derivazione, che verrà presentato prossimamente.

\begin{esempio}
Sia $G=(V,T,O,E)$, con $V=\{E,I\}$ e $T=\{a,b,0,1,(,),+,*\}$ 
quindi ho le seguenti regole, è di tipo 3:
\begin{enumerate}
\item $E\to I$
\item $E\to E+E$
\item $E\to E*E$
\item $E\to (E)$
\item $I\to a$
\item $I\to b$
\item $I\to Ia$
\item $I\to Ib$
\item $I\to I0$
\item $I\to I1$
\end{enumerate}
voglio ottenere $a*(a+b00)$ 
sostituisco sempre a destra (right most derivation)
$$E\to E*E\to E*(E)\to E*(E+E)\to E*(E+I)\to E+(E+I0)$$
$$\to R+(I+b00)\to E*(a+b00)\to I*(a+b00)\to a*(a+b00)$$

usiamo ora \textit{l'inferenza ricorsiva}:
\begin{center}
\begin{tabular}{|c|c|c|c|c|}
  \toprule passo & stringa ricorsiva & var & prod & passo stringa impiegata\\
\midrule 1 & a & I & 5 & $\backslash$ \\
\hline
2 & b & I & 6 & $\backslash$ \\ 
\hline
3 & b0 & I & 9 & 2\\
\hline
4 & b00 & I & 9 & 3\\
\hline
5 & a & E & 1 & 1 \\
\hline
6 & b00 & E & 1 & 4\\
\hline
7 & a+b00 & E & 2 & 5,6\\
\hline
8 & (a+b00) & E & 4 & 7\\
\hline
9 &a*(a+b00) & E & 3 & 5, 8\\
\hline
\end{tabular}
\end{center}
\end{esempio}
  
  %Alberi Sintattici
  Introduciamo ora un'importante forma grafica per vedere le regole di derivazioni applicate per formare una stringa
  \begin{defi}
    Data una grammatica context-free $G$, un albero sintattico per $G$ è un albero composto come:
    \begin{itemize}
       \item ogni nodo interno è etichettato con una variabile $X \in V$, con la radice etichettata con $S$.
       \item ogni foglia è etichettata con una variabile, un simbolo terminale o $\epsilon$; se una foglia viene etichettata
             con $\epsilon$ allora dev essere l'unico figlio del padre.
       \item se un nodo è etichettato con $A$ e i suoi figli sono etichettati come $x_1, x_2, \dots, x_k$, allora $A \to x_1x_2 \dots x_k$
             è una regola di produzione della grammatica.
    \end{itemize}
  \end{defi}
  Eseguendo il prodotto delle stringhe foglia otteniamo una stringa $w$ appartenente alla grammatica di cui abbiamo svolto l'albero sintattico
  
  %Esempio di Albero Sintattico
  \begin{esempio}
    Prendendo il linguaggio $L$ definito come segue:$L = \{w \in \{a,b,c,d\}^* | w = a^ncb^mcd^{n+m} n \geq 0,m > 0\}$
    stabiliamo una grammatica context-free e fare l'albero sintattico di $aacbbbcddddd \in L$:
    le regole di produzione di $L$ sono $P_g = \{S \Rightarrow aSd | cY, Y \Rightarrow bcd | bYd\}$ per cui la grammatica è:
    \begin{equation*}
      G = (\{S,Y\},\{a,b,c,d\},P_g,S)
    \end{equation*}
    L'albero sintattico di $aacbbbcddddd$ è il seguente:
    
  \end{esempio}

  \section{Equivalenza tra le derivazioni}
  In questo paragrafo consideriamo un importante teorema sulle equivalenze tra le varie modalità di derivazione, definito come segue:
  \begin{thm}
    Data una grammatica context-free $G = (V, T,P_g,S)$ abbiamo che le seguenti modalità di derivazione sono equivalenti:
    \begin{enumerate}
    \item la procedura di inferenza ricorsiva che determina che una stringa di terminali $w$ appartiene ad un linguaggio di variabili $A$
    \item $A \Rightarrow ^ * w$
    \item $A \Rightarrow _{lm} ^ * w$
    \item $A \Rightarrow _{rm} ^ * w$
    \item esiste un albero sintattico con radice $A$ e come prodotto di foglie la stringa $w$.
    \end{enumerate}
  \end{thm}
  %Immagine per vedere la modalità di equivalenza

  I primi due sono facilmente dimostrabili dato che le derivazione a sinistra e a destra sono delle derivazioni invece iniziamo a dimostrare:
  \begin{thm}

    pippo
  \end{thm}


  \section{Grammatiche e Linguaggi Ambigui}
  In questo paragrafo analizziamo le grammatiche e i linguaggi ambigui, incominciando da un esempio per catturare gli aspetti essenziali,
  che poi verranno definiti formalmente.

  Incominciamo per cui a considerare le espressioni algebriche definite su $T = \{+, *, a, b, (, )\}$ con le seguenti regole di produzione:
  \begin{equation*}
    P_g = \{E \Rightarrow I | E + E | E * E | (E), I \Rightarrow a | b | aI | bI \}
  \end{equation*}
  La grammatica per definire gli identificatori $I$ è una grammatica di tipo 3, cosa che analizzeremo nel prossimo capitolo, però
  questa grammatica è ambigua dato che preso ad esempio $a + b * a$ avremmo due derivazioni left-most, che sono le seguenti:
  %Alberi sintattici di a + b * a

  Diamo ora una definizione formale di grammatica ambigua:
  \begin{defi}
    Si dice che una grammatica è ambigua se e solo se esiste una stringa $w \in L$ tale per cui $w$ ammette due derivazioni $lm$ e/o $rm$
    diverse oppure ammette due alberi sintattici definiti sulla stessa grammatica $G$
  \end{defi}
  L'obiettivo è quello di avere grammatiche non ambigue perchè esse permettono di definire in maniera univoca, per cui automamilizzabile
  con una procedura, l'insieme delle stringhe del linguaggio e fortunatamente nella maggior parte dei casi, quando il linguaggio non è ambiguo,
  data una grammatica ambigua è possibile definire una grammatica non ambigua $G'$ tale che $L(G) = L(G')$.

  \begin{defi}
    Un linguaggio $L \subseteq \sum^*$ è detto \emph{ambiguo} se per ogni grammatica $G$, tale per cui $L = L_G$, risulta che $G$ è ambigua.
  \end{defi}
  La grammatica data delle espressioni algebriche ha due cause di ambiguità:
  \begin{enumerate}
  \item la precedenza degli operatori non viene rispettata in quanto andrebbe raggruppato prima il silmbolo di $*$ rispetto a $+$
  \item una sequenza di uguali operatori può essere reggruppata sia da sinistra che da destra e si stabilisce per convenzione
    che si raggruppa da sinistra a destra.
  \end{enumerate}
  Considerate queste cause di ambiguità, per eliminarla si introducono altre variabili definite come:
  \begin{enumerate}
  \item un fattore è un espressione che non può essere spezzata da nessun operatore, sia il $*$ che il $+$, per cui gli unici fattori
    nel nostro linguaggio delle espressioni sono gli identificatori e ogni espressioni dentro le parentesi.
  \item un termine è un espressione che non può spezzata dall'operatore $+$.
  \item un espressione può riferirsi a qualsiasi possibile espressione, incluse quelle che possono essere spezzate da un adiacente $*$ o $+$
  \end{enumerate}
  La grammatica non ambigua del linguaggio delle espressioni è la seguente:
  \begin{equation*}
    P_{exp} = \{E \to T | E + T, T \to F | T * F,F \to I | (E), I \to a | b | aI| bI | 0I | 1I\}
  \end{equation*}

  Mostriamo ora un esempio di linguaggio ambiguo $L$ definito come segue:
  \begin{equation*}
    L = \{w = a^nb^nc^md^m | n,m \geq 1\} \cup \{z = a^nb^mc^md^n | n,m \geq 1 \}
  \end{equation*}
  Le regole di produzione che generano il seguente linguaggio sono le seguenti:
  \begin{equation*}
    P_L = \{S \to XY | Z, X \to ab | aXb, Y \to cd | cYd, Z \to aZd | aRd, R \to bc | bRc\}
  \end{equation*}
  Questa grammatica è ovviamente ambigua dato che per esempio la stringa $aabbccdd$ ha due derivazioni sinistre:

  La dimostrazione che il linguaggio è ambiguo è molto complessa ma il concetto alla base è quello di provare che tutte le grammatiche
  del linguaggio generano un numero finito di stringhe con almeno due derivazioni sinistre/destre e nel nostro caso definiamo che il
  linguaggio è ambiguo perche per $n = m$ qualsiasi grammatica usiamo per definire il linguaggio avremmo i due sottolinguaggi che generano
  la stessa stringa in maniera diversa.

  \chapter{Linguaggi Regolari}
  Per definire le stringhe appartenenti ai linguaggi regolari, di tipo 3, vi può utilizzare le \emph{grammatiche regolari},
  in cui vengono definite delle regole per stabilire se e quando una stringa appartiene al linguaggio, e le \emph{espressioni regolari}

  Incominciamo a considerare le grammatiche regolari, sottoinsieme delle grammatiche di tipo 2 secondo la gerarchia di Choumsky,
  utilizzate per generare i linguaggi regolari.
  
  Prevedono dei maggiori vincoli sulle produzioni infatti:
  \begin{enumerate}
  \item $\epsilon$ può apparire solo nel corpo del simbolo start ossia può essere solo $S \to \epsilon$.
  \item le produzioni sono del seguente tipo, dove $a \in T$ e $A, B \in V$,:
    \begin{itemize}
    \item $A \to aB$ oppure $A \to a$(lineari a destra)
    \item $A \to Ba$ oppure $A \to a$(lineari a sinistra)
      \end{itemize}
  \item non si può avere produzioni sia lineari sinistre che lineari destre
  \end{enumerate}

  Esempio: i due insiemi di produzioni per definire gli identificatori di un espressione algebriche sono le seguenti:
  \begin{align*}
    P_1 = \{I \to a | b | Ia| Ib| I0 | I1\} \quad \text{lineare a destra}
    P_2 = \{I \to aJ | bJ, J \to aJ | bJ | 0J | 1J | 0 | 1 | a | b\} \quad \text{lineare a sinistra}
  \end{align*}

  Esempio:data la grammatica $G(\{S\},\{0,1\},\{S \to \epsilon | 0S | 1S\},S)$ capire che linguaggio rappresenta:
  $L = \{\epsilon, 0, 1, 00, 01, 10, 11, 000, 001, 010, 011, 100, 101, 110, 111, \dots \}$

  Esempio:Produrre le produzioni lineari sinistra e lineari destra del seguente linguaggio:
  \begin{equation*}
    L = \{w \in \{a, b\}* | w = a^nb^m n,m \geq 0\}
  \end{equation*}

  Esempio: Produrre le produzioni lineari sinistra e lineari destra del seguente linguaggio:
  \begin{equation*}
    L = \{w \in \{a,b,c,d,e\}* | w = ab^ncd^me n \geq 0, m > 0\}
  \end{equation*}

  Esempio: Fornire le produzioni lineari sinistra e lineari destra del seguente linguaggio:
  \begin{equation*}
    L = \{w \in \{0,1\}* | \text{w contiene almeno uno 0 oppure almeno un 1}\}
  \end{equation*}

  \section{Espressioni Regolari}
  Le espressioni regolari permettono di definire, utilizzando una notazione algebrica, un linguaggio regolare e vengono utilizzate
  per estrarre parole da un testo ed altre notevole applicazioni, che verranno analizzate nel corso dei paragrafi.

  Per riuscire a definire in maniera formale le espressioni regolari dobbiamo definire le seguenti operazioni sui linguaggi regolari:
  \begin{itemize}
  \item Unione: dati $L,M \subseteq \Sigma*$ si definisce $L \cup M = \{w \in \Sigma* | w \in L \vee w \in M\}$
    Esempio: $L = \{001, 10, 111\}$ e $M = \{\epsilon, 001\}$ risulta $L \cup M = \{\epsilon, 10, 001, 111\}$

  \item Concatenazione: dati due linguaggi $L, M \subseteq \Sigma*$ si ha $L \circ M = LM$, ossia linguaggio formato da tutte le strighe
    ottenute concatenando le stringhe in $L$ con le stringhe in $M$.
    Esempio: $L = \{001,10\}$ e $M = \{\epsilon, 111\}$ risulta $L \circ M = \{001, 10, 001111, 10111\}$
    
  \item Chiusura di Kleene: dato un linguaggio $L$ si ha $L*$ definito induttivamente come:
    \begin{equation*}
    \begin{split}
      L^0 = \{epsilon\}
      L^1 = L \\
      L^2 = L \circ L \\
      \dots \\
      L^i = L^{i-1} \circ L\\
      L^* = L^0 \cup L^1 \cup L^2 \cup \dots \\
      L^+ = L^* - \{epsilon\} \\
    \end{split}
    \end{equation*}
    Esempio: dato $L = \{0,1\}$ abbiamo:
    \begin{equation*}
    \begin{split}
      L^0 = \{epsilon\}\\
      L^1 = \{0, 1\}\\
      L^2 = \{00, 01, 10, 11\}\\
      L^3 = \{0000, 0001, 0010, 0011, 0100, 0101, 0110, 0111, 1000, 1001, 1010, 1011, 1100, 1101, 1110, 1111\} \\
    \end{split}
    \end{equation*}
    Il linguaggio $L*$ è generalmente un linguaggio infinito in quanto è l'unione di un numero infinito di linguaggi finiti
    ma esistono due linguaggi la cui chiusura è finita, che analizziamo ora:
    \begin{itemize}
    \item il linguaggio $L = \{0\}$ la sua chiusura di kleene è finita dato che si ha:
      \begin{equation*}
      \begin{split}
        L^0 = \epsilon \\
        L^1 = L\\
        L^2 = L \circ L = L\\
        L^3 = L^2 \circ L = L\\
        L^i = L^{i-1} \circ L = L \\
        L^* = L^0 \cup L^1 \cup L^2 \cup \dots = L \cup {epsilon} = L\\
      \end{split}
      \end{equation*}
    \item il linguaggio $L = \emptyset$ la sua chiusera di kleene è finita in quanto:
      \begin{equation*}
      \begin{split}
      L^0 = \epsilon \\
      L^1 = \emptyset \\
      L^2 = \emptyset \circ \emptyset = \emptyset \\
      L^i = \emptyset^i \circ \emptyset = \emptyset \\
      L^* = \emptyset \cup \{\epsilon\} = \{epsilon\} \\
      \end{split}
      \end{equation*}
    \end{itemize}
  \end{itemize}
  
    Dopo aver definito le operazioni sui linguaggi regolari, definiamo ora:
    \begin{defi}
      Si definisce \emph{espressione regolare} induttivamente come segue, considerando anche il linguaggio che generano:
      \begin{description}
      \item [caso base]: la base consiste in 3 parti:
        \begin{enumerate}
        \item $\epsilon$ e $\emptyset$ sono RegEx e generano $L(\epsilon) = \{epsilon\}$, $L(\emptyset) = \emptyset$
        \item se $a$ è un simbolo allora $a$ è una Regex e questa espressione genera $L(a) = \{a\}$
        \item una variabile, rappresentanti linguaggi regolari, sono RegEx, e generano $L(L) = L$
          \end{enumerate}

      \item [caso induttivo]: la parte induttiva delle espressioni regolari sono composte da 4 tipologie:
        \begin{enumerate}
        \item se $E$ e $F$ sono delle RegEx allora $E + F$ è una RegEx e rappresentano $L(E + F) = L(E) \cup L(F)$
        \item se $E$ e $F$ sono delle RegEx allora $E \circ F = EF$ è una RegEx per cui rappresentano $L(EF) = L(E)L(F)$
        \item se $E$ è una RegEx allora $E*$ è una RegEx, che denota la chiusura di $L(E)$ infatti $L(E^*) = (L(E))^*$
        \item se $E$ è una RegEx allora $(E)$ è una RegEx in cui $L((E)) = L(E)$.
        \end{enumerate}
      \end{description}
   \end{defi}
            
\end{document}
