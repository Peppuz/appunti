\begin{esempio}
trovo regex per l'insieme di stringhe in $\{0,1\}^*$ che consistono in 0 e 1 alternati:\\
$$01\to \{01\}$$
$$(01)^*\to \{\varepsilon, 01, 0101,010101,...\}$$
$$(01)^*+(10)^*\to\{\varepsilon, 01,10,0101,1010,...\}$$
ma posso volere diverse quantità di 0 e 1, sempre mantenendo l'alternanza, metto o uno 0 o un 1 davanti a quanto ottenuto appena sopra:
$$(01)^*+(10)^*+0(10)^*+1(01)^*\to \{\varepsilon,01,10,010,101,...\}$$
non è comunque l'unica soluzione, si può avere:
$$(\varepsilon+1)(01)^*(\varepsilon+0)\to \{\varepsilon,01,10,010,101,...\}$$
oppure ancora:
$$(\varepsilon+0)(10)^*(\varepsilon+1)$$
\end{esempio}
\end{definizione}
Forse queste sono da inserire
\item $(L^*)^*=L^*$
\item $\emptyset^*=\varepsilon$ infatti $L(\emptyset)=\{\varepsilon\}\cup L(\emptyset)\cup L(\emptyset)\cdot L(\emptyset)\cup...=\{\varepsilon\}\cup \emptyset\cup \emptyset...=\varepsilon$
\item $\varepsilon^*=\varepsilon$ infatti $L(\varepsilon^*)=\{\varepsilon\}\cup L(\varepsilon)\cup L(\varepsilon)=\{\varepsilon\}\cup \{\varepsilon\}\cup ...=\{\varepsilon\}=L(\varepsilon)$
\item $L^+=L\cdot L^*=L^*\cdot L$ (quindi con almeno un elemento che non sia la stringa vuota)
\item $L^*=l^++\varepsilon$  
\end{itemize}
\begin{esempio}
Ho $ER=(0+1)^*0^*(01)^*$:
\begin{itemize}
\item 001 fa parte del linguaggio? Si: $\varepsilon\cdot 0\cdot 01$
\item 1001 fa parte del linguaggio? Si: $1\cdot 0\cdot 01$
\item 0101 fa parte del linguaggio? Si: $\varepsilon\cdot\varepsilon \cdot 0101$
\item 0 fa parte del linguaggio? Si: $\varepsilon\cdot 0\cdot \varepsilon$
\item 10 fa parte del linguaggio? Si: $1\cdot 0\cdot \varepsilon$
\end{itemize}
$$L((0+1)^*)=(L(0+1))^*=(L(0)+L(1))^*=(\{0\}\cup \{1\})^*=(\{0,1\})^*=\{0,1\}^*$$
ovvero tutte le combinazioni di 0 e 1
\end{esempio}
Si ricorda che:
$$(0+1)^*\neq 0^*+1^*$$
\newpage
\begin{esempio}
ho $ER=((01)^*\cdot 10\cdot (0+1)^*)^*$
\begin{itemize}
\item 0101 fa parte del linguaggio? No
\item 01000 fa parte del linguaggio? No 
\item 01011 fa parte del linguaggio? No
\item 10111 fa parte del linguaggio? Si, $\varepsilon\cdot 10\cdot 111$
\item 101010 fa parte del linguaggio? Si, prendo $10\cdot 1010$
\item 101101 fa parte del linguaggio? Si, $\varepsilon\cdot 10\cdot 1$ due volte
\item 0101100011 fa parte del linguaggio? Si, $0101\cdot 10\cdot 0011$ (0011 lo posso prendere da $(0+1)^*$)
\end{itemize}
\end{esempio}
\begin{esempio}
ho $ER=((01)^*\cdot 10\cdot (0+1))^*$
\begin{itemize}
\item 0101 fa parte del linguaggio? No
\item 01000 fa parte del linguaggio? No 
\item 01011 fa parte del linguaggio? No
\item 10111 fa parte del linguaggio? No
\item 101010 fa parte del linguaggio? No
\item 101101 fa parte del linguaggio? Si, $\varepsilon\cdot 10\cdot 1$ due volte
\item 0101100011 fa parte del linguaggio? No
\end{itemize}
\end{esempio}
\begin{esempio}
Da $L\subseteq\{0,1\}|\mbox{ stringhe contenenti almeno una volta 01}$
quindi:
$$(0+1)^*01(0+1)^*$$
\end{esempio}
\begin{esempio}
ho $ER=(00^*1^*)^*$, quindi:
$$L=\{\varepsilon,0,01,000,001,010,011\}=\{\varepsilon\}\cup\{w\in\{0,1\}^* |\mbox{ w che inizia con 0}\}$$
\end{esempio}
\begin{esempio}
ho $ER=a(a+b)^*b$, quindi:
$$L=\{w\in\{a,b\}^*|\mbox{ w inizia con a e termina con b}\}$$
\end{esempio}
\begin{esempio}
ho $ER=(0^*1^*)^*000(0+1)^*$, quindi, sapendo che $\{0,1\}^*$ mi permette tutte le combinazioni che voglio come $(0+1)^*$:
$$L=\{w\in\{0,1\}^*|\mbox{ w come voglio con tre 0 consecutivi}\}$$
\end{esempio}
\begin{esempio}
ho $ER=a(a+b)^*c(a+b)^*c(a+b)^*b$, quindi:
$$L=\{w\in\{a,b,c\}^*|\mbox{ w inizia con a, termina con b  e contiene almeno due c, }$$
$$\mbox{eventtualmente non adiacenti}\}$$
\end{esempio}
\begin{esempio}
Da $L\subseteq\{0,1\}|\mbox{ ogni 1 è seguito da 0, a meno che non sia l'ultimo carattere}$, ovvero 11 non compare
quindi:
$$(10+0)^*(\varepsilon+1)^*$$
\end{esempio}
