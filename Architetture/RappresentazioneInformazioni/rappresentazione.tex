%Capitolo sulle rappresentazione delle informazioni
\chapter{Rappresentazione delle Informazioni}
Nei calcolatori viene utilizzata la codifica binaria, in quanto il passaggio o meno
della corrente è a due valori e ciò coincide con il numero di simboli nella codifica binaria.

Nella realtà i numeri hanno una cardinalità infinita mentre purtroppo i calcolatori,
avendo dei limiti di memoria e di componestistica, possono rappresentare soltanto
un numero finito di numeri, pari a $base ^ n$, stabilito al momento della progettazione della macchina.
Le informazioni, di qualsiasi tipo siano, vengono rappresentati sui calcolatori
come una sequenza di numeri per questo iniziamo prima l'analisi di come vengono rappresentati i numeri.

\section{Rappresentazione Numeri interi e negativi}
Per poter rappresentare un numero sia nella realtà che nei calcolatori si usa per
convenzione la rappresentazione posizionale in cui la cifra più a destra rappresenta
la potenza più alta ossia $274 = 2 * 100 + 7 * 10 + 4 * 1$.\newline
La forma generale di un numero è $\displaystyle \sum _{i=-k} ^ n d_i * base^i$ e le base più comunemente
utilizzate sono:
\begin{itemize}
  \item base 10(decimale):base utilizzata comunemente dagli umani per rappresentare
        i numeri mentre non viene usata nei calcolatori; le cifre usate sono da 0 a 9
  \item base 2(binaria): base usata comunemente nei calcolatori elettronci e prevede
        soltanto le cifre 0 e 1;un unica cifra in base binaria viene chiamata \emph{bit}.
  \item base 8(ottale): si utilizzano soltanto le prime 8 cifre da 0 a 7
  \item base 16(esadecimale):utilizzata solitamente per rappresentare le celle di memoria e
        prevede come cifre: 0,1,2,3,4,5,6,7,8,9,A,B,C,D,E,F.
\end{itemize}

%Esempi di rappresentazione dei numeri

La conversione tra le basi potenze di due sono immediate infatti basta espandere/restringere
le cifre per rappresentare le cifre dell'altra base infatti ad esempio:
%Inserire esempio di conversione tra base

La conversione tra un numero binario ed un numero decimale prevede qualche calcolo in più:
il metodo più utilizzato e valido per tutti i numeri è quello di sottrarre il numero
con la più grande potenza di 2 minore del numero e ripetendo lo stesso procedimento
con la differenza fino ad arrivare a 0.\newline
Si inserisce la cifra 1 nelle posizioni che corrispondono alle potenze di 2 usate.
%Inserire esempio di conversione binaria/decimale

Per rappresentare i numeri negativi nei calcolatori vi sono 4 modalità:
\begin{description}
  \item[modulo e segno] si utilizza il bit più a sinistra per rappresentare il segno
        (0 per + e 1 per -) e i restanti $n-1$ bit sono il valore assoluto del numero.\newline
        Non viene utilizzata nei calcolatori attuali in quanto si ha la problematica
        di rappresentare il doppio zero.
        %Inserire esempio di Modulo e Segno
  \item[complemento a 1] prevede anch'esso di utilizzare il bit del segno(0 per i positivi e 1 per i negativi)
        e la negazione di un numero si ottiene invertendo tutti bit, compreso quello di segno.
        Ha anch'esso la problematica del doppio zero per cui non viene utilizzato nei calcolatori attuali.
        %Esempio complemento a 1
  \item[complemento a 2] prevede il bit di segno e l'opposto di un numero si ottiene
        invertendo tutti i bit e poi alla fine aggiungere 1 al numero invertito.
        Con il complemento a due si risolve il problema del doppio zero ma si inserisce
        uno sfasamento di 1 tra i numeri negativi e i numeri positivi rappresentabili.
        %Esempio complemento a 2
  \item[notazione in eccesso di $2^{m-1}$] si rappresent il numero memorizzando come
        somma del numero e $2^{m-1}$ ed è poco utilizzato in quanto confusionario e
        porta facilmente ad errori ed incomprensioni nella lettura del numero.
        %Esempio notazione in eccesso
\end{description}

\section{Rappresentazione numeri in virgola fissa e mobile}
%Fare Conversione della parte frazionaria, della precisione e dell'errore di approssimazione
Per rappresentare i numeri con la virgola nei calcolatori vi sono due modalità:
\begin{description}
  \item[virgola fissa] Si riservano un numero di bit per la rappresentazione della
        parte intera e un altro quantitativo di bit per rappresentare la parte frazionaria.\newline
        Questa modalità non permette di rappresentare molti numeri e l'errore di approssimazione
        è uguale in tutte le posizioni dell'asse ed ha senso se si sa il range
        preciso di valori da rappresentare.\newline
        Per convertire la parte frazionaria in base B in base 2 si procede per
        moltiplicazioni successive come si può vedere nell'esempio:
        %Fare esempio conversione parte frazionaria

        %Esempio rappresentazione in virgola fissa
  \item[virgola mobile] sono comunemente utilizzati nei linguaggi di programmazione
        per rappresentare i numeri con la virgola e si utilizza la notazione scientifica
        $n = f * base^{exp}$ con $f$ indicante la parte frazionaria, per la rappresentazione.\newline
        Essendoci infiniti modi per rappresentare un numero in notazione scientifica,
        ma in un calcolatore, avendo una capacità finita, si stabilisce una convenzione
        per permetterne di rappresentare la maggiore combinazione di numeri.

        %Fare foto per rappresentare i numeri(Usa il libro Tanenbaum)

        Al contrario della realtà i numeri in virgola mobile non sono densi per cui
        alcuni numeri non possono essere rappresentati in maniera esatta e si commette
        per ciò un certo errore di approssimazione.\newline
        A differenza della rappresentazione in virgola fissa, l'errore complessivo varia
        al variare del numero mentre l'errore relativo, ossia la percentuale di errore commessa,
        rimane costante in tutto l'asse rappresentato.

        %Fare standard IEEE754
\end{description}

%Fare le operazioni
\section{Rappresentazione dei caratteri e stringhe}
I caratteri e le stringhe, sequenza di caratteri, vengono rappresentati nei calcolatori
come numeri e attraverso una serie di convenzioni sono trasformati nel carattere desiderato.\newline
Il primo standard per rappresentare i caratteri è l'ASCII(cerca nome completo),
presente all'inizio a 7bit, in cui si riesce soltanto a convertire le lettere inglesi,
i numeri e i caratteri speciali; successivamente è stato introdotto l'ASCII a 8bit,
che causò dei problemi di comunicazione in quanto i caratteri aggiuntivi all'ASCII normale
dipendevano e cambiavano in base alla nazione del computer.
Per permettere di rappresentare un numero maggiore di caratteri e lingue è stato introdotta
la codifica UNICODE, utilizzata ad esempio in Java, che permette la gestione dei caratteri
delle principali lingue e vi sono diverse versioni di UNICODE come ad esempio vi è
un unicode per rappresentare le emoji.

%Mettere immagine ASCII e UNICODE
