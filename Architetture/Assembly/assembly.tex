%Capitolo sulla Programmazione Assembly
\chapter{Le istruzioni del Computer e il linguaggio Assembly}
Tutti i calcolatori, dai più recenti e prestazionali ai più vetusti ed opsolescenti,
eseguono ed effettuano le loro attività interpretando soltanto il linguaggio macchina,
composto solo da sequenze di numeri binari.\newline
Per permettere lo sviluppo dei programmi sempre più complessi ed agevolare il lavoro
dei programmatori è stato introdotto il linguaggio Assembly, simbolica rappresentazione
delle istruzioni in linguaggio macchina, il quale permette di utilizzare delle label
per identificare particolari celle di memoria che tengono informazioni o dati.

Nonostante il miglioramento rispetto al linguaggio macchina, l'Assembly non è ancora
sufficiente facile da essere utilizzato dai programmatori per cui nel corso degli anni
sono stati introdotti dei linguaggi ad alto livello, come ad esempio il C,
per migliorare la produttività, la chiarezza e la leggibilità, in cui le istruzioni
somigliano all'inglese scritto rispetto a sequenze interminabili di bit; i programmi scritti
in linguaggi ad alto livello vengono poi convertiti in Assembly dal compilatore,
come verrà visto nel paragrafo sulla catena programmativa.

I vantaggi del linguaggio Assembly rispetto ai linguaggi ad alto livello risiedono
nel poter ottenere una maggiore ottimizzazione e migliori prestazioni, a discapito però
della portabilità, in quanto il linguaggio assembly è dipendente dalla macchina,
e della produttività, dato che per ottenere un risultato in assembly bisogna scrivere
un maggior numero di codice rispetto ai linguaggi ad alto livello.

Noi nel corso Architettura degli Elaboratori utilizziamo una macchina MIPS, già
analizzata in un precedente capitolo, in particolare una macchina MIPS a 32 bit.\newline
Per poter testare e verificare il funzionamento dei programmi Assembly utilizziamo
due simulatori:Qtspim, tool che permette soltanto l'esecuzione di un programma Assembly,
mentre Mars è un ide per assembly che permette di scrivere,eseguire ed analizzare l'effetto di ogni operazione.\newline
I programmi Assembly, vengono caricati in memoria per venire eseguiti per cui ogni istruzione
ha un proprio indirizzo in memoria.

Un programma scritto in linguaggio Assembly MIPS contiene un segmento \emph{data},
in cui è presente la rappresentazione binaria dei dati utilizzati nel programma,
e un segmento \emph{text}, in cui sono presenti le istruzioni del programma in cui
deve essere il label \emph{main} per indicare l'inizio del programma.

Nel linguaggio MIPS sono presenti 3 tipologie di istruzioni, tutte da 32 bit:
\begin{description}
  \item[R type] sono le istruzioni di tipo aritmetico logico che prevede la seguente distribuzione:\newline
        \begin{tabular}{|c|c|c|c|c|c|}
          \hline
           op(6) & rs(5) & rt(5) & rd(5) & shamt(5) & funct(6)\\
           \hline
        \end{tabular}\newline
        I registri $rs$ e $rt$ rappresentano i due operandi mentre il registro $rd$
        indica il registro dove memorizzare il risultato;il campo $shamt$ indica lo shift
        effettuato, campo usato solo nelle istruzioni di shift.\newline
        Il campo $op$ e $funct$ specificano l'operazione base dell'istruzione e la variante
        dell'operazione rispetto all'opcode, specificato da $op$.
  \item[I type] sono utilizzate per rappresentare le operazioni di lettura e scrittura
        in memoria, delle operazioni immediate e delle istruzioni di branch.
        La distribuzione della istruzione nei 32 bit è la seguente:\newline
        \begin{tabular}{|c|c|c|c|}
          \hline
           op(6) & rs(5) & rt(5) & constant or address(16)\\
           \hline
        \end{tabular}\newline
        Il campo $constant or address$ permette di specificare un numero costante
        o un indirizzo di memoria.
  \item[J type] è utilizzata per rappresentare le istruzioni di Jump e prevede la seguente codifica:
        \begin{tabular}{|c|c|}
          \hline
          op(6) & address(26)\\
          \hline
        \end{tabular}
\end{description}

\section{Istruzioni aritmetiche, logiche e di accesso alla memoria}
Nel Mips32 le istruzioni aritmetiche prevedono l'utilizzo di solo 3 registri
in cui due indicano gli operandi dell'operazione e un registro è utilizzato per
salvare il risultato;nel linguaggio assembly del MIPS32 si possono soltanto utilizzare
i valori presenti nei registri per effettuare le operazioni aritmetiche.
Le istruzioni aritmetiche, come anche quelle logiche, sono operazioni di tipo R
e le più utilizzate sono le seguenti:
\begin{description}
  \item[add rd,rs,rt] effettua la somma tra i registri rs e rt e la salva nel registro rd.
  \item[sub rd,rs,rt] effettua la sottrazione tra i registri rs e rt e la salva nel registro rd.
  \item[mul rd,rs,rt] effettua la moltiplicazione tra i registri rs e rt e la salva in rd.
  \item[div rd,rs,rt] effettua la divisione tra i due registri rs e rt e lo salva nel registro rd
  \item[sll rd,rt,shamt] sposta a sinistra di n bit(shamt) il registro rt e lo salva in rd;rappresenta la moltiplicazione $2 ^ n$.
  \item[srl rd,rt,shamt] sposta a destra di n bit(shamt) il registro rt e lo salva in rd e rappresenta la divisione per $2^n$.
  \item[and rd,rs,rt] effettua l'and logico bit a bit tra i registri rs e rt e lo salva nel registro rd.
  \item[or rd,rs,rt] effettua l'or logico bit a bit tra i registri rs e rt e lo salva nel registro rd.
  \item[nor rd,rs,rt] effettua il nor bit a bit tra i registri rs e rt;per implementare il $not(A)$ si effettua $A \ NOR 0$.
\end{description}

%Inserire esempi istruzioni Aritmetiche
\lstinputlisting{Assembly/somma.asm}
La syscall utilizzata alla fine viene utilizzata per effettuare operazioni di input-
output nel simulatore e queste istruzioni verranno introdotte nel paragrafo sulle procedure.

In una macchina MIPS a 32 bit, come già accennato, sono presenti soltanto 32 registri
i quali non permettono l'utilizzo di dati più complessi di quelli primitivi, come i numeri e i caratteri,
per cui per ovviare si utilizza la memoria per salvare e memorizzare i dati complessi, come array e strutture.
Per permettere di utilizzare e scrivere i dati in memoria per eseguire le operazioni aritmetiche
il MIPS32 fornisce le istruzioni di lettura e scrittura chiamate \emph{load} e \emph{store}.

%Inserire grafico organizzazione memoria!!!!

Per caricare dalla memoria una word, ossia un dato a 32 bit, si utilizza l'istruzione $lw$
il quale ha la seguente sintassi: $lw \ rt,offset(rs)$
L'istruzione $lw$ è allineata a 4 bytes per cui cercare di accedere ad un indirizzo
non allineato a 4bytes genera un eccezione mentre l'indirizzo da cui leggere i dati
è determinato da $address = offset + rs$.

L'istruzione di salvataggio in memoria di una word è l'istruzione $sw$ che ha la
stessa struttura e sintassi dell'istruzione $lw$.
In caso si abbia la necessità di caricare e/o salvare dati in memoria più piccoli
o grandi di una word ci sono le relative istruzioni per bytes,halfword e double word
da vedere nella documentazione MIPS disponibile nell'appendice(INSERIRE RIFERIMENTO AD APPENDICE).

Esempio:\newline
\lstinputlisting{Assembly/massimo.asm}
I registri a differenza dei valori in memoria permettono di usare le istruzioni Aritmetiche
e logiche, per cui andrebbero utilizzati i registri per le variabili utilizzate spesso
in un programma per rendere l'esecuzione del programma il più veloce possibile e
con minor consumo di energia.

Un istruzione similare a quella di caricamento e salvataggio in memoria, ossia hanno
la stessa tipologia di codifica la I-type, è l'addizione immediata, fornita dall'istruzione $addi \ rt,rs,constant$.\newline
Vi sono le versioni immediate di tutte le operazioni aritmetiche e logiche, tranne il nor.

\section{Istruzioni di controllo e di ripetizione}
A differenza dei normali calcolatori, i computer sono in grado di prendere delle decisioni,
ossia la capacità di eseguire differenti istruzioni in base a determinati valori dati in input.\newline
Questa capacità di prendere decisioni nei linguaggi ad alto livello viene rappresentata
dalle istruzioni $if-else$ mentre MIPS prevede due istruzioni per fare tutto ciò:
\begin{description}
  \item[beq register1,register2,L1] questa istruzione, chiamata \emph{branch if equals} punta all'istruzione nel L1
        in caso i due registri siano uguali.
  \item[bne register1,register2,L1] l'istruzione, chiamata \emph{branch if not equal} punta al label L1 in caso i due
        registri abbiano valori diversi.
\end{description}
Questa due istruzioni branch sono chiamate \emph{salti condizionali}, istruzioni
che permettono di alterare il flusso di esecuzione in base al soddisfamento di una data condizione.

Esempio: Scrivere un programma MIPS Assembly che dice se due stringhe sono uguali


Le istruzioni di ripetizioni permettono la ripetizione di una serie di istruzioni
fino al soddisfamento di una serie di condizioni;nei linguaggi ad alto livello viene
effettuata tramite le istruzioni while,do-while e for mentre nel MIPS viene implementato
nel seguente modo, con ad esempio un programma che determina la dimensione di una stringa.
%Inserire codice dimensione di una stringa
\lstinputlisting{Assembly/conta.asm}

Le istruzioni dentro il label $loop$ sono chiamate \emph{blocco di istruzioni} ossia
istruzioni in cui non compare alcun branch, eccetto eventualmente alla fine, e
senza alcun label di branch, eccetto eventualmente all'inizio.
Il test di uguaglianza e disuguaglianza sono i test più popolari ma a volte si ha
bisogno di sapere se una variabile è minore di un altra e ciò nel MIPS viene verificato
tramite \emph{slt}(set less than) in cui è memorizzato 1 se il primo registro è minore
dell'altro; dell'istruzione slt esiste la variante immediate ed unsigned per permettere
il confronto immediato con una costante e il confronto tra due numeri positivi.

%Paragrafo sulle procedure
\section{Procedure in linguaggio Assembly}
Le procedure sono una sequenza di istruzioni con un nome che risolvono un determinato compito
e possono essere riutilizzate e servono per implementare l'astrazione e la mantenibilità.
Le procedure implementano il concetto matematico di funzione e possono avere dei parametri
formali e possono ritornare o meno dei valori alla parte di programma che lo chiama.

Le procedure, qualsiasi operazione svolgano, hanno le seguente esecuzione:
\begin{enumerate}
  \item Posizionano i parametri in maniera tale che si possa accedervi.
  \item Sposta il controllo del programma alla procedura.
  \item Si ricavano le risorse di memoria necessarie per l'esecuzione della procedura.
  \item si esegue il compito desiderato della procedura.
  \item si salva il valore di ritorno in una posizione accessibile dal chiamante.
  \item sposta il controllo del programma al punto in cui è stata chiamata la procedura.
\end{enumerate}.
In quanto i registri sono il posto migliore per salvare i dati, il MIPS ha previsto
dei registri appositi per le procedure all'interno dei 32 disponibili:
\begin{enumerate}
  \item \$a0-\$a3: 4 registri per salvare 4 argomenti passati come parametri alla procedura.
  \item \$v0-\$v1:2 registri per tenere 2 valori di ritorno della procedura
  \item \$ra: registro per tenere l'indirizzo del punto in cui viene chiamata la procedura
\end{enumerate}
Il linguaggio assembly del MIPS prevede un istruzione chiamata $jal ProcedureAddress$,
per puntare ad un indirizzo e salvare simultaneamente l'indirizzo dell'istruzione successiva nel registro \$ra
e ovviamente codesta istruzione è utile ed usata per chiamare una procedura.

Per effettuare il ritorno al punto di origine e terminare l'esecuzione della procedura
il linguaggio assembly del MIPS prevede l'istruzione $jr register$ per effettuare
il salto al contenuto del registro; nel caso di ritorno da una procedura si usa $jr \$ra$.

In caso si abbiano bisogno di più di 4 registri per gli argomenti e/o più di due registri
per i valori di ritorno si utilizza lo stack per salvare i registri aggiuntivi necessari
e lo stack prevede un puntatore all'ultimo elemento inserito, a cui il MIPS ha previsto
un registro, chiamato \$sp, per memorizzare l'indirizzo del puntatore dello stack.
Lo stack è una struttura dati LIFO(Last in First Out) in cui l'inserimento, chiamato \emph{push},
nel MIPS consiste nel diminuire il valore dello stack pointer in quanto lo stack cresce dall'alto verso il basso,
mentre la rimozione, chiamata \emph{pop}, avviene aumentando il valore dello stack pointer.

%Inserire esempio di procedura in linguaggio Assembly

Per evitare di salvare e ripristinare i registri i cui i valori possono non venire
più usati, come ad esempio i registri temporanei, il linguaggio MIPS separa i registri in due gruppi:
\begin{enumerate}
  \item i registri non preservati dalla procedura come ad esempio i registri temporanei \$t0-\$t9
  \item i registri preservati dalle procedure(in caso la procedura li usa devono essere ripristinati)
        come ad esempio i registri \$s0-\$s7
\end{enumerate}

\section{Procedure annidate}
Le procedure che fino ad ora abbiamo visto sono delle procedure foglie, ossia procedure
che non chiamano altre, però alcune volte per poter svolgere la propria attività
una procedura ne chiama un altra e ciò viene definita procedura annidata; un particolare
tipo di procedura annidata è la ricorsione, in cui una procedura chiama se stessa.

Con le procedure annidate bisogna stare attenti ai registri infatti gli argomenti
\$a0-\$a4 e il registro \$ra creano conflitto tra le diverse procedure per cui bisogna
salvarli nello stack per poter mantenere i risultati dopo la chiamata alla procedura ausiliaria,
come si può notare nell'esempio:
%Inserire esempio Procedura annidata

Nello stack sono anche presenti le variabili locali non rappresentate da dei registri,
come gli array locali o le strutture e il segmento dello stack contenente
i registri salvati dalla procedura e le variabili locali si chiama \emph{record di attivazione}.
Alcuni software MIPS usano il registro \$fp(frame pointer), il quale punta alla prima
parola del record della procedura ed è utile rispetto allo stack pointer in quanto
non cambia durante la procedura per cui il riferimento ad una variabile ha lo stesso offset,
rendendo più leggibile e mantenibile il codice assembly della procedura.
%Inserire esempio dell'Assembly
