%Capitolo sul Datapath nel MIPS32
\chapter{Datapath}
Dopo aver analizzato le istruzioni e come implementare un programma in linguaggio assembly,
consideriamo ora il processore e il datapath delle istruzioni, ossia il percorso
utilizzato per l'esecuzione delle istruzioni nel computer.
Noi consideriamo per sviluppare il datapath una subset delle istruzioni MIPS in grado di effettuare
solo le seguenti istruzioni:
\begin{itemize}
  \item le istruzioni di accesso alla memoria load word(\emph{lw}) e store word(\emph{sw}).
  \item le istruzioni aritmetico-logiche \emph{add,sub,AND,OR} e \emph{slt}.
  \item le istruzioni branch equal(\emph{beq}) e and jump(\emph{j})
\end{itemize}
Le altre operazioni, come ad esempio l'istruzione \emph{multiply}, operano in maniera
similare e possono essere facilmente implementate effettuando delle piccole modifiche
al circuito del datapath.

\section{Fasi di un istruzione}
Ogni istruzione, qualsiasi essa sia, ha 3 fasi per svolgere il suo operato:
\begin{enumerate}
  \item \textbf{fetch} in cui tramite il \emph{program counter} viene prelevata l'
                       istruzione dalla memoria.Al termine della fase di fetch
                       viene aggiornato il PC, tramite PC = PC + 4, per far puntare all'istruzione successiva.
  \item \textbf{decode} in cui viene decodificata l'istruzione e predisposti i valori
                        necessari all'esecuzione dell'istruzione.
  \item \textbf{execute} in cui viene eseguita l'istruzione prevista dal programma.
\end{enumerate}
Le prime due fasi sono uguali per tutte le istruzioni mentre la fase di esecuzione,
come si poteva facilmente immaginare, varia in base all'istruzione ma comunque le
istruzioni delle 3 tipologie sono similari nell'esecuzioni così viene resa più veloce
ed efficente l'esecuzione e l'implementazione del datapath.

%Fase di fetch grafico


%Fase di decode grafico

La fase di esecuzione cambia in base al tipo di istruzione per cui verrà vista nel
prossimo paragrafo dove andrà analizzato il datapath delle istruzioni.
