%Paragrafo sulle reti Logiche(AND OR NOT ed ecc...
\chapter{Reti Logiche}
Per poter capire ed analizzare le componentistiche di un calcolatore elettronico
bisogna prima considerare le porte logiche attraverso cui sono costruiti tutti i
circuiti presenti nei calcolatori.

L'elettronica utilizzata nei calcolatori è \emph{digitale}, ossia opera con due valori
logici \emph{low} and \emph{high voltage} e gli altri valori sono temporanei e si
verificano nella transizione tra i due valori, per questo nei calcolatori si utilizza
la rappresentazione binaria in cui il valore \emph{high} rappresenta un valore vero(1)
mentre il valore \emph{low} indica il valore falso(0).

Le porte logiche attraverso cui vengono costruiti tutti i circuiti utilizzati nei calcolatori
si dividono in due categorie:\emph{reti combinatorie}, circuiti non contenenti memoria
e i cui valori ottenuti in output dipendono soltanto dei valori in input, e \emph{reti sequenziali},
circuiti in cui si ha la presenza di memoria e i valori in output dipendono dai valori
presenti in memoria e dai valori in input.


\section{Reti combinatorie}
Le reti combinatorie, come viste nell'introduzione del capitolo, sono circuiti "smemorati"
ossia i valori in output dipendono soltanto dai valori in input e la rappresentazione
dei valori assunti dal circuiti può essere fatta in due modi:tavole di verità e algebra booleana,
analizzati e studiati nel corso "Fondamenti dell'Informatica".

L'elemento base secondo cui vengono costruite tutti i circuiti sono le porte logiche base,
in cui viene implementato e rappresentato le funzioni logiche base:\emph{AND},\emph{OR} e \emph{NOT}.
%Disegno porte logiche AND OR e NOT

È prassi comune aggiungere delle bolle per invertire un dato invece di disegnare un invertitore
come si nota nel seguente circuito:
%Circuito NOT(NOT(A) OR B)

Tutte le porte logiche possono essere rappresentate solo utlizzando un tipo di porta,
prassi utilizzata nella costruzione di tutti i calcolatori, ossia le porte \emph{NOR} e/o \emph{NAND}
in cui si invertono le porte AND e/o OR e sfruttando le proprietà dell'algebra booleana
si rappresentano anche gli altri circuiti logici.

I circuti combinatori utilizzati ed analizzati sono i seguenti:
\begin{itemize}
  \item \textbf{Decoder}: circuito logico in cui viene tradotto i $n$ bit di input
                        in un segnale indicante il valore binario dei bit forniti.
                        %DISEGNO DECODER
  \item \textbf{Multiplex}:circuito in cui, dato un input, viene restituito come output
        l'elemento in input scelto da un "controllore" e viene utilizzato
        per scegliere l'elemento da utilizzare in base a dei valori di selezione.
        %DISEGNO MULTIPLEX
  \item \textbf{PLA(Programmable Logic Array)} è il circuito per rappresentare le funzioni logiche
        a due livelli(somma-prodotto/prodotto-somma) e possono essere comprate e
        venire "programmate" per eseguire e rappresentare nuovi circuiti.
        %IMMAGINE FUNZIONAMENTO E STRUTTURA PLA
  \item ROM(Read Only Memory): circuito combinatorio in cui viene implementato la tavola
        di verità di una funzione logica ed è totalmente costruita tramite i decoder.
        Il suo nome presenta la parola \emph{memory} in quanto è anche una memoria
        i cui valori sono configurati e inseriti soltanto alla costruzione del circuito.\newline
        Vi sono delle evoluzioni delle ROM che permettono la programmabilità e la
        modifica dei valori presenti nella ROM ma noi non li analizziamo.
  \item ALU(Arithmetic Logic Unit): è un circuito fondamentale nei calcolatori in cui
        si implementano le operazioni logiche eseguite da un calcolatore;in quasi
        tutte le istruzioni dei programmi viene utilizzata come sarà poi visto nel capitolo
        relativo al datapath.\newline
        Essendo un circuito fondamentale e complesso si utilizza un sottoparagrafo
        per essere il più chiari e compatti possibile.
\end{itemize}

\subsection{Arithmetic Logic Unit}
L'ALU è un circuito deputato ad eseguire, come si evince dal nome, le operazioni
aritmetiche logiche fondamentali nel percorso dati delle istruzioni dei programmi.\newline
La dimensione dei valori input e output nell'ALU dipende dalla macchina: nel MIPS32
si necessita di un ALU a $32$ bit però per comprendere l'ALU iniziamo prima a presentare
l'ALU a $1$ bit.

Le operazioni logiche nell'ALU sono facilmente implementabili in quanto si usano le
porte logiche elementari $AND$ e $OR$;nel MIPS è stato deciso di non implementare
il NOT ma si usa il NOR, il cui circuito verrà visto poi nel proseguo dell'analisi sull'ALU.

Un'altra operazione importante da implementare in una ALU è il circuito per effettuare
l'addizione, la cui l'implementazione come circuito non la facciamo in quanto si può
facilmente effettuare utilizzando la tavola di verità e le equazioni logiche.\newline
L'addizione ha 3 ingressi in input, i due addendi e il carryIn(riporto in ingresso),
e due ingressi in Output, il risultato e il carryOut(riporto in uscita).

Per scegliere l'operazione da svolgere si usa un multiplex come si nota nel seguente circuito:
%CIRCUITO OP.LOGICHE ed ADDER

La sottrazione tra due numeri può essere anch'essa facilmente implementabile considerando
che $a-b = a + (-b) = a + (b^_ + 1) = a + b^_ + 1$ e per questa ragione si utilizza
il complemento a due nella rappresentazione dei numeri binari negativi nel hardware di tutti i calcolatori.

Il MIPS ha deciso di fornire la funzione $NOR$ al posto della NOT, che può essere
implementata sfruttando le altre operazioni considerando il teorema di DeMorgan
$(a + b) = a * b$;il circuito aggiungendo la NOR e la sottrazione è la seguente:
%CIRCUITO ALU A 1 BIT


Nei calcolatori attuali si usano ALU a 32/64 bit 
\section{Reti sequenziali}
