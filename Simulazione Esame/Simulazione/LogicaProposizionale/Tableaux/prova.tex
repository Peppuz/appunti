%Esercizio Tableaux Logica Proposizionale Compitino di Prova 2017/2018
8)
\begin{equation*}
\begin{prooftree}
\hypo{F ((P \rightarrow R) \land (Q \rightarrow R)) \rightarrow (P \lor Q)}
\infer1{T (P \rightarrow R) \land (Q \rightarrow R),F P \lor Q}
\infer1{T (P \rightarrow R) \land (Q \rightarrow R),FP,FQ}
\infer1{T P \rightarrow R,T Q \rightarrow R,FP,FQ}
\infer1{FP,FP,FQ,T Q \rightarrow R/TR,T Q \rightarrow R,FP,FQ}
\end{prooftree}
\end{equation*}
Il tableaux refutazione non può mai chiudere per cui per stabilire la tipologia
della funzione bisogna effettuare il T-Tableaux:
\begin{equation*}
\begin{prooftree}
\hypo{T ((P \rightarrow R) \land (Q \rightarrow R)) \rightarrow (P \lor Q)}
\infer1{F (P \rightarrow R) \land (Q \rightarrow R)/T P \lor Q}
\infer1{F (P \rightarrow R) \land (Q \rightarrow R)/TP/TQ}
\end{prooftree}
\end{equation*}
Il T-Tableaux non può mai chiudere per cui la formula è soddisfacibile non tautologica.
