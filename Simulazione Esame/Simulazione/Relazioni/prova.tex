%Esercizio sulle Relazioni compitino di Prova 2017/2018
2)
\begin{equation*}
\begin{split}
R \subset A \times B & = \{(0,0),(1,2),(2,4)\} \\
R ^{-1} \subset B \times A & = \{(0,0),(2,1),(4,2)\} \\
\end{split}
\end{equation*}

%Grafo bipartito relazione R^{-1}
\begin{tikzpicture}
\GraphInit[vstyle = normal]
\tikzset{EdgeStyle/.style = {->,bend right=30}}
\Vertex[label = -2,x = 0,y = 0]{a1}
\Vertex[label = -1,x = 0,y = -2]{a2}
\Vertex[label = 0,x = 0,y = -4]{a3}
\Vertex[label = 1,x = 0,y = -6]{a4}
\Vertex[label = 2,x = 0,y = -8]{a5}
\Vertex[label = 0,x = 3,y = 0]{b1}
\Vertex[label = 1,x = 3,y = -2]{b2}
\Vertex[label = 2,x = 3,y = -4]{b3}
\Vertex[label = 3,x = 3,y = -6]{b4}
\Vertex[label = 4,x = 3,y = -8]{b5}
\Vertex[label = 5,x = 3,y = -10]{b6}
\Edge(b1,a3)
\Edge(b3,a4)
\Edge(b5,a5)
\end{tikzpicture}

$R$ e $R^{-1}$ sono entrambi delle funzioni
