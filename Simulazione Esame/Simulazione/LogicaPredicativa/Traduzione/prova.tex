%Esercizio Traduzione Linguaggio Naturale Compitino di Prova 2017/2018
9)
Frase:Ogni treno ha un numero identificativo

Costanti:non presenti \newline
Predicati:$Treno(x)$,$Avere(x)$\newline
Funzioni:$id(x)$
\begin{equation*}
    \forall x (Treno(x) \rightarrow Avere(id(x)))
\end{equation*}

Il treno IC714 parte da Milano Centrale

Costanti:$IC714$,$Milano Centrale$ \newline
Predicati:$Treno(x)$,$Partenza(x,y$)\newline
Funzioni:non presenti
\begin{equation*}
Treno(IC714) \rightarrow Partenza(IC714,Milano Centrale)
\end{equation*}

Qualche treno che parte da Milano Centrale in direzione Roma è in ritardo

Costanti:$Milano Centrale$,$Roma$ \newline
Predicati:$Treno(x)$,$Partenza(x,y)$,$Arrivo(x,y)$,$Ritardo(x)$ \newline
Funzioni:non presenti
\begin{equation*}
\exists x (Treno(x) \land Partenza(x,Milano Centrale) \land Arrivo(x,Roma) \rightarrow Ritardo(x))    
\end{equation*}
