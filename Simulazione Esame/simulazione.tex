%Appunti in Latex per gli esercizi di Fondamenti dell'Informatica
\documentclass[a4paper,11px]{report}
\usepackage[T1]{fontenc}
\usepackage[utf8]{inputenc}
\usepackage[italian]{babel}
\usepackage{amsmath}%Package per la matematica
\usepackage{amsthm}%Package per i teoremi matematici
\usepackage{amsfonts}%Package per i font matematici
\usepackage{ebproof}%Package per fare i Tableaux
\usepackage{tkz-graph}%Package per disegnare i grafi
\usepackage{forest}%Package per disegnare gli alberi
\setlength{\parindent}{0pt}%Toglie rientro dai Capoversi
\newtheorem{defi}{Definizione}%Definizione per avere la gestione delle definizioni
\newtheorem{prop}{Proposizione}[chapter]
\newtheorem{thm}{Teorema}[chapter]
\newcommand{\numberset}{\mathbb}
\newcommand{\N}{\numberset{N}}
\newcommand{\Z}{\numberset{Z}}
\newcommand{\Q}{\numberset{Q}}
\newcommand{\R}{\numberset{R}}

\begin{document}

%Esercizio Induzione Compitino di Prova 2017/2018
6)
\begin{thm}
    $\displaystyle n! \geq 2^{(n-1)} \quad \forall n \in \N$
\end{thm}
%Dimostrazione
\begin{proof}
Per $n = 0$ si ha $0! \geq 2^{0-1}$ ossia $ 1 \geq 1/2$ che è sempre verificato

Se $n! \geq 2^{(n-1)}$ allora $(n+1)! \geq 2^n$
\begin{equation*}
\begin{split}
(n+1)! \geq 2^{n+1-1} & = n! (n+1) \geq 2^{n-1} * 2 \\
                      & = n! \frac{(n+1)}{2} \geq 2^{n-1} \\
\end{split}
\end{equation*}
Essendo $\frac{(n+1)}{2} \geq 0 \forall n \in \N$ e $n! \geq 2^{n-1}$ verificato per ipotesi
si ricava che la proposizione è verificata.
\end{proof}
%Compitino di Prova
%Compitino di Gennaio 2018

%Inserire immagine prova

%Esercizio sugli insiemi Compito Gennaio 2018
1)
%Insiemi
%Esercizio Funzioni compito Gennaio 2018
%Relazioni
%Esercizio Relazioni compito di Gennaio 2018
3)
a)
%Fare le matrici booleane

b)
$R_1$ coincide con $R$ in quanto $R$ è già riflessiva.
%Funzioni
%Esercizio Alberi Compito Gennaio 2018
%Albero
%Esercizio Poset compito Gennaio 2018
%Poset e Reticolo
%Esercizio Induzione compito Gennaio 2018
%Induzione
%Esercizio Tavole di Verità compito  Gennaio 2018
%Tavole di Verità
%Esercizio Tableaux Proposizionali compito Gennaio 2018
%Tableaux proposizionali
%Esercizio Traduzione compito Gennaio 2018
%Traduzione linguaggio Naturale
%Esercizio Tableaux Predicativi compito Gennaio 2018
%Tableaux Predicativi
%Compitino Gennaio 2018

\end{document}
