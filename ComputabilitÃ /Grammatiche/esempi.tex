\begin{esempio}
$L=\{a^n c b^{n-1}|n\geq 2\}$, con $a^n c b^{n-1}=a^{n-1}acb^{n-1}$. $S\to aSb|aacb$. Quindi:
$$S\to aSb\to aaaccbb\in L$$
\end{esempio}
\begin{esempio}
cerco CFG per $L=\{a^n c^k b^n|\,n,\,k>0\}$. $a$ e $b$ devono essere uguali, uso quindi una grammatica context free, mentre $c$ genera un linguaggio regolare.\\
Si ha la grammatica $G=\{V,T,P,S)$, $V=\{S,C\}$, $T=\{a,b,c\}$ e si hanno $S\to aSb|aCb$ e $C\to cC|c$. dimostro che $aaaccbbb\in L, n=3,\, k=2$:
$$S\to aSb \to aaSbb\to aaaCbbb\to aaacCbbb\to aaaccbbb$$
\end{esempio}

\begin{esempio}
scrivere CFG per $L=\{a^nb^nc^kb^k|\, n,\,k\geq 0\}
$
$$=\{w\in\{a,b,c,d\}^*|\,a^nb^nc^kb^k|\, n,\,k\geq 0\}$$
quindi L concatena due linguaggi $L1$ e $L2$, $X=\{a^nb^n\}$ e $Y=\{c^kd^k\}$: 
$$X\to aXb | \varepsilon$$
$$Y\to cYd | \varepsilon$$
$$S\to XY$$
voglio derivare $abcd$:
$$S\to XY \to XcYd\to aXbcYd\to aXbc\varepsilon d\to a\varepsilon bc\varepsilon d\to abcd$$
voglio derivare $cd$
$$S\to XY\to Y\to cYd\to cd$$
\end{esempio}
Quindi se ho $w\in L1, L2$, ovvero appartenente ad una concatenazione di linguaggi prima uso le regole di un linguaggio, poi dell'altro e infine ottengo il risultato finale.\\
\begin{esempio}
scrivere CFG per $L=\{a^nb^kc^kd^n|\, n>0,\, k\geq 0\}
$.
$$S\to aSd|\, aXd$$
$$X\to bXc| \varepsilon$$
derivo $aabcdd$:
$$S\to aSd\to aaXdd\to aabXcdd\to aabcdd$$
\end{esempio}
\begin{esempio}
scrivere CFG per $L=\{a^ncb^nc^mad^m|\, n>0,\, m\geq 1\}
$.
$$S\to XY$$
$$X\to aXb|c$$
$$Y\to cUd| cad$$
$$S\to XY\to cY\to ccad$$
\end{esempio}
\begin{esempio}
scrivere CFG per $L=\{a^{n+m}xc^nyd^m|\, n,\, m\geq 0\}
$. $a^{n+m}=a^na^m \mbox{ o } a^ma^n$. Si hanno 2 casi:
\begin{enumerate}
\item $L=\{a^na^m xc^nyd^m|\, n,\, m\geq 0\}
$
\item $L=\{a^ma^n xc^nyd^m|\, n,\, m\geq 0\}
$
\end{enumerate}
ma solo  $L=\{a^ma^n xc^nyd^m|\, n,\, m\geq 0\}
$ può generare una CFG (dove non si possono fare incroci, solo concatenazioni e inclusioni/innesti). 
$$S\to aSd| Y$$
$$Y\to Xy$$
$$X\to aXc|x$$ 
si può fare in 2:
$$S\to aSd| Xy$$
$$X\to aXc|x$$ 
derivo con $m=n=1$, $aaxcyd$:
$$S\to aSd\to aXyd\to aaXcyd\to aaxcyd$$
\end{esempio}
\begin{esempio}
scrivere CFG per $L=\{a^nb^m|\, n\geq m \geq 0\}
$.$$L=\{\varepsilon, a, ab, aa, aab, aabb, aaa, aaab, aaabb, aaabbb,...\}$$
Se $n\geq m$ allora $\exists k\geq 0 \to n=m+k$. Quindi:
$$l=\{a^{m+k}b^m|m,k\geq0\}$$ si può scrivere in 2 modi:
\begin{enumerate}
\item $l=\{a^ma^kb^m|m,k\geq0\}$ quindi con innesto
\item $l=\{a^ka^mb^m|m,k\geq0\}$quindi con concatenazione
\end{enumerate}
entrambi possibili per una CFG:
\begin{enumerate}
\item 
$$S\to XY$$
$$X\to aX|\varepsilon \mbox{ si può anche scrivere } X\to Xa|\varepsilon$$
$$Y\to aYb|\varepsilon$$ 
oppure 
$$S\to aS|X$$
$$X\to aXb| \varepsilon$$
\item 
$$S\to aSb|\varepsilon$$
$$X\to aX|\varepsilon$$
\end{enumerate}
\end{esempio}
\begin{esempio}
scrivere CFG per $L=\{a^nb^{m+n}c^h|\, m>h\geq0,\, n\geq0\}
$.\\
Se $n>h$ allora $\exists k \to n= h+k$, quindi:
$$L=\{a^nb^{m+h+k}c^h|\, m>h\geq0,\, n\geq0\}$$. ovvero:
$$L=\{a^nb^nb^kb^hc^h|\, m\geq 0, k>0, h\geq 0\}$$
si ha:
$$S\to XYZ$$
$$X\to aXb|\varepsilon$$
$$Y\to Yb|b$$
$$Z\to bZc|\varepsilon$$
si può anche fare:
$$S\to XY$$
$$X\to aXb|\varepsilon$$
$$Y\to bYc|Z$$
$$Z\to bZ|b$$
\end{esempio}
\begin{esempio}
scrivere CFG per $L=\{a^nb^mc^k|\, k>n+m,\, n,m\geq 0\}
$.\\
per $n=m=0,\, k=1$ avrò la stringa $c$.
se $k>n+m$ allora $\exists l>0\to k=n+m+l$ quindi:
$$L=\{a^nb^mc^{n+m+l}|\, l>0,\, n,m\geq 0\}
$$
$$=L=\{a^nb^mc^nc^mc^l|\, l>0,\, n,m\geq 0\}$$
sistemando:
$$=L=\{a^nb^mc^lc^mcnl|\, l>0,\, n,m\geq 0\}$$
quindi:
$$S\to aSc|X$$
$$X\to bXc|Y$$
$$Y\to cY|c$$
\end{esempio}
\newpage
\begin{esempio}
scrivere CFG per $L=\{a^nxc^{n+m}y^hz^kd^{m+h}|\, n,m,k,h\geq 0\}
$.\\
ovvero:
$$L=\{a^nxc^nc^my^hz^kd^hd^m|\, n,m,k,h\geq 0\}$$
quindi avrò:
$$S\to XY$$
$$X\to aXc|x$$
$$Y\to cYd|W$$
$$W\to yWd|X$$
$$Z\to zZ|\varepsilon$$
\end{esempio}
\begin{esempio}
Sia $L=\{w\in\{a,b\}^*|\, \mbox{ w contiene lo stesso numero di a e b}\}$:
\[ S\to aSbS|\,bSaS|\, \epsilon \]
dimostro per induzione che è corretto:
\begin{itemize}
\item \textbf{caso base:} $|w|=0\to w=\varepsilon$
\item \textbf{caso passo:} si supponga che $G$ produca tutte le stringhe (di lunghezza $<$ di $n$) di $\{a,b\}^*$ con lo stesso numero di \textit{a} e \textit{b} e dimostro che produce anche quelle di lunghezza $n$, sia:
$$w\in \{a,b\}^* \mid\, |w|=n \mbox{ con\textit{ a} e \textit{b} in egual numero, }m(a)=m(b) \mbox{ con m() che indica il numero di caratteri}$$
quindi si ha che:
$$w=aw_1bw_2\mbox{ o } w=bw_1aw_2$$
sia.
$$k_1=m(a)\in w_1=m(b)\in w_1$$
$$k_2=m(a)\in w_2=m(b)\in w_2$$
allora:
$$k_1+k_2+1=m(a)\in w= m(b)\in W$$
sapendo che $|w_1|<n$ e $|w_2|<n$ allora $w_1$ e $w_2$ sono egnerati da G per ipotesi induttiva
\end{itemize}
\end{esempio}
Trovo una grammatica lineare destra e una sinistra per $L=\{ab^ncd^me|\,n\geq 0\,,m> 0\}$:
\begin{itemize}
\item \textbf{lineare a destra:} si ha  si ha $G=(\{S,A,B,E\},\{a,b,c,d,e\},P,S)$ e quindi:
$$S\to aA$$
$$A\to bA|\,cB$$
$$B\to dB|\, dE$$
$$E\to e$$
\item \textbf{lineare a sinistra:} si ha  si ha $G=(\{S,X,Y,Z\},\{a,b,c,d,e\},P,S)$ e quindi:
$$S\to Xe$$
$$A\to Xd|\,Yd$$
$$B\to Zc$$
$$E\to a|\,Zb$$
\end{itemize}
quindi se per esempio ho la stringa "ciao" si ha:
\begin{itemize}
\item \textbf{lineare a destra:}
$$S\to Ao$$
$$A\to Ba$$
$$B\to Ei$$
$$E\to c$$
\item \textbf{lineare a sinistra:}
$$S\to cA$$
$$A\to iB$$
$$B\to aE$$
$$E\to o$$
\end{itemize}
\end{esempio}
\begin{esempio}
A partire da $G=(\{S,T\},\{0,1\},P,S)$ con:
$$S\to\varepsilon|\,0S|\,1T$$
$$T\to 0T|\,1S$$

trovo come è fatto $L(G)$:
$$L(G)=\{w\in\{0,1\}^*|\, w \mbox{ ha un numero di 1 pari}\}$$
\end{esempio}
\begin{esempio}
fornire una grammatica regolare a destra e sinistra per:
$$L=\{w\in\{0,1\}^*|\, w \mbox{ ha almeno uno 0 o almeno un 1}\}$$
Si ah che tutte le stringhe tranne quella vuota ciontengono uno 0 o un 1
quindi  $G=(\{S\},\{0,1\},P,S)$:
\begin{itemize}
\item \textbf{lineare a destra:}
$$S\to 0|\,1|\,0S|\,1S$$
\item \textbf{lineare a sinistra:}
$$S\to 0|\,1|\,S0|\,S1$$
\end{itemize}
\end{esempio}
\begin{esempio}
Trovo automa per: $$L=\{w\in\{a,b\}^*|\mbox{ w che contiene un numero dispari di b}\}$$
\begin{center}
\begin{tikzpicture}[shorten >=1pt,node distance=3cm,on grid,auto]
	\node[state, initial] (q_0) {$q_0$};
	\node[state, accepting] (q_1) [right=of q_0] {$q_1$};
	\path[->]
	(q_0) edge [bend left = 25] node {b} (q_1)
	      edge [loop below] node {a} ()
	(q_1) edge [bend left = 25] node {b} (q_0)
	      edge [loop below] node {a} ();
\end{tikzpicture}
\end{center}
ovvero se da $q_0$ vado a $q_1$ sono obbligato ab generare una sola $b$, dato che il nodo accettnate è $q_1$. In entrambi i nodi posso generare quante $a$ voglio e posso tornare da $q_1$ a $q_0$ per generare altre $b$.
\end{esempio}
\begin{esempio}
Trovo automa per: $$L=\{w\in\{0,1\}^*| w= 0^n1^m\}$$
vediamo i vari casi:
\begin{itemize}
\item $n,m\geq 0$:
\begin{center}
\begin{tikzpicture}[shorten >=1pt,node distance=2cm,on grid,auto]
   \node[state,initial, accepting] (q_0)   {$q_0$};
   \node[state, accepting] (q_1) [right=of q_0] {$q_1$};
   \node[state] (q_E) [right=of q_1] {$q_E$};
   \path[->]
   (q_0) edge  node {1} (q_1)
         edge [loop below] node {0} ()
   (q_1) edge  node  {} (q_E)
         edge [loop below] node {1} ()
   (q_E) edge [loop below] node {0,1} ();
\end{tikzpicture}
\end{center}
ovvero posso non generare nulla e uscire subito con $q_0$, generare solo un 1 e passare a $q_1$ e uscire oppure generare 0 e 1 a piacere con l'ultimo stato o generare 0 a piacere dal primo e 1 a piacere dal secondo.
\item $n\geq 0 \,\,m>0$:
\begin{center}
\begin{tikzpicture}[shorten >=1pt,node distance=2cm,on grid,auto]
   \node[state,initial] (q_0)   {$q_0$};
   \node[state, accepting] (q_1) [right=of q_0] {$q_1$};
   \node[state] (q_E) [right=of q_1] {$q_E$};
   \path[->]
   (q_0) edge  node {1} (q_1)
         edge [loop below] node {0} ()
   (q_1) edge  node  {} (q_E)
         edge [loop below] node {1} ()
   (q_E) edge [loop below] node {0,1} ();
\end{tikzpicture}
\end{center}
ovvero come l'esempio sopra solo che non posso uscire in $q_0$ in quanto almeno un 1 deve essere per forza generato
\item $n> 0\,\, m\geq 0$:
\begin{center}
\begin{tikzpicture}[shorten >=1pt,node distance=2cm,on grid,auto]
   \node[state,initial] (q_0)   {$q_0$};
   \node[state, accepting] (q_1) [right=of q_0] {$q_1$};
   \node[state, accepting] (q_2) [right=of q_1] {$q_2$};
   \node[state] (q_E) [right=of q_2] {$q_E$};
   \path[->]
   (q_0) edge  node {0} (q_1)
         edge [bend right] node {1} (q_E)
   (q_1) edge  node {1} (q_2)
         edge [loop above] node {0} ()
   (q_2) edge node {0} (q_E)
         edge [loop above] node {1} ()
   (q_E) edge [loop below] node {0,1} ();
\end{tikzpicture}
\end{center}
\textit{CHIARIRE}
\item $n,m>0$:
\begin{center}
\begin{tikzpicture}[shorten >=1pt,node distance=2cm,on grid,auto]
   \node[state,initial] (q_0)   {$q_0$};
   \node[state] (q_1) [right=of q_0] {$q_1$};
   \node[state, accepting] (q_2) [right=of q_1] {$q_2$};
   \node[state] (q_E) [right=of q_2] {$q_E$};
   \path[->]
   (q_0) edge  node {0} (q_1)
         edge [bend right] node {1} (q_E)
   (q_1) edge  node {1} (q_2)
         edge [loop above] node {0} ()
   (q_2) edge node {0} (q_E)
         edge [loop above] node {1} ()
   (q_E) edge [loop below] node {0,1} ();
\end{tikzpicture}
\end{center}
\textit{CHIARIRE}
\end{itemize}
\end{esempio}
\newpage
\begin{esempio}
Trovo automa per: $$L=\{w\in\{a,b\}^*|\mbox{ w che contiene un numero pari di a e dispari di b}\}$$
\begin{center}
\begin{tikzpicture}[shorten >=1pt,node distance=2cm,on grid,auto]
	\node[state, initial] (q_0) {$q_{pp}$};
	\node[state] (q_1) [right=of q_0] {$q_{dp}$};
	\node[state, accepting] (q_2) [below=of q_0] {$q_{pd}$};
	\node[state] (q_3) [right=of q_2] {$q_{dd}$};
	\path[->]
	(q_0) edge [bend left = 25] node {a} (q_1)
	      edge [bend right = 25] node [left] {b} (q_2)
	(q_1) edge [bend left = 25] node {a} (q_0)
	      edge [bend right = 25] node [left] {b} (q_3)
	(q_2) edge [bend right = 25] node [right] {b} (q_0)
	      edge [bend left = 25] node {a} (q_3)
	(q_3) edge [bend right = 25] node [right] {b} (q_1)
	      edge [bend left = 25] node {a} (q_2);
\end{tikzpicture}
\end{center}
\end{esempio}
\begin{esempio}
Trovo automa per: $$L=\{w\in\{a,b\}^*|\mbox{ w che contiene un numero pari di a seguito da uno dispari di b}\}$$
$$L=\{a^{2n}b^{2k+1}|j,k\geq 0\}$$
\begin{center}
\begin{tikzpicture}[shorten >=1pt,node distance=2cm,on grid,auto]
	\node[state, initial] (q_0) {$q_{0}$};
	\node[state, accepting] (q_1) [right=of q_0] {$q_{1}$};
	\node[state] (q_2) [below=of q_0] {$q_{2}$};
	\node[state] (q_3) [right=of q_2] {$q_S$};
	\node[state] (q_4) [right = of q_3] {$q_E$};
	\path[->]
	(q_0) edge [bend left = 25] node {b} (q_1)
	      edge [bend right = 25] node [left] {a} (q_2)
	(q_1) edge [bend left = 25] node {a} (q_4)
	      edge [bend right = 25] node [left] {b} (q_3)
	(q_2) edge [bend right = 25] node [below] {b} (q_4)
	      edge [bend right = 25] node [right] {a} (q_0)
	(q_3) edge [bend right = 25] node [right] {b} (q_1)
	      edge [bend left = 25] node {a} (q_4);
\end{tikzpicture}
\end{center}
ovvero in tabella:
\begin{center}
\begin{tabular}{c|c|c}
$\delta$ & a & b \\
\hline
$\to\,q_0$ & $q_1$ & $q_2$\\
\hline
$q_1$ & $q_0$ & $q_E$\\
\hline
$*\,q_2$ & $q_E$ & $q_3$\\
\hline
$q_S$ & $q_E$ & $q_2$\\
\hline
$q_E$ & $q_E$ & $q_E$
\end{tabular}
\end{center}
\end{esempio}
\begin{esempio}
Trovo automa per: $$L=\{a^{2k+1}b^{2h}|\, h,k\geq 0\}$$
\begin{center}
\begin{tikzpicture}[shorten >=1pt,node distance=2cm,on grid,auto]
	\node[state, initial] (q_0) {$q_{0}$};
	\node[state, accepting] (q_1) [right=of q_0] {$q_{1}$};
	\node[state] (q_3) [right=of q_1] {$q_{3}$};
	\node[state] (q_2) [below= of q_1] {$q_{2}$};
	\node[state, accepting] (q_4) [right = of q_2] {$q_4$};
	\node[state] (q_5) [right=of q_4] {$q_E$};
	\path[->]
	(q_0) edge  node [bend left = 25] {a} (q_1)
	(q_1) edge [bend left = 25] node {a} (q_2)
	      edge node [bend left= 25] {b} (q_3)
	(q_2) edge [bend left = 25] node [left] {a} (q_1)
	(q_3) edge [bend right = 25] node [left] {b} (q_4)
	(q_4) edge [bend right = 25] node {} (q_3);
\end{tikzpicture}
\end{center}
\end{esempio}
\begin{esempio}
Trovo automa per: $$L=\{a^{2n+1}b^{2k+1}|\, n,k\geq 0\}$$
\begin{center}
\begin{tikzpicture}[shorten >=1pt,node distance=2cm,on grid,auto]
	\node[state, initial] (q_0) {$q_{0}$};
	\node[state] (q_1) [right=of q_0] {$q_{1}$};
	\node[state, accepting] (q_3) [right=of q_1] {$q_{3}$};
	\node[state] (q_2) [below= of q_1] {$q_{2}$};
	\node[state] (q_4) [right = of q_2] {$q_4$};
	\node[state] (q_5) [right=of q_4] {$q_E$};
	\path[->]
	(q_0) edge  node [bend left = 25] {a} (q_1)
	(q_1) edge [bend left = 25] node {a} (q_2)
	      edge node [bend left= 25] {b} (q_3)
	(q_2) edge [bend left = 25] node [left] {a} (q_1)
	(q_3) edge [bend right = 25] node [left] {b} (q_4)
	(q_4) edge [bend right = 25] node [right] {b} (q_3);
\end{tikzpicture}
\end{center}
\end{esempio}
\begin{esempio}
Trovo automa per: $$L=\{x010y|\,x,y\in\{o,1\}^*\}$$
\begin{center}
\begin{tikzpicture}[shorten >=1pt,node distance=2cm,on grid,auto]
	\node[state, initial] (q_0) {$q_{0}$};
	\node[state] (q_1) [right=of q_0] {$q_{1}$};
	\node[state] (q_2) [right=of q_1] {$q_{2}$};
	\node[state, accepting] (q_3) [right= of q_2] {$q_{3}$};
	\path[->]
	(q_0) edge  node {0} (q_1)
	      edge [loop above] node {1} ()
	(q_1) edge  node {1} (q_2)
	      edge [loop above] node {0} ()
	(q_2) edge  node {0} (q_3)
	(q_3) edge [loop below] node {0,1} ();
\end{tikzpicture}
\end{center}
\end{esempio}
