\subsubsection{esercizi}
\begin{esempio}
automa DFA per $w=x010y,\,x,y\in\{0,1\}^*$ :\\
la stringa più corta è 010
\begin{center}
\begin{tikzpicture}[shorten >=1pt,node distance=2cm,on grid,auto]
\node[state, initial] (q_0) {$q_{0}$};
\node[state] (q_1) [right=of q_0] {$q_{1}$};
\node[state] (q_2) [right=of q_1] {$q_{2}$};
\node[state, accepting] (q_3) [right=of q_2] {$q_{3}$};
\path[->]
(q_0) edge  node {0} (q_1)
      edge [loop above] node {1} ()
(q_1) edge  node {1} (q_2)
      edge [loop above] node {0} ()
(q_2) edge  node {0} (q_3)
      edge  [bend left = 45] node {1}(q_0)
(q_3) edge [loop above] node {0,1} ();
\end{tikzpicture}
\end{center}
\end{esempio}
\begin{esempio}
automa DFA per $a^{2k+1}b^{2h},\, h,k\geq 0$ :\\
concatenazione di a dispari e b pari:
\begin{center}
\begin{tikzpicture}[shorten >=1pt,node distance=2cm,on grid,auto]
\node[state, initial] (q_0) {$q_{0}$};
\node[state, accepting] (q_1) [right=of q_0] {$q_{1}$};
\node[state] (q_3) [right=of q_1] {$q_{3}$};
\node[state] (q_2) [below= of q_1] {$q_{2}$};
\node[state, accepting] (q_4) [right = of q_2] {$q_4$};
\node[state] (q_5) [right=of q_4] {$q_E$};
\path[->]
(q_0) edge  node [bend left = 25] {a} (q_1)
(q_1) edge [bend left = 25] node {a} (q_2)
      edge node [bend left= 25] {b} (q_3)
(q_2) edge [bend left = 25] node [left] {a} (q_1)
(q_3) edge [bend right = 25] node [left] {b} (q_4)
(q_4) edge [bend right = 25] node [right] {b} (q_3)
(q_2) edge  [bend right = 55] node [below] {b} (q_5)
(q_3) edge  [bend left = 25] node {a} (q_5)
(q_4) edge  [bend right = 25] node [below] {a} (q_5)
(q_5) edge [loop right] node {a,b} ();
\end{tikzpicture}
\end{center}
\end{esempio}
\begin{esempio}
cerco DFA per stringhe inizianti con a e finenti con b, con occorrenze di b singole o a coppie, nessuna regola per c.\\
per esempio $abbcb$ è nel linguaggio
\begin{center}
\begin{tikzpicture}[shorten >=1pt,node distance=2cm,on grid,auto]
\node[state, initial] (q_0) {$q_{0}$};
\node[state] (q_1) [right=of q_0] {$q_{1}$};
\node[state, accepting] (q_2) [right=of q_1] {$q_{2}$};
\node[state, accepting] (q_3) [right= of q_3] {$q_{3}$};
\node[state] (q_5) [below=of q_0] {$q_E$};
\path[->]
(q_0) edge  node [bend left = 25] {a} (q_1)
    edge  node [bend left = 25] {b,c} (q_5)
(q_1) edge  node {b} (q_2)
      edge [loop] node {a,c} ()
(q_2) edge [bend left = 25] node {a,c} (q_1)
      edge  node  {b} (q_3)
(q_3) edge [bend left = 65] node [below] {b} (q_5)
      edge [bend left = 55] node {a,c} (q_1)
(q_5) edge [loop left] node {a,b,c} ();
\end{tikzpicture}
\end{center}
\end{esempio}
\newpage
\begin{esempio}
cerco DFA per  stringhe di bit non contegano 000
\begin{center}
\begin{tikzpicture}[shorten >=1pt,node distance=2cm,on grid,auto]
  \node[state, initial, accepting] (q_0) {$q_0$};
  \node[state, accepting] (q_1) [right=of q_0]{$q_1$};
  \node[state, accepting] (q_2) [right=of q_1]{$q_2$};
  \node[state] (q_e) [right=of q_2]{$q_E$};
  \path[->]
  (q_0) edge  [bend left=25] node {0} (q_1)
        edge [loop] node {1} ()
  (q_1) edge node {0} (q_2)
        edge [bend left=25] node {1} (q_0)
  (q_2) edge node {0} (q_e)
        edge  [bend left=55] node{1} (q_0)
  (q_e) edge [loop ] node {0,1} ();
\end{tikzpicture}
\end{center}
\end{esempio}
\begin{esempio}
cerco DFA per  stringhe di bit non contegano 000 almeno una volta
\begin{center}
\begin{tikzpicture}[shorten >=1pt,node distance=2cm,on grid,auto]
  \node[state, initial] (q_0) {$q_0$};
  \node[state] (q_1) [right=of q_0]{$q_1$};
  \node[state] (q_2) [right=of q_1]{$q_2$};
  \node[state, accepting] (q_e) [right=of q_2]{$q_E$};
  \path[->]
  (q_0) edge  [bend left=25] node {0} (q_1)
        edge [loop] node {1} ()
  (q_1) edge node {0} (q_2)
        edge [bend left=25] node {1} (q_0)
  (q_2) edge node {0} (q_e)
        edge  [bend left=55] node{1} (q_0)
  (q_e) edge [loop ] node {0,1} ();
\end{tikzpicture}
\end{center}
\end{esempio}
\begin{esempio}
cerco DFA per  stringhe di bit che contengono 000 solo una volta
\begin{center}
\begin{tikzpicture}[shorten >=1pt,node distance=2cm,on grid,auto]
  \node[state, initial] (q_0) {$q_0$};
  \node[state] (q_1) [right=of q_0]{$q_1$};
  \node[state] (q_2) [right=of q_1]{$q_2$};
  \node[state, accepting] (q_3) [right=of q_2]{$q_3$};
  \node[state,accepting] (q_4) [below=of q_3] {$q_3$};
  \node[state] (q_5) [below=of q_4]{$q_5$};
  \node[state] (q_6) [below=of q_5]{$q_6$};
  \node[state] (q_e) [right=of q_3]{$q_E$};
  \path[->]
  (q_0) edge  [bend left=25] node {0} (q_1)
        edge [loop] node {1} ()
  (q_1) edge node {0} (q_2)
        edge [bend left=25] node {1} (q_0)
  (q_2) edge node {0} (q_3)
        edge  [bend left=45] node{1} (q_0)
  (q_3) edge node {0} (q_e)
        edge node {1} (q_4)
  (q_4) edge [bend left=25] node {0} (q_5)
      edge [loop right] node {1} ()
  (q_5) edge [bend left=25] node {1} (q_4)
        edge node {0} (q_6)
  (q_6) edge [bend right=25] node {0} (q_e)
        edge [bend left=55] node {0} (q_4)
  (q_e) edge [loop ] node {0,1} ();
\end{tikzpicture}
\end{center}
\end{esempio}
\newpage
\begin{esempio}
Trasformare il seguente NFA in un DFA:
\begin{center}
\begin{tikzpicture}[shorten >=1pt,node distance=3cm,on grid,auto]
\node[state, initial, accepting] (q_0) {$q_0$};
\node[state, accepting] (q_1) [above right=of q_0] {$q_1$};
\node[state] (q_2) [below right =of q_0] {$q_2$};
\node[state] (q_3) [below right=of q_1] {$q_3$};
\path[->]
(q_0) edge  [bend left=25] node {a} (q_1)
      edge  [bend right=25] node [below] {b} (q_2)
(q_1) edge  [bend left=25] node {a} (q_2)
      edge [loop ] node {a} ()
(q_2) edge  [bend left=25] node {b} (q_1)
      edge  [bend right=15] node [below] {b} (q_3)
(q_3) edge   node [above] {a} (q_1)
      edge  [bend right=15] node [above] {a} (q_2);
\end{tikzpicture}
\end{center}
abbiamo quindi:
$$\delta_D(\{q_0\},a)=\delta_N(q_0,a)=\{q_1,q_2\}$$
$$\delta_D(\{q_0\},b)=\delta_N(q_1,b)=\emptyset$$
$$\delta_D(\{q_1,q_2\},a)=\delta_N(q_1,a)\cup \delta_N(q_2,a)=\{q_1,q_2\}cup \emptyset=\{q_1,q_2\}$$
$$\delta_D(\{q_1,q_2\},b)=\delta_N(q_1,b)\cup \delta_N(q_2,b)=\emptyset\cup \{q_1,q_3\}=\{q_1,q_3\}$$
$$...$$
ottengo quindi:
\begin{center}
\begin{tabular}{c|c|c}
DFA & a & b \\
\hline
$*\to\,\{q_0\}$ & $\{q_1,q_2\}$ & $\emptyset$ \\
\hline
$*\{q_1,q_2\}$ & $\{q_1,q_2\}$ & $\{q_1,q_3\}$ \\
\hline
$\emptyset$ & $\emptyset$ & $\emptyset$ \\
\hline
$*\{q_1,q_3\}$ & $\{q_1,q_2\}$ & $\emptyset$
\end{tabular}
\end{center}
Posso ora rinominare:
\begin{itemize}
\item $A=\{q_0\}$
\item $B=\{q_1,q_2\}$
\item $C=\emptyset$
\item $D=\{q_1,q_3\}$
\end{itemize}
\newpage
ottengo quindi il seguente DFA:
\begin{center}
\begin{tikzpicture}[shorten >=1pt,node distance=3cm,on grid,auto]
\node[state, initial, accepting] (q_0) {$A$};
\node[state, accepting] (q_1) [right=of q_0] {$B$};
\node[state] (q_2) [below=of q_0] {$C$};
\node[state,accepting] (q_3) [right=of q_2] {$D$};
\path[->]
(q_0) edge  node {a} (q_1)
      edge  node {b} (q_2)
(q_1) edge  [bend left=25] node {b} (q_3)
      edge  [loop] node {a} ()
(q_2) edge  [loop below] node {a,b} ()
(q_3) edge  [bend left=25] node {a} (q_1)
      edge  node  {a} (q_2);
\end{tikzpicture}
\end{center}
\end{esempio}
\begin{esempio}
Trasformare il seguente $\varepsilon$-NFA in un DFA:
\begin{center}
\begin{tikzpicture}[shorten >=1pt,node distance=3cm,on grid,auto]
\node[state, initial] (q_0) {$p$};
\node[state] (q_1) [right=of q_0] {$q$};
\node[state, accepting] (q_2) [below right =of q_0] {$r$};
\path[->]
(q_0) edge  [bend left=25] node [left] {c} (q_2)
      edge  [bend right=25] node {b} (q_1)
      edge [loop ] node {a} ()
(q_1) edge  [bend right=25] node {$\varepsilon$} (q_0)
      edge  [bend right=25] node [left] {b} (q_2)
      edge  [loop ] node {a} ()
(q_2) edge  [bend left=25] node {c} (q_0)
      edge  [bend right=25] node [right] {$\varepsilon$} (q_1)
      edge  [loop below] node {a} ();
\end{tikzpicture}
\end{center}
vediamo le ECLOSE:
$$ECLOSE(p)=\{p\}$$
$$ECLOSE(q)=\{p,q\}$$
$$ECLOSE(r)=\{p,q,r\}$$
si ottiene quindi:
\begin{center}
\begin{tabular}{c|c|c|c}
& a & b & c\\
\hline
$to\,\{p\}$ & $\{p\}$ & $\{p,q\}$ & $\{p,q,r\}$\\
\hline
$\{p,q\}$ & $\{p,q\}$ & $\{p,q,r\}$ & $\{p,q,r\}$\\
\hline
$*\{p,q,r\}$ & $\{p,q,r\}$ & $\{p,q,r\}$ & $\{p,q,r\}$
\end{tabular}
\end{center}
infatti, per esempio:
$$\delta_D(\{p\}, a)= ECLOSE (\delta_(p,a))=ECLOSE (\{p\})=ECLOSE (p)=\{p\}$$
$$\delta_D(\{p,q\}, a)= ECLOSE (\delta_N(p,a)\cup \delta_N(q,a))=$$
$$ECLOSE (\{p\}\cup \{q\})=ECLOSE(P)\cup ECLOSE(q)=\{p\}\cup\{p,q\}=\{p,q\}$$
$$...$$
si hanno quindi le seguenti rinominazioni:
\begin{itemize}
\item $A=\{p\}$
\item $b=\{p,q\}$
\item $C=\{p,q,r\}$
\end{itemize}
ovvero:
\begin{center}
\begin{tikzpicture}[shorten >=1pt,node distance=3cm,on grid,auto]
\node[state, initial] (q_0) {$A$};
\node[state] (q_1) [right=of q_0] {$B$};
\node[state, accepting] (q_2) [below=of q_0] {$C$};
\path[->]
(q_0) edge  node {b} (q_1)
      edge  node {c} (q_2)
      edge  [loop] node {a} ()
(q_1) edge  node {b,c} (q_2)
      edge  [loop] node {a} ()
(q_2) edge  [loop below] node {a,b,c} ();
\end{tikzpicture}
\end{center}
\end{esempio}
