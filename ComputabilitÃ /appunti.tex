%Appunti del Corso Computabilità
\documentclass[a4paper]{book}
\usepackage[T1]{fontenc}%per rappresentare i font italiani, come le lettere accentate, con la giusta spaziatura
\usepackage[utf8]{inputenc}%per poter inserire nel testo .tex i caratteri unicode8
\usepackage[italian]{babel}%per poter effettuare la giusta sillabazione della lingua italiana
\usepackage{classicthesis}%necessario per usare lo stile arsclassica
\usepackage{arsclassica}%per poter usare lo stile arsclassica usato nell'Arte di imparare il Latex
\usepackage{amsmath}%per poter rappresentare ed utilizzare al meglio gli ambienti e le formule matematiche
\usepackage{amssymb}%per rappresentare alcuni simboli particolari matematici
\usepackage{amsthm}%per definire e poter effettuare le dimostrazioni matematiche
\usepackage{amsfonts}%per poter avere i font matematici
\usepackage{amstext}%per avere una gestione del testo nell'ambiente matematico
\usepackage{booktabs}%per la corretta gestione delle tabelle
\usepackage{graphics}%per effettuare i grafici
\usepackage{rotating}%per effettuare le rotazioni delle immagini e grafici
\usepackage{microtype}%per effettuare un aggiustamento della spaziatura tra caratteri e del font
\usepackage{tikz}%Pacchetto per fare i disegni in particolare con la libreria automata farò gli automi
\usetikzlibrary{automata,positioning}%libreria per disegnare automi a stati finiti

\theoremstyle{definition}%per avere lo stile tondo quando uso un ambiente definito da newtheorem
\newtheorem*{defi}{Def}%Definizione per avere la gestione delle definizioni
\newtheorem{prop}{Prop}[chapter]
\newtheorem{lem}{Lemma:}
\newtheorem{thm}{Thm:}[chapter]
\newtheorem{corol}{Corol}[chapter]
\newtheorem*{esempio}{Esempio}
\newcommand{\numberset}{\mathbb}
\newcommand{\N}{\numberset{N}}
<<<<<<< HEAD

\begin{document}

=======
\newcommand{\Z}{\numberset{Z}}
\newcommand{\Q}{\numberset{Q}}
\newcommand{\R}{\numberset{R}}
\newcommand{\C}{\numberset{C}}

\begin{document}
>>>>>>> 6dce03e48312df68955becb5927d04e110d17342
\title{Appunti del corso Linguaggi e Computabilità}
\author{Marco Natali}
\date{}
\maketitle

Il corso di Linguaggi e Computabilità riguarda l'informatica teorica e si occupa di definire la calcolabilità di un problema,
di definire le grammatiche e i linguaggi formali con l'ausilio di anche di automi e macchine di Turing.

Incominciamo con la definizione dei componenti basilari attraverso cui svilupperemo poi i concetti del corso
\begin{defi}
<<<<<<< HEAD
  Si definisce come \emph{alfabeto}, indicato con $\Sigma$, una sequenza di simboli, attraverso cui possiamo stabilire un alfabeto.\newline
  Si definisce invece come stringa, una sequenza finita di simboli appartenenti ad un alfabeto $\Sigma$ ed esisterà sempre la stringa $\epsilon$,
  indicante la stringa vuota.
\end{defi}
\begin{esempio}
  \[
  \begin{aligned}
    \Sigma = \{0, 1\}\ \text{e} \Sigma = \{a, b, c\} \\
    w = 10110 \, \, \, z = abccabbcc \\
  \end{aligned}
  \]
\end{esempio}
\begin{defi}
È possibile fornire una definizione induttiva di stringa, partendo dalla stringa $\epsilon$:
=======
  Si definisce come \emph{alfabeto}, indicato con $\Sigma$, una sequenza di simboli, attraverso cui possiamo stabilire un alfabeto.
\end{defi}
\begin{esempio}
  $\Sigma = \{0, 1\}$ e $\Sigma = \{a, b, c\}$
\end{esempio}

\begin{defi}
  Si definisce come stringa, indicata solitamente con $w$, una sequenza di simboli appartenenti ad un alfabeto $\Sigma$
\end{defi}
Esempio: su un alfabeto $\Sigma = \{0, 1\}$ definiamo le seguenti stringhe:
  $w = 10110 \,\,\, z = 10111111$

Considerando un qualsiasi alfabeto $\Sigma$, esisterà sempre la stringa $\epsilon$, rappresentante la stringa vuota per cui
attraverso questa considerazione si definisce una stringa in maniera induttiva come:
\begin{defi}
 Una stringa viene definita induttiva come:
>>>>>>> 6dce03e48312df68955becb5927d04e110d17342
    \begin{description}
      \item [caso base]: $\epsilon$ è una stringa vuota
      \item [caso passo]: se $w$ è una stringa, allora anche $a \circ w $ è una stringa
    \end{description}
\end{defi}
<<<<<<< HEAD
Dopo aver fornito le definizioni per le stringhe, definiamo le seguenti operazioni definite su le stringhe:
\begin{itemize}
   \item insieme di stringhe: definiamo come $\Sigma ^ k$ l'insieme di stringhe su $\Sigma$ con $k$ caratteri come segue:
          \[
          \begin{aligned}
          \Sigma ^ 0 & = \{ \epsilon\} \\
          \Sigma ^ 1 & \neq \Sigma \text{ma sono l'insieme delle stringhe di un carattere} \\
          \Sigma ^ 2 & = \text{insieme di stringhe di due caratteri} \\
          \dots \dots \\
          \Sigma ^ k & = \text{insieme di stringhe di k caratteri} \\
          \end{aligned}
          \]
        Le due più importanti insiemi di stringhe, usate per rappresentare l'insieme di stringhe di qualsiasi lunghezza, sono:
        \[
        \begin{aligned}
          \Sigma ^ * & = \cup _{i = 0} \Sigma ^ i \\
          \Sigma ^ + & = \Sigma ^ * - \{\epsilon\} \\
        \end{aligned}
        \]

  \item $|w|: \Sigma ^ * \to \N$: rappresenta la lunghezza di una stringa, ossiail numero di caratteri presenti in una stringa,
        e la definizione di lunghezza avviene induttivamente come segue:
=======
Dopo aver definito le stringhe, definiamo le seguenti operazioni definite su le stringhe:
  \begin{itemize}
  \item lunghezza $|w|: \Sigma ^ * \to \N$: rappresenta il numero di caratteri presenti in una stringa, con $\Sigma ^ *$ indicante
        una qualsiasi stringa e la definizione di lunghezza avviene induttivamente come segue:
>>>>>>> 6dce03e48312df68955becb5927d04e110d17342
        \begin{description}
        \item [base]: la lunghezza di $|\epsilon| = 0$
        \item [passo]: se $|w| = n$ con $n \in \N$ e sia $a \in \Sigma$ allora $|a \circ w| = 1 + |w| = n + 1$.
        \end{description}
        \begin{esempio}
          $w = abcdec  \quad |w| = 6$
        \end{esempio}
<<<<<<< HEAD

  \item $\circ: \sum ^ * \times \sum ^ * \to \sum ^ *$: rappresenta la concatenazione, ossial'aggiunta dei caratteri della seconda stringa
=======
  \item insieme di stringhe: definiamo come $\Sigma ^ k$ l'insieme di stringhe su $\Sigma$ con $k$ caratteri come segue:
        \begin{align}
          \Sigma ^ 0 & = \{ \epsilon\} \\
          \Sigma ^ 1 & = \Sigma \\
          \Sigma ^ 2 & = insieme di stringhe di due caratteri \\
          \dots \dots \\
          \Sigma ^ k & = insieme di stringhe di k caratteri \\
        \end{align}
        Le due più importanti insiemi di stringhe, usate per rappresentare l'insieme di stringhe di qualsiasi lunghezza, sono:
        \begin{align}
          \Sigma ^ * & = \cup _{i = 0} \Sigma ^ i \\
          \Sigma ^ + & = \Sigma ^ * - \{\epsilon\} \\
        \end{align}

  \item concatenazione $\circ: \sum ^ * \times \sum ^ * \to \sum ^ *$: rappresenta l'aggiunta dei caratteri della seconda stringa
>>>>>>> 6dce03e48312df68955becb5927d04e110d17342
        al termine della prima stringa ma vediamo ora una definizione più formale:
        \begin{defi}
          Date due stringhe $x = a_1 a_2 \dots a_n$ e $y = b_1 b_2 \dots b_n$ si definisce $x \circ y = a_1 a_2 \dots a_n b_1 b_2 \dots b_n$
          con $|x \circ y| = |x| + |y| = n + m$.
        \end{defi}
        La concatenazione possiede le seguenti proprietà:
        \begin{itemize}
        \item è associativa: $\forall x,y,z \in \sum^* x \circ (y \circ z) = (x \circ y) \circ z$.
        \item non è commutativa infatti presi due stringhe $x,y$ diverse risulta $x \circ y \neq y \circ x$.
        \item possiede l'elemento neutro $\epsilon$ per cui $x \circ \epsilon = x = \epsilon \circ x$.
        \end{itemize}
        \begin{esempio}
          \begin{equation*}
          \begin{split}
            w = 1011100 \quad z = 1011110  \\
<<<<<<< HEAD
          w \circ z & = 10111001011110 \\
          z \circ w  &= 10111101011100 \\
=======
          w \circ z = 10111001011110 \\
          z \circ w = 10111101011100 \\
>>>>>>> 6dce03e48312df68955becb5927d04e110d17342
          \end{split}
          \end{equation*}
        \end{esempio}
  \end{itemize}
<<<<<<< HEAD
Attraverso le seguenti operazioni si può stabilire che $(\Sigma ^ *,\circ,\epsilon)$ è un monoide libero su $\Sigma$.

Dopo aver considerato e definito l'alfabeto e le stringhe, bisogna definire i linguaggi, concetto su cui si basa tutto il nostro
corso e l'informatica teorica:
\begin{defi}
=======
  Attraverso le seguenti operazioni si può stabilire che $(\Sigma ^ *,\circ,\epsilon)$ è un monoide libero su $\Sigma$.

  \begin{defi}
>>>>>>> 6dce03e48312df68955becb5927d04e110d17342
    Definiamo come \emph{Linguaggio} un'insieme di stringhe scelte su $\Sigma ^ *$ scelte per far parte del linguaggio
    ossia $L \subseteq \Sigma ^ *$.
    Le componenti di un linguaggio sono:
    \begin{description}
    \item [alfabeto] : insieme di simboli su cui definiamo poi il lessico e la sintassi
    \item [lessico]: definisce il vocabolario del linguaggio e viene definito tramite una grammatica di tipo 3
    \item [sintassi]: definisce come le varie frasi del linguaggio devono essere disposte nel linguaggio e ciò
      viene definito da una grammatica di tipo 2.
    \item [semantica]: il significato attribuito alle frasi del linguaggio però nei linguaggi formali deriva dalla sintassi
      anche se in questo corso la semantica non verrà affrontata.
    \end{description}
<<<<<<< HEAD
\end{defi}
I linguaggi possono essere riconosciuti attraverso degli automi oppure generati medianti delle grammatiche.\newline
Un'altra cosa importante è che si hanno sottoinsiemi particolari di linguaggi, come l'insieme vuoto, che resta comunqe un linguaggio,
il \textbf{linguaggio vuoto} e $\emptyset \in \Sigma^k,\,|\emptyset|=0$ che è diverso dal linguaggio che contiene la stringa vuota $|\epsilon|=1$,
inoltre $\Sigma^*\subseteq \Sigma^*$ che ha lunghezza infinita.\newline
Vediamo qualche esempio di linguaggio:
\begin{itemize}
\item il linguaggio di tutte le stringhe che consistono in $n$ 0 seguiti da $n$ 1:
  \[ \{\epsilon,01,0011,000111,...\} \]
\item l'insieme delle stringhe con un uguale numero di 0 e di 1:
  \[ \{\epsilon,01,10.0011,0101.1001,..\} \]
\item l'insieme dei numeri binari il cui valore è un numero primo:
  \[ \{\epsilon,10 , 11, 101, 111,1011,...\} \]
\item $\Sigma^*$ è un linguaggio per ogni alfabeto $\Sigma$
\item $\emptyset$, il linguaggio vuoto, e $\{\epsilon\}$ sono un linguaggio rispetto a qualunque alfabeto
\end{itemize}
La prima modalità per definire un linguaggio è attraverso la definizione di una grammatica, per stabilire tutte e sole
le stringhe del linguaggio.
In questo paragrafo forniremo soltanto una definizione informale e definiamo le diverse tipologie di gerarchie, senza
effettuarne una trattazione formale, cosa che avverrà nel prossimo capitolo.
\begin{defi}
    Si definisce come \emph{Grammatica} un'insieme di regole che delineano le stringhe ammissibili del linguaggio e
    possono essere di due tipologie:
    \begin{itemize}
      \item grammatica generativa : insieme di regole che permettono di generare tutte le stringhe di un lunguaggio partendo dalle stringhe base
      \item grammatica analitica : analizza le stringhe passate in input e stabilisce l'appartenenza o meno al linguaggio
    \end{itemize}
\end{defi}
  Nel 1956 il linguista Choumsky introdusse e definì la gerarchia delle grammatiche e linguaggi, che ha avuto una notevole importanza
  nell'informatica teorica anche se il suo intento era quello di catalogare le varie tipologie di linguaggi naturali:
\begin{description}
  \item [grammatiche di tipo 0]: non si hanno restrizioni sulle regole di produzione, $\alpha\to\beta$ e sono linguaggi ricorsivamente numerabili,
    rappresentati dalle \emph{macchine di Turing}, deterministiche o non deterministiche (la macchina di Turing è un automa).

  \item [grammatiche di tipo 1]:  il lato destro della produzione ha lunghezza almeno uguale a quello sinistro e si chiamano anche
        grammatiche dipendenti dal contesto e come automa hanno \emph{la macchina di Turing che lavora in spazio lineare}:
        \[ \alpha_1A\alpha_2\to \alpha_1B\alpha_2 \]
        con $\alpha_1$ e $\alpha_2$ detti \emph{contesto} e $\alpha_1,\,\alpha_2,\, \beta\in (V\cup T)^*$

  \item [grammatiche di tipo 2]: sono quelle libere dal contesto, context free ed usano come riconoscitore gli automi a pila.
        Come regola ha $A\to\beta$ con $A\in V$ e $\beta\in V\cup T)^*$.

  \item [grammatiche di tipo 3]: sono le grammatiche \emph{regolari}, i cui linguaggi vengono riconosciuti dagli automi a stati finiti.\newline
        Come regole ha $A\to\alpha B$ (o $A\to B\alpha$) e $A\to\alpha$  con $A,B\in V$ e $\alpha\in T$.
\end{description}
Durante questo corso effettueremo una trattazione delle grammatiche di tipo 3 e 2, anche se analizzeremo solo un esempio di grammatica di tipo 1;
iniziamo nel prossimo capitolo a considerare le grammatiche di tipo 2, chiamate grammatiche context-free.

\chapter{Grammatiche di tipo 2}
Iniziamo ora a considerare le grammatiche di tipo 2 partendo da un esempio e poi fornendo una definizione formale.
\begin{esempio}
  Dato il linguaggio delle stringhe palindrome $L _{pal} = \{w \in \Sigma ^ * : w = w ^ R\}$ dove $w^R$ rappresenta la stringa
  reversa per cui ad esempio $'OTTO' \in L_{pal}$ mentre $'PAPA'$ non appartiene al linguaggio.
=======
  \end{defi}

  \begin{defi}
    Si definisce come \emph{Grammatica} un'insieme di regole che delineano le stringhe ammissibili del linguaggio e
    possono essere di due tipologie:
    \begin{description}
      \item [grammatica generativa] : insieme di regole che permettono di generare tutte le stringhe di un lunguaggio partendo dalle stringhe base
      \item [grammatica analitica] : analizza le stringhe passate in input e stabilisce l'appartenenza o meno al linguaggio
    \end{description}
    \end{defi}
  Nel 1956 il linguista Choumsky introdusse e definì la gerarchia delle grammatiche, che ha avuto una notevole importanza
  nell'informatica teorica anche se il suo intento era quello di catalogare le varie tipologie di linguaggi naturali:
  \begin{description}
  \item [grammatiche di tipo 3]: sono i linguaggi regolari, generati da grammatiche regolari e riconosciuti da automi a stati finiti.
    Una spiegazione dettagliata avverrà nel seguito dei paragrafi/capitoli.

  \item [grammatiche di tipo 2]: grammatiche libere dal contesto, riconosciute da automi a pila; una spiegazione più dettagliata
    avverrà nei paragrafi successivi.

  \item [grammatiche di tipo 1]: grammatica dipendente dal contesto in cui per definizione i lati destri delle produzioni delle
    grammatiche non posso essere più lunghi dei rispetti lati sinistri, ossia $\alpha \to \beta |\alpha| \geq |\beta|$.
    Per rappresentarli e stabilire se una stringa appartiene al linguaggio si usano gli \emph{automi lineari}, che non sono oggetto del corso.

  \item [grammatiche di tipo 0]: grammatiche in cui non vi è alcun vincolo per le produzioni della grammatica e può essere
         rappresentato tramite una macchina di turing nondeterministica però questa tipologia di grammatica non verrà affrontata.
  \end{description}

\chapter{Grammatiche di tipo 2}
Iniziamo ora a considerare le grammatiche di tipo 2 incominciando da un esempio e poi definendola in maniera formale.
\begin{esempio}
  Dato il linguaggio delle stringhe palindrome $L _{pal} = \{w \in \Sigma ^ * : w = w ^ R\}$ dove $w^R$ rappresenta la stringa
  reversa per cui ad esempio $''OTTO'' \in L _ {pal}$ mentre $''PAPA''$ non appartiene al linguaggio.
>>>>>>> 6dce03e48312df68955becb5927d04e110d17342

  Definiamo in maniera più formale e meno ambigua i componenti di $L_{pal}$:
  \begin{description}
  \item [caso base]: $\epsilon, 0, 1 \in L_{pal}$
  \item [caso induttivo]: se $w \in L_{pal}$ allora $0w0$ e $1w1$ appartengono a $L_{pal}$ e nient'altro appartiene al linguaggio.
  \end{description}
  Le regole di derivazione per $L_{pal}$ sono le seguenti, e derivano dalla definizione dei componenti:
  \begin{itemize}
  \item $P \to \epsilon$
  \item $P \to 0$
  \item $P \to 1$
  \item $P \to 0P0$
  \item $P \to 1P1$
  \end{itemize}
<<<<<<< HEAD
  Una versione più concisa delle regole di derivazione è la seguente:
  \[ P_g\{P \to \epsilon | 0 | 1 | 0P0 | 1P1\} \]
=======
>>>>>>> 6dce03e48312df68955becb5927d04e110d17342
  Dopo aver dato le regole di derivazione dobbiamo capire se le seguente regole definiscono tutte e sole le stringhe del linguaggio per
  cui ci chiediamo per esempio se $010 \in L_{pal}$ e $10001 \in L_{pal}$?
  \begin{align}
    P \Rightarrow 0P0 \Rightarrow 010  \quad \text{la stringa appartiene correttamente al linguaggio}\\
    P \Rightarrow 1P1 \Rightarrow 10P01 \Rightarrow 10001 \quad \text{la stringa appartiene correttamente al linguaggio}\\
  \end{align}
<<<<<<< HEAD
=======

>>>>>>> 6dce03e48312df68955becb5927d04e110d17342
  Attraverso l'applicazione di una serie di regole di derivazione otteniamo una stringa $w \in \Sigma ^*$ se e solo $w \in L$
  della grammatica definita.
\end{esempio}

%Definizione di Grammatica
\begin{defi}
  Si definisce una grammatica context free come $G = (V, T, P_g, S)$, i cui componenti sono:
  \begin{itemize}
  \item $V$ rappresenta l'insieme delle variabili usate per rappresentare un linguaggio
  \item $T$ rappresenta l'insieme dei simboli terminali, ossia l'insieme dei simboli attraverso cui sono definite le stringhe
    del linguaggio per questo di solito coincide con l'alfabeto del linguaggio.

  \item $S$ rappresenta la variabile di inizio della grammatica, ossia la variabile attraverso cui si definisce la grammatica
    per cui le altre variabili sono classi ausiliari di stringhe che aiutano a definire le stringhe del linguaggio.

  \item $P_g$ indica l'insieme di regole, che rappresentano la definizione ricorsiva del linguaggio, della seguente forma:
    \begin{equation*}
        P_g = \{ X \to \beta | \beta \in (V \cup  T)^* \ \text{e} \ X \in V\}
    \end{equation*}
<<<<<<< HEAD
    la variabile $X$ rappresenta la testa della produzione mentre $\beta$ è il corpo
  \end{itemize}
  Non si possono applicare delle regole in parallelo, ma soltanto una alla volta.
=======
    la variabile $X$ rappresenta la testa della produzione mentre $\beta$ indica il corpo
  \end{itemize}
>>>>>>> 6dce03e48312df68955becb5927d04e110d17342
\end{defi}

  Nell'esempio del linguaggio palidromo la grammatica che lo genera è
  \begin{equation*}
    G = (\{P\},\{0, 1\}, \{P \to \epsilon,P \to 0,P \to 1,P \to 0P0,P \to 1P1\}, S\}
  \end{equation*}

  %Concetto di derivazione
  Dopo aver definito in maniera formale di grammatica introduciamo il concetto di derivazione, per stabilire se una stringa appartiene
  o meno al linguaggio.
<<<<<<< HEAD
  \begin{defi}
    Sia $G = (V, T,P_g,S)$ una grammatica context-free, sia $\alpha A \beta$ una stringa di terminali e variabili con $A \in V$ e
    sia infine $A \to \gamma$ una regola di derivazione allora $\alpha A \beta \Rightarrow _g \alpha \gamma \beta$.\newline
    Si indica $\Rightarrow _g ^*$, il simbolo di applicazione di zero,uno o più step di derivazione, definito nel seguente modo:
=======

  \begin{defi}
    Sia $G = (V, T,P_g,S)$ una grammatica context-free, sia $\alpha A \beta$ una stringa di terminali e variabili con $A \in V$ e
    sia infine $A \to \gamma$ una regola di derivazione allora $\alpha A \beta \Rightarrow _g \alpha \gamma \beta$.
  \end{defi}
  \begin{defi}
    Si indica $\Rightarrow _g ^*$, il simbolo di applicazione di zero,uno o più step di derivazione, definiti nel seguente modo:
>>>>>>> 6dce03e48312df68955becb5927d04e110d17342
    \begin{description}
    \item [caso base]: per ogni stringa $\alpha$ di terminali e variabili, si ha $\alpha \Rightarrow _g ^* \alpha$
    \item [caso passo]: se $\alpha \Rightarrow _g ^* \beta$ e $\beta \Rightarrow _ g \gamma$ allora $\alpha \Rightarrow _g ^* \gamma$
    \end{description}
<<<<<<< HEAD
    Si può anche dire che $\alpha\to_G *\, \beta$ se e solo se esiste una sequenza di stringhe $\gamma_1,...,\gamma_n$ con $n\geq 1$
    tale che $\alpha=\gamma_1$, $\beta=\gamma_n$ e $\forall i,\, 1<i<n-1$ si ha che $\gamma_1\to \gamma_{i+1}$ per cui
    la derivazione in 0 o più passi è la chiusura transitiva della derivazione.
  \end{defi}
  Le stringhe che otteniamo sono delle forme sentenziali, ossia stringhe appartenenti a $(V \cup T)^*$ e un particolare
  sottoinsieme, in cui le stringhe sono composte soltanto da terminali, definisce le stringhe del linguaggio.
=======
  \end{defi}
  %Esempio

  Le stringhe che otteniamo sono delle forme sentenziali, ossia stringhe appartenenti a $(V \cup T)^*$ e un particolare
  sottoinsieme, in cui le stringhe sono composte da letterali, definisce le stringhe del linguaggio.
>>>>>>> 6dce03e48312df68955becb5927d04e110d17342

  Sempre considerando l'esempio del linguaggio delle stringhe palindrome la derivazione di $10011001$ è la seguente:
  \begin{equation*}
    P \Rightarrow 1P1 \Rightarrow 10P01 \Rightarrow 100P001 \Rightarrow 1001P1001 \Rightarrow 10011001
  \end{equation*}

  Al fine di ridurre il numero di scelte nella derivazione di una stringa introduciamo ora:
  \begin{description}
  \item [left derivation]: sostituiamo la variabile più a sinistra nell'applicazione di una regola di derivazione
                           e ciò viene rappresentato con il simbolo $\Rightarrow _{lm}$.

  \item [right derivation]: sostituiamo la variabile più a destra nell'applicazione di una regola di derivazione
                            ed essoviene rappresentato con il simbolo $\Rightarrow _{rm}$.
  \end{description}

  \begin{defi}
    Data una grammatica context-free $G = (V, T, P_g, S)$ si definisce un linguaggio context-free $L$ come:
    \begin{equation*}
      L_g = \{ w \in T^* | S \Rightarrow _g ^* w \}
    \end{equation*}
  \end{defi}

  Nei linguaggi context-free si può soltanto effettuare la concatenazione e l'annidamento dei sottolinguaggi come vediamo
  nei seguenti esempi:

  %Esempio di linguaggio Context-free
  \begin{esempio}
  Mostriamo ora un esempio di linguaggio definito come la concatenazione di due linguaggi
  \begin{equation*}
     L_g = \{ w \in \{0,1\}^* | w = 0^m 1^{m+1}01^n001^n \ n\geq 0\}
  \end{equation*}
<<<<<<< HEAD
  Si può vedere che ci 3 blocchi che concatenati generano il linguaggio $L_g$ per cui le regole di derivazione sono
  \[ P_g = \{S \to X0Y, X \to 1 | 0X1, Y \to 1001 | 1Y1\} \]
  \end{esempio}
=======
  Per sapere come costruire la grammatica del linguaggio dobbiamo visualizzare la struttura della stringa
  \begin{equation*}
    0^m11^m 0 1^n001^n  \quad \text{lo spazio rappresenta i blocchi su cui viene formata la stringa}
  \end{equation*}
  Si può vedere che i 3 blocchi sono 3 linguaggi che concatenati generano il linguaggio $L_g$ per cui le regole
  di derivazione di $L_g$ sono $P_g = \{S \to X0Y, X \to 1 | 0X1, Y \to 1001 | 1Y1\}$.
  \end{esempio}

>>>>>>> 6dce03e48312df68955becb5927d04e110d17342
  \begin{esempio}
    Mostriamo ora un esempio di linguaggio definito come l'annidamento di linguaggi:
    \begin{equation*}
      L = \{ w = \{a,b,c,d\}^* | w = a^nb^mc^md^n \, m > 0, n \geq 0\}
    \end{equation*}
    Le regole di produzione del seguente linguaggio sono $P_g = \{S \to aSd | Y, Y \to bYc | bc\}$
    e come si nota il linguaggio viene definito come una sequenza di a e d intrammezzate da un blocco Y, formato da b e c,
    e ciò è l'annidamento tra diversi blocchi di una stringa che formano le stringhe del linguaggio.
  \end{esempio}

  %Inferenza Ricorsiva
<<<<<<< HEAD
  Un'altra modalità per stabilire l'appartenza di una stringa di un linguaggio è l'\emph{inferenza ricorsiva}, in cui, a differenza delle altre
  modalità, applica le regole dal corpo alla testa, ossia concateniamo ogni terminale che appare nel corpo e inferiamo che la stringa trovata
=======
  Un'altra modalità per stabilire l'appartenza di una stringa di un linguaggio è \emph{inferenza ricorsiva}, il quale a differenza delle altre
  applica le regole dal corpo alla testa, ossia concateniamo  ogni terminale che appare nel corpo e inferiamo che la stringa trovata
>>>>>>> 6dce03e48312df68955becb5927d04e110d17342
  è nel linguaggio delle variabili, presenti in testa alle regole.

  Viene poco utilizzato in quanto è più naturale e chiaro pensare secondo la derivazione per cui ne vediamo soltanto un esempio
  e lo usiamo in un importante teorema sull'equivalenza delle modalità di derivazione, che verrà presentato prossimamente.
<<<<<<< HEAD
  \begin{esempio}
    Sia $G=(V,T,O,E)$, con $V=\{E,I\}$ e $T=\{a,b,0,1,(,),+,*\}$
    quindi ho le seguenti regole, è di tipo 3:
    \begin{enumerate}
    \item $E\to I | E+E | E*E | (E)$
    \item $I\to a | b | Ia | Ib | I0 | I1$
    \end{enumerate}
    Voglio ottenere $a*(a+b00)$ e sostituisco sempre a destra (right most derivation):
    \[E\to E*E\to E*(E)\to E*(E+E)\to E*(E+I)\to E+(E+I0)\]
    \[\to R+(I+b00)\to E*(a+b00)\to I*(a+b00)\to a*(a+b00)\]
    usiamo ora \textit{l'inferenza ricorsiva}:
    \begin{table}
      \centering
    \caption{Inferenza ricorsiva di $a*(a+b00)$}
      \begin{tabular}{|c|c|c|c|c|}
        \toprule passo & stringa ricorsiva & var & prod & passo stringa impiegata\\
        \midrule 1 & a & I & 5 & $\backslash$ \\
        \hline
        2 & b & I & 6 & $\backslash$ \\
        \hline
        3 & b0 & I & 9 & 2\\
        \hline
        4 & b00 & I & 9 & 3\\
        \hline
        5 & a & E & 1 & 1 \\
        \hline
        6 & b00 & E & 1 & 4\\
        \hline
        7 & a+b00 & E & 2 & 5,6\\
        \hline
        8 & (a+b00) & E & 4 & 7\\
        \hline
        9 &a*(a+b00) & E & 3 & 5, 8\\
        \hline
      \end{tabular}
    \end{table}
  \end{esempio}

  %Alberi Sintattici
\section{Alberi di Derivazione}
Introduciamo ora un'importante forma grafica per vedere le regole di derivazioni applicate per formare una stringa
\begin{defi}
    Data una grammatica context-free $G$, un albero di derivazione per $G$ è un albero composto come:
=======

\begin{esempio}
Sia $G=(V,T,O,E)$, con $V=\{E,I\}$ e $T=\{a,b,0,1,(,),+,*\}$
quindi ho le seguenti regole, è di tipo 3:
\begin{enumerate}
\item $E\to I$
\item $E\to E+E$
\item $E\to E*E$
\item $E\to (E)$
\item $I\to a$
\item $I\to b$
\item $I\to Ia$
\item $I\to Ib$
\item $I\to I0$
\item $I\to I1$
\end{enumerate}
voglio ottenere $a*(a+b00)$
sostituisco sempre a destra (right most derivation)
$$E\to E*E\to E*(E)\to E*(E+E)\to E*(E+I)\to E+(E+I0)$$
$$\to R+(I+b00)\to E*(a+b00)\to I*(a+b00)\to a*(a+b00)$$

usiamo ora \textit{l'inferenza ricorsiva}:
\begin{center}
\begin{tabular}{|c|c|c|c|c|}
  \toprule passo & stringa ricorsiva & var & prod & passo stringa impiegata\\
\midrule 1 & a & I & 5 & $\backslash$ \\
\hline
2 & b & I & 6 & $\backslash$ \\
\hline
3 & b0 & I & 9 & 2\\
\hline
4 & b00 & I & 9 & 3\\
\hline
5 & a & E & 1 & 1 \\
\hline
6 & b00 & E & 1 & 4\\
\hline
7 & a+b00 & E & 2 & 5,6\\
\hline
8 & (a+b00) & E & 4 & 7\\
\hline
9 &a*(a+b00) & E & 3 & 5, 8\\
\hline
\end{tabular}
\end{center}
\end{esempio}

  %Alberi Sintattici
  Introduciamo ora un'importante forma grafica per vedere le regole di derivazioni applicate per formare una stringa
  \begin{defi}
    Data una grammatica context-free $G$, un albero sintattico per $G$ è un albero composto come:
>>>>>>> 6dce03e48312df68955becb5927d04e110d17342
    \begin{itemize}
       \item ogni nodo interno è etichettato con una variabile $X \in V$, con la radice etichettata con $S$.
       \item ogni foglia è etichettata con una variabile, un simbolo terminale o $\epsilon$; se una foglia viene etichettata
             con $\epsilon$ allora dev essere l'unico figlio del padre.
       \item se un nodo è etichettato con $A$ e i suoi figli sono etichettati come $x_1, x_2, \dots, x_k$, allora $A \to x_1x_2 \dots x_k$
             è una regola di produzione della grammatica.
    \end{itemize}
  \end{defi}
<<<<<<< HEAD
Eseguendo la concatenazione delle stringhe foglia otteniamo una stringa $w$ appartenente alla grammatica di cui abbiamo svolto l'albero sintattico
=======
  Eseguendo il prodotto delle stringhe foglia otteniamo una stringa $w$ appartenente alla grammatica di cui abbiamo svolto l'albero sintattico

>>>>>>> 6dce03e48312df68955becb5927d04e110d17342
  %Esempio di Albero Sintattico
  \begin{esempio}
    Prendendo il linguaggio $L$ definito come segue:$L = \{w \in \{a,b,c,d\}^* | w = a^ncb^mcd^{n+m} n \geq 0,m > 0\}$
    stabiliamo una grammatica context-free e fare l'albero sintattico di $aacbbbcddddd \in L$:
    le regole di produzione di $L$ sono $P_g = \{S \Rightarrow aSd | cY, Y \Rightarrow bcd | bYd\}$ per cui la grammatica è:
    \begin{equation*}
      G = (\{S,Y\},\{a,b,c,d\},P_g,S)
    \end{equation*}
    L'albero sintattico di $aacbbbcddddd$ è il seguente:

  \end{esempio}

  \section{Equivalenza tra le derivazioni}
  In questo paragrafo consideriamo un importante teorema sulle equivalenze tra le varie modalità di derivazione, definito come segue:
  \begin{thm}
    Data una grammatica context-free $G = (V, T,P_g,S)$ abbiamo che le seguenti modalità di derivazione sono equivalenti:
    \begin{enumerate}
    \item la procedura di inferenza ricorsiva che determina che una stringa di terminali $w$ appartiene ad un linguaggio di variabili $A$
    \item $A \Rightarrow ^ * w$
    \item $A \Rightarrow _{lm} ^ * w$
    \item $A \Rightarrow _{rm} ^ * w$
    \item esiste un albero sintattico con radice $A$ e come prodotto di foglie la stringa $w$.
    \end{enumerate}
  \end{thm}
<<<<<<< HEAD
  \begin{esempio}
Usiamo l'esempio delle stringhe palindrome:
$$P\to 0P0|\,1P1|\epsilon$$
sia il seguente albero sintatico:

\begin{esempio}
Si ha:
$$E\to I|\, E+E|\, E*E|\, (E)$$
$$I\to a|\,b|\,Ia|\,Ib|\,I0|\,I1$$
un albero sintattico per $a*(a+b00)$ può essere:



queste 5 proposizioni si implicano l'uni l'altra:
\begin{center}
\begin{tikzpicture}
    \node (top) at (0,0) {5};
 	\node (a) at(-1,-0.5) {3};
 	\node (b) at(0,-1) {4};
 	\node (c) at(-2.0,-1.85) {2};
 	\node (d) at(1.5,-2) {1};
    \draw [->] (top) -- (a);
    \draw [->] (top) -- (b);
    \draw [->] (a) -- (c);
    \draw [->] (b) -- (c);
    \draw [->] (c) -- (d);
    \draw [->] (d) -- (top);
\end{tikzpicture}
\end{center}
Le prime due sono banali dato che una derivazione sinistra/destra sono anche delle derivazioni quindi risulta verificato.
Vediamo le altre dimostrazioni di implicazioni:
\begin{proof}[da 1 a 5]
si procede per induzione:
\begin{itemize}
\item \textbf{caso base:} ho un livello solo (una sola riga), $\exists A\to w$:
$$\overset{A}{\overset{\triangle}w}$$
\item \textbf{caso passo:} suppongo vero per un numero di righe $\leq n$, lo dimsotro per $n+1$ righe:
$$A\to X_1,X_2,...,X_k$$
$$w=w_1,w_2,...,w_k$$
ovvero, in meno di $n+1$ livelli:
\end{itemize}
\end{proof}
\begin{proof}[da 5 a 3]
procedo per induzione:
\begin{itemize}
\item \textbf{caso base (n=1): }$\exists A\to w\mbox{ quindi } A\to_{lm}w$, come prima si ha un solo livello:
$$\overset{A}{\overset{\triangle}w}$$
\item \textbf{caso passo: }suppongo che la proprierà valga per ogni albero di profondità minore uguale a $n$, dimostro che valga per gli alberi profondi $n+1$:
$$A\to X_1,X_2,...,X_k$$
$$w=w_1,w_2,...,w_k$$
ovvero, in meno di $n+1$ livelli:
\begin{center}

%\psframebox[linestyle=none,framesep=10pt]{%
%pstree{\LFTw{t}{\fontspec{Noto Sans}[Script=Latin]A}}{\pstree{\Tp[edge=none]}{%
%  \LFTw{t}{\fontspec{Noto Sans}[Script=Latin]$\overset{X_1}{\overset{\triangle}w_1}$}
%  \LFTw{t}{\fontspec{Noto Sans}[Script=Latin]$\overset{X_2}{\overset{\triangle}w_2}$}
%  \LFTw{t}{\fontspec{Noto Sans}[Script=Latin]$\vdots$}
%  \LFTw{t}{\fontspec{Noto Sans}[Script=Latin]$\overset{X_k}{\overset{\triangle}w_k}$}}}}
\end{center}
$$A\to_{lm} X_1,X_2,...,X_k$$
$$x_1\to^*_{lm}w_1 \mbox{ per ipotesi induttiva si ha un albero al più di n livelli}$$
quindi:
$$A\to_{lm}X_1,...,X_k\to^*_{lm}w_1,X_2,...,X_k\to^*_{lm}...\to^*_{lm}w_1,...,w_k=w$$
\end{itemize}
\begin{esempio}
$$E\to I\to Ib\to ab$$
$$\alpha E\beta\to\alpha I\beta\to \alpha Ib\beta\to \alpha ab\beta,\,\,\,\alpha,\beta\in(V\cup T)^*$$
\end{esempio}
\end{proof}
\begin{esempio}
Mostro l'esistenza di una derivazione sinistra dell'albero sintattico di $a*(a+b00)$:
$$E\to^*_{lm}E*E\to^*_{lm}I*E\to^*_{lm}a*E\to^*_{lm}a*(E)\to^*_{lm}a*(E+E)\to^*_{lm}$$
$$a*(I+E)\to^*_{lm}a*(a+E)\to^*_{lm}a*(a+I)\to^*_{lm}a+(a+I0)\to^*_{lm}a*(a+I00)\to^*_{lm}a*(a+b00)$$
\end{esempio}

\section{Grammatiche e Linguaggi Ambigui}
Fino ad ora abbiamo considerato delle grammatiche uniche, ossia in grado di generare in maniera univoca un linguaggio ma non è sempre così,
infatti in questo paragrafo consideramo il problema dell'ambiguità nelle grammatiche e nelle grammatiche.

Diamo ora una definizione formale di grammatica ambigua:
=======
  %Immagine per vedere la modalità di equivalenza

  I primi due sono facilmente dimostrabili dato che le derivazione a sinistra e a destra sono delle derivazioni invece iniziamo a dimostrare:
  \begin{thm}

    pippo
  \end{thm}


  \section{Grammatiche e Linguaggi Ambigui}
  In questo paragrafo analizziamo le grammatiche e i linguaggi ambigui, incominciando da un esempio per catturare gli aspetti essenziali,
  che poi verranno definiti formalmente.

  Incominciamo per cui a considerare le espressioni algebriche definite su $T = \{+, *, a, b, (, )\}$ con le seguenti regole di produzione:
  \begin{equation*}
    P_g = \{E \Rightarrow I | E + E | E * E | (E), I \Rightarrow a | b | aI | bI \}
  \end{equation*}
  La grammatica per definire gli identificatori $I$ è una grammatica di tipo 3, cosa che analizzeremo nel prossimo capitolo, però
  questa grammatica è ambigua dato che preso ad esempio $a + b * a$ avremmo due derivazioni left-most, che sono le seguenti:
  %Alberi sintattici di a + b * a

  Diamo ora una definizione formale di grammatica ambigua:
>>>>>>> 6dce03e48312df68955becb5927d04e110d17342
  \begin{defi}
    Si dice che una grammatica è ambigua se e solo se esiste una stringa $w \in L$ tale per cui $w$ ammette due derivazioni $lm$ e/o $rm$
    diverse oppure ammette due alberi sintattici definiti sulla stessa grammatica $G$
  \end{defi}
  L'obiettivo è quello di avere grammatiche non ambigue perchè esse permettono di definire in maniera univoca, per cui automamilizzabile
  con una procedura, l'insieme delle stringhe del linguaggio e fortunatamente nella maggior parte dei casi, quando il linguaggio non è ambiguo,
  data una grammatica ambigua è possibile definire una grammatica non ambigua $G'$ tale che $L(G) = L(G')$.
<<<<<<< HEAD
\begin{esempio}
vediamo un esempio:

La grammatica data delle espressioni algebriche ha due cause di ambiguità:
\begin{enumerate}
  \item la precedenza degli operatori non viene rispettata in quanto andrebbe raggruppato prima il silmbolo di $*$ rispetto a $+$
  \item una sequenza di uguali operatori può essere reggruppata sia da sinistra che da destra e si stabilisce per convenzione
    che si raggruppa da sinistra a destra.
\end{enumerate}
=======

  \begin{defi}
    Un linguaggio $L \subseteq \sum^*$ è detto \emph{ambiguo} se per ogni grammatica $G$, tale per cui $L = L_G$, risulta che $G$ è ambigua.
  \end{defi}
  La grammatica data delle espressioni algebriche ha due cause di ambiguità:
  \begin{enumerate}
  \item la precedenza degli operatori non viene rispettata in quanto andrebbe raggruppato prima il silmbolo di $*$ rispetto a $+$
  \item una sequenza di uguali operatori può essere reggruppata sia da sinistra che da destra e si stabilisce per convenzione
    che si raggruppa da sinistra a destra.
  \end{enumerate}
>>>>>>> 6dce03e48312df68955becb5927d04e110d17342
  Considerate queste cause di ambiguità, per eliminarla si introducono altre variabili definite come:
  \begin{enumerate}
  \item un fattore è un espressione che non può essere spezzata da nessun operatore, sia il $*$ che il $+$, per cui gli unici fattori
    nel nostro linguaggio delle espressioni sono gli identificatori e ogni espressioni dentro le parentesi.
  \item un termine è un espressione che non può spezzata dall'operatore $+$.
  \item un espressione può riferirsi a qualsiasi possibile espressione, incluse quelle che possono essere spezzate da un adiacente $*$ o $+$
  \end{enumerate}
  La grammatica non ambigua del linguaggio delle espressioni è la seguente:
  \begin{equation*}
    P_{exp} = \{E \to T | E + T, T \to F | T * F,F \to I | (E), I \to a | b | aI| bI | 0I | 1I\}
  \end{equation*}
<<<<<<< HEAD
\end{esempio}
Possono esserci più derivazioni di una stringa ma l'importante è che non ci siano alberi sintattici diversi e capire se una CFG è ambigua
è un problema indecidibile, per cui molto complesso ed oneroso.
\begin{esempio}
vediamo un esempio:
\[ S\to \epsilon|\,SS|\, iS|\, iSeS \]
con S=statement, i=if e e=else.
Considero due derivazioni:
\begin{enumerate}
\item $S\to iSeS\to iiSeS\to iie$:
  Fare albero sintattico!!!!
\end{enumerate}
Si ha quindi una grammatica ambigua
\end{esempio}
Per risolvere codesto problema nei linguaggi di programmazione i compilatori assumono la consecutine di associare l'else all'ultimo if

 \begin{defi}
    Un linguaggio $L \subseteq \sum^*$ è detto \emph{ambiguo} se per ogni grammatica $G$, tale per cui $L = L_G$, risulta che $G$ è ambigua.
  \end{defi}
\begin{esempio}
Sia $L=\{a^nb^nc^md^m|\, n,m\geq 1\}\cup \{a^nbmnc^md^n|\, n,m\geq 1\}$\\
si ha quindi un CFL formato dall'unione di due CFL. $L$ è inerentemente ambiguo e generato dalla seguente grammatica:
\begin{itemize}
\item $S\to AB|\,C$
\item $A\to aAb|\,ab$
\item $B\to cBd|\, cd$
\item $C\to aCd|\, aDd$
\item $D\to bDc|\, bc$
\end{itemize}
si possono avere due derivazioni:
\begin{enumerate}
\item $S\to_{lm}AB\to_{lm} aAbB\to_{lm} aabbB\to_{lm}aabbcBd\to_{lm}aabbccdd$
\item $S\to_{lm} C\to_{lm} aCd\to_{lm}aaBdd\to_{lm}aabBcdd\to_{lm}aabbccdd$
\end{enumerate}
a generare problemi sono le stringhe con n=m perché possono essere prodotte in due modi diversi da entrambi i sottolinguaggi. Dato che l'intersezione tra i due sottolinguaggi non è vuota si ha che $L$ è ambiguo
\end{esempio}

\section{Linguaggi e grammatiche dipendenti dal contesto}
vediamo un esempio di grammatica dipendente dal contesto:
\[L=\{a^nb^nc^n|\, n\geq 1\} \text{la cui grammatiche che lo genera è la seguente:\} \]
\[G=\{V,T,P,S\}=\{(S,B,C,X)\}=\{(a,b,c),P,S\}\].\newline
Ecco le regole di produzione, e le grammatiche di tipo 1 posso scambiare variabili a differenza delle context free:
\begin{enumerate}
\item $S\to aSBC$
\item $S\to aBC$
\item $CB\to XB$
\item $XB\to XC$
\item $XC\to BC$
\item $aB\to ab$
\item $bB\to bb$
\item $bC\to bc$
\item $cC\to cc$
\end{enumerate}
vediamo un esempio di derivazione:
per $n=1$ ho $abc$ ovvero $S \to aBC \to abC \to abc$.\newline
con $n=2$ ho $aabbcc$:
\[ S\to aSBC\to aaBCBC\to aaBXBC\to aaBXCC\to aaBBCC\to aabBCC\to aabbCC\to aabbcC\to aabbcc \]
Vedere dimostrazione pag 14 di Lorenzo Soligo
%vedere dimostrazione pag 14 soligo
\end{esempio}
\begin{esempio}
vediamo un esempio di grammatica dipendente dal contesto:
$$L=\{a^nb^mc^nd^m|\, n,m\geq 1\}$$
Si ha:
$$G=(\{S,X,C,D,Z\},\{a,b,c,d\},P,S)$$
con le seguenti regole di produzione:
\begin{itemize}
\item $S\to aSc|\, aXc$
\item $X\to bXD|\, bD$
\item $DC\to CD$
\item $DC\to DZ$
\item $DZ\to CZ$
\item $XZ\to CD$
\item $bC\to bc$
\item $cC\to cc$
\item $cD\to cd$
\item $dD\to dd$
\end{itemize}
provo a derivare $aabbbccddd$ quindi con $n=2,\,m=3$:\\
$$S\to aSC\to aaXCC\to aabXDCC\to aabbXDDCC\to $$
$$aabbbDDDCC\to aabbbCCDDD\to aabbbccddd$$
\end{esempio}
\begin{esempio}
Sia $L=\{w\in\{a,b\}^*|\, \mbox{ w contiene lo stesso numero di a e b}\}$:
$$S\to aSbS|\,bSaS|\, \varepsilon$$
dimostro per induzione che è corretto:
\begin{itemize}
\item \textbf{caso base:} $|w|=0\to w=\varepsilon$
\item \textbf{caso passo:} si supponga che $G$ produca tutte le stringhe (di lunghezza $<$ di $n$) di $\{a,b\}^*$ con lo stesso numero di \textit{a} e \textit{b} e dimostro che produce anche quelle di lunghezza $n$, sia:
$$w\in \{a,b\}^* \mid\, |w|=n \mbox{ con\textit{ a} e \textit{b} in egual numero, }m(a)=m(b) \mbox{ con m() che indica il numero di caratteri}$$
quindi si ha che:
$$w=aw_1bw_2\mbox{ o } w=bw_1aw_2$$
sia.
$$k_1=m(a)\in w_1=m(b)\in w_1$$
$$k_2=m(a)\in w_2=m(b)\in w_2$$
allora:
$$k_1+k_2+1=m(a)\in w= m(b)\in W$$
sapendo che $|w_1|<n$ e $|w_2|<n$ allora $w_1$ e $w_2$ sono egnerati da G per ipotesi induttiva
\end{itemize}
\end{esempio}

\chapter{Linguaggi Regolari}
  Per definire le stringhe appartenenti ai linguaggi regolari, di tipo 3, vi può utilizzare le \emph{grammatiche regolari},
  in cui vengono definite delle regole per stabilire se e quando una stringa appartiene al linguaggio, oppure le \emph{espressioni regolari}.
=======

  Mostriamo ora un esempio di linguaggio ambiguo $L$ definito come segue:
  \begin{equation*}
    L = \{w = a^nb^nc^md^m | n,m \geq 1\} \cup \{z = a^nb^mc^md^n | n,m \geq 1 \}
  \end{equation*}
  Le regole di produzione che generano il seguente linguaggio sono le seguenti:
  \begin{equation*}
    P_L = \{S \to XY | Z, X \to ab | aXb, Y \to cd | cYd, Z \to aZd | aRd, R \to bc | bRc\}
  \end{equation*}
  Questa grammatica è ovviamente ambigua dato che per esempio la stringa $aabbccdd$ ha due derivazioni sinistre:

  La dimostrazione che il linguaggio è ambiguo è molto complessa ma il concetto alla base è quello di provare che tutte le grammatiche
  del linguaggio generano un numero finito di stringhe con almeno due derivazioni sinistre/destre e nel nostro caso definiamo che il
  linguaggio è ambiguo perche per $n = m$ qualsiasi grammatica usiamo per definire il linguaggio avremmo i due sottolinguaggi che generano
  la stessa stringa in maniera diversa.

  \chapter{Linguaggi Regolari}
  Per definire le stringhe appartenenti ai linguaggi regolari, di tipo 3, vi può utilizzare le \emph{grammatiche regolari},
  in cui vengono definite delle regole per stabilire se e quando una stringa appartiene al linguaggio, e le \emph{espressioni regolari}
>>>>>>> 6dce03e48312df68955becb5927d04e110d17342

  Incominciamo a considerare le grammatiche regolari, sottoinsieme delle grammatiche di tipo 2 secondo la gerarchia di Choumsky,
  utilizzate per generare i linguaggi regolari.

<<<<<<< HEAD
  Si ha la solita grammatica $G = (V, T, P, S)$ con però vincoli su $P$:
  \begin{itemize}
  \item $\epsilon$ si può ottenere solo con $S\to \epsilon$
  \item le produzioni sono tutte lineari a destra ($A\to aA$ o $A\to a$) o a sinistra ($A\to Ba$ o $A\to a$)
  \end{itemize}
\begin{esempio}
$I\to a|\,b|\,Ia|\,Ib|\,I0|\,I1$ è una grammatica con le produzioni lineari a sinistra.\newline
Potremmo pensarlo a destra $I\to a|\,b|\,aI|\,bI|\,0I|\,1I$.\newline
Vediamo esempi di produzione con queste grammatiche:
\begin{itemize}
\item con $I\to a|\,b|\,Ia|\,Ib|\,I0|\,I1$ possiamo derivare $ab01b0$:
\[ I\to I0\to Ib0\to I1b0\to I01b0\to Ib01b0\to ab01b0 \]
\item con $I\to a|\,b|\,aI|\,bI|\,0I|\,1I$ invece non riusciamo a generare nulla:
\[ I\to 0I\to 0a \]
\end{itemize}
Definisco quindi un'altra grammatica (con una nuova categoria sintattica):
\[ \begin{aligned}
  I\to aJ|\, bJ \\
  J\to a|\,b|\,aJ|\,bJ|\,0J|\,1J\\
  \end{aligned} \]
che però non mi permette di terminare le stringhe con 0 e 1, la modifico ancora otterdendo:
\[ \begin{aligned}
  I\to aJ|\, bJ\\
  J\to a|\,b|\,aJ|\,bJ|\,0J|\,1J|\,0|\,1\\
  \end{aligned} \]
e questo è il modo corretto per passare da lineare sinistra a lineare destra
\end{esempio}

\begin{esempio}
Sia $G=(\{S\},\{0,1\},P,S)$ con $S\to \epsilon|\,0|\,1|\,0S|\,1S$ e si ha quindi:
\[ L(G)=\{0,1\}^* \]
Si hanno comunque due proposizioni ridondanti, riducendo trovo:
\[ S\to \epsilon|\, 0S|\,1S \]
con solo produzioni lineari a destra mentre usando le produzioni lineari a sinistra ottengo:
\[ S\to \epsilon|\, S0|\,S1 \]
\end{esempio}
\begin{esempio}
Trovo una grammatica lineare destra e una sinistra per $L=\{a^nb^m|\,n,m\geq 0\}$:
\begin{itemize}
\item \textbf{lineare a destra:} si ha $G=(\{S,B\},\{a,b\},P,S)$ e quindi:
$$S\to \varepsilon|\,aS|\,bB$$
$$B\to bB|\,b$$
ma non si possono generare stringhe di sole $b$, infatti:
$$S\to aS\to abB\to abbB\to abbb$$
ma aggiungere $\varepsilon$ a B \textbf{non è lecito}. posso però produrre la stessa stringa da due derivazioni diverse:
$$S\to \varepsilon|\,aS|\,bB|\,b$$
$$B\to bB|\,b$$
che risulta quindi la nostra lineare a destra
\item \textbf{lineare a sinistra:} si ha $G=(\{S,A\},\{a,b\},P,S)$ e quindi:
$$S\to \varepsilon|\,Sb|\,Ab|\,a$$
$$A\to Aa|\,a$$
\end{itemize}
\end{esempio}
\begin{esempio}
Trovo una grammatica lineare destra e una sinistra per $L=\{ab^ncd^me|\,n\geq 0\,,m> 0\}$:
\begin{itemize}
\item \textbf{lineare a destra:} si ha  si ha $G=(\{S,A,B,E\},\{a,b,c,d,e\},P,S)$ e quindi:
$$S\to aA$$
$$A\to bA|\,cB$$
$$B\to dB|\, dE$$
$$E\to e$$
\item \textbf{lineare a sinistra:} si ha  si ha $G=(\{S,X,Y,Z\},\{a,b,c,d,e\},P,S)$ e quindi:
$$S\to Xe$$
$$A\to Xd|\,Yd$$
$$B\to Zc$$
$$E\to a|\,Zb$$
\end{itemize}
quindi se per esempio ho la stringa "ciao" si ha:
\begin{itemize}
\item \textbf{lineare a destra:}
$$S\to Ao$$
$$A\to Ba$$
$$B\to Ei$$
$$E\to c$$
\item \textbf{lineare a sinistra:}
$$S\to cA$$
$$A\to iB$$
$$B\to aE$$
$$E\to o$$
\end{itemize}
\end{esempio}
\begin{esempio}
A partire da $G=(\{S,T\},\{0,1\},P,S)$ con:
$$S\to\varepsilon|\,0S|\,1T$$
$$T\to 0T|\,1S$$

trovo come è fatto $L(G)$:
$$L(G)=\{w\in\{0,1\}^*|\, w \mbox{ ha un numero di 1 pari}\}$$
\end{esempio}
\begin{esempio}
fornire una grammatica regolare a destra e sinistra per:
$$L=\{w\in\{0,1\}^*|\, w \mbox{ ha almeno uno 0 o almeno un 1}\}$$
Si ah che tutte le stringhe tranne quella vuota ciontengono uno 0 o un 1
quindi  $G=(\{S\},\{0,1\},P,S)$:
\begin{itemize}
\item \textbf{lineare a destra:}
$$S\to 0|\,1|\,0S|\,1S$$
\item \textbf{lineare a sinistra:}
$$S\to 0|\,1|\,S0|\,S1$$
\end{itemize}
\end{esempio}

\section{Espressioni Regolari}
=======
  Prevedono dei maggiori vincoli sulle produzioni infatti:
  \begin{enumerate}
  \item $\epsilon$ può apparire solo nel corpo del simbolo start ossia può essere solo $S \to \epsilon$.
  \item le produzioni sono del seguente tipo, dove $a \in T$ e $A, B \in V$,:
    \begin{itemize}
    \item $A \to aB$ oppure $A \to a$(lineari a destra)
    \item $A \to Ba$ oppure $A \to a$(lineari a sinistra)
      \end{itemize}
  \item non si può avere produzioni sia lineari sinistre che lineari destre
  \end{enumerate}

  Esempio: i due insiemi di produzioni per definire gli identificatori di un espressione algebriche sono le seguenti:
  \begin{align*}
    P_1 = \{I \to a | b | Ia| Ib| I0 | I1\} \quad \text{lineare a destra}
    P_2 = \{I \to aJ | bJ, J \to aJ | bJ | 0J | 1J | 0 | 1 | a | b\} \quad \text{lineare a sinistra}
  \end{align*}

  Esempio:data la grammatica $G(\{S\},\{0,1\},\{S \to \epsilon | 0S | 1S\},S)$ capire che linguaggio rappresenta:
  $L = \{\epsilon, 0, 1, 00, 01, 10, 11, 000, 001, 010, 011, 100, 101, 110, 111, \dots \}$

  Esempio:Produrre le produzioni lineari sinistra e lineari destra del seguente linguaggio:
  \begin{equation*}
    L = \{w \in \{a, b\}* | w = a^nb^m n,m \geq 0\}
  \end{equation*}

  Esempio: Produrre le produzioni lineari sinistra e lineari destra del seguente linguaggio:
  \begin{equation*}
    L = \{w \in \{a,b,c,d,e\}* | w = ab^ncd^me n \geq 0, m > 0\}
  \end{equation*}

  Esempio: Fornire le produzioni lineari sinistra e lineari destra del seguente linguaggio:
  \begin{equation*}
    L = \{w \in \{0,1\}* | \text{w contiene almeno uno 0 oppure almeno un 1}\}
  \end{equation*}

  \section{Espressioni Regolari}
>>>>>>> 6dce03e48312df68955becb5927d04e110d17342
  Le espressioni regolari permettono di definire, utilizzando una notazione algebrica, un linguaggio regolare e vengono utilizzate
  per estrarre parole da un testo ed altre notevole applicazioni, che verranno analizzate nel corso dei paragrafi.

  Per riuscire a definire in maniera formale le espressioni regolari dobbiamo definire le seguenti operazioni sui linguaggi regolari:
  \begin{itemize}
<<<<<<< HEAD
  \item Unione: dati due linguaggi $L,M \subseteq \Sigma*$ si definisce $L \cup M$ come:
    \[ L \cup M = \{w \in \Sigma* | w \in L \lor w \in M\} \]
    Esempio: $L = \{001, 10, 111\}$ e $M = \{\epsilon, 001\}$ risulta $L \cup M = \{\epsilon, 10, 001, 111\}$.\newline
    Risulta verificata la seguente equivalenza $L\cup M \equiv M\cup L$.

  \item Concatenazione: dati due linguaggi $L, M \subseteq \Sigma*$ si ha $L \circ M = LM$, ossia il linguaggio formato da tutte le strighe
    ottenute concatenando le stringhe in $L$ con le stringhe in $M$.
    Esempio: $L = \{001,10\}$ e $M = \{\epsilon, 111\}$ risulta $L \circ M = \{001, 10, 001111, 10111\}$.
=======
  \item Unione: dati $L,M \subseteq \Sigma*$ si definisce $L \cup M = \{w \in \Sigma* | w \in L \vee w \in M\}$
    Esempio: $L = \{001, 10, 111\}$ e $M = \{\epsilon, 001\}$ risulta $L \cup M = \{\epsilon, 10, 001, 111\}$

  \item Concatenazione: dati due linguaggi $L, M \subseteq \Sigma*$ si ha $L \circ M = LM$, ossia linguaggio formato da tutte le strighe
    ottenute concatenando le stringhe in $L$ con le stringhe in $M$.
    Esempio: $L = \{001,10\}$ e $M = \{\epsilon, 111\}$ risulta $L \circ M = \{001, 10, 001111, 10111\}$
>>>>>>> 6dce03e48312df68955becb5927d04e110d17342

  \item Chiusura di Kleene: dato un linguaggio $L$ si ha $L*$ definito induttivamente come:
    \begin{equation*}
    \begin{split}
<<<<<<< HEAD
      L^0 = \{\epsilon\} \\
=======
      L^0 = \{epsilon\}
>>>>>>> 6dce03e48312df68955becb5927d04e110d17342
      L^1 = L \\
      L^2 = L \circ L \\
      \dots \\
      L^i = L^{i-1} \circ L\\
      L^* = L^0 \cup L^1 \cup L^2 \cup \dots \\
      L^+ = L^* - \{epsilon\} \\
    \end{split}
    \end{equation*}
    Esempio: dato $L = \{0,1\}$ abbiamo:
<<<<<<< HEAD
    \[
=======
    \begin{equation*}
>>>>>>> 6dce03e48312df68955becb5927d04e110d17342
    \begin{split}
      L^0 = \{epsilon\}\\
      L^1 = \{0, 1\}\\
      L^2 = \{00, 01, 10, 11\}\\
<<<<<<< HEAD
      L^3 = \{000, 001, 010, 011, 100, 101, 110, 111\} \\
    \end{split}
    \]
=======
      L^3 = \{0000, 0001, 0010, 0011, 0100, 0101, 0110, 0111, 1000, 1001, 1010, 1011, 1100, 1101, 1110, 1111\} \\
    \end{split}
    \end{equation*}
>>>>>>> 6dce03e48312df68955becb5927d04e110d17342
    Il linguaggio $L*$ è generalmente un linguaggio infinito in quanto è l'unione di un numero infinito di linguaggi finiti
    ma esistono due linguaggi la cui chiusura è finita, che analizziamo ora:
    \begin{itemize}
    \item il linguaggio $L = \{0\}$ la sua chiusura di kleene è finita dato che si ha:
<<<<<<< HEAD
      \[
=======
      \begin{equation*}
>>>>>>> 6dce03e48312df68955becb5927d04e110d17342
      \begin{split}
        L^0 = \epsilon \\
        L^1 = L\\
        L^2 = L \circ L = L\\
        L^3 = L^2 \circ L = L\\
        L^i = L^{i-1} \circ L = L \\
        L^* = L^0 \cup L^1 \cup L^2 \cup \dots = L \cup {epsilon} = L\\
      \end{split}
<<<<<<< HEAD
      \]

    \item il linguaggio $L = \emptyset$ la sua chiusera di kleene è finita in quanto:
      \[
=======
      \end{equation*}
    \item il linguaggio $L = \emptyset$ la sua chiusera di kleene è finita in quanto:
      \begin{equation*}
>>>>>>> 6dce03e48312df68955becb5927d04e110d17342
      \begin{split}
      L^0 = \epsilon \\
      L^1 = \emptyset \\
      L^2 = \emptyset \circ \emptyset = \emptyset \\
      L^i = \emptyset^i \circ \emptyset = \emptyset \\
      L^* = \emptyset \cup \{\epsilon\} = \{epsilon\} \\
      \end{split}
<<<<<<< HEAD
      \]
    \end{itemize}%itemize dei linguaggi la cui chiusura è finita
  \end{itemize}%itemize delle operazioni dei linguaggi regolari

  Dopo aver definito le operazioni sui linguaggi regolari, definiamo ora le espressioni regolari, molto utilizzate per rappresentare in
  maniera algebrica la grammatica dei linguaggi regolari:
\begin{defi}
Si definisce \emph{espressione regolare} induttivamente come segue, considerando anche il linguaggio che generano:
   \begin{description}
   \item [caso base]: la base consiste in 3 parti:
        \begin{enumerate}
        \item $\epsilon$ e $\emptyset$ sono Regex e generano $L(\epsilon) = \{epsilon\}$, $L(\emptyset) = \emptyset$
        \item se $a$ è un simbolo allora $a$ è una Regex e questa espressione genera $L(a) = \{a\}$
        \item una variabile, rappresentanti linguaggi regolari, sono Regex, e generano $L(L) = L$
        \end{enumerate}

   \item [caso induttivo]: la parte induttiva delle espressioni regolari sono composte da 4 tipologie:
        \begin{enumerate}
        \item se $E$ e $F$ sono delle Regex allora $E + F$ è una Regex e rappresentano $L(E + F) = L(E) \cup L(F)$
        \item se $E$ e $F$ sono delle Regex allora $E \circ F = EF$ è una Regex per cui rappresentano $L(EF) = L(E)L(F)$
        \item se $E$ è una Regex allora $E*$ è una RegEx, che denota la chiusura di $L(E)$ infatti $L(E^*) = (L(E))^*$
        \item se $E$ è una Regex allora $(E)$ è una Regex in cui $L((E)) = L(E)$.
        \end{enumerate}
  \end{description}
\end{defi}

Esempio: Data l'espressione regolare $Regex = 01$ si ha allora $L(01) = L(0) \circ L(1) = 0 \circ 1 = 01$.

Al fine di ridurre la lunghezza delle espressioni regolari per migliorarne la leggibilità e la comprensione, si introducono
delle proprietà algebriche iniziando prima dalla definizione di espressioni equivalenti:
\begin{defi}
  Due espressioni regolari sono equivalenti se denotano lo stesso linguaggio.
  Due espressioni regolari con variabili sono equivalenti se e solo se sono equivalenti per ogni assignamento alle variabili.
\end{defi}
Per sapere quali operazioni in una Regex viene eseguita, si introduce la precedenza degli operatori, eseguiti da sinistra a destra:
\begin{itemize}
  \item $^*$ e si applica alla sequenza più piccola a sinistra che sia anche un'espressione regolare
  \item $\circ$  applicato da sinistra a destra
  \item $+$  viene valutato da sinistra a destra
  \item la parentesi $( )$ permette di isolare il contenuto dentro e stabilire l'ordine di applicazione
\end{itemize}

Gli operatori delle espressioni regolari possiedono le seguenti proprietà:
\begin{itemize}
\item Unione: l'unione si può vedere come l'addizione nell'aritmetica dato che possiede le stesse proprietà infatti:
  \begin{itemize}
  \item \textbf{Commutatività}: dati due linguaggi $L$ e $M$ risulta $L + M = M + L$
  \item \textbf{Associavità}: dati tre linguaggi $L, M$ e $N$ risulta $(L + M) + N = L + (M + N)$
  \item \textbf{Identità}: l'identità per l'unione è l'insieme $\emptyset$ infatti risulta verificato $L + \emptyset = L = \emptyset + L$
  \item \textbf{Idempotenza}: dato un linguaggio $L$ risulta $L + L = L$.
  \end{itemize}
\item la Concatenazione presenta delle analogie con la moltiplicazione infatti possiede le seguenti proprietà:
  \begin{itemize}
  \item \textbf{Associavità}: dati tre linguaggi $L, M$ e $N$ risulta $(LM)N = L(MN)$
  \item \textbf{Identità}: l'identità per la concatenazione è l'$\epsilon$ attraverso cui risulta $\epsilon L = L = L \epsilon$
  \item \textbf{Annichilatore}: l'annichilatore per la concatenazione è $\emptyset$ in quanto risulta $\emptyset L = \emptyset = L \emptyset$.
    L'annichilatore risulta molto utile per effettuare delle utili ed importanti semplificazioni.
  \end{itemize}
  L'unione e la concatenazione possiedono le proprietà distributive della concatenazione rispetto all'unione:
  \begin{itemize}
  \item $L(M + N) = LM + LN$ \quad (legge distribuitiva sinistra della concatenazione rispetto all'unione)
  \item $(M + N)L = ML + NL$ \quad (legge distribuitiva destra della concatenazione rispetto all'unione)
  \end{itemize}
\item la chiusura possiede le seguenti proprietà:
  \begin{itemize}
  \item $(L^*)^* = L^*$
  \item $\emptyset ^ * = \epsilon$
  \item $\epsilon ^ * = \epsilon$
  \item $L^+ = LL^* = L^* L$
  \item $L^* = L^+ + \epsilon$
  \item $L? = \epsilon + L$
  \end{itemize}
\end{itemize}

Esempio:Scrivere la regex del seguente linguaggio:
\[ L = \{ w \in \{a,b,c,d\}^* | w = a^n bcd^m \ \text{con} n > 0, m\geq 0\} \]
Dato che si ha $n > 0$ ci sarà per forza una $a$ per cui per generare $a^n$ si ha $aa^*$ mentre essendo $m \geq 0$ per
definire $d^m$ si usa $d*$.\newline
Fatte codeste considerazioni si ottiene Regex = $aa^*bcd*$.

Scrivere la regex del seguente linguaggio $L = \{ w \in {0,1\}^* | w = \text{stringhe in cui compare 10}\}$.\newline
  La regex per generare il linguaggio è $(0+1)^*10(0+1)^*$ in quanto (0+1)^* mi genera una qualsiasi stringa di $0$ e $1$ mentre
  $10$ mi assicura di avere tutte e sole le strighe in cui compare $10$.

\chapter{Automi a stati finiti}
Un automa a stati finiti
Un automa a stati finiti ha un insieme di stati e un controllo che si muove da stato a stato in risposta a input esterni. Si ha una distinzione:
\begin{itemize}
\item \textbf{automi deterministici:} dove l'automa non può essere in più di uno stato per volta
\item \textbf{automi non deterministici:} dove l'automa può trovarsi in più stati contemporaneamente
\end{itemize}
\subsection{Automi deterministici}
Un automa a stati finiti deterministico (\textit{DFA}), un automa che dopo aver letto una qualunque sequenza di input si trova in un singolo stato. Il termine \textit{deterministico} concerne il fatto che per ogni input esiste un solo stato verso il quale l'automa passa dal suo stato corrente. Un automa a stati finiti deterministico consiste nelle seguenti parti:
\begin{itemize}
\item un insieme finito di stati, spesso indicato con $Q$
\item un insieme finito di simboli di input , spesso indicato con $\Sigma$
\item una funzione di transizione, che prende come argomento uno stato e un simbolo di input e restituisce uno stato. La funzione di transizione sarà indicata comunemente con $\delta$. Nella rappresentazione grafica informale di automi $\delta$ è rappresentata dagli archi tra gli stati e dalle etichette sugli archi. Se $q$ è uno stato e $a$ è un simbolo di input, $\delta(q,a)$ è lo stato $p$ tale che esiste un arco etichettato con $a$ da $q$ a $p^2$
\item uno stato iniziale, uno degli stati in $Q$
\item un insieme di stati finali, o accettanti , $F$. L'insieme $F$ è un sottoinsieme di $Q$
\end{itemize}
Nel complesso un DFA è rappresentato in maniera concisa con l'enumerazione dei suoi elementi, quindi con la quintupla:
$$A=(Q,\Sigma,\delta,q_0,F)$$
con $A$ nome del DFA, $Q$ insiem degli stati, $\Delta$ rappresentante i simboli di input, $\delta$ la sua funzione di transizione, $q_0$ il suo stato iniziale e $F$ l'insieme degli stati accettanti.\\
Vediamo come decidere se accettare o meno una stringa (sequenza di caratteri) in input mediante un DFA.\\
Ho una sequenza in input $a_1...a_n$. Parto dallo stato iniziale $q_0$, consultando la funzione di transizione $\delta$, per esempio  $\delta(q_0,a_1)=q_1$ e trovo lo stato in cui il DFA entra dopo aver letto $a_1$. Poi passo a $\delta(q_1,a_2)=q_2$ e così via, $\delta(q_{i-1},a_i)=q_i$ fino a ottenere $q_n$. Se $q_n$ è elemento di $F$ allora $a_1...a_n$ viene accettato, altrimenti viene rifiutato.
\begin{esempio}
specifico DFA che accetta tutte le strighe binarie in cui compare la sequenza 01:
$$L=\{w|w \mbox{ è della forma x01y, con x e y pari a 0 o 1} \}=\{01,11010,100011,...\}$$
o anche:
$$L=\{x01y| x,y\in\{0,1\}^* \}$$
abbiamo quindi:
$$\Sigma=\{0,1\}$$
ragioniamo sul fatto che $A$:
\begin{enumerate}
\item se ha "già visto" 01, accetterà qualsiasi input
\item pur non avendo ancora visto 01, l'input più recente è stato 0, cosicché se ora vede un 1 avrà visto 01
\item non ha ancora visto 01, ma l'input più recente è nullo (siamo all'inizio), in tal caso A non accetta finché non vede
uno 0 e subito dopo un 1
\end{enumerate}
la terza condizione rappresenta lo stato iniziale. All'inizio bisogna vedere uno 0 e poi un 1. Ma se nello stato $q_0$ si vede per primo un 1 allora non abbiamo fatto alcun passo verso 01, e dunque dobbiamo permanere nello stato $q_0$, $\delta(q_0,1)=q_0$. D'altra parte se nello stato iniziale vedo 0 siamo nella seconda condizione, uso quindi $q_2$ per questa condizione, si avrà quindi $\delta(q_0,0)=q_2$. Vedo ora le transizoni di $q_2$, se vedo 0 ho che 0 è l'ultimo simbolo incontrato quindi uso nuovamente $q_2$, $\delta(q_2,0)=q_2$, in attesa di un 1. Se arriva 1 passo allo stato accertante $q_1$ corrispondente alla prima condizione, $\delta(q_2,1)=q_1$. Ora abbiamo incontrato 01 quindi può succedere qualsiasi cosa e dopo qualsiasi cosa accada potremo nuovamente aspettarci qualsiasi cosa, ovvero $\delta(q_1,0)=\delta(q_1,1)=q_1$. Si deduce quindi che:
$$Q=\{q_0,q_1,q_2\} \mbox{ e } F=\{q_1\}$$
quindi:
$$A=\{\{q_0,q_1,q_2\} ,\{0,1\}, \delta, q_0, \{q_1\} \}$$
con in totale le seguenti transizioni:
$$\delta(q_0,1)=q_0$$
$$\delta(q_0,0)=q_2$$
$$\delta(q_2,0)=q_2$$
$$\delta(q_2,1)=q_1$$
$$\delta(q_1,0)=q_1$$
$$\delta(q_1,1)=q_1$$
posso rappresentarle in maniera tabulare, con lo stato inizale indicato da $\to$ e quelli accettanti con $*$:
\begin{center}
\begin{tabular}{c|c|c}
$\delta$ & 0 & 1 \\
\hline
$\to\,q_0$ & $q_1$ & $q_0$\\
\hline
$*\,q_1$ & $q_1$ & $q_1$\\
\hline
$q_2$ & $q_2$ & $q_1$
\end{tabular}
\end{center}
o col diagramma di transizione:
\begin{center}
\begin{tikzpicture}[shorten >=1pt,node distance=2cm,on grid,auto]
   \node[state,initial] (q_0)   {$q_0$};
   \node[state] (q_2) [right=of q_0] {$q_2$};
   \node[state, accepting] (q_1) [right=of q_2] {$q_1$};
   \path[->]
   (q_0) edge  node {0} (q_2)
         edge [loop below] node {1} ()
   (q_2) edge  node  {1} (q_1)
         edge [loop below] node {0} ()
   (q_1) edge [loop below] node {0,1} ();
\end{tikzpicture}

\end{center}
\end{esempio}
\newpage
\begin{esempio}
Trovo automa per: $$L=\{w\in\{a,b\}^*|\mbox{ w che contiene un numero pari di b}\}$$
\begin{center}
\begin{tikzpicture}[shorten >=1pt,node distance=3cm,on grid,auto]
	\node[state, initial, accepting] (q_0) {$q_0$};
	\node[state] (q_1) [right=of q_0] {$q_1$};
	\path[->]
	(q_0) edge [bend left = 25] node {b} (q_1)
	      edge [loop below] node {a} ()
	(q_1) edge [bend left = 25] node {b} (q_0)
	      edge [loop below] node {a} ();
\end{tikzpicture}
\end{center}
ovvero se da $q_0$ vado a $q_1$ sono obbligato ab generare due $b$, dato che il nodo accettnate è $q_0$. In entrambi i nodi posso generare quante $a$ voglio.
\end{esempio}
\begin{esempio}
Trovo automa per: $$L=\{w\in\{a,b\}^*|\mbox{ w che contiene un numero dispari di b}\}$$
\begin{center}
\begin{tikzpicture}[shorten >=1pt,node distance=3cm,on grid,auto]
	\node[state, initial] (q_0) {$q_0$};
	\node[state, accepting] (q_1) [right=of q_0] {$q_1$};
	\path[->]
	(q_0) edge [bend left = 25] node {b} (q_1)
	      edge [loop below] node {a} ()
	(q_1) edge [bend left = 25] node {b} (q_0)
	      edge [loop below] node {a} ();
\end{tikzpicture}
\end{center}
ovvero se da $q_0$ vado a $q_1$ sono obbligato ab generare una sola $b$, dato che il nodo accettnate è $q_1$. In entrambi i nodi posso generare quante $a$ voglio e posso tornare da $q_1$ a $q_0$ per generare altre $b$.
\end{esempio}
\begin{esempio}
Trovo automa per: $$L=\{w\in\{0,1\}^*| w= 0^n1^m\}$$
vediamo i vari casi:
\begin{itemize}
\item $n,m\geq 0$:
\begin{center}
\begin{tikzpicture}[shorten >=1pt,node distance=2cm,on grid,auto]
   \node[state,initial, accepting] (q_0)   {$q_0$};
   \node[state, accepting] (q_1) [right=of q_0] {$q_1$};
   \node[state] (q_E) [right=of q_1] {$q_E$};
   \path[->]
   (q_0) edge  node {1} (q_1)
         edge [loop below] node {0} ()
   (q_1) edge  node  {} (q_E)
         edge [loop below] node {1} ()
   (q_E) edge [loop below] node {0,1} ();
\end{tikzpicture}
\end{center}
ovvero posso non generare nulla e uscire subito con $q_0$, generare solo un 1 e passare a $q_1$ e uscire oppure generare 0 e 1 a piacere con l'ultimo stato o generare 0 a piacere dal primo e 1 a piacere dal secondo.
\item $n\geq 0 \,\,m>0$:
\begin{center}
\begin{tikzpicture}[shorten >=1pt,node distance=2cm,on grid,auto]
   \node[state,initial] (q_0)   {$q_0$};
   \node[state, accepting] (q_1) [right=of q_0] {$q_1$};
   \node[state] (q_E) [right=of q_1] {$q_E$};
   \path[->]
   (q_0) edge  node {1} (q_1)
         edge [loop below] node {0} ()
   (q_1) edge  node  {} (q_E)
         edge [loop below] node {1} ()
   (q_E) edge [loop below] node {0,1} ();
\end{tikzpicture}
\end{center}
ovvero come l'esempio sopra solo che non posso uscire in $q_0$ in quanto almeno un 1 deve essere per forza generato
\item $n> 0\,\, m\geq 0$:
\begin{center}
\begin{tikzpicture}[shorten >=1pt,node distance=2cm,on grid,auto]
   \node[state,initial] (q_0)   {$q_0$};
   \node[state, accepting] (q_1) [right=of q_0] {$q_1$};
   \node[state, accepting] (q_2) [right=of q_1] {$q_2$};
   \node[state] (q_E) [right=of q_2] {$q_E$};
   \path[->]
   (q_0) edge  node {0} (q_1)
         edge [bend right] node {1} (q_E)
   (q_1) edge  node {1} (q_2)
         edge [loop above] node {0} ()
   (q_2) edge node {0} (q_E)
         edge [loop above] node {1} ()
   (q_E) edge [loop below] node {0,1} ();
\end{tikzpicture}
\end{center}
\textit{CHIARIRE}
\item $n,m>0$:
\begin{center}
\begin{tikzpicture}[shorten >=1pt,node distance=2cm,on grid,auto]
   \node[state,initial] (q_0)   {$q_0$};
   \node[state] (q_1) [right=of q_0] {$q_1$};
   \node[state, accepting] (q_2) [right=of q_1] {$q_2$};
   \node[state] (q_E) [right=of q_2] {$q_E$};
   \path[->]
   (q_0) edge  node {0} (q_1)
         edge [bend right] node {1} (q_E)
   (q_1) edge  node {1} (q_2)
         edge [loop above] node {0} ()
   (q_2) edge node {0} (q_E)
         edge [loop above] node {1} ()
   (q_E) edge [loop below] node {0,1} ();
\end{tikzpicture}
\end{center}
\textit{CHIARIRE}
\end{itemize}
\end{esempio}
\newpage
\begin{esempio}
Trovo automa per: $$L=\{w\in\{a,b\}^*|\mbox{ w che contiene un numero pari di a e dispari di b}\}$$
\begin{center}
\begin{tikzpicture}[shorten >=1pt,node distance=2cm,on grid,auto]
	\node[state, initial] (q_0) {$q_{pp}$};
	\node[state] (q_1) [right=of q_0] {$q_{dp}$};
	\node[state, accepting] (q_2) [below=of q_0] {$q_{pd}$};
	\node[state] (q_3) [right=of q_2] {$q_{dd}$};
	\path[->]
	(q_0) edge [bend left = 25] node {a} (q_1)
	      edge [bend right = 25] node [left] {b} (q_2)
	(q_1) edge [bend left = 25] node {a} (q_0)
	      edge [bend right = 25] node [left] {b} (q_3)
	(q_2) edge [bend right = 25] node [right] {b} (q_0)
	      edge [bend left = 25] node {a} (q_3)
	(q_3) edge [bend right = 25] node [right] {b} (q_1)
	      edge [bend left = 25] node {a} (q_2);
\end{tikzpicture}
\end{center}
\end{esempio}
\begin{esempio}
Trovo automa per: $$L=\{w\in\{a,b\}^*|\mbox{ w che contiene un numero pari di a seguito da uno dispari di b}\}$$
$$L=\{a^{2n}b^{2k+1}|j,k\geq 0\}$$
\begin{center}
\begin{tikzpicture}[shorten >=1pt,node distance=2cm,on grid,auto]
	\node[state, initial] (q_0) {$q_{0}$};
	\node[state, accepting] (q_1) [right=of q_0] {$q_{1}$};
	\node[state] (q_2) [below=of q_0] {$q_{2}$};
	\node[state] (q_3) [right=of q_2] {$q_S$};
	\node[state] (q_4) [right = of q_3] {$q_E$};
	\path[->]
	(q_0) edge [bend left = 25] node {b} (q_1)
	      edge [bend right = 25] node [left] {a} (q_2)
	(q_1) edge [bend left = 25] node {a} (q_4)
	      edge [bend right = 25] node [left] {b} (q_3)
	(q_2) edge [bend right = 25] node [below] {b} (q_4)
	      edge [bend right = 25] node [right] {a} (q_0)
	(q_3) edge [bend right = 25] node [right] {b} (q_1)
	      edge [bend left = 25] node {a} (q_4);
\end{tikzpicture}
\end{center}
ovvero in tabella:
\begin{center}
\begin{tabular}{c|c|c}
$\delta$ & a & b \\
\hline
$\to\,q_0$ & $q_1$ & $q_2$\\
\hline
$q_1$ & $q_0$ & $q_E$\\
\hline
$*\,q_2$ & $q_E$ & $q_3$\\
\hline
$q_S$ & $q_E$ & $q_2$\\
\hline
$q_E$ & $q_E$ & $q_E$
\end{tabular}
\end{center}
\end{esempio}
\begin{esempio}
Trovo automa per: $$L=\{a^{2k+1}b^{2h}|\, h,k\geq 0\}$$
\begin{center}
\begin{tikzpicture}[shorten >=1pt,node distance=2cm,on grid,auto]
	\node[state, initial] (q_0) {$q_{0}$};
	\node[state, accepting] (q_1) [right=of q_0] {$q_{1}$};
	\node[state] (q_3) [right=of q_1] {$q_{3}$};
	\node[state] (q_2) [below= of q_1] {$q_{2}$};
	\node[state, accepting] (q_4) [right = of q_2] {$q_4$};
	\node[state] (q_5) [right=of q_4] {$q_E$};
	\path[->]
	(q_0) edge  node [bend left = 25] {a} (q_1)
	(q_1) edge [bend left = 25] node {a} (q_2)
	      edge node [bend left= 25] {b} (q_3)
	(q_2) edge [bend left = 25] node [left] {a} (q_1)
	(q_3) edge [bend right = 25] node [left] {b} (q_4)
	(q_4) edge [bend right = 25] node {} (q_3);
\end{tikzpicture}
\end{center}
\end{esempio}
\begin{esempio}
Trovo automa per: $$L=\{a^{2n+1}b^{2k+1}|\, n,k\geq 0\}$$
\begin{center}
\begin{tikzpicture}[shorten >=1pt,node distance=2cm,on grid,auto]
	\node[state, initial] (q_0) {$q_{0}$};
	\node[state] (q_1) [right=of q_0] {$q_{1}$};
	\node[state, accepting] (q_3) [right=of q_1] {$q_{3}$};
	\node[state] (q_2) [below= of q_1] {$q_{2}$};
	\node[state] (q_4) [right = of q_2] {$q_4$};
	\node[state] (q_5) [right=of q_4] {$q_E$};
	\path[->]
	(q_0) edge  node [bend left = 25] {a} (q_1)
	(q_1) edge [bend left = 25] node {a} (q_2)
	      edge node [bend left= 25] {b} (q_3)
	(q_2) edge [bend left = 25] node [left] {a} (q_1)
	(q_3) edge [bend right = 25] node [left] {b} (q_4)
	(q_4) edge [bend right = 25] node [right] {b} (q_3);
\end{tikzpicture}
\end{center}
\end{esempio}
\begin{esempio}
Trovo automa per: $$L=\{x010y|\,x,y\in\{o,1\}^*\}$$
\begin{center}
\begin{tikzpicture}[shorten >=1pt,node distance=2cm,on grid,auto]
	\node[state, initial] (q_0) {$q_{0}$};
	\node[state] (q_1) [right=of q_0] {$q_{1}$};
	\node[state] (q_2) [right=of q_1] {$q_{2}$};
	\node[state, accepting] (q_3) [right= of q_2] {$q_{3}$};
	\path[->]
	(q_0) edge  node {0} (q_1)
	      edge [loop above] node {1} ()
	(q_1) edge  node {1} (q_2)
	      edge [loop above] node {0} ()
	(q_2) edge  node {0} (q_3)
	(q_3) edge [loop below] node {0,1} ();
\end{tikzpicture}
\end{center}
\end{esempio}
\subsection{Automi non deterministici}
In sigla \textit{NFA} sono delle quintuple $A=(Q, \Sigma, \delta, q,F$ ma ora la delta prende uno stato e un simbolo e ci manda in un insieme di stati:
$$\delta:Q\times \Sigma\to 2^Q$$
delta è delta cappuccio
$$DFA:\,\,\, L(A)=\{w\in \Sigma^*|\,\delta(q_0,w)\in F\}$$
$$NFA:\,\,\, L(A)=\{w\in \Sigma^*|\,\delta(q_0,w)\cap F \neq \emptyset\}$$
=======
      \end{equation*}
    \end{itemize}
  \end{itemize}

    Dopo aver definito le operazioni sui linguaggi regolari, definiamo ora:
    \begin{defi}
      Si definisce \emph{espressione regolare} induttivamente come segue, considerando anche il linguaggio che generano:
      \begin{description}
      \item [caso base]: la base consiste in 3 parti:
        \begin{enumerate}
        \item $\epsilon$ e $\emptyset$ sono RegEx e generano $L(\epsilon) = \{epsilon\}$, $L(\emptyset) = \emptyset$
        \item se $a$ è un simbolo allora $a$ è una Regex e questa espressione genera $L(a) = \{a\}$
        \item una variabile, rappresentanti linguaggi regolari, sono RegEx, e generano $L(L) = L$
          \end{enumerate}

      \item [caso induttivo]: la parte induttiva delle espressioni regolari sono composte da 4 tipologie:
        \begin{enumerate}
        \item se $E$ e $F$ sono delle RegEx allora $E + F$ è una RegEx e rappresentano $L(E + F) = L(E) \cup L(F)$
        \item se $E$ e $F$ sono delle RegEx allora $E \circ F = EF$ è una RegEx per cui rappresentano $L(EF) = L(E)L(F)$
        \item se $E$ è una RegEx allora $E*$ è una RegEx, che denota la chiusura di $L(E)$ infatti $L(E^*) = (L(E))^*$
        \item se $E$ è una RegEx allora $(E)$ è una RegEx in cui $L((E)) = L(E)$.
        \end{enumerate}
      \end{description}
   \end{defi}

>>>>>>> 6dce03e48312df68955becb5927d04e110d17342
\end{document}
