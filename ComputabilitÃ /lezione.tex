Automi a stati finiti deterministici
dispositivi di calcolo di riconoscimento di un linguaggio di tipo 3(Regolari)

Def: Un automa a stati finiti è un quintupla $A = (Q,\Sigma,\delta, q_0, F)$

dove $Q$ è un insieme finito non vuoto di stati,
$\Sigma$ è un alfabeto di ingresso,
$\delta$ è la funzione di transizione, ossia stabilisce come deve comportarsi l'automa durante un passo.
$q_0 \in Q$ è lo stato iniziale della computazione,
$F \subseteq Q$ è l'insieme degli stati finali accettati dall'automa.

La funzione $\delta: Q \times \Sigma \to Q$

Esempio pagina 44:
